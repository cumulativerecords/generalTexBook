\chapter{About This Book}

This book was made by GNU/Linux computers, under supervision of a BCBA.

The second edition has advantages over the first. Its user license guarantees the freedoms specified under the Creative Commons 4.0 - Attribution - Sharealike - Non-commercial (CC 4.0-BY-SA-NC). The license provides readers with the freedom to use, copy, modify, and distribute the book non-commercially. This is a good thing for behavior analysts because we like to customize everything. Modified and copied versions retain the same freedoms as the original work. There is no need to ask the publisher for permission to reprint the book's contents. 

However, no amount of licensing is useful if users have difficulty accessing the full manuscript text in an editable format. This version uses a typesetting system that users can easily customize. It is maintained by Cumulative Records Documentation Society (CRDS), a 501c3 nonprofit. The nonprofit has secured a lifelong sponsorship from GitHub to host the full manuscript and source code. This means you can re-brand with your company name, change the contents or sections, etc. It also means the book can be improved by creating an Issue in GitHub, which a maintainer from the nonprofit can address. Such changes will improve the main version of the book, distributed to all. All previous versions will remain available. The technology has incredible applications for behavior analysts. CRDS is a BACB-authorized continuing education provider and will offer workshops to help users take advantage of these capabilities.

These advancements make the second edition far superior to the traditional "all rights reserved" copyright used in the first edition printings. This edition also offers superior typesetting technology. Content is better due to richer connections to the literature, more performance measures for assessment, and more generality in examples.

The book is distributed freely with paper copies sold at a reasonable price. The profits go to a public charity (CRDS) to advance the contents of the book.

\section{History of the TrainABA Supervision Curriculum Series}
This book originated from a project from TrainABA, a startup organization from 2013-2016. Its goal was to function as a publisher and resource for behavior analysis supervision. It was unsuccessful. When TrainABA closed, the publisher released its works under a Creative Commons 4.0 - Attribution - Sharealike - Non-commercial license. Some of its works survived as a project, such as the free Moodle Course, manuscripts, and SAFMEDs app. These works were donated to CRDS to be developed as a community edition for public use. 

\section{Publisher}
The publisher is CRDS, a 501(c)3 nonprofit based in Los Angeles, California, USA. CRDS produces archive-quality continuing education materials for public use. CRDS employs technical producers and project maintainers to develop and distribute works. CRDS survives on the generosity of its members. If your company uses these materials, we ask that you donate a reasonable amount to support the cause. The donation is to make sure these high-quality materials will continue to be available to your company in the future. To make a donation, or to become a member, please visit \url{http://cumulativerecords.org}.

\section{Collaboration Tools}
CRDS built this book using collaboration tools from software developers. Anyone can contribute or suggest changes for free. There will be a permanent public record of any such collaboration. We encourage readers to report errors using the Issue Tracker on our GitHub repository. The location is: \url{https://github.com/cumulativerecords/trainaba-v1-ed2/issues}

\subsection{Creating New Materials from This Book}
Readers can and should extend the book's contents (e.g., build a slideshow to be used where one works or teaches). All readers are invited to suggest changes to this book using the GitHub repository. For readers who have modified the contents to be used where they work or teach, we ask that you submit your materials to CRDS so that we can make them available to other readers. We believe this will afford us the opportunity to have one or two well-developed versions of a work, which are compatible with the original book. We believe one organized version is better than multiple partially-developed, incompatible but similar works. 

%Readers who create materials retain credit for their contributions. In many cases, CRDS will supply content creators with a technical producer who will help them organize the materials for larger audiences at no charge. Organizations who use CRDS materials regularly for personnel development provide donations.

%\subsection{Online Documentation}
%The materials in this book will likely be distributed as online documentation from \url{https://ReadTheDocs.org}. More information about this project will be made available at a later release. 

\section{Versions}
The typesetting system used to compile this work is very flexible. It can compile a similar version for nearly any page size with a very simple change in code. It is designed to provide maximum flexibility to readers, who are often supervisors and educators with a need to use only certain sections of this work. By downloading the source code, readers are able to pick and choose which sections of the book to compile. They can rebrand the book to indicate that they modified the original version. We encourage readers to tinker with the source code to download modified versions of the work. It is surprisingly easy to make a mobile-friendly version of this book. One can also make a new version for each month in a supervision setting, for example. CRDS is available to provide customizations. To request a custom size or version, contact CRDS at \url{http://cumulativerecords.org/contact}.




