%Frontmatter
%
%
%SECTION ONE
%Getting Started
%
\chapter{Introduction}
\epigraph{``This book was made by GNU/Linux computers under supervision of a BCBA.''}{\textit{Host Name: gnu-recrdsX200\\Ser. No. 9817325619287}}

The license and technology make a TrainABA book worth using. The benefit to TrainABA supervision curriculum is the freedoms its license provides. Behavior analysts can readily assign definitions, examples, performance measures, related tasks, and recommended readings for each task list item. Contents are available in plain text. Modifications are encouraged under the license terms. The book interiors are organized with the \LaTeX{} typesetting system to compile custom PDF documents. Later sections of this book describe the Creative Commons 4.0 (CC 4.0) license and how to use it. 

TrainABA supervision curriculum was designed in 2015 as a ``vanilla base''  to be customized by users for specific needs. At the time, the best available plan was to build a project that could distribute updates and releases. The vanilla model was an alternative to using random, apocryphal materials. Unknown sources use unknown licenses, inadvertently inviting infringement of intellectual property. Apocryphal materials are not distributed. When users cannot track modified versions, improved versions may go unnoticed because search engines prioritize older versions. For the second edition, TrainABA moved the project to Cumulative Records Documentation Society (CRDS), a 501(c)(3) nonprofit. 

TrainABA has long-term support thanks to its licenses, CRDS, and GitHub. CRDS provides staff to manage the development of TrainABA materials. GitHub, a software development company, provided CRDS with a lifelong donation to its platform. GitHub's platform gives TrainABA a distribution platform, issue tracker, and version control with releases.

Customizations are expected in professional settings. Supervisors customize materials for cohorts. Interns modify sections as preparation for the exam. Behavioral service providers operate by service hours where speed counts. The license used by TrainABA allows users to bootstrap their own materials by repurposing TrainABA book content, add to it, and redistribute those modifications for non-commercial use. The modified contents can be made available to the general behavior analysis community. All of these activities are lawful under the freedoms specified in TrainABA's license. Organizations can rely on TrainABA curriculum as a long-term strategy due to the license. TrainABA relies on donations to CRDS to continue developing TrainABA. 

The CC 4.0 license offers freedoms for professionals. Traditional copyright licenses require users to obtain written permission from publishers for every use of copyrighted materials. Behavior analysts tend to avoid this step in professional settings. The CC 4.0 license is a better fit. It allows anyone to copy, modify, and redistribute materials for non-commercial use. 

This book represents a second edition of the TrainABA Supervision Curriculum system released in 2015.

The items listed below reflect the initial features of the TrainABA Supervision Curriculum systems. All of these have been made available for a number of years under open source licenses. The volumes include:

\begin{enumerate}
\item TrainABA Supervision Curriculum: BCBA Reference Manual (Volume 1)
\item TrainABA Supervision Curriculum: Independent Fieldwork (Volume 2)
\item TrainABA Supervision Curriculum: RBT Credential (Volume 3)
\end{enumerate}

Target outcomes include: 
\begin{itemize}
\item Grow management teams with less challenges
\item Get from start to finish with a page-by-page, week-by-week program (from supervision contract to  fieldwork to BACB\textregistered{}{} application) 
\item (Removed because the BACB\textregistered{}{} created an alternative) Find systems to help you track supervision hours and signature forms to email to the BACB\textregistered{}
\item Use Individual and Group meeting agendas
\item Save time by bootstrapping materials from TrainABA contents 
\item Check pre-assigned homework
\item Track ongoing progress on the 4th Edition Task List™ assessment 
\item Prepare for BACB\textregistered{}-exam with test topics built into fieldwork
\item Organize essential supervision materials and meetings in one place, accessible by mobile or desktop devices
\end{itemize}

TrainABA was a BACB\textregistered{}-Approved Continuing Education (ACE) provider from 2014-2016. It specialized in responding to supervision-related problems that were complex and time-consuming. A different ACE provider was selected to provide TrainABA-related continuing education events beginning in 2019. Its name is Cumulative Records Documentation Society (CRDS). 

The 4th Edition Task List™ face sheets were organized by ``segments'' to make it easier to complete and check homework assignments. Segments were a special term created for the first edition. ``Modules'' is a more universal term conveying the same meaning. The second edition uses the term modules in place of segments. 

TrainABA Supervision Curriculum: Independent Fieldwork contains exercises for the supervisee to complete in a week-by-week progression that cover all 168 items on the 4th Edition Task List with the following:
\begin{enumerate}
\item Individual meeting agendas
\item Group meeting agendas
\item Ongoing homework assignments
\item 4th Edition Task List™ assessment
\end{enumerate}

Remember that documentation guidelines are not included in the TrainABA supervision curriculum. Follow procedures provided directly by the board.
 
\section{About the Second Edition}
The second edition has advantages over the first. Its user license guarantees the freedoms specified under the Creative Commons 4.0 - Attribution - Sharealike - Non-commercial (CC 4.0-BY-SA-NC). The license provides readers with the freedom to use, copy, modify, and distribute the book non-commercially. This is a good thing for behavior analysts because we like to customize everything. Modified and copied versions retain the same freedoms as the original work. There is no need to ask the publisher for permission to reprint the book's contents. 

However, no amount of licensing is useful if users have difficulty accessing the full manuscript text in an editable format. This version uses a typesetting system that users can easily customize. It is maintained by Cumulative Records Documentation Society (CRDS), a 501(c)(3) nonprofit. The nonprofit has secured a lifelong sponsorship from GitHub to host the full manuscript and source code. This means you can re-brand with your company name, change the contents or sections, etc. It also means the book can be improved by creating an Issue in GitHub, which a maintainer from the nonprofit can address. Such changes will improve the main version of the book, distributed to all. All previous versions will remain available. The technology has incredible applications for behavior analysts. CRDS is a BACB-authorized continuing education provider and will offer workshops to help users take advantage of these capabilities.

These advancements make the second edition far superior to the traditional ``all rights reserved'' copyright used in the first edition printings. This edition also offers superior typesetting technology. Content is better due to richer connections to the literature, more performance measures for assessment, and more generality in examples.

The book is distributed freely with paper copies sold at a reasonable price. The profits go to a public charity (CRDS) to advance the contents of the book.

\section{History of the TrainABA Supervision Curriculum Series}
This book originated from a project from TrainABA, a startup organization from 2013-2016. Its goal was to function as a publisher and resource for behavior analysis supervision. It was unsuccessful. When TrainABA closed, the publisher released its works under a Creative Commons 4.0 - Attribution - Sharealike - Non-commercial license. Some of its works survived as a project, such as the free Moodle Course, manuscripts, and SAFMEDs app. These works were donated to CRDS to be developed as a community edition for public use. 

\section{Publisher}
The publisher is CRDS, a 501(c)(3) nonprofit based in Los Angeles, California, USA. CRDS produces archive-quality continuing education materials for public use. CRDS employs technical producers and project maintainers to develop and distribute works. CRDS survives on the generosity of its members. If your company uses these materials, we ask that you donate a reasonable amount to support the cause. The donation is to make sure these high-quality materials will continue to be available to your company in the future. To make a donation, or to become a member, please visit \url{http://cumulativerecords.org}.

\section{Collaboration Tools}
CRDS built this book using collaboration tools from software developers. Anyone can contribute or suggest changes for free. There will be a permanent public record of any such collaboration. We encourage readers to report errors using the Issue Tracker on our GitHub repository. The location is: \url{https://github.com/cumulativerecords/trainaba-v1-ed2/issues}

\subsection{Creating New Materials from This Book}
Readers can and should extend the book's contents (e.g., build a slideshow to be used where one works or teaches). All readers are invited to suggest changes to this book using the GitHub repository. For readers who have modified the contents to be used where they work or teach, we ask that you submit your materials to CRDS so that we can make them available to other readers. We believe this will afford us the opportunity to have one or two well-developed versions of a work, which are compatible with the original book. We believe one organized version is better than multiple partially-developed, incompatible but similar works. 

%Readers who create materials retain credit for their contributions. In many cases, CRDS will supply content creators with a technical producer who will help them organize the materials for larger audiences at no charge. Organizations who use CRDS materials regularly for personnel development provide donations.

%\subsection{Online Documentation}
%The materials in this book will likely be distributed as online documentation from \url{https://ReadTheDocs.org}. More information about this project will be made available at a later release. 

\section{Versions}
The typesetting system used to compile this work is very flexible. It can compile a similar version for nearly any page size with a very simple change in code. It is designed to provide maximum flexibility to readers, who are often supervisors and educators with a need to use only certain sections of this work. By downloading the source code, readers are able to pick and choose which sections of the book to compile. They can rebrand the book to indicate that they modified the original version. Readers are invited to tinker with the source code to download modified versions of the work. It is surprisingly easy to make a mobile-friendly version of this book, for example. One can also make a new version for each month in a supervision setting. CRDS is available to provide customizations for organizations and universities. To request a custom version, contact CRDS at \url{http://cumulativerecords.org/contact}.
\chapter{Supervision Taxonomy}
%
In this chapter:
\begin{enumerate}
\item What is ABA Supervision?
\item Taxonomy of ABA Supervision
\item BACB\textregistered{} Influence in the Development of Formal Behavior Analysis Supervision
\item Case Study: Associated Aardvarks for Autism
\item Resources for Supervised Experience
\end{enumerate}

\section{What is ABA Supervision?}
Most definitions indicate that it is the act of providing supervision or oversight, involving directing in relation to execution and performance.

\subsection{Supervision in Management}
``Supervision'' is most used in relation to business administration, in the realm of management. Human supervisors are employed to supervise humans and systems. 
 
\subsection{Supervision of Humans and Systems}
It would be unusual to use the term ``supervisor'' to describe a machine that monitors information (including that generated by a human), or to describe a human that oversees information generated solely by a machine. The typical use of the word, ``supervisor'' implies a human to human interaction related to execution or performance. 
  
\subsection{Supervision in ABA Settings}
The words, ``execution'' and ``performance'' are commonly used in business settings and could be used within the context of behavior analysis. However, more precise language further locates ``supervision'' in the field of applied behavior analysis. A manager at a typical business corporation ``supervises'' an employee to monitor performance and execution of tasks. In ABA settings, managers would more likely say they, ``supervise staff to measure how effectively they implement a behavioral plan.'' 
 
\subsection{Direct vs. Indirect ABA Supervision}
One might also describe ABA supervision as, ``Overseeing direct implementation'' or ``following the procedures. An ABA subordinate may be evaluated on procedural integrity, procedural drift, and other direct measures of staff performance. The client's progress may be a direct measure of the subordinate's performance. However, client progress is an indirect measure of supervision efficacy. 
 
\subsection{Management Problems in ABA Organizations}
The proliferation of autism behavior services has accompanied the rise in management problems at private practice ABA agencies, centers, and special education school settings. Management at such organizations have not widely adopted management practices to solve the repetitive problems occurring at the workplace in daily operations. 
 
What maintains management problems? The organization's decision makers: 
\begin{enumerate}
\item Do not know how to identify the problem.
\item Do not know how to solve it.
\item Have attempted solutions in the past, believing they exhausted possible solutions.
\item Lack resources such as consultants and management workshops.
\end{enumerate}

The above reasons are potential barriers if all resources are equal. However, more important barriers include time constraints, lack of money to pay consultants, and a tendency to focus on work that is expressly billable as opposed to management strategies, which require time, creativity, and a willingness to gather and analyze data over time to find ``what works'' at a particular organization using an experimental approach. 
 
\subsection{What Is Being Done}
It is perhaps surprising that behavior analysis professionals are not leading the managerial movement to solve such problems, given the curious nature of behavior analysis professionals and their propensity for solving problems in the world around them.

\section{A Taxonomy of ABA Supervision}
In the 8 Hour Supervisor Training workshops, TrainABA moderators generally introduce the supervision taxonomy by addressing the ``big picture'' of policy-level issues governing the professional practice of applied behavior analysis. The presenter posits that such policy both necessitated and helped define how ABA supervision would be practiced. Policy varies across countries, states, and provinces, and various funding sources share properties with specific differences. 

The development of ABA supervision is an ongoing, collaborative process. It has evolved along with growing numbers of behavioral service providers. ABA supervision existed prior to the formation of the Behavior Analyst Certification Board\textregistered{} (BACB\textregistered{}) in 1999. It has set the standard for professional ABA services.

After introductions, the workshop presenter often asks the following set of questions:

\begin{enumerate}
\item How Does Policy Influence How Funding Sources Choose Providers? 
\item How do funding source requirements influence the professional practice of applied behavior analysis services?
\end{enumerate}

In the USA, the answer is generally that policy is written and a licensing body enforces compliance. However, the professional practice of behavior analysis is currently experiencing an early stage developmental period. More states are passing legislation. Some states are still in the process of licensure for ABA professionals. 

Generally, the process involves policy language for licensure that acknowledges the BACB\textregistered{} certification credential and identifies an established licensing board to regulate practitioners. Other practices, such as psychology, have their own licensing boards and may opt to include behavior analysts within their board. 

Other service delivery professions, such as psychology, have similar requirements as the BACB\textregistered{}, such as required education, supervised experience (1500 hours), ethical compliance code, continuing education requirements, etc. 

As states adapt to the growing demand for professional behavior analytic services, many have acknowledged the BACB\textregistered{} certification as a requirement for billing. The BACB\textregistered{} is not a licensing body but serves as the central regulatory body for certified professional behavior analysts. It is the authoritative body for certification and credentialing in professional ABA services around the world. The BACB\textregistered{} is an influential global organization. Its international impact has been possible, in part, because it is not bound by a specific state or federal government. Such would not be the case if the BACB\textregistered{} was created as a licensing organization in Florida, its state of origin. The BACB\textregistered{} is currently headquartered in Littleton, Colorado, USA. Its strong influence in shaping the practice of professional behavior analysis services merits a prominent role in a taxonomy for ABA supervision today. 

\begin{itemize}
\item Licensing body standardizes practitioner KSAs, ethics, practice (medicine, psychology, counseling, etc.).
\item If no licensing body available, some authoritative body for certification or credentialing assumes that role (BACB in 1998).
\end{itemize}

\section{Development of Behavior Analysis Supervision}

The BACB\textregistered{} established a model for the professional delivery of behavior analytic services for insurance providers in 2012. It involved a hierarchy upon which a BCBA or BCBA-D oversees a BCaBA, who supervises a behavior technician. It is common practice for companies to omit the BCaBA. In such cases, the BCBA or BCBA-D may oversee the behavior technicians directly.

Individuals with BACB\textregistered{} certification are not required to supervise. Some certified practitioners work directly with clients, particularly in group home settings and other consulting situations where monthly hours are low and hiring a direct implementation professional would not be appropriate. However, the global rise in autism diagnoses has warranted a high demand for appropriate structure of professional behavior analytic services that serve children with autism. Such services are delivered in homes, schools, and centers. Applied behavior analysis practitioners typically train and supervise professionals who implement behavior analytic programming directly with staff. 

Generally, the certificant acts as a supervisor who analyzes data, conducts most or all elements of the assessment, designs and develops behavioral programming, and reports on progress. It is typical for ABA certificants to function in a supervisory role under such a service delivery model. However, not every certificant supervises staff.

Most, but not all certificants, supervise clinical staff. Some work directly with clients or in research roles.

The BACB\textregistered{} established the BCBA, BCBA-D and BCaBA credentials in its early years. 

In the summer 2014, the BACB\textregistered{} introduced the Registered Behavior Technician (RBT) Credential. This was a standardized credential for individuals who provided direct implementation of behavior analysis programs. 

It should be noted that some confusion over terminology has arisen among practitioners as the RBT credential is becoming more common. 

A rule-of-thumb:

``Certification'' is for Supervisors and ``credentialing'' is for Direct Implementation staff.

The ABA Supervision taxonomy, therefore, applies to individuals who either hold or are candidates for BACB\textregistered{} certification. RBT Credentialees are supervised by individuals who hold a BACB\textregistered{} certification. However, credentialees do not supervise.

We draw this distinction to help define and locate the meaning of an ABA supervisor. The following chart identifies the basic difference in requirements for supervisors – ABA certification – versus those they supervise – RBT credentialees.

(This book is NOT for developing behavior technicians.)

Behavior Technicians 


ABA Credential

    • RBT 
%
%
%
Training

    • High school diploma\\
    • RBT 40 hour Training\\
    • RBT Assessment\\
    • Fingerprints/RBT application\\


Ongoing Quality Assurance

    • Ethical/Disciplinary Standards\\
    • Ongoing supervision\\

*BCaBAs require ongoing supervision from a BCBA or BCBA-D

The following chart depicts the typical arrangement for ABA service delivery for an organization whose staff hold BACB\textregistered{} credentials.


In the above list, where is the candidate? \textit{Hint: ``Candidate'' refers to the intern accruing hours toward credential requirements.}

Suppose the position is called the supervisor intern. Is the supervisor intern billable? Herein lies the problem – or solution – for many ABA professionals around the world.

Consider an \textit{intern supervisor}, a clinical supervisor who does not hold a certification and is actively accruing supervision hours while working for a behavioral services provider. The position exists in underserved areas where demand for behavioral services is higher than the supply of certified practitioners. 

How does a behavioral service provider establish billing practices for the worker? Policy may be available to guide decision-making. Funding sources may offer guidelines. The challenge at the policy-level is to provide enough legislation to safeguard clients with qualified service providers. Legislation that is too rigid makes it difficult for companies to ``keep the lights on,'' or meet minimum expenses to turn a profit. Often, practitioners assume that a company is making a lot of money because they see clients, employees, laptops, trainings, catered lunches, and office space. Yet organizations in the ABA industry must exercise restraint and deliberation for their business practices. For example, what happens when a company can only be paid for services provided by a BCBA, yet all the BCBAs in the area are gainfully employed? 

Say a local university observes the growing demand for professional behavior analysts has increased. That university creates a certification program as a hybrid or standalone master's degree that satisfies the university hours required by the BACB. A few years pass. The university has graduated its first class. Yet the university does not have a practicum and it is the responsibility of the graduates to complete their supervised experience hours. What would the students do? They would reach out to local companies for employment – after all, it will soon be time to repay student loans – in hopes that a company can provide a training system for the individual.

The recent graduate may face a mixed landscape, shaped by the contingencies the company faces for promoting and/or billing for that supervisor intern. There are a few common scenarios:
\begin{itemize}
\item Two local funding sources reimburses companies ONLY for supervision hours performed by a BCBA.
\item One source formally allows the agency to have the supervision hours performed at the rate of a Behavior Technician (lower rate).
\item The other source does not reimburse work performed by a non-credentialed supervisor. They will discontinue services unless all supervisor hours are performed by a credential holder.
\end{itemize}

\section{Case Study: Associated Aardvarks for Autism (AAA)}

\textbf{Associated Aardvarks for Autism (AAA)} was a fictitious ABA agency. Their director was tasked with designing an internship program for recent graduates of a local ABA master's program. AAA would select one intern to pilot the program. The internship was meant to provide 100 percent of hours toward supervised experience for the BCBA\textregistered{} credential. 

AAA had certain outcomes in mind. At a minimum, they expected the internship program to produce one supervisor. AAA hoped the supervisor, learn the company culture through internship, would indirectly help with personnel issues. This connection was clear to the CEO and HR managers, though it was harder to explain to clinical leadership. Through internship, upper management expected the new supervisor to learn the ``AAA way'' technicians should perform the job. Examples included customer service behaviors. Upper management expected the intern to remain grateful to the company for its investment in their professional development. Recruitment attempts failed to find a comparable supervisor. A good behavior analyst supervisor was hard to find. 

\subsection{AAA's Accountant}
AAA's director asked the accounting department to create Table 2.1, which summarized cashflow for the internship. The accountant assumed they would hire the intern at \$45,000. That included a 50-week year at 30 hour/week, for which half of the hours were to be provided as direct implementation of behavioral programming. The engagement culminated in a complete 1,500-hour internship, satisfying the requirements toward the BACB\textregistered{}'s credentialing application. 

The 30 hours per week led to the quickest possible completion of the supervised experience hours for the ABA credential internship. The accountant interviewed the clinical director and was informed that the BACB\textregistered{} permitted up to 50\% of intern hours for direct implementation. What were the other hours? The accountant asked about scheduling meetings, billing, drive-time, and planning meetings that were not clinical. Unfortunately, those things were considered ``non-clinical'' of ``administrative'' hours. They could not be counted toward the intern's credentialing hours.

The accountant knew that at least half of the intern's hours needed to be clinical but could not be direct implementation of behavioral programming. For that reason, the accountant only calculated 15 hours per week of billable work for the intern to serve as a behavior technician. He knew the other half included program development, report writing, parent education, and staff training. The company had one funding source that allowed interns to bill for these services. However, two other funding sources required the full credential for reimbursement. For that reason, the accountant described the ``other 50\%,'' as he called it, as a gray area category that may or may not be billable. But how could he help offset more of the cost of internship, so the company would not have to pay so much money out of pocket?

\subsection{Funding Sources}
The accountant considered the types of reimbursement contingencies the company faced. The most lenient funding sources allowed the company to bill for the intern's program development and report writing, generally under the supervision of a credentialed individual. Other funding sources will not pay for any supervisor hours performed by a non-credentialed ABA professional. The most conservative funding sources only reimburse for supervision performed in the presence of the client by a credentialed ABA professional. 

In the AAA Company for Table 2.1, the funding source allowed only direct implementation hours to be billed by the intern. These hours are performed by a behavior technician and take the form of direct service.

The accountant recommended AAA to minimize the out-of-pocket expense of the internship by assigning a salaried individual as the supervisor for the 75 hours a supervisor would need to spend supervising that intern.  The director asked how much money that would save. 

\subsection{Estimating Supervision Costs}
The accountant created Table 2.2 to estimate costs if the intern's supervisor would have been paid hourly. The table reflected \$45 per hour for the supervisor wage. It was a safe estimate considering the BACB\textregistered{} and Association of Professional Behavior Analysts (APBA) 2012 study which showed that most supervisor hours were reimbursed \$40-50 per hour. The accountant found those data in a BACB\textregistered{} newsletter from 2012. The director forwarded those emails to him regularly. He was glad to have the opportunity to show he had read them. 

\subsubsection{Billing for Parent Education Groups}
AAA also recognized that the intern could run Parent Education Groups as part of their 50\% of non-direct implementation hours. Fortunately, AAA could be reimbursed for these hours. It was not a huge savings, but it neutralized some of the costs of the intern's hourly rate. The Parent Education Groups only added 4 hours per week to the Intern's workload. 

\subsection{Consulting the Clinical Director}
The accountant approached the clinical director for ideas on the rest of the internship hours. It looked like AAA needed to meet the 30 hours per week in the agreement but they were short. Adding 1.5 hours per week of supervision, 4 hours for parent education, and 15 hours of direct implementation left a deficit pf 9.5 hours to fill. The company met those hours by scheduling the intern for staff training, along with program development and report writing for clients on the intern's direct implementation caseload.

\subsection{Putting It Together}
As a result, the intern was able to satisfy a requirement toward the BACB\textregistered{} credential application and was paid \$45,000 for the year. Admittedly, it was not a huge amount of money for someone with a master's degree. The intern felt it was fair because she was only asked to work 30 hours per week. There were some travel time hours and expenses, which were handled separately in compliance with law. The company ultimately lost \$12,750 for the year.

AAA knew they would lose the money this year but hoped the intern would pass the credentialing exam soon and stay with AAA, billing at a full supervisor rate. That would allow AAA to earn a higher reimbursement rate for the hours the supervisor worked. More importantly, it meant AAA could add some clients from their waiting list, placing these clients on the new supervisor's caseload. 

\subsection{A Risk for the Company}
AAA recognized that the main incentive for a company to sponsor an intern was the possibility of serving more clients once the intern earned the credential. It was a gamble for AAA. Not every intern passed the credentialing exam. In this case, they requested transcripts from the possible interns they were considering. They wanted an intern with the highest possible grades because they believed previous academic performance suggested a history of work habits and a higher likelihood of having acquired the skills needed to pass the credential exam.

\subsection{Final Decision}
AAA knew that other ABA agencies in the area were using contracts to keep interns at their companies for long enough to recoup the cost of the internship. They weighed the pros and cons of contracts but chose to revisit that issue at a later date.

Ultimately, upper management was ambivalent about the internship. The director said, ``If someone told you to pay \$12,750 today and there was only a 58\% you could get that money back in 2 years, would you invest?'' She was referring to passing rates on the exam. The clinical director looked up the university pass rates for graduates of that local program and found that 60\% of graduates passed the exam on the first try. 

These data did not impress the director. The decision was made to offer the internship as a trial. The clinical director selected a salaried supervisor and put pressure on her to make sure the intern learned the BACB\textregistered{} 4th Edition Task List fully. ``If she doesn't pass the exam, I'm holding you responsible,'' said the clinical director. The supervisor accepted the challenge and implemented the procedures in Section 2 of the TrainABA supervised experience book. Years later, the intern had passed the exam on her first attempt and was successfully managing a caseload of 12 clients. She was a success story. AAA realized that not all internship stories have happy endings.



\section{Resources for Supervised Experience}
See the following pages for Table 2.1 and 2.2 to see what the accountant gave the director at AAA.
Resources for the Supervised Experience Process 
Items in Bold are required by the BACB\textregistered{} for credentialing. Non-bold items are supplemental materials. \\

Supervisee\\
\begin{enumerate}
\item Contract
\item BACB.com module (see notes on registration)
\item Clients (generally)
\item 4th Ed. Task List
\item Experience Verification Forms
\item Supplementary Materials
\item Homework
\end{enumerate}

Supervisor\\
\begin{enumerate}
\item Contract
\item BACB.com module
\item Clients
\item 4th Ed. Task List (Assessment and Meeting Agendas)
\item Experience Verification Forms
\item Supplementary Materials
\item Homework
\item Create Performance Management Plan (Professional Development)
\item Ongoing Payment (from employer, university, contractor)
\item Time Retainer
\item Technology (journals, online videos, etc.)
\item Communication (synchronous and/or asynchronous)
\end{enumerate}

\subsection{Beyond the Taxonomy}
Have you located yourself as an ABA supervisor in the taxonomy? Can you write the steps of the Supervised Experience Process? If not, please review the charts above. The goal of this chapter was to identify the type of supervision you offer, or plan to offer, as a supervisor in the field of applied behavior analysis.

The first edition TrainABA book listed prerequisites needed for supervising certification candidates. It included checklists to identify requirements. After publication, the BACB\textregistered{} disseminated more supervision materials. Use the BACB\textregistered{} materials for official purposes, as they are subject to change without warning at the BACB\textregistered{}'s sole discretion.


%
% for Supervision} of BACB\textregistered{} Experience Hours\\
Information below reflects the BACB\textregistered{}'s recent supervision standards, effective January 1, 2015.

\subsubsection{BACB\textregistered{} Rules for Supervision}
\begin{enumerate}
\item Each supervisee must have a valid supervision contract. Multiple exemplars and comprehensive guidelines are available at bacb.com\textregistered{}
\item    2. Each supervisor must have completed both of the following by December 31, 2014.
\begin{enumerate}
\item Complete 8 Hour Supervisor Training from a BACB\textregistered{} ACE provider (Available from TrainABA as a live webinar)
\item Complete an online, competency-based supervision module on BACB.com
\item Complete 3 CEUs for supervision for every recertification cycle
\end{enumerate}
\item Each supervision period is 2 consecutive weeks
\item Ratio of Independent Fieldwork to Direct Supervision must be no less than 5\% by the end of the 2 week period (You MUST provide Direct Supervision 5\% or more of their Independent Fieldwork by the end of each 2-week period.)
\item Per 2 week supervision period, no more than 50\% of supervision can be direct care. The other 50\%+ must be behavior analytic in nature
\item Start/end dates may not be more than 5 years apart.
\item Supervision must be face to face. Real-time video is okay. Think of Google Hangouts, FaceTime, Skype, etc.
\item 5\% of 1500 hours = 75 hours of independent fieldwork experience
\item Supervision hours may be counted toward total experience hours
\item No more than 50 percent of supervision (per 2-week period) can be in a group format
\item Group maximum = 10 supervisees
\item You do not need to provide Direct Supervision every week
\item Must meet at least once for every 2 week period
\item Content must be behavior analytic (Do not discuss billing, travel time, non-clinical scheduling, etc.)
\end{enumerate}
    
Mathematical Assumptions\\
\begin{itemize}
\item Supervisors must provide 5\% of 1,500 Independent Fieldwork hours = 75 hours
\item Supervision period is two weeks in duration
\item (75 hours total) DIVIDED BY (maximum of 3 hours per 2-week period) = 25 meetings, one per 2 weeks
\item Up to 50\% of supervised experience hours can be delivered in group format
\item Therefore, deliver group supervision meetings that are 1.5 hours in duration, once per 2 week period
\item Also provide individual supervision for 1.5 hours in duration for each 2 week period
\item Given the math above, Train ABA recommends you make group supervision meetings 1.5 hours long (90 minutes) for full time staff. We built the agendas around the 90 minute model
\item If your supervisees do not work 30 hours per week during both weeks of the 2-week supervision period, you will need to adjust the math to provide exactly 5\% of the hours they provided. See rules below.
\end{itemize}
    
Rules for Calculating How Many Hours Your Supervisee Has Completed\\
\begin{enumerate}
\item Your supervisee must work at least 10 and up to 30 hours during both weeks of each 2-week supervision period
\item You must provide supervision for 5\% of these hours
\item You do not need to provide exactly 50\% of group supervision every 2-week period, but we use that model for this protocol because it makes the math easier
\item When your supervisee works less than the expected amount of hours for a week or 2-week period, adjust your supervision hours to equal 5\% of their hours worked
\item If they work more than 30 hours in one week, the company can pay wages but the BACB\textregistered{} will not recognize extra hours
\end{enumerate}

\subsection{Registration Process}
As of the first edition of this book, the BACB\textregistered{} requirement for interns to ``register'' before beginning experience hours was considered new. Registration was first mentioned in the BACB\textregistered{}'s September 2012 Newsletter. By the date of this second edition book, all supervisors are likely registered and know to advise supervisees to register.\\

As of January 1, 2015, the registration process had two steps.\\
\begin{enumerate}
\item Create a login at bacb.com\textregistered{}
\item In that login, complete the same Supervision Policies Module required by individuals who wish to supervise those accruing Experience Hours. In plain English, your supervisees must complete the same supervision module as you. Additionally, they are expected to do it at the outset of supervision. Some supervises may not know of this requirement. Please advise your supervisees to complete this module immediately. It takes approximately 1.5 hours and is available free of charge.
\end{enumerate}

NOTE: Interns complete the module only once.\\

Note that TrainABA is not endorsed by the BACB\textregistered{}. The information presented is not meant to represent the opinion of the BACB\textregistered{} and any references to the BACB\textregistered{} or bacb.com are used with respect to their copyrights.\\
%
%
%
%
%
%
%
%
%
%
%
%
%
%
%
%
%
%
%
%
%
%
\chapter{Refresher: Supervisor Workshop}

This chapter contains key takeaways from the general 8-Hour Supervisor Training Workshop curriculum.\\

The BACB\textregistered{} 8-Hour Supervisor Training Workshop curriculum has 6 sections:
\begin{enumerate}
\item Purpose of Supervision
\item  Features of Supervision
\item  Behavioral Skills Training
\item  Delivering Performance Feedback
\item  Evaluating the Effects of Supervision
\item  Ongoing Professional Feedback
\end{enumerate}

These sections are summarized briefly in visuals and charts below. This chapter is meant to serve as a refresher for the concepts presented in the 8-Hour Supervisor Training workshop. It is not a substitute for the workshop. These materials were taken from the TrainABA 8-Hour Supervisor Training Workshop. 

\section{Purpose of Supervision}
\begin{displayquote}
``\textit{The purpose of supervision is to improve and maintain the behavior-analytic, professional and ethical repertoires of the supervisee and facilitate the delivery of high-quality services to his/her clients.}''\\
--BACB\textregistered{}{} 8-Hour Curriculum Training Outline, 2012
\end{displayquote}
\section{Features of Supervision}
The following items were considered appropriate supervision activities according to the BACB's initial supervision curriculum.
\begin{itemize}
\item Focus on developing new ABA skills
\item Use BACB\textregistered{}{} Fourth Edition Task List
\item Follow 7 Dimensions of Behavior Analysis (BATCAGE) (Baer, Wolf, \& Risley, 1968)
\item Give supervisees multiple sites, varied experiences, different supervisors
\item Conducting assessments to determine the need for behavioral intervention
\item Designing, implementing, \& systematically monitoring skill-acquisition and behavior- reduction programs
\item Oversee implementation of behavior-analytic programs by others
\item Training, designing behavioral systems, and performance management
\item Using behavioral skills training to Model and rehearse various behavior analytic skills and procedures.
\item Engaging in role-play scenarios in natural and contrived situations for various skills
\item Other items directly related to ABA
\end{itemize}

The following items were considered inappropriate supervision activities according to the BACB's initial supervision curriculum. These items represent inappropriate supervision activities. They are non-examples of content for group supervision meetings.

\begin{itemize}
\item Attending meetings with little or no behavior-analytic content
\item Scheduling, travel time, billing
\item Using unproven or non-behavior analytic interventions 
\item Non-behavioral administrative activities, non-behavioral assessments (diagnostic or intellectual assessments)
\end{itemize}

\section{Features of Supervision}
(Chart pending).
%Recertification Requirements Before January 1, 2015
%
%Recertification Requirements After January 1, 2015
%
%
\section{Using Behavioral Skills Training}
%
Why is Behavioral Skills Training (BST) popular in ABA supervision now?

In 2012, the Behavior Analyst Certification Board\textregistered{} created a document called the ``Supervision Training Curriculum Outline.'' It contained the required topics for Approved-Continuing Education (ACE) providers who would provide the 8-Hour Supervisor Training curriculum. Section (3) of this (6) section document was titled, ``Behavioral Skills Training (BST).'' 

BST is found in various JABA articles and books by behavior analysts. Perhaps the best example of BST is found in Raymond Miltenberger's 2011 textbook, ``Behavior Modification Principles and Procedures.'' 

An 8-step BST procedure is outlined on the following page.
%
%
\subsection{Behavioral Skills Training} 
(Note: This chart needs to be revised for the LaTeX edition.)
\begin{enumerate}
\item Provide a rationale for why the target skills are to be trained
\item Provide a succinct, written description (instructions) of the target skills 
\item Scripts are included in this document. Be sure to provide a script to employees.
\item Provide a detailed, vocal description (instructions) of the target skills
\item        a. Trainer reads script aloud to trainee
\item    4. Demonstrate (model) each of the target skills 
\item    a. Trainer is first to role play, demonstrating correct behavior for trainee
\item        b. Include examples and non-examples 
\item        c. If training scenario is a non-example, trainer deviates from script and scenario is terminated with positive feedback.
\item    5. Require trainee to practice (rehearse) each target skill
\item        a.  Trainee role plays scenarios from the list
\item        b. Include examples and non-examples 
\item        c. If training scenario is a non-example, trainee deviates from script and scenario is terminated with positive feedback
\item    6. Provide positive and corrective feedback to supervisee
\item        a. Provide it vocally, immediately following trainee role play
\item        b. Deliver positive feedback to trainee throughout training, aiming for 4:1 ratio
\item        c. Deliver corrective feedback directly.  
\item    7. Repeat the previous step until supervisee performs each target skill correctly 
\item    8. Assess application and generalization of skills to new targets, clients, and settings, when appropriate
\end{enumerate}

\section{Delivering Performance Feedback}
Corrective Feedback
\begin{enumerate} 
\item        1. Provide an empathy statement 
\item        2. Describe ineffective performance 
\item        3. Provide a rationale for desired change in performance 
\item        4. Provide instructions and demonstration for how to improve designated performance 
\item        5. Provide opportunities to practice the desired performance 
\item        6. Provide immediate feedback 
\end{enumerate}

\section{Evaluating the Effects of Supervision}
Evaluate supervision with evidence-based, intervention specific criteria for:
\begin{itemize}
\item    • Client performance
\item    • Staff performance
\item    • Supervisory behavior 
\end{itemize}
%
\subsection{Ongoing Professional Development}
\textbf{1.03 Professional Development (+RBT)}
\textit{Behavior analysts who engage in assessment, therapy, teaching, research, organizational consulting, or other professional activities maintain a reasonable level of awareness of current scientific and professional information in their fields of activity, and undertake ongoing efforts to maintain competence in the skills they use by reading the appropriate literature, attending conferences and conventions, participating in workshops, and/or obtaining Behavior Analyst Certification Board certification.}
--BACB Professional and Ethical Compliance Code, Ver. 9/23/2014.
%
The supervisor should be able to describe the following methods for his/her ongoing professional development as a supervisor
\begin{itemize}
\item Creating a continuous learning community to enhance supervisory and training behavior
\item Regular review of resources and research for best practices in supervision
\end{itemize}
%
The supervisor should be able to describe the following methods for his/her ongoing professional development as a supervisor and to the supervisee:
\begin{itemize}
\item Supervisory study groups
\item Attending conferences
\item Seeking peer review
\item Seeking mentorship
\item Regular review of resources and research relevant to supervisee's area of practice
\item Seeking consultation when necessary
\end{itemize}