\clearpage \section[\fourjThirteen{}]{\fourjThirteen{}%
              \sectionmark{J-13 Select behavioral cusps...}}
\sectionmark{J-13 Select behavioral cusps...}
\subsection{Definition}
``A behavior that has consequences beyond the change itself, some of which may be considered important... What makes a behavior change a cusp is that it exposes the individual's repertoire to new environments, especially new reinforcers and punishers, new contingencies, new responses, new stimulus controls, and new communities of maintaining or destructive contingencies. When some or all of these events happen, the individual's repertoire expands; it encounters a differentially selective maintenance of the new as well as some old repertoires, and perhaps that leads to some further cusps'' (Rosales-Ruiz \& Baer, 1997, p. 534).

Bosch and Fuqua (2001) cited 5 criteria for a behavior to be considered a behavioral cusp. ``They stated that a behavior might be a cusp if it meets one or more of five criteria: ``(a) access to new reinforcers, contingencies, and environments; (b) social validity; (c) generativeness; (d) competition with inappropriate responses; and (e) number and the relative importance of people affected'' (Bosch \& Fuqua, 2001, p. 125 via Cooper et al., 2007, p. 59).
%
\subsection{Examples}
\begin{enumerate}
\item Common behavioral cusps include crawling, reading, imitation skills, walking, talking, and writing. These skills set the stage for the client to develop and learn many other skills (i.e., reading allows a client to access leisure material, access information, and is necessary in regards to reading street and safety signs, and accessing various other forms of reinforcement).
\end{enumerate}
%
\subsection{Assessment}
\begin{enumerate}
\item Have supervisee identify 5 behaviors that he/she feels are behavioral cusps. Have him/her describe why the behavior is a behavioral cusp.
\item Have supervisee read Rosales-Ruiz \& Baer (1997) article on behavioral cusps. Have him/her summarize the main points of this article.
%
\end{enumerate}
%
\subsection{Relevant Literature}
\begin{refsection}
\nocite{cooper2007applied,
        rosales1997behavioral,
        bosch2001behavioral}
\printbibliography[heading=none]
\end{refsection}
%
\subsection{Related Tasks}
\fourgEight{}\\
\fouriOne{}\\
\fouriSix{}\\
\fourjOne{}\\
\fourjEight{}\\
\fourjFourteen{}\\
\fourFKTen{}\\
