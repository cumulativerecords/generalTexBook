\clearpage \section{\fourFKThirtyFour{}}
\subsection{Definition}
Conditional discrimination – ``refer to a concept related to stimulus equivalence and can be created using matching-to-sample procedures.  Conditional discriminations operate within a four-term contingency that accounts for the environmental context, such that the contingency appears as the following: contextual stimulus $\rightarrow$ SD $\rightarrow$ response $\rightarrow$ reinforcement'' (Cooper, Heron, \& Heward, 2007, p. 400).\\

The contextual events operating within this four-term contingency become conditional discriminations (Sidman, 1994).   Moreover, a specific conditional discrimination implies that there is a specific conditional relation, referring to the direct outcome of the reinforcement contingency (Carrigan \& Sidman, 1992).  Different conditional relations are reflected in the properties of stimulus equivalence (e.g., reflexivity, symmetry, and transitivity).  In order for new conditional discriminations to emerge, at least one conditional discrimination must be directly trained.  This initial discrimination is often arbitrary (e.g., training A > B implies that B $<$ A). 
%
\subsection{Examples}
\begin{enumerate}
\item In a matching-to-sample trial, ``A'' is given as the conditional sample and is then presented in a series of five letters ``z,'' ``k,'' ``a,'' ``t,'' and ``b.''  By correctly matching ``A'' to ``a'' the learner is discriminating stimuli and selecting the correct comparison that will be reinforced.  The learner's response will not be reinforced if a stimulus other than ``a'' is selected.  In this example, the conditional sample ``A'' reflects the contextual stimulus as it creates a context for which stimulus to discriminate.  
%
\end{enumerate}
%
\subsection{Assessment}
\begin{enumerate}
\item Conditional discriminations can be a difficult concept to understand without having adequate knowledge of stimulus control, stimulus equivalence, and matching-to-sample procedures.  Therefore, in order to assess for learning, the supervisee should be asked to define and provide examples of conditional discriminations within the context of explaining stimulus control, stimulus equivalence, and matching-to-sample procedures.
\item Once the supervisee can explain conditional discriminations, applied knowledge can be assessed by asking supervisee to demonstrate direct training of conditional discriminations by using matching-to-sample and stimulus equivalence procedures.  The supervisor will model these procedures, have supervisee demonstrate the skill and provide feedback based on their performance.  
%
\end{enumerate}
%
\subsection{Relevant Literature}
\begin{refsection}
\nocite{bush1989contextual,
        carrigan1992conditional,
        cooper2007applied,
        johnson1993conditional,
        sidman1994equivalence}
\printbibliography[heading=none]
\end{refsection}
%
\subsection{Related Tasks}
\foureSix{}\\
\fourFKEleven{}\\
\fourFKTwentyFour{}\\
\fourFKThirtyFive{}\\
\fourFKThirtySeven{}\\
\fourjFourteen{}\\
