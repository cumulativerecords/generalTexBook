\clearpage \section[\fouriTwo{}]{\fouriTwo{}%
              \sectionmark{I-02 Define environmental var...}}
\sectionmark{I-02 Define environmental var...}
\subsection{Definition}
This task relates to the importance of defining environmental variables in observable and measurable terms.
\begin{enumerate}
\item As Cooper, Heron, and Heward (2007) state, in order to achieve a high level of treatment integrity in an experiment, it is of utmost importance to ``develop complete and precise operational definitions of the treatment procedures'' (Cooper, Heron, \& Heward, 2007, p. 235). In the same way that it is critical to define target behavior in observable and measurable terms, so is the case with defining environmental variables. 
\item Baer, Wolf, and Risley (1968) stress that the ``technological'' dimension of Applied Behavior Analysis refers simply to the fact that ``the techniques making up a particular behavioral application are completely identified and described''(Baer, Wolf, \& Risley, 1968, p. 95). As such, the techniques, or environmental variables being manipulated, must be defined in observable and measurable terms to meet the technological dimension of applied behavior analysis (Cooper, Heron, \& Heward, 2007). 
\item However, historically, operationally defining independent variables has not been conducted to the standard required for a science of behavior that seeks to achieve the technological dimension of applied behavior analysis*. It has also not been done to the same standard as that of the dependent variables (Johnston \& Pennypacker, 1980; Peterson, Homer \& Wonderlich, 1982; Gresham, Gansle \& Noell, 1993). In 1982 Peterson, Homer and Wonderlich called for researchers to measure the independent variables in a more stringent manner. Unfortunately, an assessment of this area later on by Gresham, Gansle and Noell (1993) found that this had not been accomplished.
\item Defining environmental variables in observable and measurable terms
\item It is believed that environmental variable definitions should be written to meet the same standards as those required to be met by target behavior definitions (Gresham, Gansle \& Noell, 1993). They should be ``clear, concise, unambiguous, and objective'' (Cooper, Heron \& Heward, 2007, p. 235).
\item Gresham et al. (1993, p. 261) suggest that independent variable definitions can be made along four dimensions: spatial, verbal, physical and temporal. 
\end{enumerate}

\subsection{Examples}
Gresham et al. (1993, pp. 261-262) give an example of an adequate definition of an independent variable, a time-out procedure, provided by Mace, Page, Ivancic and O'Brien (1986).
\begin{enumerate}
\item Immediately following the occurrence of a target behavior (temporal dimension), (b) the therapist said, ``No, go to time-out'' (verbal dimension), (c) led the child by the arm to a prepositioned time-out chair (physical dimension), and (d) seated the child facing the corner (spatial dimension). (e) If the child's buttocks were raised from the time-out chair or if the child's head was turned more than 45° (spatial dimension), the therapist used the least amount of force necessary to guide compliance with the time-out procedure (physical dimension). (f) At the end of 2 min (temporal dimension), the therapist turned the time-out chair 45° from the corner (physical and spatial dimensions) and walked away (physical dimension).
\item Gresham et al., (1993, p. 262) argued that a failure to define operational variables along these four dimensions, as done so by Mace, Page, Ivancic and O'Brien (1986), makes ``replication and external validation of behavior-analytic investigations difficult.''
\end{enumerate}
%
\subsection{Assessment}
\begin{enumerate}
\item Ask your Supervisee to explain why it's important as a behavior analyst to define environmental variable in observable and measurable terms. 
\item Ask your Supervisee to write an operational definition for the following independent variable:
\item Verbal Praise (Answer = (a) Immediately following the occurrence of a target behavior (temporal dimension), (b) the therapist delivered verbal praise, for example, ``great job/nice work/well done'' (verbal dimension), (c) but did not provide any physical contact such as a hi-five or pat on the back (physical dimension). The therapist was within 2 to 20 feet of the client at all times during the intervention (spatial dimension).
\end{enumerate}
%
\subsection{Relevant Literature}
\begin{refsection}
\nocite{baer1968some,
        cooper2007applied,
        gresham1993treatment,
        johnston2010strategies,
        peterson1982integrity}
\printbibliography[heading=none]
\end{refsection}
%
\subsection{Related Tasks}
\fourbEleven{}\\
\fouriOne{}\\
\fouriFour{}\\
\fourjOne{}\\
\fourFKSeven{}\\
\fourFKEleven{}\\
\fourFKThirtyThree{}\\
%
Footnotes\\
*See Baer, Wolf \& Risley (1968) for more information on the seven dimensions of Applied Behavior Analysis.
