\clearpage \section[\fouraTwelve{}]{\fouraTwelve{}%
              \sectionmark{A-12 Design... continuous}}
\sectionmark{A-12 Design... continuous}
\subsection{Definition} 
	Event recording – ``measurement procedure for obtaining a tally or count of the number of times a behavior occurs'' (Cooper, Heron, \& Heward, 2007, p. 695).
%  
\subsection{Examples}
Examples of contexts that are likely to be appropriate for an event recording procedure:
\begin{enumerate}
\item Property destruction that typically occurs two to five times a week. 
\item Correct responses to the question, ``What do you want?'' when asked in at least six distributed trials each day. 
\end{enumerate}
Examples of contexts that may be inappropriate for an event recording procedure: 
\begin{enumerate}
\item Vocal Stereotypy that occurs on and off so rapidly that an observer would not be able to accurately determine the start and end of the stereotypy.
\item Aggressive behavior in a classroom setting with one teacher, who must conduct instruction, interact with other students, and count hitting behavior that often includes multiple students.  
\item An observer may increase counting accuracy by using a counting device with low-technology (i.e., masking tape around the wrist for tally marks or a golf-stroke counter or with high technology (i.e., iPad or laptop direct observation programs). 
\end{enumerate}
%
\subsection{Assessment}
\begin{enumerate}
\item Ask the supervisee to describe situations that would be appropriate to use event recording procedures. 
\item Ask the supervisee to describe situations that may not be appropriate to use an event recording procedure.
\item Ask the supervisee to use an event recording procedure to measure 3 responses at one time. Ask the student to design a datasheet and method for counting that will help the observer record the responses accurately. 
\end{enumerate}
%
\subsection{Relevant Literature}
\begin{refsection}
\nocite{cooper2007applied,kelly1977review,sasso1992use}
\printbibliography[heading=none]
\end{refsection}
%
\subsection{Related Tasks} 
\fouraNine{}\\ 
\fourhOne{}\\
\fouriOne{}\\
\fourFKFourtySeven{}\\
\fourFKFourtyEight{}\\
