\clearpage \section{\fourFKFour{}}
\subsection{Definition} 
Empiricism - ``the practice of objective observation of the phenomena of interest'' (Cooper et al., 2007, p. 5).

According to Fisher (2011), scientists make observations about the world by using information available to the senses. Sensory evidence is the primary source of information and should maintain the attitude of empiricism by believing what they observe the world to be and not what they have been taught that it should be.

\subsection{Examples}
\begin{enumerate}
\item Mr. Johnson, a BCBA, conducts a functional analysis to determine the function of Billy's aggressive behavior in class. He completes rating scales, interviews, and other indirect assessment procedures, but doesn't use these to guess the reinforcer for Billy's aggression. Mr. Johnson uses the indirect assessment procedures to inform his experiment. He designs a pairwise functional analysis and runs out several phases of direct observation until results are conclusive. He concludes that Billy's aggressive behavior is sensitive to attention as a maintaining variable. At the IEP meeting, Billy's parents applaud Mr. Johnson's empiricism for completing such a thorough assessment and analyzing all the possible factors before determining a function. 
\item Mr. Riley is a district BCBA and has been asked to conduct a functional behavior assessment for Mary in regards to her aggressive behavior. Mr. Riley hypothesizes that Mary is engaging in aggressive behavior to get access to her dolls because all little girls like dolls. Based on his reasoning he has already decided that Mary's aggression is maintained by access to dolls. Since he already has a strong hypothesis for the function of aggression, Mr. Riley writes a report and creates a treatment for Mary.

\end{enumerate}
%
\subsection{Assessment}
\begin{enumerate}
\item Have supervisee describe the term empiricism and how it relates to applied behavior analysis. Have him/her identify ways that they can make sure that their work is empirically based. 
\item Have supervisee read an article on decreasing problematic behavior. Have him/her identify what makes this article empirically sound. 
\item Have supervisee list practices that are not empirically based and then identify what the individual could do to make sure that they were practicing appropriate empiricism. 
\end{enumerate}
%
\subsection{Relevant Literature}
\begin{refsection}
\nocite{cooper2007applied,
        baer1968some,
        fisher2014handbook,
        schmidt1992data}
\printbibliography[heading=none]
\end{refsection}
%
\subsection{Related Tasks}
\fourbOne{}\\
\fourhOne{}\\
\fourhThree{}\\
\fouriOne{}\\
\fouriThree{}\\
\fouriFive{}\\
\fourjOne{}\\
\fourjFifteen{}\\
\fourkSeven{}\\
\fourFKTen{}\\
