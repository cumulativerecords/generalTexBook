\clearpage \section[\fourkNine{}]{\fourkNine{}%
              \sectionmark{K-09 Secure the support...}}
\sectionmark{K-09 Secure the support...}
\subsection{Definition}
Foxx (1996, p. 230) stated that in programming successful behavior change interventions, ``10\% is knowing what to do; 90\% is getting people to do it... Many programs are unsuccessful because these percentages have been reversed'' (Cooper et al., 2007, p. 652). 

Being explicit yet simplistic in describing programs and protocols will help secure support from other individuals in a client's environment. If a behavior change procedure or program is too difficult, technical, or places unreasonable demands on the other individuals involved, they are less likely to implement these programs. In addition, adequate training of behavior procedures should be provided to ensure proper implementation by those interacting with the client in the natural environment. Specifically, training pertaining to the delivery of reinforcers, which maintain the individual's newly acquired behavioral repertoires.

Jarmolowicz et al. (2008) compared the effectiveness of conversational language instructions and technical language instructions when explaining how to implement a treatment to caregivers. They found that the caregivers that were given conversational language instruction implemented the treatment more accurately.
%
\subsection{Examples}
\begin{enumerate}
\item Richard is trying to generalize skills learned in the special education classroom for one of his students. He went to each teacher to explain how this will help the student in their class and answered any questions they may have about the programs. In addition, he conducted a training on the specific program and offered to consult with each teacher in order to make sure generalization was successful and the repertoire was maintained.
%
\end{enumerate}
%
\subsection{Assessment}
\begin{enumerate}
\item Have supervisee identify ways they can build a rapport with other service providers and help support them when they have a student who needs to generalize and maintain skills in new settings.
\item Have supervisee choose a particular behavior change program or strategy. Have him/her describe and explain this program to an individual who does not have a background in applied behavior analysis. 
\item Have supervisee choose a specific behavior change program. Have him/her practice explaining the benefits of this program to others in order to get them on board with implementing this program.
\item Give supervisee a complex behavior change program. Have him/her simplify this program and create guidelines and instructions that they could give to an individual who does not have knowledge of applied behavior analytic strategies and techniques. 
\end{enumerate}
%
\subsection{Relevant Literature}
\begin{refsection}
\nocite{cooper2007applied,
        jarmolowicz2008effects,
        stokes1974programming}
\printbibliography[heading=none]
\end{refsection}
%
\subsection{Related Tasks}
\fourhOne{}\\
\fourjOne{}\\
\fourkThree{}\\
\fourkFour{}\\
\fourkSix{}\\
\fourkEight{}\\
