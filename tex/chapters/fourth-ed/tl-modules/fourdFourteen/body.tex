\clearpage \section{\fourdFourteen{}}
\subsection{Definition}
Listener – ``someone who provides reinforcement for verbal behavior'' (Cooper, Heron, \& Heward, 2007, p. 698)

Part of being involved in a verbal community is reinforcing the behavior of speakers. There are several methods for training someone to respond as a listener. Skills such as vocal imitation (echoic), following instructions, answering questions (intraverbal), conversation skills (intraverbal), indicating objects, etc., all require listener behavior. 

Sundberg discusses a distinction between intraverbal and listener responding. ``If the child's response were verbal, then it would classified as intraverbal... but if the response were nonverbal it would be classified as listener behavior (or often termed receptive language or receptive labeling)'' (Sundberg, 2008, pp. 11-12).

There are multiple protocols for receptive language training (cf., Fabrizio \& Moors, 2001; Leaf \& McEachin, 1999; Lovaas, 2003; Maurice, Green, \& Luce, 1996). In a review of teaching receptive language to children with autism, Pelios and Sucharzewski (2004) point out that one must consider antecedent manipulations (e.g., within-stimulus prompts, keeping stimulus short, using topographically dissimilar responses) and consequence manipulations (e.g., rich reinforcement schedules, token economies, performance based breaks).  Also, they recommend systematically programming specific antecedent and consequence manipulations and requiring specific response requirements.

\subsection{Examples}
\begin{enumerate}
\item A teacher conducting receptive language training tells her student ``sit down'' the student sits down and the teacher praises the student. The teacher then says ``clap hands'' the student claps hands and the teacher praises the student.
\item A teacher presents an array of fruit and says to the student, ``give me the apple.'' The student gives the teacher the apple and the teacher gives the student a token. 
\end{enumerate}
%
\subsection{Assessment}
\begin{enumerate}
\item Have supervisee read relevant literature on receptive language/listener training.
\item Role play with your supervisee teacher (speaker) and student (listener) examples of receptive language training protocols.
\end{enumerate}
%
\subsection{Relevant Literature}
\begin{refsection}
\nocite{cooper2007applied,
    fabrizio2001brief,
    leaf1999work,
    lovaas2003teaching,
    maurice1996behavioral,
    pelios2004teaching,
    schlinger2008listening,
    sundberg2008vb-mapp}
\printbibliography[heading=none]
\end{refsection}
%
\subsection{Related Tasks}
\fourdTen{}\\
\fourdThirteen{}\\
\fourFKFourtyThree{}\\
\fourFKFourtyFour{}\\
\fourFKFourtyFive{}\\
\fourFKFourtySix{}\\
