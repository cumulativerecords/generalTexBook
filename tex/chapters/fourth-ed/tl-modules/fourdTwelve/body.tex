\clearpage \section{\fourdTwelve{}}
\subsection{Definition}
Practitioners may use a variety of techniques to teach language when working with clients.  Tact training is one such technique in which a consumer may be taught to label ``objects, actions, properties of objects and actions, prepositional relations, abstractions, private events, and so on''  (Sundberg as cited in Cooper, Howard, \& Heron, 2007, p. 544). ``The goal of teaching is to bring a verbal response under nonverbal stimulus control''  (Sundberg as cited in Cooper, Howard, \& Heron, 2007, p.544). 

Initially, a practitioner pairs a nonverbal stimulus (such as snow falling outside of one's window) with an echoic model ``snow.''  The imitation of this verbal model is differentially reinforced.  Over time this echoic is faded out so that only the presence of the nonverbal stimulus (the snow) sets the occasion for the consumer to label ``snow'' in the absence of a verbal model. A time delay procedure, in which the practitioner gradually delays the presentation of the echoic model after the nonverbal stimulus appears, may be helpful in fading out the verbal model.  

When using tact training, the trainer should:
\begin{enumerate}
\item Ensure that the listener is attending.  Make sure that they are looking in your direction, are making eye contact and that the environment isn't too noisy or distracting.
\item Pair the presentation of nonverbal stimulus that you would like to train with an echoic model.  
\item Pause to allow the listener to process the information and wait for a response.
\item Provide differential reinforcement for responses that are closer and closer approximations to the verbal model. Note:  It may be difficult to differentially reinforce the tacting of events that cannot be shared by both instructor and student, such as private events like pain, as the instructor may not be able to adequately able to label them (Cooper, Howard, \& Heron, 2007).  For this reason it is recommended that initial tact training be done with objects or actions that can be directly observed.
\item Once the client is able to imitate the verbal model in the presence of the stimulus, gradually fade out the verbal model so that only the stimulus itself sets the occasion for the response. 
\end{enumerate}
%
\subsection{Assessment}
\begin{enumerate}
\item Ask the supervisee to state what tact training can be used to teach.
\item Ask the supervisee to give examples of tact training.
\item Ask the supervisee to discuss how tact training should be delivered.
\item Ask the supervisee to state how the verbal model can be successfully faded out.
\end{enumerate}
%
\subsection{Relevant Literature}
\begin{refsection}
\nocite{skinner1957verbal,
    sundberg2001benefits,
    sundberg1998teaching}
\printbibliography[heading=none]
\end{refsection}
%could not verify. used sundberg2001benefits instead.
%Sundberg, M. L., \& Michael, J. (2001). The value of Skinner's analysis of verbal behavior for teaching children with autism. Behavior Modification, 25, 698-724.
%
\subsection{Related Tasks}
\fourdTwelve{}\\
\fourFKFourtyFive{}\\
