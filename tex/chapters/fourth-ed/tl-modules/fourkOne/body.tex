\clearpage \section[\fourkOne{}]{\fourkOne{}%
              \sectionmark{K-01 Provide for ongoing doc...}}
\sectionmark{K-01 Provide for ongoing doc...}
\subsection{Definition}
Behavior analysts follow guidelines related to documentation in 2.10 and 2.11 of the Behavior Analyst Certification Board professional and ethical compliance code for behavior analysts Boundaries of Competence:

2.10 Documenting Professional Work and Research.

(a) Behavior analysts appropriately document their professional work in order to facilitate provision of services later by them or by other professionals, to ensure accountability, and to meet other requirements of organizations or the law. 

(b) Behavior analysts have a responsibility to create and maintain documentation in the kind of detail and quality that would be consistent with best practices and the law. 

2.11 Records and Data. 

(a) Behavior analysts create, maintain, disseminate, store, retain, and dispose of records and data relating to their research, practice, and other work in accordance with applicable laws, regulations, and policies; in a manner that permits compliance with the requirements of this Code; and in a manner that allows for appropriate transition of service oversight at any moment in time. 

(b) Behavior analysts must retain records and data for at least seven (7) years and as otherwise required by law. (BACB, 2014, p.9)

Written documentation of work products (such as raw data sheets, reports and spreadsheets) must be kept be kept in secure locations in the event that this information may need to be transferred to other professionals that are also supporting the client. Ongoing documentation and record-keeping also ensures accountability for services rendered. Documentation should be thorough and well-maintained; should the behavioral analysts' professional services be involved in legal proceedings, documentation must be detailed and comprehensive enough to meet judicial scrutiny (Bailey \& Burch, 2011).

Record disposal must sufficiently eliminate all confidential records that may reveal client's private health information. Electronic transfer of client's identifying information and records under any insecure medium (e.g. public areas, fax and email) are prohibited by the Health Insurance Portability and Accountability Act (Cooper, Heron, \& Heward, 2007). 
%
\subsection{Assessment}
\begin{enumerate}
\item Provide hypothetical scenarios and ask your supervisee if documentation of behavioral services was sufficient.
\item Ask your supervisee where current work products are stored. Request to see written documentation of work products assess to see if documentation is thorough and well-maintained. 
\item Use behavioral skills training to practice documenting sessions with clients or observations of staff members.
%
\end{enumerate}
%
\subsection{Relevant Literature}
\begin{refsection}
\nocite{bailey2013ethics,
        bac2014professional,
        cooper2007applied}
\printbibliography[heading=none]
\end{refsection}
