\clearpage \section{\fouraSix{}}
\subsection{Definition}
Percent of occurrence - ``A ratio formed by combining the same dimensional quantities such as count or time; expressed as a number of parts per 100; typically expressed as a ratio of the number of responses of a certain type per total number of responses. A percentage represents a proportional quantity per 100'' (Cooper, Heron \& Heward, 2007, p. 701).\\

Percent of occurrence demonstrates proportional relations effectively. For example, it can be used to indicate how many times an individual engaged in a target response given a set number of opportunities available).\\

There are several important limitations. The method has no dimensional quantities (e.g. does not indicate how many target responses were emitted nor how many opportunities were given). When there are few response opportunities (e.g. fewer than 20), percent occurrence measures may skew performance (e.g. An individual  answering 1 out of 2 problems correct on a math test will receive the same score of 50\% as an individual answering 25 out of 50 problems correct). It also imposes an artificial ceiling of measurement (e.g. 100\% may be subjective; suggesting that a learner performing at 100\% cannot improve). (Cooper et al., 2007).

\subsection{Examples}
\begin{enumerate}
\item  Twelve strangers walk past an elderly man on the street. He greets three of them and ignores the rest. The percent of occurrence of greeting strangers is 25\%.
\begin{itemize}
\item To compute: Divide number of greetings emitted by the man (3) by the total number of opportunities to greet (12) and multiply that product by 100 to yield a percentage (3/12= 0.25 x 100= 25\%).
\end{itemize}
\item (Non-example) Twelve strangers walk by an elderly man. He greets three of them and ignores the rest. The percent of occurrence of greeting strangers is 0.25. 
\end{enumerate}
%
\subsection{Assessment}
\begin{enumerate}
\item Provide hypothetical situations and ask your supervisee if using percent of occurrence measures are appropriate
\item Provide various hypothetical situations and ask your supervisee to calculate percent of occurrence
\item Have supervisee graph percent of occurrence measured on the job or in a role play
\end{enumerate}
%
\subsection{Relevant Literature}
\begin{refsection}
\nocite{cooper2007applied}
\printbibliography[heading=none]
\end{refsection}
%
\subsection{Related Tasks}
\fouriOne{}\\
\fourhOne{}\\
\fourFKFourtySeven\\
