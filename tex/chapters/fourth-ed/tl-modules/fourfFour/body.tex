\clearpage \section{\fourfFour{}}
\subsection{Definition}
Precision teaching is a methodology of teaching which involves measuring performance of a learner and making changes based on these data. Precision teaching was proposed by Ogden Lindsley in 1964 in an article ``Direct Measurement and Prosthesis of Retarded Behavior.'' 

There are four steps of precision teaching process: 
\begin{enumerate}
\item Pinpoint - means to describe an actual movement which a learner needs to perform in a specific time interval to show an improvement on learning behavior.
\item Record - means that a learner or a teacher collect data on pinpointed behavior regularly and display these data graphically using the Standard Celeration Chart (the SCC). 
\item Change - means that the teacher analyzes the data using guidelines of analyzing the SCC (Graf \& Lindsley, 2002) and quickly makes changes in a current instructional method if needed. 
\item Try Again - means that the teacher keeps exploring the best instructional methods for those learners who need more help and provides the learners with the opportunities to practice until the mastery level. 
\end{enumerate}

Precision teaching follows four principles described by Kubina (2012): ``(i) a focus on observable behavior, (ii) the use of frequency as data metric, (iii) graphing student performance data on a Standard Celeration Chart, and (iv) making decisions based on performance data'' (As cited in Cooper, Heron, Heward, 2007, p. 142).
%
\subsection{Examples}
\begin{enumerate}
\item A teacher's objective is to teach a student to say and write an answer when vocally asked simple math problems questions. The teacher will record how many times a student answers correctly in one minute interval. These data will be collected over the next several weeks and progress will be charted using a Standard Celebration Chart. Decisions about the teaching procedure will be made depending on the performance data.
\item (Non-example) A teacher's objective is to teach students to solve simple math problems.  Percent correct are collected across the quarter. At the end of the quarter a report card will be sent to the student's home.
\end{enumerate}
%
\subsection{Assessment}
\begin{enumerate}
\item Have supervisee identify the benefits of adding precision teaching to a curriculum.
\item Have supervisee identify and describe the key features of precision teaching. 
\item Have supervisee describe the parts of a Standard Celeration Chart. Have him/her discuss the benefits of using this graphic display to track data.
\end{enumerate}
%
\subsection{Relevant Literature}
\begin{refsection}
\nocite{cooper2007applied,
        cooper2000tutoring,
        kerr2003precision,
        hughes2007using,
        kubina2012precision,
        kubina2002benefits,
        potts1993ogden}
\printbibliography[heading=none]
\end{refsection}
%
\subsection{Related Tasks}
\fouraOne{}\\
\fouraSeven{}\\
\fourfThree{}\\
\fourhOne{}\\
\fourhTwo{}\\
\fourhThree{}\\
\fourhFour{}\\
\fourhFive{}\\
\fourjFifteen{}\\
\fourFKThirtyThree{}\\
