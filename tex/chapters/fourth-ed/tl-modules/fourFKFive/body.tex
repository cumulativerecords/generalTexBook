\clearpage \section{\fourFKFive{}}
\subsection{Definition}
Parsimony - The concept ``that ``simple, logical explanations must be ruled out, experimentally or conceptually, before more complex, or abstract experimentations are considered'' (Cooper, Heron, \& Heward, p. 22). 

Behavior analysts attempt to identify the simplest explanation for an individual's observed responses and then apply the least complex intervention that results in improved behavior. 
%
\subsection{Examples}
\begin{enumerate}
\item A non-verbal client hits her head repeatedly for a period of days each month. Although the analyst considered multiple environmental antecedent and consequent factors that might influence the client's behavior, she first looked at the calendar to see if the client's head hitting each month corresponded to her monthly menstrual cycle. She found that head-hitting occurred the last days immediately before her period began and the first day of her period. The analyst asked a nurse to review the data and recommend a medical intervention before the analyst continued to assess the influence of external environmental factors. 
\item An analyst was asked to design strategies for staff when responding to a client's aggressive behavior after asking him to brush his teeth. He reviewed the data staff had recorded about self-care behaviors and saw that aggression was a relatively new behavior during teeth-brushing. He learned that the client became aggressive toward staff shortly after they began buying a discounted toothpaste instead of the client's usual brand. When the analyst offered the client a choice between the two brands, the client chose his old brand and aggression did not occur after asking him to brush his teeth.
\end{enumerate}
%
\subsection{Assessment}
\begin{enumerate}
\item Give the supervisee 3 scenarios and ask the supervisee to consider what might be a parsimonious (simplest that works) first approach for each situation.
\item Example of a scenario for assessment: An adult client sometimes asks to go outside before breakfast. He screams and refuses to eat when he is made to sit at the table instead of going outside. Staff believe he should eat a good breakfast as part of his regular morning routine before he begins activities. A parsimonious response from the analyst might be to suggest that staff add a choice step before breakfast in which staff ask the client if he would like to go outside for 5 minutes before he eats. 
%
\end{enumerate}
%
\subsection{Relevant Literature}
\begin{refsection}
\nocite{cooper2007applied,
        etzel1979simplest}
\printbibliography[heading=none]
\end{refsection}
%
\subsection{Related Tasks}
\fourFKOne{}\\ %FK-01: Lawfulness of behavior.
\fourFKTwo{}\\ %* FK-02: Selectionism (phylogenic, ontogenic, cultural)
\fourFKThree{}\\%* FK-03: Determinism
\fourFKFour{}\\%* FK-04: Empiricism
\fourFKSix{}\\%* FK-06: Pragmatism
