\clearpage \section{\foureTwo{}}
\subsection{Definition}
Discrimination training procedures involve ``reinforcing or punishing a response in the presence of one stimulus and extinguishing it or allowing it to recover in the presence of another stimulus'' (Malott \& Trojan Suarez, 2004, p. 485). There are typically two competing contingencies when discrimination training occurs.  The first contingency involves an S-delta (i.e., signals that a specified response will not be reinforced or punished when in the presence of a specified stimulus). The second contingency involves a discriminative stimulus (i.e., signals that a specified response will be reinforced in the presence of a specific stimulus condition). When discrimination training occurs, a specified response will no longer be reinforced in the presence of an S-delta, however, that same response will be reinforced in the presence of a discriminative stimulus. The goal of discrimination training is to reinforce responses in certain stimulus conditions so that they occur more frequently when those stimulus conditions are present and over time, the response will no longer occur in the presence of the S-delta. When discrimination training is successful, the learner can discriminate which antecedent stimulus conditions will result in greater reinforcement for a given response. 
%
\subsection{Examples}
\begin{enumerate}
\item Discrimination training can be used to teach an individual appropriate times to take breaks, when or where it is acceptable to engage in self-stimulatory behaviors, what items in the kitchen can be accessed without asking for permission, and so on. Discrimination training procedures are evident in basic instructional lessons such as teaching a child to identify colors to seemingly natural situations such as only scheduling clients on days allowed by funding sources because this results in you being paid for your services. 
\item Carl's teacher determines attention is reinforcing his speaking out in class. Carl's teacher teaches Carl to ask questions when there is a green card present on the board, and not to ask questions when there is a red card on the board. She does this by only delivering attention to Carl when the green card is present on the board, and ignoring Carl when the red card is present. 
\end{enumerate}
%
\subsection{Assessment}
\begin{enumerate}
\item Ask supervisee to provide examples of how discrimination procedures can be used with a specific client or to teach a specific skill.
\item Ask supervisee to clearly operationalize the S-delta and discriminative stimulus contingencies that will be utilized during a specific discrimination training procedure.
\item Observe supervisee describe discrimination procedures to a client, colleague, etc.
\end{enumerate}
%
\subsection{Relevant Literature}
\begin{refsection}
\nocite{cooper2007applied,
        malott2003principles,
        taylor2014discrimination}
\printbibliography[heading=none]
\end{refsection}
%
\subsection{Related Tasks}
\fourdEight{}\\
\foureOne{}\\
\foureThree{}\\
\foureThirteen{}\\
\fourjEleven{}\\
\fourFKEleven{}\\
\fourFKTwentyFour{}\\
\fourFKTwentyFive{}\\
\fourFKThirtyFive{}\\
