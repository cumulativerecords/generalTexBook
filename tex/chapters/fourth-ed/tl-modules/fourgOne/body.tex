\clearpage \section[\fourgOne{}]{\fourgOne{}%
              \sectionmark{G-01 Review records...}}
\sectionmark{G-01 Review records...}
\subsection{Definition}
An important part of understanding the client you are working with is to take time to review the treatment history of this individual. There are records related to a medical history that would be beneficial to see. This would include medical procedures, medications, and any current health concerns. It's also important to get the history of treatment related to psychological/behavioral intervention as well. 

Gresham, Watson, Stewart, \& Skinner in 2001 recommend that FBAs should include a record review to understand more about the history of the client as well as information regarding previous interventions. 

Matson (2010) summarized this position by saying ``The behavior analyst should carefully review records relating to previous attempts to change potential target behaviors. Records indicating previous success with related target behaviors or a history of limited impact on a behaviour despite well-planned and faithfully implemented change programmes may be useful in selecting change targets which can be achieved within meaningful timescales'' (p. 35).

\subsection{Assessment}
\begin{enumerate}
\item Have the supervisee describe where he/she would go to get access to these records. Pick a client and have him/her actually show you where the information is located.
\item Have the supervisee describe the reasons why it's important to do a records review when getting a new client or starting a new behavioral intervention.
\end{enumerate}
%
\subsection{Relevant Literature}
\begin{refsection}
\nocite{watson2001functional,
        matson2009applied}
\printbibliography[heading=none]
\end{refsection}
%
\subsection{Related Tasks}
\fourgOne{}\\
\fourgTwo{}\\
\fourgThree{}\\
\fourgFour{}\\
\fourgFive{}\\
\fourgSix{}\\
\fourgSeven{}\\
\fouriThree{}\\
\fouriFour{}\\
\fourkOne{}\\
