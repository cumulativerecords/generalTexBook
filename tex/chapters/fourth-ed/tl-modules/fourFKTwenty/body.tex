\clearpage \section{\fourFKTwenty{}}
\subsection{Definition}
Conditioned punisher – ``a stimulus that functions as a punisher as the result of being paired with unconditioned or conditioned punishers'' (Cooper, Heron, \& Heward, 2007, p. 40).

Conditioned punishment as defined by Hake and Azrin (1965) is a process that ``results when it can be shown (1) there is little or no punishment effect before the stimulus is paired with an unconditioned punisher, but (2) a punishment effect occurs after (3) the stimulus has been paired, or is being paired, with an unconditioned punisher'' (p. 279).  Evidence of conditioned punishment was suggested in early research when a reduction in a response was observed following the process of pairing a stimulus with an electric shock followed by discontinuing the shock and making the stimulus contingent upon a selected response (Hake \& Azrin, 1965).

\subsection{Examples}
\begin{enumerate}
\item Similar to classical conditioning, a tone (neutral stimulus) is repeatedly paired with an electric shock (unconditioned punisher) whenever a dog barks, in time the tone (conditioned punisher) suppresses the bark in the absence of the electric shock.
\item A child engages in aggression. A parent responds to aggression by taking away their child's favorite video game contingent on every instance of aggression.  The parent begins to pair removal of the video games with a reprimand.  The reprimand may function as a conditioned punisher if aggression continues to decrease following the presentation of a reprimand without taking away the video games. This process illustrates conditioned punishment.
\item Conditioned punishers may be referred to as learned or secondary punishers.
%
\end{enumerate}
%
\subsection{Assessment}
\begin{enumerate}
\item Ask supervisee to explain the process of conditioned punishment.
\item Ask supervisee to define a conditioned punisher.
\item Ask supervisee to provide examples of conditioned punishment and a conditioned punisher.
\item Ask supervisee to identify examples of conditioned punishment in a client's environment.
%
\end{enumerate}
%
\subsection{Relevant Literature}
\begin{refsection}
\nocite{bailey2013ethics,
        cooper2007applied,
        hake1965conditioned,
        iwata1988development}
\printbibliography[heading=none]
\end{refsection}
%
\subsection{Related Tasks}
\fourcTwo{}\\
\fourdFifteen{}\\
\fourdSixteen{}\\
\fourdSeventeen{}\\
\fourdEighteen{}\\
\fourdNineteen{}\\
\fourFKFourteen{}\\
\fourFKSeventeen{}\\
\fourFKEighteen{}\\
\fourFKNineteen{}\\
\fourFKTwentyOne{}\\
