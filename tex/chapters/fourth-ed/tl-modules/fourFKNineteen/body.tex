\clearpage \section{\fourFKNineteen{}}
\subsection{Definition}
Unconditioned punisher – ``A stimulus change that decreases the frequency of any behavior that immediately precedes it irrespective of the organism's learning history with the stimulus'' (Cooper, Heron, \& Heward, 2007, p. 707).*
%
\subsection{Examples}
\begin{enumerate}
\item Bright lights, loud sounds, extreme temperatures, certain tastes (sour, bitter), physical restraint, loss of bodily support, extreme muscular efforts, etc.
%
\end{enumerate}
%
\subsection{Assessment}
\begin{enumerate}
\item Ask the supervisee to describe an example of unconditioned punishment.
\item Use the supervisee to describe the difference between unconditioned punishment and an unconditioned punisher.
\item Ask the supervisee to list as many unconditioned punishers as possible in one minute.
%
\end{enumerate}
%
\subsection{Relevant Literature}
\begin{refsection}
\nocite{cooper2007applied,
        herman1964punishment}
\printbibliography[heading=none]
\end{refsection} 
%    
%
\subsection{Related Tasks}
\fourdSeventeen{}\\
\fourdSixteen{}\\
\fourdNineteen{}\\
\foureEleven{}\\
\fourgSeven{}\\
\fourjTen{}\\
\fourFKTwenty{}\\
%
\subsection{Footnotes}
*Conditioned punishers are products of the evolutionary development of the species (Cooper, Heron, \& Heward, 2007).\\
*Conditioned punishers are also called primary or unlearned punishers (Cooper, Heron, \& Heward, 2007).\\
