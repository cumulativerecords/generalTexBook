\clearpage \section[\fourdTwenty{}]{\fourdTwenty{}%
              \sectionmark{D-20 Use response-independent...}}
\subsection{Definition}
Noncontingent reinforcement (NCR) – ``A procedure in which stimuli with known reinforcing properties are presented on fixed-time (FT) or variable time (VT) schedules completely independent of behavior; often used as an antecedent intervention to reduce problem behavior'' (Cooper, Heron \& Heward, 2007, p. 700).

Noncontingent reinforcement is sometimes used in applied research in an experimental design called the NCR reversal technique. This design involves a baseline phase, NCR phase (where a potential reinforcer is delivered on a fixed or variable time schedule independent of the target behavior), and a phase where the reinforcer is delivered contingent on a set behavioral criteria. The phases are repeated as necessary to indicate experimental control over the dependent variable. The NCR and baseline conditions function as a comparative measure to validate the independent variable in these studies.

Differential reinforcement procedures may limit access to reinforcement if appropriate behavior occurs at low rates. NCR gives consistent access to reinforcement. 
%
\subsection{Examples}
\begin{enumerate}
\item A DRO program was trialed for 3 weeks to decrease Jimmy's verbal protesting during group activities. Based on data collected, the DRO program was deemed ineffective for decreasing Jimmy's verbal protesting. Mr. Jones took data and found that Jimmy could quietly engage in group activities for 3 minutes before starting to protest. Mr. Jones decided to implement a 2 minute NCR program in which he would give Jimmy a sticker every 2 minutes regardless of the presence of interfering behaviors.
\item (Non-Example) Mr. Michael was concerned with Barry's aggressive behavior during group activities. He decided to give him a sticker for every 2 minutes that he did not engage in aggressive behavior.
\end{enumerate}
%
\subsection{Assessment}
\begin{enumerate}
\item Have supervisee create an NCR program and explain procedures to other supervisees.
\item Have Supervisee identify the difference between differential reinforcement and noncontingent reinforcement and give examples of both.
\item Have supervisee give examples of NCR used in his/her professional and nonprofessional life.
\end{enumerate}
%
\subsection{Relevant Literature}
\begin{refsection}
\nocite{cautela1984general,
       cooper2007applied,
       hagopian1994schedule,
       ingvarsson2008some,
       wilder2005noncontingent}
\printbibliography[heading=none]
\end{refsection}
%
\subsection{Related Tasks}
\fourbFour{}\\
\fourcOne{}\\
\fourdTwo{}\\
\fourdTwentyOne{}\\
\fourjTwo{}\\
