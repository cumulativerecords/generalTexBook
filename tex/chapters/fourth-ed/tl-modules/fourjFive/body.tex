\clearpage \section[\fourjFive{}]{\fourjFive{}%
              \sectionmark{J-05 Select... repertoires.}}
\sectionmark{J-05 Select... repertoires.}
\subsection{Definition}
\begin{enumerate}
\item \textit{Importance of considering the client's current repertoires.} Basing intervention strategies on the client's current repertoires is a key foundation of what behavior analysts do. It is imperative that prior to implementing any type of intervention or strategy with a client, the behavior analyst is extremely clear about what the client already does and can therefore, consider possible intervention strategies. Noell, Call, and Ardoin (2011) state that ``one of the considerable challenges in teaching arises from identifying not only the behaviors that are prerequisites for the target response, but also the level of skill proficiency needed to set the occasion for teaching the target skill'' (Noell, Call, \& Ardoin, 2011, cited from Fisher, Piazza, \& Roane, 2011, p. 251).
%
\item \textit{Importance of accurate assessments.} In order to assess a client's current repertoires, it is imperative that these repertoires are properly assessed. For example, assessments to evaluate the presence of a particular skill or repertoire should take place in a variety of different environments, with many different examples of stimuli, with the antecedent presented in a variety of different ways, and with many different people presenting the skill. Novel examples of the skill should also be tested. Noell et al. (2011) emphasize this point by suggesting that ``assessment of behavior under varied conditions in a manner that tests consequences should be an element of any pre-teaching assessment'' (Noell, Call \& Ardoin, 2011, cited from Fisher, Piazza \& Roane, 2011, p. 255). 

\item \textit{After completing assessments of the client's current repertoires.} Once the assessment stage is complete, it is then appropriate to select possible intervention strategies. As Noell et al. (2011) propose it is important at this point that behavior analysts ``keep the long-term view in mind'' (Noell, Call \& Ardoin, 2011, cited from Fisher, Piazza \&Roane, 2011, p. 266).  We should be attempting to ``not bring individual operants under stimulus control'' but instead ``help clients and students develop the complex, flexible repertoires that are adaptive, that remain in contact with reinforcement, and that confer adaptive advantage and endure'' (Noell, Call \& Ardoin, 2011, cf. Fisher, Piazza \& Roane, 2011, p. 266).
%
\end{enumerate}
%
\subsection{Assessment}
\begin{enumerate}
\item Ask your Supervisee to explain why it's important as a behavior analyst to select interventions based on the client's current repertoires. 
\item Ask your Supervisee to conduct an assessment with a client, if possible, and design potential interventions or strategies. Give feedback as appropriate.
\item Have your Supervisee take the lead on assessing and developing interventions with your supervision. 
%
\end{enumerate}
%
\subsection{Relevant Literature}
\begin{refsection}
\nocite{fisher2014handbook,
        noell2011building,
        shapiro2011academic}
\printbibliography[heading=none]
\end{refsection}
%
\subsection{Related Tasks}
\fourdNine{}\\
\fourgThree{}\\
\fouriThree{}\\
\fourjThree{}\\
\fouriFour{}\\
\fourjTwo{}\\
\fourjSix{}\\
\fourjSeven{}\\
\fourjEight{}\\
