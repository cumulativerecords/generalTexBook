\clearpage \section{\fourFKFourtyFour{}}
\subsection{Definition}
Mand - ``An elementary verbal operant that is evoked by an MO and followed by specific reinforcement'' (Cooper, Heron, \& Heward, 2007, p. 699).\\

The form of the response is specific and under control of motivating operations. Response topography can vary: vocal, sign language, augmentative communication, pushing, reaching, hitting, etc.
%
\subsection{Examples}
\begin{enumerate}
\item ``I want a cookie.'' (This is a mand for an item. Mands can include verbs, use of adjectives, prepositions, pronouns etc.)
\item A child says ``watch me'' after learning how to ride a bike independently (mand for attention)
\item Asking questions like ``what's your name? or ``where's the phone?'' (mand for information)
\item Child says, ``No!'' when parent is about to use blender (mand for avoidance of an aversive)
%

%
\end{enumerate}
%
\subsection{Assessment}
\begin{enumerate}
\item Ask your Supervisee to recall how they asked for supervision
\item Ask your Supervisee to list the types of mands they would emit if they were lost in a foreign county and needed directions to a local gas station
\item Ask you supervisee to list 5 ways they use mands in an inappropriate way (eg. complain about work to get attention)
%
\end{enumerate}
%
\subsection{Relevant Literature}
\begin{refsection}
\nocite{cooper2007applied,
        laraway2003motivating,
        michael1988establishing,
        sundberg2001benefits,
        sweeney2007transferring}
\printbibliography[heading=none]
\end{refsection}
%
\subsection{Related Tasks}
\fourdNine{}\\
\fourdEleven{}\\
\fourFKTwentySeven{}\\
\fourFKTwentyEight{}\\
