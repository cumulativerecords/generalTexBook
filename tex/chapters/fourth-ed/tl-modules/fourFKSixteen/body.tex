\clearpage \section{\fourFKSixteen{}}
\subsection{Definition}
Respondent behavior - ``A response component of a reflex; behavior that is elicited, or induced, by antecedent stimuli'' (Cooper, Heron, \& Heward, 2007, p. 703).

Operant behavior – ``Behavior that is selected, maintained, and brought under stimulus control as a function of its consequences'' (Cooper, Heron, \& Heward, 2007, p. 701).

Operant and respondent behavior interact very commonly. They may occur concurrently when a stimulus both evokes an operant response while at the same time elicits a respondent response on the part of the organism. The procedures involved with what we call operant or respondent conditioning are names of procedures for the ease of use of our field. There are respondent and operant interactions occurring whenever an organism behaves. 

Pierce and Cheney (2013) describe it this way: ``When biologically relevant stimuli such as food are contingent on an organism's operant behavior, species-characteristic, innate behavior is occasionally elicited at the same time'' (p. 194). The presence of stimuli that have been paired with aversive or appetitive stimulation will elicit respondent behavior at the same time operant behavior is occurring to access or avoid those stimuli.

``The neural capacity for operant conditioning arose on the basis of species history; organisms that changed their behavior as result of life experience had an advantage over animals that did not do so'' (Pierce \& Cheney, 2013, p. 194).

Certain respondent behavior interacts with operant behavior. The effects are often described as motivating operations. For instance, behavior changes before and after meal times, with or without medications, after traumatic events, or disruptions in family life. 

\subsection{Examples}
\begin{enumerate}
\item  After a traumatic event involving physical abuse, every time a male walks into the room, your client ``freezes'' and does not follow instructions. This could be due to elicited behavior (``freezing'' in the presence of conditioned aversive stimuli) in competition with operant behavior (following instructions).
\item A medication, when consumed, will elicit respondent behavior that makes certain things more or less aversive. Consider if your client starts taking a medication to decrease aggression maintained by access to toys. The effect of the medication may decrease the likelihood that toys function as a reinforcer in effect decreasing the amount of aggression. It may increase the likelihood that food functions as a reinforcer.
\end{enumerate}
%
\subsection{Assessment}
\begin{enumerate}
\item Ask your supervisee to describe how respondent behavior can interact with operant behavior.
\item Ask your supervisee to give an example of when this might occur with one of his/her clients during a specific treatment procedure.
\end{enumerate}
%
\subsection{Relevant Literature}
\begin{refsection}
\nocite{cooper2007applied,
        pierce2013behavior,
        davis1977operant}
\printbibliography[heading=none]
\end{refsection}
%
\subsection{Related Tasks}
\fourgTwo{}\\
\fourgFive{}\\
\fourFKSeven{}\\
\fourFKThirteen{}\\
\fourFKFourteen{}\\
\fourFKFifteen{}\\
