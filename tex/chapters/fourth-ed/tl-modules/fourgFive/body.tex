\clearpage \section[\fourgFive{}]{\fourgFive{}%
              \sectionmark{G-05 Describe and explain behavior...}}
\sectionmark{G-05 Describe and explain behavior...}
\subsection{Definition}
Behavior analysts must have a strong verbal repertoire when speaking about behavior analysis.  This includes using behavior-analytic language when describing and explaining behavior, including private events.  

Skinner's radical behaviorism rejected psychological models of behavior that relied on mentalistic explanations. Mentalistic approaches attributed the origination and cause of behavior to ``inner'' dimensions or mental entities (i.e., hypothetical constructs and explanatory fictions such as the unconscious or psyche).  Mentalistic explanations of behavior often neglect the consideration and analysis of controlling variables in the environment and use circular reasoning to explain the cause and effect of behavior.  Understanding the philosophy of radical behaviorism and the principles of behavior can assist behavior analysts in explaining behavior in behavior-analytic terms.

For example, you are conducting a functional behavior assessment in a school setting for a student who engages in high rates of aggression in the classroom.  The teacher tells you that the student's aggression occurs because the student is frustrated and lives in an unpleasant environment at home.  This is a mentalistic explanation.  After several observations, you have determined that when academic demands are placed, the student engages in aggression and their aggression is reinforced by escape (i.e., academic demands are removed). This is a behavior-analytic explanation of behavior that accounts for behavior as it is a function of environmental variables.  

Using behavior-analytic language to explain and describe behavior can be difficult as we are often exposed to mentalistic explanations (e.g., ``wanting to'' or ``felt like it'' as causes of behavior).  Read Malott and Trojan-Suarez (2004) for a discussion about circular reasoning and talking about behavior.

\subsection{Assessment}
\begin{enumerate}
\item Ask supervisee to explain and describe behavior in behavior-analytic terms.
\item Provide supervisee with examples of mentalistic explanations of behavior and ask supervisee to provide behavior-analytic explanations.
\item Observe supervisee explain behavior in behavior-analytic terms to a colleague or client.
%
\end{enumerate}
%
\subsection{Relevant Literature}
\begin{refsection}
\nocite{cooper2007applied,
        malott2008principles,
        moore2008conceptual}
\printbibliography[heading=none]
\end{refsection}
%
\subsection{Related Tasks}
\fourFKSeven{}\\
\fourFKEight{}\\
\fourFKThirtyOne{}\\
\fourFKThirtyThree{}\\
\fourgFour{}\\
\fouriOne{}\\
\fouriTwo{}\\
