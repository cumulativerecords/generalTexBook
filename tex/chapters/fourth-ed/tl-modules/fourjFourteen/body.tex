\clearpage \section[\fourjFourteen{}]{\fourjFourteen{}%
              \sectionmark{J-14 Arrange instructional proc...}}
\sectionmark{J-14 Arrange instructional proc...}
\subsection{Definition}
Generative learning involves applying learning to novel contexts without being explicitly taught and is related to language and cognition. Deriving relations is based on the stimulus equivalence paradigm and procedures (Sidman, 1971). A small number of taught relations among stimuli may generate numerous derived relations (Wulfert \& Hayes, 1988). Readers are encouraged to understand stimulus equivalence prior to arranging instructional procedures to promote generative learning. Experimental procedures often utilize matching-to-sample tasks or a computerized program called Implicit Relational Assessment Procedure (IRAP) to teach derived relations. It has been extensively studied in children with autism (Kilroe, Murphy, Barnes-Holmes, \& Barnes-Holmes, 2014). The general instructional procedure involves providing explicit reinforcement for a series of conditional discriminations, after which untrained relations (i.e., derived relations) will emerge and can be subsequently reinforced. 

Stimulus equivalence is one of several empirically supported examples of derived relations, as relations can be derived based on opposition, temporality, analogy, comparison, and distinction (Stewart, McElwee, \& Ming, 2013). Relational Frame Theory (RFT) was developed as a behavior analytic account of human language and cognition (Hayes, Barnes-Holmes, \& Roche, 2001) and addresses the need for a theoretical explanation for generative learning, Resources for learning RFT are included in the relevant literature section.  

\subsection{Examples}
\begin{enumerate}
\item  A learner is taught that spoken word ``apple'' = picture of an apple = written word ``apple'' = picture of an apple. These two relations are directly taught. However, through this explicit training, the learner can derive that the spoken word ``apple'' = written word ``apple.'' In this example, reinforcement should occur for successful matching of the two trained relations.  Once the third relation is derived, the response should be reinforced. 
\end{enumerate}
%
\subsection{Assessment}
\begin{enumerate}
\item Ask supervisee to provide a definition of derived relations.
\item Ask supervisee to provide examples of derived relations.
\item Supervisor should ensure that supervisee thoroughly understands stimulus equivalence, matching-to-sample procedures, and conditional discriminations, in order to arrange instructional procedures to promote generative learning.  Once this foundational learning has occurred, supervisor can ask supervisee to demonstrate teaching procedures that will facilitate generative learning.
\end{enumerate}
%
\subsection{Relevant Literature}
\begin{refsection}
\nocite{hayes2001relational,
        sidman1994equivalence,
        stewart2013language,
        torneke2010learning,
        wulfert1988transfer}
\printbibliography[heading=none]
\end{refsection}
%
\subsection{Related Tasks}
\foureSix{}\\
\fourFKEleven{}\\
\fourFKTwentyFour{}\\
\fourFKThirtyFour{}\\
