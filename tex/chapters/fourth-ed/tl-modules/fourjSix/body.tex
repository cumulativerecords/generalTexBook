\clearpage \section[\fourjSix{}]{\fourjSix{}%
              \sectionmark{J-06 Select... environments.}}
\sectionmark{J-06 Select... environments.}
\subsection{Definition}
``Achieving optimal generalized outcomes requires thoughtful, systematic planning. This planning begins with two major steps: (1) selecting target behaviors that will meet natural contingencies of reinforcement, and (2) specifying all desired variations of the target behavior and the settings/situations in which it should (and should not) occur after instruction has ended'' (Cooper, Heron, \& Heward, 2007, p. 623). In other words, an intervention must be selected that will allow the client to access reinforcement in a specific environment. If that is not possible, then alternative interventions should be explored. 

Ayllon and Azrin (1968) state that an important rule of thumb is to choose interventions that will help produce reinforcement after the intervention is discontinued. The intervention should support the student until they can access naturally existing contingencies (i.e., verbal praise from a teacher) and then more intensive, contrived contingencies should be systematically faded. The goal of most intervention programs is to teach a skill and then fade support so the client can implement that skill across settings. 

Cooper, Heron, \& Heward (2007, p. 626) identify 5 strategic approaches to promote generalized behavior change.
\begin{enumerate}
\item Teach the full range of relevant stimulus conditions and response requirements (i.e., teaching sufficient stimulus and response examples based on the setting)
\item Make the instructional setting similar to the generalization setting. (i.e., program common stimuli and teach loosely)
\item Maximize the target behavior's contact with reinforcement in the generalization setting. (i.e., ask people in the generalization setting to reinforce the target behavior, teach the learner to recruit reinforcement, and teach the target behavior to levels of performance required by natural existing contingencies of reinforcement.)
\item Mediate generalization (i.e., teach self-management skills \& contrive mediating stimulus)
\item Train to generalize. (i.e., reinforce response variability and instruct learner to generalize)
\end{enumerate}
%
\subsection{Examples}
\begin{enumerate}
\item George set up a token economy for Bill that systematically increased the number of responses needed to earn a token. After some time, Bill was earning tokens for completing an entire worksheet rather than earning a token for each question answered. This allowed Bill to independently complete worksheets in his general education classroom without a paraprofessional by his side giving him tokens after each answer. 
%
\end{enumerate}
%
\subsection{Assessment}
\begin{enumerate}
\item Give supervisee a reinforcement program. Have him/her create a fading procedure for this program to increase the number of responses required to earn a token. 
\item Have supervisee list the five strategies for promoting generalized behavior change and have him/her give examples of each.
\item Have supervisee describe and differentiate between contrived contingencies and naturally existing contingencies and give several examples of each.
%
\end{enumerate}
%
\subsection{Relevant Literature}
\begin{refsection}
\nocite{ayllon1968token,
        baer1999plan,
        cooper2007applied,
        snell2006instruction,
        stokes1977implicit,
        stokes1989operant}
\printbibliography[heading=none]
\end{refsection}
%
\subsection{Related Tasks}
\fourgEight{}\\
\fourjSeven{}\\
\fourjEight{}\\
\fourjEleven{}\\
\fourjTwelve{}\\
\fourkSeven{}\\
\fourkNine{}\\
