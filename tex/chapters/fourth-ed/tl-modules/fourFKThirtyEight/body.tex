\clearpage \section{\fourFKThirtyEight{}}
\subsection{Definition} 
Behavioral contrast - ``The  phenomenon in which a change in one component of a multiple schedule that increases or decreases the rate of responding on that component is accompanied by a change in the response rate in the opposite direction on the other, unaltered component of the schedule; behavior punished in one situation may increase in other situations where it's not punished... Contrast effects of punishment can be minimized, or prevented altogether, by consistently punishing occurrences of the target behavior in all relevant settings and stimulus conditions, withholding or at least minimizing the person's access to reinforcement for the target behavior, and providing alternative desirable behaviors'' (Cooper, Heron, \& Heward, 2007, p. 337)
%
\subsection{Examples}
\begin{enumerate}
\item Rich sneaks candy from home and eats it in class. His teacher catches him one day and he stops eating candy in the classroom. However, he now asks to go to the bathroom every morning and eats candy in the bathroom where the teacher cannot see him. 
%
\end{enumerate}
%
\subsection{Assessment}
\begin{enumerate}
\item Have supervisee describe the principles of punishment and reinforcement and how behavioral contrast relates to each.
\item Have supervisee explain why rates may fluctuate based on punishment and reinforcement contingencies in multiple schedules.
\item Have supervisee give examples of behavioral contrast from their experiences and what they have done in those circumstances.
%
\end{enumerate}
%
\subsection{Relevant Literature}
\begin{refsection}
\nocite{cooper2007applied,
        gross1981behavioral,
        koegel1980behavioral,
        nevin1992behavioral,
        reynolds1961behavioral}
\printbibliography[heading=none]
\end{refsection}
%
\subsection{Related Tasks}
\fourcOne{}\\
\fourcTwo{}\\
\fourdOne{}\\\
\fourdTwo{}\\\
\fourdFifteen{}\\
\fourdSixteen{}\\
\fourdSeventeen{}\\
\fourdNineteen{}\\
\foureSeven{}\\
\fourFKEighteen{}\\
\fourFKTwenty{}\\
\fourFKTwentyOne{}\\
