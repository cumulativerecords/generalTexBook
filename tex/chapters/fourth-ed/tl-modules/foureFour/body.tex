\clearpage \section[\foureFour{}]{\foureFour{}%
              \sectionmark{E-04 Use contingency contracting...}}
\subsection{Definition}
Contingency contract -  ``...also called a behavioral contract, is a document that specifies a contingent relationship between the completion of a specified behavior and access to, or delivery of, a specified reward such as free time, a letter grade or access to a preferred activity'' (Cooper, Heron \& Heward, 2007, p. 551). 

Contingency contracts have several components: 
\begin{enumerate}
\item Outlines the task to be completed- includes an objective definition of the task, who must complete the task and when the task must be completed.
\item Specifies the reward contingent on task completion- includes description of the reward, who will deliver the reward, who will measure, whether the task has been completed to criterion, and when the reward will be received.
\item Outlines how performance will be measured and what data will be taken. 
\end{enumerate}

Contingency contracts can be highly effective if used properly because the individual whose behavior is to be changed is involved in the process from the start. Studies have shown that contingency contracting ``...has been identified as an important step toward self-management of behavior'' (Miller \& Kelley, 1994, p. 74) because by helping to determine the parameters of the task and outlining what and when rewards should be given, reinforcer assessments have already been identified and the individual is already motivated to engage in the target behavior, which can greatly increase compliance. However, Cooper, Heron and Heward (2007) caution against using contingency contracts with all populations. There must be set criteria that the individual must already possess in order for contingency contracts to be effective. The target behavior must already be in the individual's repertoire and the individual must already be able to discriminate when and which environments are appropriate for the response to occur. Additionally, the individual's behavior must be able to ``come under the control of the visual or oral statements (rules) of the contract'' (Cooper, Heron, \& Heward, 2007, p. 558).  The individual does not need to be proficient in reading so long as the contract is asdapted using symbols, icons, photographs, etc. and the individual thoroughly understands the reinforcement contingency. 
%
\subsection{Assessment}
\begin{enumerate}
\item Have the supervisee design a contingency contract.
\item Have the supervisee describe each component of the contract.
\end{enumerate}
%
\subsection{Relevant Literature}
\begin{refsection}
\nocite{cooper2007applied,miller1994use}
\printbibliography[heading=none]
\end{refsection}
%
\subsection{Related Tasks}
\fourkTwo{}\\
\fourFKFourtyTwo{}\\
