\clearpage \section[\fourjTwo{}]{\fourjTwo{}%
              \sectionmark{J-02 Identify potential interv...}}
\sectionmark{J-02 Identify potential interv...}
\subsection{Definition}
``Interventions should be functionally equivalent to problem behavior'' (Cooper, Heron, \& Heward, 2007, p. 513). In other words, the intervention should serve as a more appropriate way of accessing a specific function than the interfering behavior. ``One effective way to design interventions is to review confirmed hypotheses to determine how the ABC contingency can be altered to promote more positive behavior'' (Cooper, et al., 2007, p. 513).  In other words if you can change the antecedents or consequences associated with a problem behavior, you may be able to decrease occurrences of that behavior. 

Wolf (1978) recommended that interventions should be assessed based on appropriateness and the potential social significance of the outcomes. Specific social validity assessments can be conducted ``to help choose and guide (behavior change) program developments and applications'' (Baer \& Schwartz, 1991, p. 231).

Results of an FBA can help determine which interventions would not be appropriate to decrease the target behavior. Once assessments are completed, monitor the progress of the interventions put in place and conduct follow up assessments regularly. Any intervention program should be based on techniques found in behavior analytic literature. This ensures that the intervention is a proven technology and has stood up to experimental manipulation and decreased or increased similar types of behavior.

In regard to assessments that are not based on the function of the behavior, (i.e., adaptive assessments, verbal behavior assessments, cognitive assessments) a profile of strengths and weaknesses as well as standard scores and rankings are typically provided once the assessment is completed. The areas that present as weaknesses should typically be addressed first based on their social significance to the client. Once these areas have been identified, appropriate interventions can be put into place to build the client's missing repertoires and skill base. 
 
\subsection{Examples}
\begin{enumerate}
\item George completed an FBA on Sam's instances of aggression in the classroom. The hypothesized function for this behavior was escape from demands. George created an intervention plan that would allow Sam to functionally ask for breaks to briefly escape work demands. 
\end{enumerate}
%
\subsection{Assessment}
\begin{enumerate}
\item Assign a specific behavior as well as a hypothesized function of that behavior. Have supervisee research and identify 3 potential interventions to decrease the target behavior based on the hypothesized function. 
\item Have supervisee explain under what circumstances they would use each intervention.
%
\end{enumerate}
%
\subsection{Relevant Literature}
\begin{refsection}
\nocite{baer1991if,
        cooper2007applied,
        watson2001functional,
        lerman1994transfer}
\printbibliography[heading=none]
\end{refsection}
%
\subsection{Related Tasks}
\foureOne{}\\
\fouriOne{}\\
\fouriTwo{}\\
\fouriFour{}\\
\fouriSix{}\\
\fourjOne{}\\
\fourjTwo{}\\
\fourjTen{}\\
