\clearpage \section{\fourFKTwentySeven{}}
\subsection{Definition}
Conditioned motivating operation - ``A motivating operation whose value-altering effect depends on a learning history'' (Michael, as cited in Cooper, Heron, \& Heward, 2007, p. 384).\\

Three types of conditioned motivating operations (CMOs): surrogate (CMO-S), reflexive (CMO-R), and transitive (CMO-T)\\



\subsection{Examples}
Surrogate CMOs replace and have the same effect as the motivating operation that it was previously paired with.\\

Example: A rat is placed in a distinctive environment when food deprived. This is repeated a number of times. Over time, the rat is placed in the same environment when they have not been deprived of food. The distinctive environment and it's relation to a state of food deprivation results in an increase in the value of food as a reinforcer and an increase in the frequency of behavior with a history of producing food. In this example, the distinctive environment is paired with a unconditioned motivating operation (food deprivation). Over time, the distinctive environment functions as a motivating operation in the absence of food deprivation. 

Reflexive CMOs create a circumstance in which its own removal serves as the reinforcement.\\

Example: The presence of instructional materials often precedes the presentation of instructional tasks. If an individual engages in behavior maintained by access to escape from instructional tasks, in time they may engage in escape maintained behavior in the presence of instructional materials and the removal of these materials may function as a reinforcer. In this example, the instructional materials serve as a CMO-R.

Transitive CMOs make other stimuli more effective reinforcers.\\

Example: A locked door functions as a CMO-T to establish a key as a reinforcer.
%
%
\subsection{Assessment}
\begin{enumerate}
\item Have supervisee describe the three types of CMOs. Have him/her give examples of each. 
\item Have supervisee explain the definitions for conditioned and unconditioned motivating operations in simple terms that someone who does not have ABA experience can understand.
\item Have supervisee explain how to weaken the effects of each of the three types of CMO. 
%
\end{enumerate}
%
\subsection{Relevant Literature}
\begin{refsection}
\nocite{catania1993coming,
        clark1958effect,
        cooper2007applied,
        endicott2007contriving,
        hesse1993establishing,
        iwata2000current,
        lotfizadeh2012motivating,
        michael1993establishing}
\printbibliography[heading=none]
\end{refsection}
%
\subsection{Related Tasks}
\fourdOne{}\\
\foureOne{}\\
\fouriTwo{}\\
\fourFKTwo{}\\
\fourFKThirteen{}\\
\fourFKFourteen{}\\
\fourFKSeventeen{}\\
\fourFKNineteen{}\\
\fourFKTwentySix{}\\
\fourFKTwentyEight{}\\
\fourFKTwentyNine{}\\
\fourFKThirty{}\\
