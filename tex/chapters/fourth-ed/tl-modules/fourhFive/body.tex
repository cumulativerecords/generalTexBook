\clearpage \section[\fourhFive{}]{\fourhFive{}%
              \sectionmark{H-05 Evaluate temporal relations...}}
\sectionmark{H-05 Evaluate temporal relations...}
\subsection{Definition}
Behavior analysts can analyze data across several temporal relations prior to visual inspection. ``The manner in which data are aggregated before transforming them into a visual display serves an equally influential role in data analysis'' (Fahmie \& Hanley, 2008, p. 320).  Such aggregation occurs with the use of within-session, between-session in time series data. 

Between-session analysis involves plotting total number of occurrences of a dependent variable within some unit of time (i.e., sessions) and visually inspecting point-by-point (i.e., session-by-session). Another prevalent type of aggregation occurrences of behavior is within-session data (likely due to its universal application). 

Within-sessing data can be analyzed via the observation of data as it changes throughout the duration of the session or at specific times during the session. Fahmie and Hanley (2008) outlined eight conditions under which within-session data are valuable:
\begin{enumerate}
\item Description of naturally occurring behavioral relations (descriptive assessment) 
\item Determination of behavioral function (functional analyses)
\item Detection of within-session trends
\item Safeguard clients from any risks associated with prolonged session exposure
\item Creation of sufficient data for analysis following abbreviated data collection
\item Determination of observation session duration
\item Clarification of counterintuitive response patterns
\item Understanding behavioral processes
\end{enumerate}

There are several methods of within-session data analysis. In the descriptive assessment literature, within-session data are calculated via conditional probabilities to determine possible temporal relations between behavior and environmental events (e.g., occurrence/nonoccurrence of putative reinforcer delivery) (Vollmer, Borrero, Wright, Camp, \& Lalli, 2001). In the functional analysis literature, within-session data have been used to compare the utility of two types of functional analyses (e.g., trial-based versus multi-element) (Kahng \& Iwata, 1999; LaRue et al., 2010). Moreover, within-session data have been used in an effort to further analyze unclear results following an unclear analysis of full session data (Call \& Mevers, 2014; Kahng \& Iwata, 1999; Payne et al., 2014; Roane, Lerman, Kelley, \& VanCamp, 1999; Vollmer, Marcus, Ringhdahl \& Roane, 1995; Vollmer et. al., 1993). For example, Kahng \& Iwata (1999) compared full 15-minute functional analysis session data with within-session data by plotting the first session of each condition into a minute-by-minute observation period. One of their findings was that within-session data clarified unclear (absence of function) results of the full session data. 

In another example, Payne et al., (2014) analyzed within-session data in different manner by comparing data when the putative establishing operation (EO) was present versus when the putative EO was absent across the last five 10-minute sessions of each condition. The results generated from the within-session data analysis was used to develop a second experimental analysis that clarified the function of the behavior for the two participants.
%
\subsection{Assessment}
\begin{enumerate}
\item Have the supervisee read the Fahmie \& Hanley (2008) article. Then provide the supervisee with examples of different data analysis units along the continuum the authors display in Figure 1. Have them place the scenarios along the continuum and discuss. 
\item Have the supervisee describe different methods of within session data collection ( e.g. minute-by-minute, event based observation period comparisons) and their utility ( a review of the methods used in the relevant literature will assist with this task)
%
\end{enumerate}
%
\subsection{Relevant Literature}
\begin{refsection}
\nocite{call2014relative,
        fahmie2008progressing,
        hartmann1980interrupted,
        iwata2000skill,
        krause1999long,
        larue2010comparison,
        payne2014using,
        roane1999within,
        tryon1982simplified,
        vollmer2001identifying,
        vollmer1993within,
        vollmer1995progressing}
\printbibliography[heading=none]
\end{refsection}
%
\subsection{Related Tasks} 
\fourhOne{}\\
\fourhFour{}\\
\fouriFive{}\\
\fourjFifteen{}\\
\fourFKFourtySeven{}\\
