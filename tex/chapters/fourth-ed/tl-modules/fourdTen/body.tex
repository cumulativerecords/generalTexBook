\clearpage \section{\fourdTen{}}
\subsection{Definition} 
Echoics are units of verbal behavior that are, ``evoked by a verbal discriminative stimulus that has point-to point correspondence and formal similarity with the response'' (Cooper, Heron, \& Heward, 2014, p. 694).  

Repeating words, phrases, or other auditory verbal units is common for all speakers in day-to-day speech (Sundberg, 2008).

Echoic training, sometimes referred to as vocal imitation training, is a procedure in which a caregiver or teacher emits a sound and a listener echoes or repeats what has been said (Skinner, 1957).  Reinforcement (either social, tangible or other) is often delivered after the correct utterance is given.  

Echoic training can be used to teach a variety of skills such as:
\begin{enumerate}
\item Mands-such as when you give a child a full verbal model of the appropriate way to ask for another cup of milk ``I want milk'' and he repeats this phrase.
\item Tacts- such as telling a classroom full of Spanish students that the word for dog is ``perro'' and asking them to repeat this word back to you.
\item Intraverbal behavior- such as an elementary school teaching modeling the fill in of the word ``star'' after saying ``Twinkle, twinkle little... '' and pausing (Skinner, 1957).
\end{enumerate}

When using echoic training, the trainer should:
\begin{enumerate}
\item Deliver the verbal discriminative stimulus (the word, sound or phrase you intend them to repeat).
\item Provide positive reinforcement for responses that have point-to-point similarity to the target response.  
\end{enumerate}
%
\subsection{Assessment}
\begin{enumerate}
\item Ask the supervisee to state what echoic training can be used to teach.
\item Ask the supervisee to give examples of echoic training.
\item Ask the supervisee to discuss how echoic training should be delivered.
\end{enumerate}
%
\subsection{Relevant Literature}
\begin{refsection}
\nocite{drash1999using,
    cooper2007applied,
    kodak2009acquisition,
    stock2008comparison,
    sundberg2008verbal,
    skinner1957verbal}
\printbibliography[heading=none]
\end{refsection}
%
\subsection{Related Tasks}
\fourdOne{}\\
\fourdFour{}\\
\fourFKFourtyThree{}\\
