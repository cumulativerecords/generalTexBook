\clearpage \section{\fourFKTwentySix{}}
\subsection{Definition}
Unconditioned motivating operations - ``...events, operations, and stimulus conditions with value-altering motivating effects that are unlearned'' (Michael, as cited in Cooper, Heron, \& Heward, 2007, p. 377).\\
Deprivation of basic human needs such as water, food, and sleep all create ``evocative effects'' that establish these items as reinforcers.\\

Cooper et al., 2007 identifies nine unconditioned motivating operations (UMOs) including food deprivation, water deprivation, sleep deprivation, activity deprivation, oxygen deprivation, sex deprivation, becoming too warm or cold, and an increase in painful stimulation. The withholding of any of these will lead to an increase in the reinforcing value of obtaining that which has been deprived.\\

On the other hand, when there is no longer deprivation, this serves as a UMO having an abative effect on behavior, making it less likely to occur.
%
\subsection{Examples}
\begin{enumerate}
\item Roger has not slept in three days because he has been studying for his chemistry final. Sleep becomes more valuable the more deprived of sleep he gets. 
\item (Non-example) Roger lost his key to his apartment and cannot get in. The locked door serves as motivation for him to find his key to get into his apartment. The key serves as a reinforcer because his learning history identifies this as the only way to unlock his door and get into his apartment.
%
\end{enumerate}
%
\subsection{Assessment}
\begin{enumerate}
\item Have supervisee identify at least 7 unconditioned motivating operations. Have him/her describe their reinforcer-establishing effect as well as their evocative effect. (see page 379, table 16.1 in Cooper et al., 2007)
\item Have supervisee identify UMOs that decrease reinforcer effectiveness and abate relevant behavior. Have him/her describe the reinforcer-abolishing effect and the abative effect of each UMO. (See page 380, table 16.2 in Cooper et al., 2007).
\item Have supervisee explain how to weaken the effects of a UMO. (See page 380-381 in Cooper et al., 2007).
\item Have supervisee explain the difference between motivating operations and discriminative stimuli. (See page 377 in Cooper et al., 2007).
%
\end{enumerate}
%
\subsection{Relevant Literature}
\begin{refsection}
\nocite{cooper2007applied,
        laraway2001abative,
        lotfizadeh2012motivating,
        michael1982distinguishing,
        michael2000implications,
        ulrich1962reflexive}
\printbibliography[heading=none]
\end{refsection}
%
\subsection{Related Tasks}
\fourdOne{}\\
\foureOne{}\\
\fouriTwo{}\\
\fourFKTwo{}\\
\fourFKThirteen{}\\
\fourFKFourteen{}\\
\fourFKSeventeen{}\\
\fourFKNineteen{}\\
\fourFKTwentySeven{}\\
\fourFKTwentyEight{}\\
\fourFKTwentyNine{}\\
\fourFKThirty{}\\
