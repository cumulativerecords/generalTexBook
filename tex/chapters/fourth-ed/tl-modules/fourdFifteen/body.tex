\clearpage \section{\fourdFifteen{}}
\subsection{Definition}
Punisher – ``A stimulus change that decreases the future frequency of behavior that immediately precedes it'' (Cooper, Heron, \& Heward, 2007, p. 702).  

Punishers can be categorized as unconditioned or conditioned. Unconditioned punishers, or unlearned punishers, are stimuli whose presentation functions as punishment without previous pairing with any other punishers. Such punishers consist of stimulation such as pain, intense odors, visual stimulation, taste, sound, or extreme temperatures (Cooper, Heron, \& Heward, 2007).  Conditioned punishers, or learned punishers, are stimuli whose presentation has previously been paired with an unconditioned punisher or a previously conditioned punisher (Cooper, Heron, \& Heward, 2007).   For example, if a person eats yogurt and immediately gags or vomits, yogurt may become a conditioned aversive and thereby a conditioned punisher by decreasing the behavior of eating yogurt and possibly other food with a similar consistency to yogurt.  As the above examples of conditioned and unconditioned punishers show, the process of punishment is a naturally occurring phenomenon that causes behavior change.  However, punishment procedures can also be an effective means for decreasing challenging behavior that is life threatening or resistant to other forms of intervention in an ethical manner.  Iwata (1988) recommends that behavior analysts view the use of punishers as a default technology to be used when other interventions have failed.  

Regarding the selection of a punisher to use in an intervention, it is important to note that punishers are idiosyncratic. A punisher for one person maybe a reinforcer for someone else, and perhaps a neutral stimulus to another. For this reason, a punisher assessment can assist in identifying stimuli that will likely function as punishers by measuring avoidance and escape behavior following the presentation with each stimulus (Fisher et al., 1994). Once potential punishers have been identified there are some factors to consider when choosing a stimulus to use in the treatment of challenging behavior. Research has indicated that the magnitude, or amount of the punisher, should be delivered at the optimum level at the outset of the intervention (Azrin \& Holz, 1966; Thompson et al., 1999).  Furthermore, in keeping with ethical considerations the selection of the least intrusive punisher(s) is recommended.  Typically intrusiveness is outlined by hierarchically arranging interventions according to the degree to which the intervention limits individual freedom, intrudes into the child's life, or produces discomfort, pain, or distress (Luiselli, 2008). Pairing procedures may be beneficial in assisting with the identification of less intrusive punishers by establishing less intrusive conditioned punishers (Vorndran \& Lerman, 2006).  Lastly, it should be noted that Lerman and Vorndran (2002) highlighted the need for further basic and applied research on punishment due to a need for identifying strategies to enhance the effectiveness of least intrusive punishment procedures. 
%
\subsection{Assessment}
\begin{enumerate}
\item Ask supervisee to give examples of an unconditioned punisher
\item Ask supervisee to give examples of a conditioned  punisher
\item Ask supervisee to identify ethical considerations regarding the use of punishment and selecting punishers 
\item Ask supervisee to list the characteristics that should be considered when selecting a punisher 
\end{enumerate}
%
\subsection{Relevant Literature}
\begin{refsection}
\nocite{azrin1966punishment,
        bac2014professional,
        cooper2007applied,
        fisher1994preliminary,
        iwata1988development,
        lerman2002status,
        luiselli2008effective,
        thompson1999effects,
        vorndran2006establishing}
\printbibliography[heading=none]
\end{refsection}
\subsection{Related Tasks}
\fourdSixteen{}\\
\fourdSeventeen{}\\
\fourjTwo{}\\
\fourjTen{}\\
\fourFKNineteen{}\\
\fourFKTwenty{}\\
\fourFKTwentyOne{}\\
\fourFKTwentyThree{}\\
