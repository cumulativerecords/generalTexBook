\chapter{\foursecb{}}
\clearpage \section[\fourbOne{}]{\fourbOne{}%
            \sectionmark{B-01 Use the dimensions...}}
\sectionmark{B-01 Use the dimensions...}
``Baer, Wolf, and Risley (1968) recommended that applied behavior analysis should be applied, behavioral, analytic, technological, conceptually systematic, effective, and capable of appropriately generalized outcomes...'' In 1987 Baer and colleagues reported that the ``seven self-conscious guides to behavior analytic conduct'' they had offered 20 years earlier ``remain functional; they still connote the current dimensions of the work usually called applied behavior analysis'' (Baer, et al., cited in Cooper et al., 2007, p. 16).
\begin{enumerate}
\item Applied - The applied dimension relates to choosing target behaviors to change that are socially significant. 
\item Behavioral - The behavioral dimension refers to the target behavior being systematically chosen for intervention based on its significance and this behavior must be measurable.  Baer et al. (1968, p. 93) summarized this point by stating, ``Since the behavior of an individual is composed of physical events, its scientific study requires their precise measure.''
\item Analytic - The analytic dimension refers to ``... a functional relation between the manipulated events and a reliable change in some measurable dimension of the targeted behavior'' (Cooper et al. 2007, p. 17). Baer et al. (1968, p. 94) stated that, ``An experimenter has achieved an analysis of behavior when he can exercise control over it.''
\item Technological - ``A study in applied behavior analysis is technological when all of its operative procedures are identified and described with sufficient detail and clarity'' (Cooper et al., 2007, p. 17).
\item Conceptually systematic - Conceptual systems refer to the application of behavior analytic principles to create behavior change. ``The field of applied behavior analysis will probably advance best if the published descriptions of its procedures are not only precisely technological, but also strive for relevance to principle'' (Baer et. al. 1968, p. 96).
\item Effective - ``An effective application of behavioral techniques must improve the behavior under investigation to a practical degree'' (Cooper et al., 2007, p. 17).
\item Generality - The final dimension of applied behavior analysis outlined by Baer et al. (1968) was generality. ``A behavior change has generality if it lasts over time, appears in environments other than the one in which the intervention that initially produced it was implemented, and/or spreads to other behaviors not directly treated by the intervention.'' (Cooper et al., 2007, p. 18).
\end{enumerate}

\subsection{Examples}
\begin{enumerate}
\item Tim was evaluating the effectiveness of an intervention to decrease inappropriate comments for a first grade student on his case load. The data indicated that the behavior had decreased across all settings including when the child was home, displaying generalization of the intervention. He also noticed that the intervention was analytic because the data indicated that on days there was a substitute who was not thoroughly trained on the intervention there was a significant increase in rates of inappropriate commenting. Finally, Tim deemed the behavior of inappropriately commenting to be socially significant because it impeded the student from effectively accessing the classroom curriculum. 
\item Students learning the 7 dimensions of ABA often use the acronyms GET-A-CAB or BAT-CAGE to remember them.
\end{enumerate}
%
\subsection{Assessment}
\begin{enumerate}
\item Have supervisee create SAFMEDs cards for each of the dimensions of ABA.
\item Have supervisee identify, define, and give examples of each of the 7 dimensions of Applied Behavior Analysis mentioned in the Baer, Wolf, \& Risley (1968) article. 
\item In reference to the ``applied'' dimension of ABA, have supervisee list 5 types of behavior that they feel are social significant in their life. Have him/her describe why these are socially significant types of behavior. 
\item Have supervisee identify types of socially significant behavior they target with clients.
\end{enumerate}
%
\subsection{Relevant Literature}
\begin{refsection}
\nocite{cooper2007applied,
    baer1968some,
    baer1987some,
    stokes1977implicit,
    wolf1978social}
\printbibliography[heading=none]
\end{refsection}
%
\subsection{Related Tasks}
\fourbFour{}\\
\fourbFive{}\\
\fourbSix{}\\
\fourbSeven{}\\
\fourbNine{}\\
\fourbEleven{}\\
\fourhFour{}\\
\fouriOne{}\\
