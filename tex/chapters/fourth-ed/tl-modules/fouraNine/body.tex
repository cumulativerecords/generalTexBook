\clearpage \section[\fouraNine{}]{\fouraNine{}%
              \sectionmark{A-09 Evaluate the acc...}}
\sectionmark{A-09 Evaluate the acc...}
Evaluating the accuracy and reliability of measurement procedures involves ``measuring the measurement system'' (Cooper, Heron, \& Heward, 2007, p. 110). As human error is the biggest threat to the accuracy and reliability of data, measurements must be evaluated to determine trustworthiness.  Accuracy of measurement is determined when the observed values equal the true values.  Establishing a true value requires the use of a different measurement procedure than the one used to record the observed value.  This often makes it difficult to determine a true value for many of the behaviors of interest.  Measures of reliability should be used when a true value cannot be established.  Reliability of measurement is determined when the same value is given across repeated measures of the same event, thus reliability reflects consistency. 
%
\subsection{Examples}
\begin{enumerate}
\item Accuracy: You and a friend decide to go on a 5-mile run.  Your friend tells you that she can monitor the distance because her legs always start to hurt once she runs 5 miles.  You, being a data-driven behavior analyst, decide that your friend's measurement procedure might not be the most accurate so you use your smart phone app to track the distance.  Your measurement system will likely reveal a better estimate of the true value of the distance you ran.
\item Reliability: Using the example for accuracy, both measures can be reliable if at the end of the run your friend tells you that you must have run 5 miles because her legs hurt and your app indicates you ran 5 miles.   
\end{enumerate}
%
\subsection{Assessment}
\begin{enumerate}
\item Ask supervisee to evaluate the accuracy and reliability of a measurement procedure that is being usedmm with a client.
\item Ask supervisee to provide definitions and examples of accuracy and reliability.
\end{enumerate}
%
\subsection{Relevant Literature}
\begin{refsection}
\nocite{cooper2007applied,johnston1993strategies}
\printbibliography[heading=none]
\end{refsection}
%
\subsection{Related Tasks}
\fouraOne{}\\
\fouraTwo{}\\
\fouraThree{}\\
\fouraFour{}\\
\fouraFive{}\\
\fouraSix{}\\
\fouraSeven{}\\
\fouraEight{}\\
\fourhOne{}\\
\fourhTwo{}\\
