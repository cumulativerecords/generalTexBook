\clearpage \section{\fourFKThirtySeven{}}
\subsection{Definition} 
Stimulus generalization - ``When an antecedent stimulus has a history of evoking a response that has been reinforced in its presence, the same type of behavior tends to be evoked by stimuli that share similar physical properties with the controlling antecedent stimulus'' (Cooper, Heron, \& Heward, 2007, p. 705).
%
\subsection{Examples}
\begin{enumerate}
\item  A child says, ``mommy'' in the presence of her mother, but also says ``mommy'' when she sees her grandmother or daycare provider.
%
\end{enumerate}
%
\subsection{Assessment}
Using the example above:
\begin{enumerate}
\item Have intern identify the difference between response generalization and stimulus generalization.
\item Have intern identify the difference between stimulus generalization and stimulus discrimination. 
\item Have intern identify the qualities of each and give examples.
\item Have intern give examples from their workplace of stimulus generalization.
\end{enumerate}
%
\subsection{Relevant Literature}
\begin{refsection}
\nocite{baer1968some,
        cooper2007applied,
        cuvo2003stimulus,
        guttman1956discriminability,
        johnston1979relation,
        stokes1977implicit}
\printbibliography[heading=none]
\end{refsection}
%
\subsection{Related Tasks}
\fourbOne{}\\
\foureSix{}\\
\fourjEleven{}\\
\fourjFourteen{}\\
\fourFKTen{}\\
\fourFKEleven{}\\
\fourFKTwelve{}\\
\fourFKThirtySix{}\\
