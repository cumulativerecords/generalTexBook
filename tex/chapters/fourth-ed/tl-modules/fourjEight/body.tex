\clearpage \section[\fourjEight{}]{\fourjEight{}%
              \sectionmark{J-08 Select... social validity...}}
\sectionmark{J-08 Select... social validity...}
\subsection{Definition}
A distinguishing characteristic of applied behavior analysis is assessing an individual's functioning within the context of natural environments. This applied aspect focuses the behavior analyst on identifying meaningful goals and acceptable methods for intervention that will increase the individual's independence and level of functioning in natural settings. The behavior analyst sets intervention goals that comply with stated preferences of the individual client, goals of those who live and work with the individual, and consider how typical individuals function in similar environments. Analysts seek goals that are socially valid and intervention methods that are not only scientifically validated strategies for accomplishing those goals, but strategies that can be expected to be implemented consistently and with fidelity by those who will apply the strategies. Although an intervention might be effective in a clinical, controlled setting, the behavior analyst must consider intervention limitations related to ``social acceptability, complexity, practicality, and cost. Regardless of their effectiveness, treatments that are perceived by practitioners, parents, and/or clients as unacceptable or undesirable for whatever reason are unlikely to be used'' (Cooper et al, 2007, p. 250).
%
\subsection{Examples}
\begin{enumerate}
\item A child hits her six month-old sister even when their parents model and reinforce appropriate behaviors toward the baby. The parents find it difficult to avoid explaining to the child, at the same time that they block her physically, reasons her behaviors are unkind and even dangerous. The behavior analyst talks to the parents about how the parent's explanations might be reinforcing the big sister aggression. The parents and the analyst increase the opportunities they have to individually attend to the child during appropriate play throughout the day. The parents ask the analyst to help them design a structured plan to teach appropriate sibling behaviors through language, modeling, literature, role play, movies, and increased reinforcement for appropriate behaviors of the sister toward her younger sibling.
\item A man hits his head and pulls at his ears with such force that he has required emergency medical care. At the beginning of treatment, the behavior analyst recommends that the man be given access to a helmet to prevent significant injury when he is not adequately staffed to stop his behavior. His family is against the man appearing in public with a protective helmet. The behavior analyst explains the reasons such equipment might be important for protecting the man from harm when his 1:1 staff person is distracted by driving a car or interacting with clerks or others in the community. The analyst and the family agree that until interventions stops the severe self-injury, the man will participate in community activities with a helmet unless a family member accompanies staff in the community with him. 
%
\end{enumerate}
\subsection{Assessment}
\begin{enumerate}
\item A behavior analyst wants to increase a non-verbal teenager's independent functioning during daily care routines by teaching him to dress, brush teeth, and bathe independently. The boy's mother says she doesn't mind physically prompting her son through those daily care routines, but states that she hates his screaming while she does it.

\item The analyst completes a functional assessment and learns the following: The boy can complete most of the steps for dressing, brushing teeth, and bathing independently, but has not learned a consistent chain of steps for each skill. The boy screams at other times during the day when his mother uses physical prompting. Ask the supervisee to consider the social validity of the behavior analyst's goals and the preferences expressed by the boy's mother. 

\item Ask the supervisee to write at least one hypothesis to explain, based on the information above, what might be the relations between the self-care skills and the screaming behavior.

\item Ask the supervisee to explain to the mother why teaching the son chained steps for each skill is important for ending the screaming behavior in non-technical language in order to gain her support for teaching self-care routines.
\end{enumerate}
%
\subsection{Relevant Literature}
\begin{refsection}
\nocite{cooper2007applied,
        fawcett1991social,
        wolf1978social}
\printbibliography[heading=none]
\end{refsection}
%
\subsection{Related Tasks}
\fourgSix{}\\
\fourgEight{}\\
\fouriSix{}\\
\fourjFour{}\\
\fourjFive{}\\
\fourjSix{}\\
\fourjTwelve{}\\
\fourkTwo{}\\
\fourkThree{}\\
\fourkNine{}\\
