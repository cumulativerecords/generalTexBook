\clearpage \section[\fourFKThirty{}]{\fourFKThirty{}%
              \sectionmark{FK-30 Distinguish... motivating}}
\sectionmark{FK-30 Distinguish... motivating}
\subsection{Definition}
Understanding motivating operations (MO) and reinforcement effects are critical components in the analysis of behavior.  To briefly explain the difference, MOs are antecedent variables that have behavior-altering effects in that they alter the current frequency of relevant behaviors whereas the process of reinforcement is a consequence-based process (as is extinction and punishment) said to have repertoire-altering effects in that the future frequency of the behavior that preceded the consequence is altered (Cooper, Heron, \& Heward, 2007).  While this explanation can help clarify the difference between MO effects and reinforcement effects, it is also important to understand the basic features of MOs.  Specifically, MOs have a value-altering effect or a behavior-altering effect as defined by Cooper et al., (2007).\\

The value-altering effect is either (a) an increase in the reinforcing effectiveness of some stimulus, object, or event, in which case the MO is an establishing operation (EO); or (b) a decrease in reinforcing effectiveness, in which case the MO is an abolishing operation (AO).  The behavior-altering effect is either (a) an increase in the current frequency of behavior that has been reinforced by some stimulus, object, or event, called an evocative effect; or (b) a decrease in the current frequency of behavior that has been reinforced by some stimulus, object, or event, called an abative effect (p. 375).
%
\subsection{Examples}
\begin{enumerate}
\item Being deprived of food or water increases the reinforcing value of food and water (i.e., a value altering effect in which the MO functions as an EO), and there will likely be an increase in the current frequency of all behavior that has previously been reinforced with food and water (i.e., an evocative behavior-altering effect).  Conversely, if a large meal was just consumed then it is unlikely that food will be reinforcing (i.e., a value-altering effect in which the MO functions as an AO) and there will be a reduction in the current frequency of all behavior previously reinforced with food (i.e., an abative behavior-altering effect).
\item This next example can illustrate the difference between MO effects and reinforcement/punishment effects (i.e., repertoire-altering effects).  Before leaving for work you realize that it is going to be a cold day.  The heater in your car does not work well so you plan ahead by putting a blanket and extra jacket in your car to use if it becomes too cold. During your drive to work, it becomes increasingly cold so you turn on your car heater, put your extra jacket on, and lay the blanket over you so you become much warmer.  In this example, there was an increase in the current frequency of all behavior that has been reinforced by becoming warmer (i.e., an evocative behavior-altering effect).  For the rest of the winter, to avoid becoming too cold on your drive to work, you leave every morning already wearing an extra jacket and put a blanket on you as soon as you get in the car (i.e., repertoire-altering effect on future behavior).  
%
\end{enumerate}
%
\subsection{Assessment}
\begin{enumerate}
\item Ask supervisee to define MOs and explain the basic characteristics.
\item Ask supervisee to discriminate between behavior-altering effects and repertoire-altering effects.
\item Ask supervisee to identify potential MOs for client's behavior.
\item Ask supervisee to provide examples of behavior-altering effects and repertoire-altering effects that are operating in a client's environment.
\end{enumerate}
%
\subsection{Relevant Literature}
\begin{refsection}
\nocite{cooper2007applied,
        iwata2000current,
        laraway2001abative,
        schlinger1987function}
\printbibliography[heading=none]
\end{refsection}
\subsection{Related Tasks}
\foureOne{}\\
\fourFKTwentyFive{}\\
\fourFKTwentySix{}\\
\fourFKTwentySeven{}\\
\fourFKTwentyEight{}\\
\fourFKTwentyNine{}\\
