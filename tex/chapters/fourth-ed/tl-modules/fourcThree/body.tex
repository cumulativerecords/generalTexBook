\clearpage \section[\fourcThree{}]{\fourcThree{}%
              \sectionmark{C-03 State... extinction.}}
\sectionmark{C-03 State... extinction.}
\subsection{Definition}
Extinction ``occurs when reinforcement of a previously reinforced behavior is discontinued; as a result the frequency of that behavior decreases in the future'' (Cooper, Heron \& Howard, 2007, p. 457). Extinction renders target behavior useless and is often a significant component contributing to the effectiveness of a behavioral program. However, extinction should be used with caution. Two common side effects may occur when extinction is utilized: an extinction burst, defined as ``an immediate increase in the frequency of the response after the removal of the positive, negative, or automatic reinforcement'' (Cooper, Heron \& Howard, 2007, p. 462), and extinction-induced aggression.

Studies have compared withdrawal of reinforcement as an aversive event and responses to extinction are similar to attack responses in laboratory subjects exposed to aversive stimulation, such as heat, shocks and physical blows (Lerman et al., 1999). Extinction should not be used as a singular intervention when self-injury or aggression is severe and cannot be prevented and appropriate safe-guards cannot be put in place. Other considerations include extinction being inappropriate in settings where maladaptive behaviors are likely to be imitated by others (e.g., classroom setting) and when extinction is not feasible (e.g. if an individual engages in physical aggression for attention, response-blocking may be enough to reinforce the individual's behavior). 

Research has shown that there are other behavioral strategies that can be utilized to mitigate the unwanted effects of extinction. These include using differential reinforcement of alternative behavior in conjunction with extinction procedures. In situations in which using extinction is not possible, research has shown that by manipulating reinforcement schedules and reinforcement parameters (e.g. quality, duration, immediacy of reinforcement) to favoring appropriate behavior rather than problem behavior, problem behavior has also been shown to decrease (Athens \& Vollmer, 2010).
%
\subsection{Assessment}
\begin{enumerate}
\item Ask your supervisee to list the possible unwanted effects of extinction.
\item Ask your supervisee in which situations should extinction not be utilized.
\item Ask your supervisee what behavioral strategies can mitigate the unwanted effects of extinction.
\item Have your supervisee determine which of his/her clients could benefit from extinction and which clients should avoid the use of extinction. 
%
\end{enumerate}
%
\subsection{Relevant Literature}
\begin{refsection}
\nocite{athens2010investigation,
    cooper2007applied,
    lerman1995prevalence,
    lerman1999side}
\printbibliography[heading=none]
\end{refsection}
%
\subsection{Related Tasks}
\fourdTwo{}\\
\fourdEighteen{}\\
\fourdNineteen{}\\
\foureOne{}\\
\foureEight{}\\
