\clearpage \section[\fourjThree{}]{\fourjThree{}%
              \sectionmark{J-03 Select... task analysis.}}
\sectionmark{J-03 Select... task analysis.}
\subsection{Definition}
What are the options for intervention strategies when it comes to teaching a chain of behavior through a task analysis?
\begin{enumerate}
\item Once a person's baseline level has been assessed (through the single or multiple-opportunity method*) to determine what components of the task analysis he/she can perform, the appropriate intervention strategy needs to be selected. Cooper, Heron \& Heward (2007) suggest there are four appropriate intervention strategies which practitioners can choose from; forward chaining, total-task chaining, and backward chaining.
\item Cooper, Heron and Heward (2007, p. 446) argue that research to date does not suggest a clear answer to the question ``which chaining strategy to use?'' As such, it is very important to examine the results of the baseline level assessment, to consider the client and how they learn best, and what the different intervention strategies can offer in different situations, in order to select the most appropriate method.
\end{enumerate}

Total-task chaining
\begin{enumerate}
\item If the client performs quite a few steps in the task analysis but is not performing them in the correct sequence, the most appropriate method to choose would probably be total-task chaining. 
\item Total-task chaining would also be an appropriate intervention strategy to select when the client has generalized motor imitation and moderate to severe disabilities (Test et al., 1990, cited from Cooper, Heron \& Heward, 2007).
\item When the chain is quite short and not too complex, this may also be an appropriate teaching method to utilize (Cooper, Heron \& Heward, 2007).
\end{enumerate}
%
Forward chaining
\begin{enumerate}
\item This approach may be more appropriate to use when the client has demonstrated more proficiency with the first couple of steps in the chain and/or the last steps in the chain are more complex to complete. 
\item It may also be useful to use this approach when it is necessary to link smaller chains into larger ones. For example, if you have a skill such as bed making and this is made up of perhaps four/five skill clusters, forward chaining is a useful method to link the skill clusters altogether (Cooper, Heron \& Heward, 2007).
\end{enumerate}
%
Backward chaining
\begin{enumerate}
\item Backward chaining may be more appropriate to use when the client has demonstrated more proficiency with the last couple of steps in the chain and/or steps that appear earlier in the chain are more complex to complete.
\end{enumerate}
%
\subsection{Assessment}
\begin{enumerate}
\item Ask your Supervisee to identify situations in which he/she might suggest forward chaining
\item Ask your Supervisee to identify situations in which he/she might suggest backward chaining
\item Ask your Supervisee to identify situations in which he/she might suggest total-task chaining
\end{enumerate}
%
\subsection{Relevant Literature}
\begin{refsection}
\nocite{cooper2007applied,
        kazdin2012behavior,
        miltenberger2008behavior,
        test1990teaching}
\printbibliography[heading=none]
\end{refsection}
%
\subsection{Related Tasks}
\fourdSix{}\\
\fourdSeven{}\\
\fourjTwo{}\\
\fourjFive{}\\
%
Footnotes\\
* Please see 4th ed. task list item D-07 for more information on conducting task analyses.\\
* Please see 4th ed. task list item D-06 for a more detailed description of the different chaining procedures.\\
