\clearpage \section{\fourFKTwentyThree{}}
\subsection{Definition}
Automatic reinforcement - ``Reinforcement that occurs independent of the social mediation of others'' (Cooper Heron, \& Heward, 2007, p. 267).\\

Automatic punishment - ``Punishment that occurs independent of the social mediation by others''(Cooper Heron, \& Heward, 2007, p. 534).
%
\subsection{Examples}  
\begin{enumerate}
\item (Automatic Reinforcement) Scratching an insect bite removes an itch; eating food when hungry removes hunger, humming may be auditory reinforcement; nonfunctional movements such as hand flapping may produce a sensation, which is automatically reinforcing; some self-injurious behavior may produce a sensation, which the individual may enjoy.
\item (Automatic Punishment) Albert bites his canker sore, causing a shocking pain.  Albert is becomes cautious as he eats with his canker sore until the canker disappears.  A dog gets a thorn in his paw.  He experiences pain when he steps down on his foot.  He begins to walk on three legs. 
\end{enumerate}
%
\subsection{Assessment}
\begin{enumerate}
\item Ask your supervisee to come up with several examples of automatic punishment and automatic reinforcement. 
\item Ask your supervisee provide examples of how extinction of the following automatically reinforced behavior may occur: child making sounds by tapping the table, child receiving kinesthetic stimulation by flapping his arms, child throws up and eats vomit for the taste, child scratching surface for tactile stimulation on fingers, child flipping light switch on and off to gain a visual sensation
\item Ask supervisee if they will likely have to treat clients with an automatic punishment function (yes-many food refusal behavior may have an automatic punishment for example).
%
\end{enumerate}
%
\subsection{Relevant Literature}
\begin{refsection}
\nocite{cooper2007applied,
        vollmer1994concept}
\printbibliography[heading=none]
\end{refsection} 
%
\subsection{Related Tasks}
\fourFKSeventeen{}\\
\fourFKNineteen{}\\
\fourFKTwentyTwo{}\\
