\clearpage \section[\fourjFifteen{}]{\fourjFifteen{}%
              \sectionmark{J-15 Base decision-making...}}
\sectionmark{J-15 Base decision-making...}
\subsection{Definition}
Behavior analysts make decisions during assessment and intervention based on data. Graphic displays (e.g., line graphs, bar graphs, and cumulative graphs) aid accurate and efficient interpretation of quantitative data and facilitate communication with others. ``The primary function of a graph is to communicate without assistance from the accompanying text'' (Spriggs \& Gast, 2010, p. 167). Line graphs are most often used by behavior analysts to show effects and possible functional relations between intervention (independent variable) and a defined behavior (dependent variable). Bar graphs are often used by behavior analysts to summarize or compare discrete aspects of recorded behavior. Cumulative graphs show the rate of change in responding across time. Although they may be used with duration or latency data, they are most often used to show frequency data. Behavior analysts often use tables to summarize data or other information. ``An informative table supplements—rather than duplicates—the text'' (APA, 2010). As with graphs, a table should communicate efficiently but include enough information to be understood alone without explanations in the text.
%
\subsection{Examples}
\begin{enumerate}
\item A behavior analyst interprets the effectiveness of a constant time delay procedure for teaching a student 10 sight words using the following line graph: 
\item A behavior analyst uses a table to summarize the number of times a child chose a specific toy during seven sessions of free-play with four toys available. 
\end{enumerate}
%
\subsection{Assessment}
\begin{enumerate}
\item Ask the supervisee to interpret results of the constant time delay intervention using only the graphic data. Would the supervisee recommend CTD for teaching discrete skills to this student? Ask the supervisee to write a title for this graph.
\item Ask the supervisee to interpret information in the table to decide which toys she would place in a free time area for this child to enjoy. Would the supervisee consider adding different types of any one toy to the playtime area based on this data? If so, which type? If not, why not? Ask the supervisee to write a title for this table.
%
\end{enumerate}
%
\subsection{Relevant Literature}
\begin{refsection}
\nocite{cooper2007applied,
        spriggs2010visual,
        american2010publication}
\printbibliography[heading=none]
\end{refsection}
%
\subsection{Related Tasks}
\fouraTen{}\\
\fouraEleven{}\\
\fouraTwelve{}\\
\fouraFourteen{}\\
\fourbThree{}\\
\fourhOne{}\\
\fourhThree{}\\
\fourhFour{}\\
\fouriFive{}\\
\fouriSeven{}\\
\fourjOne{}\\
\fourkFour{}\\
\fourFKThirtyThree{}\\
