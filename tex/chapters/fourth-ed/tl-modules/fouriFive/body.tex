\clearpage \section[\fouriFive{}]{\fouriFive{}%
              \sectionmark{I-05 Organize, Analyze...}}
\sectionmark{I-05 Organize, Analyze...}
\clearpage \section{\fouriFive{}}
\subsection{Definition}
Organize the data
\begin{enumerate}
\item Once data have been collected in their raw format, it is then important to organize the data into a format that is easy to analyze (Cooper, Heron \& Heward, 2007). The most effective way to do this, and the most common method utilized by Behavior Analysts, is to visually display the data in a graph (Cooper, Heron, \& Heward, 2007).
\item As Parsonson and Baer (1978, p. 134) said ``the function of the graph is to communicate... in an attractive manner, descriptions and summaries of the data that enable rapid and accurate analysis of the facts'' (cited from Cooper, Heron \& Heward, 2007, p. 128).
\item The visual formats most often used by behavior analysts are line graphs, bar graphs, cumulative records, semilogarithmic charts and scatterplots* (Cooper, Heron \& Heward, 2007).
\end{enumerate}
%
Analyze and interpret observed data\\

Behavior analysts use a systematic form of assessing graphically displayed data called visual analysis (Cooper, Heron \& Heward, 2007). 

Visual analysis encompasses examining each of three characteristics in a graphic display of data, both within and across the different conditions and phases of an experiment. These three characteristics are:

\begin{enumerate}
\item The level of the data
\item The extent and type of variability in the data 
\item The trends in the data
\end{enumerate}

Johnston and Pennypacker (1993b, p. 320) recommend that the viewer should carefully examine the graph's overall construction, paying attention to details such as the axis labels and the scaling of each axis, prior to attempting to interpret the data. They argue ``it's impossible to interpret graphic data without being influenced by various characteristics of the graph itself'' (cited from Cooper, Heron \& Heward, 2007, p. 149).
%
Visual analysis within conditions\\
\begin{enumerate}
\item Within given conditions, examination needs to occur to determine a few relevant factors (Cooper, Heron \& Heward, 2007):
\item The number of data points in each condition (in general, the more measurements of the dependent variable there are per unit of time, the more confidence one can have in the data).
\item Variability (A high degree of variability usually indicates little control has been achieved over the factors influencing behavior).
\item Level (examined in terms of its absolute value within a condition, the degree of stability/variability and the extent of change from one level to another).
\item Trend (the trend indicates whether a particular behavior has increased, decreased or has neither increased nor decreased within a condition).
\end{enumerate}

Visual analysis between conditions\\
\begin{enumerate}
\item After examining the data within each condition or phase of a study, visual analysis now proceeds to examining the data between conditions (Cooper, Heron \& Heward, 2007):
\item Comparison needs to be made between the different conditions of the level, trend and variability of the data (Cooper, Heron \& Heward, 2007, p. 154).
\item The data are examined in terms of the overall level of performance between conditions; generally when there is no overlap of data points between the highest values in one condition and the lowest values in another condition, there is a strong likelihood that the behavior changed from one condition to the next (Cooper, Heron \& Heward, 2007, p. 154).
\end{enumerate}

Once an ``examination and comparison of changes in level, trend and variability between conditions has occurred, a comparison needs to be made of performance across similar conditions'' (Cooper, Heron \& Heward, 2007, p. 155). If a behavior change is found to have occurred over the course of an intervention, the next question to be asked is, ``was the behavior change a result of the intervention?'' (Cooper, Heron \& Heward, 2007, p. 155).

\subsection{Assessment}
\begin{enumerate}
\item Ask your Supervisee to explain why it's important as a behavior analyst to organize and interpret observed data. 
\item Ask your Supervisee to organize a set of data and display it graphically in the most appropriate way. 
\end{enumerate}
%
\subsection{Relevant Literature}
\begin{refsection}
\nocite{cooper2007applied,
        fisher2014handbook,
        johnston1993strategies,
        parsonson1978analysis}
\printbibliography[heading=none]
\end{refsection}
%
\subsection{Related Tasks}
\fouraTen{}\\
\fouraEleven{}\\
\fourbFour{}\\
\fourbFive{}\\
\fourbSix{}\\
\fourbSeven{}\\
\fourbEight{}\\
\fourbNine{}\\
\fourjFifteen{}\\
%
Footnotes
* See Cooper, Heron \& Heward (2007), pages 129 – 154 for more information on these graphic displays.
