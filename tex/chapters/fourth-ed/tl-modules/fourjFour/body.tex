\clearpage \section[\fourjFour{}]{\fourjFour{}%
              \sectionmark{J-04 Select... client preferences.}}
\sectionmark{J-04 Select... client preferences.}
\subsection{Definition}
The importance and ethical necessity of basing intervention strategies on client preferences:
\begin{enumerate}
\item As behavior analysts, it is our ethical responsibility to continually put our client's needs first, and this includes, considering which type of intervention may be more preferred by the clients we serve. As Bailey and Burch (2011) state, one of our core ethical principles is treating others with care and compassion and this encompasses giving our clients choices (Bailey \& Burch, 2011).
%
\item Historically, consideration of client preferences is an area that within behavior analysis, perhaps has not been given as much attention as it deserves. In one area of study, Hanley, Piazza, Fisher, Contrucci \& Maglieri (1997) reported that, ``few if any studies have examined the social acceptability of or consumer preferences'' for the relevant treatment options but had instead given more weight to the opinions of the caregivers as opposed to those of the client (Hanley et al., 1997, p. 460). Another interesting train of thought has been that ``choice making is often not taught'' (Bannerman, Sheldon, Sherman, \& Harchik, 1990, p. 81).
%
\item Another reason for considering clients' preferences over treatment options is it may make the intervention more successful. Data from Miltenberger, Suda, Lennox and Lindeman (1991) indicated it was very important for successful treatment, to consider client preferences when selecting interventions. Findings from many other studies have also supported this premise (e.g., Berk, 1976; Hanley, Piazza, Fisher, \& Maglieri, 2005; Mendonca \& Brehm, 1983; Perlmuter \& Montry, 1973).
\end{enumerate}

Selecting interventions based on client preferences 
\begin{enumerate}
\item There are methods reported in the literature for determining which treatment method is more preferred by a client*. As such, once it has been established that an intervention is necessary to treat a behavior, it is imperative then to consider assessing a client's preference for one treatment option over others to assist with the behavior change program. In this way, treatment is more likely to be successful, will likely have more social validity (Schwartz \& Baer, 1991) and will be meeting more of our ethical standards as behavior analysts.
\end{enumerate}
%
\subsection{Assessment}
\begin{enumerate}
\item Ask your Supervisee to explain why it's important as behavior analysts to select interventions based on client preferences. 
\item Ask your Supervisee to investigate what different methods are available for evaluating clients' preferences for different interventions and report them back to you, along with advantages and disadvantages of each method.
\end{enumerate}
%
\subsection{Relevant Literature}
\begin{refsection}
\nocite{bailey2013ethics,
        bannerman1990balancing,
        berk1976effects,
        hanley1997evaluation,
        hanley2005effectiveness,
        mendonca1983effects,
        miltenberger1991assessing,
        perlmuter1973effect}
\printbibliography[heading=none]
\end{refsection}
\subsection{Related Tasks}
\foureEight{}\\
\fouriSeven{}\\
\fourjTwo{}\\
\fourjFive{}\\
\fourjSix{}\\
\fourjSeven{}\\
\fourjEight{}\\
%
Footnotes\\
*1 See Hanley, Piazza, Fisher, Contrucci \& Maglieri (1997) and Miltenberger, Suda, Lennox \& Lindeman (1991) for more information about how to test clients' preferences for different interventions.\\
