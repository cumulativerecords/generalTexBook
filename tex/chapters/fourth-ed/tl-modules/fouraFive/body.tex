\clearpage \section{\fouraFive{}}
\subsection{Definition}  
Interresponse time (IRT) - ``...the elapsed time between two successive responses'' (Cooper, Heron, \& Heward, 2007, p. 80).
%  
\subsection{Examples} 
\begin{enumerate}
\item Sparky loves to bark at passing cars.  He hears a car drive by the house and barks.  Thirty-seven seconds later another car passes by and Sparky barks again.  Interresponse time between barking at the vehicles is thirty-seven seconds.
\item Doodles the cat likes to scratch the furniture.  She walks over the chair and sinks her claws in.  Eleven seconds later Doodles walks over to the couch and begins to scratch again.  Interresponse time between scratches is eleven seconds.  
\item Roger the rooster doesn't know that he's only supposed to crow at dawn. He lets out crows all day long.  He is observed to crow at 3:43 in the afternoon.  He crows again at 3: 59.  Interresponse time between crows is sixteen minutes.  
\item (Non-example) Sparky's owner accidentally steps on his tail.  Sparky yelps from the pain. 
\end{enumerate}
%
\subsection{Assessment}
\begin{enumerate}
\item Ask your supervisee to identify the interresponse time in the examples above. 
\item Ask your supervisee to create another example and non-example of his/her own. 
\item Have your supervisee measure the interresponse time of a behavior on the job or in a role-play. 
\item Have your supervisee graph the interresponse time measured on the job or in a role-play.
\end{enumerate}
%
\subsection{Relevant Literature}
\begin{refsection}
\nocite{blough1963interresponse,cooper2007applied,favell1980rapid}
\printbibliography[heading=none]
\end{refsection}
%
\subsection{Related Tasks}
\fouraFive{}\\
\fouriOne{}\\
\fourhOne{}\\
\fourFKFourtySeven{}\\
