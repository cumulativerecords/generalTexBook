\clearpage \section{\fourFKFourtySix{}}
\subsection{Definition} 
Intraverbal – ``An elementary verbal operant that is evoked by a verbal discriminative stimulus and that does not have point-to-point correspondence with that verbal stimulus'' (Cooper, Heron \& Heward, 2007, p. 698).
%
\subsection{Examples}
\begin{enumerate}
\item A new employee shows up for his first day on the job. The man in the cubical next to him asks, ``What is your name?''  ``Harvey,'' the man replies. Saying ``Harvey'' is an intraverbal in that context.
\item Hanks boss stops his office to let him know that his sales were ``outstanding this week.''  ``Thanks. I really put in some long hours,'' Hank notes.  ``Thanks,'' is an intraverbal in that context.
\item (Non-example) The office phone rings. Harvey picks up the phone and answers ``Hello.'' There is no one on the line so he hangs up and keeps working. 
%
\end{enumerate}
%
\subsection{Assessment}
\begin{enumerate}
\item Ask the supervisee to define ``intraverbal''  
\item Ask the supervisee to give several examples of intraverbals.
\item Ask the supervisee to give a non-example of an intraverbal.
%
\end{enumerate}
%
\subsection{Relevant Literature}
\begin{refsection}
\nocite{cooper2007applied,
        partington1993teaching,
        skinner1957verbal}
\printbibliography[heading=none]
\end{refsection}
%
\subsection{Related Tasks}
\fourdThirteen{}\\
