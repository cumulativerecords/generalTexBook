\clearpage \section[\fourFKTwentyNine{}]{\fourFKTwentyNine{}%
              \sectionmark{FK-29 Distinguish... discriminative}}
\sectionmark{FK-29 Distinguish... discriminative}
\subsection{Definition}
Discriminative stimulus (SD) - ``A stimulus in the presence of which responses of some type have been reinforced and in the absence of which the same type of responses have occurred and not been reinforced; this history of differential reinforcement is the reason an SD increases the momentary frequency of the behavior'' (Cooper, Heron, \& Heward, 2007, p. 694).\\

Motivating operation - ``An environmental variable that (a) alters (increases or decreases) the reinforcing or punishing effectiveness of some stimulus, object, or event; and (b) alters (increases or decreases) the current frequency of all behavior that has been reinforced or punished by that stimulus, object, or event'' (Cooper et al., 2007, p. 699).\\
%
\subsection{Examples}
\begin{enumerate}
\item A child often asks their parents to play video games after school. The child's father often says ``yes'' to this request, while the child's mother says ``no'' and tells the child to get started on their homework. Over time the child continues to ask their father if they can play video games, but has stopped asking their mother. In this example the presence of the father likely functions as an SD due to the history of requests being granted in his presence, but not in the presence of their mother.
\item After playing outside for an hour, a child walks into the house and gets a drink of water. In this example, playing outside likely functions as a motivating operation, more specifically as establishing operation, in that it increases the value of water as a reinforce and increases the frequency of behavior with a history of producing water.
\item A student earns tokens throughout the school day and can trade them in for a preferred item or activity. Usually the student chooses to trade in their tokens for a small snack, accept after lunch. Usually after lunch the student chooses computer over snacks. In this example, consuming food during lunch likely functions as a motivating operation, more specifically an abolishing operation, in that it decreases the value of food as a reinforcer and decreases the frequency of behavior with a history of producing food.
%
\end{enumerate}
%
\subsection{Assessment}
\begin{enumerate}
\item Have supervisee explain the difference between SD s and MOs.
\item Have the supervisee to create additional examples of SD s and MOs.
\item Provide the supervisee with examples of responses and the reinforcers for those responses. Have the supervisee describe potential ways to increase and decrease the value of the reinforcer.
%
\end{enumerate}
%
\subsection{Relevant Literature}
\begin{refsection}
\nocite{cooper2007applied,
        michael1982distinguishing}
\printbibliography[heading=none]
\end{refsection}
%
\subsection{Related Tasks}
\foureOne{}\\
\fourgEight{}\\
\fouriTwo{}\\
\fourjFour{}\\
\fourjSix{}\\
\fourjSeven{}\\
\fourkNine{}\\
\fourFKSeven{}\\
\fourFKTwentyFour{}\\
\fourFKTwentySix{}\\
\fourFKThirtyOne{}\\
\fourFKThirtyThree{}\\
