\clearpage \section{\fourfSeven{}}
\subsection{Definition}
Functional communication training (FCT) - ``...an application of differential reinforcement of alternative behaviors (DRA) because the intervention develops an alternative communicative response as an antecedent to diminish the problem behavior'' (Fisher, Kuhn, \& Thompson, 1998, p. 543).

The alternative response can include vocalizations, sign language, communication boards and devices, picture cards, or gestures.

Carr and Durand (1985) used a two-step process to demonstrate how to deliver FCT.  First they completed a functional behavior assessment to identify the stimuli with known reinforcing properties that maintain the problem behavior, and second, they used those stimuli as reinforcers to develop an alternative behavior to replace the problem behavior.

Guidelines for the effective use of functional communication training include providing a dense schedule of reinforcement, fading prompts, and the appropriate reinforcement schedule thinning after the response is at strength.
%
\subsection{Examples}
\begin{enumerate}
\item Rob was throwing books at his teacher every time he was asked to do a math worksheet. After completing a functional analysis, Rob's teacher found throwing books was maintained by access to escape. Rob was taught to ask for a break when he was doing math instead of throwing something at his teacher. This response, paired with pre-teaching and prompt fading, helped replace the problematic behavior.
\end{enumerate}
%
\subsection{Assessment}
\begin{enumerate}
\item Have supervisee identify the advantages and disadvantages of functional communication training.
\item Have supervisee identify some common guidelines for using FCT.
\item Have supervisee describe instance when he/she used FCT in a professional setting.
\item Have supervisee describe how functional behavior assessment (FBA) and differential reinforcement of alternative behaviors (DRA) relate to the use of FCT.
%
\end{enumerate}
%
\subsection{Relevant Literature}
\begin{refsection}
\nocite{cooper2007applied,
        durand1999functional,
        carr1985reducing,
        durand1992analysis,
        fisher1998establishing,
        hanley2001reinforcement}
\printbibliography[heading=none]
\end{refsection}
%
\subsection{Related Tasks}
\fourdTwo{}\\\
\fourdThree{}\\
\fourdFour{}\\
\fourdFive{}\\
\fourdTen{}\\
\fourdEleven{}\\
\fourdTwentyOne{}\\
\foureOne{}\\
\fouriThree{}\\
\fouriFour{}\\
