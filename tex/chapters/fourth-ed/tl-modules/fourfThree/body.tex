\clearpage \section{\fourfThree{}}
\subsection{Definition}
Direct instruction - a teaching method ``...emphasizing the use of specified teacher directions, programmed instruction and presentation of materials, examples, and prompts, the use of reinforcement and mastery learning principles, regular and direct assessment, and teaching prerequisite skills'' (Callahan, Shukla-Mehta, \& Wie, 2010, p. 78).

Direct instruction was developed to improve academic skills of elementary school students with learning challenges. This model of instructional design was proposed by Engelmann and Becker (Becker, et al., 1975). Direct instruction relies on scripted lessons implemented by a directly trained teacher provided to small-group of learners.  These scripted lessons include a lot of examples and non-examples. Students respond in unison when asked by a teacher and practice skills in groups until reaching a mastery level. Students and teachers systematically measure and teachers analyze students' performance. Those who follow the DI model believe that the learner knows better; that all children can be taught. If a student fails, this is teacher's fault. Additionally, the process of teaching includes hierarchy of instruction complexity: basic skills should be taught before advancing to more complex skills. 
%
\subsection{Assessment}
\begin{enumerate}
\item Have supervisee define direct instruction. Have him/her identify and explain the characteristics and methodology associated with the curriculum.
\item Have supervisee identify the differences between direct instruction and other teaching methodologies.
\item Have supervisee list the pros and cons of implementing direct instruction. Have him/her describe if they would use direct instruction over other methodologies and curricula and why.
%
\end{enumerate}
%
\subsection{Relevant Literature}
\begin{refsection}
\nocite{callahan2010aba,
        cooper2007applied,
        moran2004evidence,
        ledford2012using,
        gersten1987long,
        gersten1988continued,
        gersten1986direct,
        graves1986effects}
\printbibliography[heading=none]
\end{refsection}
%
\subsection{Related Tasks}
\fouraSeven{}\\
\fourdThree{}\\
\fourdEight{}\\
\fourjTwo{}\\
\fourjEleven{}\\
\fourjTwelve{}\\
\fourjFourteen{}\\
