\chapter{\foursecFK{}}
\clearpage \section{\fourFKOne{}}
\subsection{Definition}
Lawfulness of behavior – ``behavior is the result of some condition that has caused it to happen'' (Malott, 2012, p. 168)

The lawfulness of behavior makes a science of behavior possible. ``Science is, of course, more than a set of attitudes. It is a search for order, for uniformities, for lawful relations among the events in nature'' (Skinner, 1953, p.13).

If behavior did not follow universal laws related to the environment that hosts it, it would not be possible to predict or control responding in a scientific way. Skinner describes the necessity for lawfulness of behavior in this quote: ``If we are to use the methods of science in the field of human affairs, we must assume that behavior is lawful and determined. We must expect to discover that what a man does is the result of specifiable conditions and that once these conditions have been discovered, we can anticipate and to some extent determine his actions'' (Skinner, 1953, p. 6).
%
%
%
\subsection{Assessment}
\begin{enumerate}
\item Ask your supervisee to describe why an understanding of lawfulness of behavior is important when designing treatments for their client.
\item Ask your supervisee to role-play a scenario in which he/she discusses lawfulness of behavior with a caretaker or teacher of a client.
%
\end{enumerate}
%
\subsection{Relevant Literature}
\begin{refsection}
\nocite{malott2015principles,
        skinner1953science}
\printbibliography[heading=none]
\end{refsection}
%
\subsection{Related Tasks}
\fourbThree{}\\
\fourFKTwo{}\\
\fourFKThree{}\\
\fourFKFour{}\\
\fourFKFive{}\\
\fourFKSix{}\\
