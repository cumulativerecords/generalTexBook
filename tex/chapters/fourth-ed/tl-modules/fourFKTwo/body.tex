\clearpage \section{\fourFKTwo{}}
\subsection{Definition} 
Selectionism - refers to selection by consequences, a scientific paradigm, which asserts that all forms of operant behavior evolve as a result of the consequences that occurred during one's lifetime.

Skinner (1981) wrote:\\

``Human behavior is the joint product of (i) the contingencies of survival responsible for the natural selection of the specific and (ii) the contingencies of reinforcement responsible for the repertoires acquired by its members, including (iii) the special contingencies maintained by the social environment. (Ultimately, of course, it is all a matter of natural selection, since operant conditioning is an evolved process, of which cultural practices are special applications.)'' (p. 502).  

Skinner's paradigm emphasizes the role of function and draws on evolutionary theory and natural selection (phylogeny). Ontogeny refers to the learning history of an individual. Skinner viewed cultural practices as an evolved process maintained by operant conditioning. Variation in behavior is required for selection by consequences, meaning the most adaptive behavioral repertoire persists because it serves a valuable function for the organism (Cooper, Heron, \& Heward, 2007). Maladaptive, unhealthy, and harmful behavior can persist because it serves a function for the individual (e.g., substance abuse, non-suicidal self-harm, etc.).

\subsection{Examples}
\begin{enumerate}
\item In evolutionary history, our ancestors ate certain foods because it had an adaptive value as it helped ensure survival (natural selection of behavior; phylogenic selection).  The food did not necessarily need to be a reinforcer but was necessary for survival.  However, in modern times, we all have food preferences and may eat food that has no nutritional value or health benefits, indicating that specific foods are eaten because of their reinforcing value (ontogenic selection).  This type of eating habit is not adaptive (e.g., think about overeating, binge eating, obesity and the subsequent health problems that can emerge from this type of eating behavior) but it is strengthened and maintained by operant conditioning, thus reflecting selection by consequences (Skinner, 1981).  
\item Cultural Selection: Pennypacker (1992) provides examples of how selection by consequences is observed in education, economics, and politics and social organization. 
%
\end{enumerate}
%
\subsection{Assessment}
\begin{enumerate}
\item Ask supervisee to define selectionism.
\item Ask supervisee to read and summarize the relevant literature, while highlighting examples that reflect selectionism.
\item Ask supervisee to provide an example of behavior maintained by selectionism.
%
\end{enumerate}
%
\subsection{Relevant Literature}
\begin{refsection}
\nocite{cooper2007applied,
        pennypacker1992behavior,
        skinner1981selection}
\printbibliography[heading=none]
\end{refsection}
%
\subsection{Related Tasks}
\fourFKFifteen{}\\
\fourFKThirtyOne{}\\
\fourFKThirtyThree{}\\
\fourFKFourtyOne{}\\
\fourFKFourtyTwo{}\\
