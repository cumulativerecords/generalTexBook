\clearpage \section[\fouriThree{}]{\fouriThree{}%
              \sectionmark{I-03 Design and implement indiv...}}
\sectionmark{I-03 Design and implement indiv...}
\subsection{Definition}
``Behavioral assessment involves a variety of methods including direct observations, interviews, checklists, and tests to identify and define targets for behavior change'' (Cooper, Heron, \& Heward, 2007, p. 49).

``Applied behavior analysis uses the methods of FBA to identify antecedent and consequent events and to use this information in designing interventions to change socially significant behaviors'' (Gresham, Watson, \& Skinner, 2001, p. 157).

``FBA is designed to obtain information about  the purpose (function) a behavior serves for a person... FBA is used to identify the type and source of reinforcement for challenging behaviors as the basis for intervention efforts...'' (Cooper et al., 2007, p. 501).

``Once the function of behavior is determined, this information is used to design interventions to reduce problem behaviors and to facilitate positive behaviors'' (Gresham et al., 2001, p. 158).

The first step in the process is to define the target behaviors that the assessment will focus on. These behaviors are typically identified by teachers, therapists, or caregivers due to their interference with learning, adaptive functioning, and overall quality of life. In following the principles of ABA, the behaviors targeted for assessment must be socially significant.

Another direct method of determining the function of a behavior is to conduct a functional analysis. This involves systematic manipulation of the environment, while controlling variables to evoke the target behavior under conditions representing each possible function. The typical functions of behavior include access to attention, access to tangible items, escape from demands, and automatic reinforcement. During a functional analysis, each function is assessed to determine if they are maintaining the target behavior. FA is considered to be the most reliable source for determining the function of a behavior, but may not be feasible in some settings due to the time it takes to conduct, safety implications (depending on the severity of the target behavior), and resources needed to conduct each experimental phase.

Other methods used to gather information about the function of target behaviors in a behavioral assessment include a thorough review of the client's previous records (academic reports, past evaluations, behavior support plans, IEP's, etc.), the use of behavioral rating scales such as the FAST (functional analysis screening tool), MAS (Motivational Assessment Scale), or PBQ (Problem Behavior Questionnaire)), structured interviews with caregivers (i.e., functional assessment interview form), direct observation in the target environment (i.e., home, school, community), behavior data collection and analysis, and A-B-C data collection and analysis. These indirect assessments should be used to inform an experimental functional analysis. They are not designed to determine the function of a response on their own.

Once the direct and indirect assessments are completed, this information is analyzed and the BCBA makes recommendations for intervention based on the results of assessments. 
%
\subsection{Examples}
\begin{enumerate}
\item Rich is completing a Functional Behavior Assessment on the aggression of one of his students. After teacher interviews, the completion of rating scales, and several observations in various settings, Rich hypothesizes that the function of aggression is escape from demands. He uses this information to create an intervention plan to decrease aggression at school. 
%
\end{enumerate}
%
\subsection{Assessment}
\begin{enumerate}
\item Have supervisee complete a behavior rating scale on an individual based on one of their behaviors. This can include the FAST, MAS, PBQ, or another common rating scale used in behavioral assessments. Once the supervisee has completed the rating scale, have them score the form and present the results. 
\item Supervisor will create role plays in which each supervisee will collect ABC data on specific topography of problem behavior. After a number of instances have been recorded, supervisee will analyze the data and formulate a hypothesized function of the problem behavior.
\item Have supervisee read Iwata, Dorsey, Slifer, Bauman, \& Richman (1982/1994) and describe the experimental conditions of a functional analysis.
%
\end{enumerate}
%
\subsection{Relevant Literature}
\begin{refsection}
\nocite{carr1993behavior,
        cooper2007applied,
        watson2001functional,
        iwata1994toward,
        o1997functional,
        sprague1998antisocial,
        witt2000functional}
\printbibliography[heading=none]
\end{refsection}
%
\subsection{Related Tasks}
\fourbThree{}\\
\foureOne{}\\
\fourgOne{}\\
\fourgFour{}\\
\fourgFive{}\\
\fourgEight{}\\
\fourhTwo{}\\
\fouriOne{}\\
\fouriTwo{}\\
\fouriFour{}\\
\fouriSix{}\\
\fourjOne{}\\
\fourjTwo{}\\
\fourjTen{}\\
