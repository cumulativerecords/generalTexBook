\clearpage \section{\fouraFour{}}
\subsection{Definition}  
Latency - ``A measure of temporal locus; the elapsed time from the onset of a stimulus (e.g., task direction, cue) to the initiation of a response'' (Cooper, Heron, \& Heward, 2007, p. 80).  

\subsection{Examples}
\begin{enumerate}
\item Hitting the snooze button or hitting the break
\item Example: Gertrude is not a morning person.  Her alarm goes off at precisely 5:30AM.  She hears the annoying wail but doesn't respond immediately.  After 32 seconds of beeping, she whacks the snooze button, rolls over and goes back to sleep.  Latency to turning off the alarm is 32 seconds.
\item Example:  Marty is driving down a country road.  Out of nowhere a herd of deer dart out in front of his car.  It takes Marty 5 seconds from the time he first sees the deer to hit the break.  Latency from the time the deer are spotted to applying pressure to the break is 5 seconds.    
\item Non-example: Gertrude is not a morning person.  Her alarm goes off at precisely 5:30 AM.  She does not respond to its annoying wailing and continues to sleep despite the noise.  The alarm stops on its own 1 hour later.
\end{enumerate}
%
\subsection{Assessment}
\begin{enumerate}
\item Ask your supervisee to identify the latency of a few responses of your choosing.  
\item Ask your supervisee to create another example and non-example of his/her own.
\item Have your supervisee measure the latency to another behavior on the job or in role-play.
\end{enumerate}
%
\subsection{Relevant Literature}
\begin{refsection}
\nocite{cooper2007applied,thomason2011response}
\printbibliography[heading=none]
\end{refsection}

\subsection{Related Tasks}
\fourhOne{}\\
\fouriOne{}\\
\fourFKFourtySeven{}\\
