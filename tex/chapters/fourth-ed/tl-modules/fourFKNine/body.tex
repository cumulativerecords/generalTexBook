\clearpage \section[\fourFKNine{}]{\fourFKNine{}%
              \sectionmark{FK-09 Distinguish... conceptual analysis}}
\sectionmark{FK-09 Distinguish... conceptual analysis}
\subsection{Definition}
The conceptual analysis of behavior is a combination of philosophical, theoretical, and historical investigations behind the science of behavior.  The modern philosophy of behavior analysis is specifically referred to as Radical Behaviorism and was coined by B.F. Skinner. Radical Behaviorism is rooted in the idea that the science of behavior is a natural science encompassed by behavioral events that happen due to the way the universe is arranged (determinism) and explained by other natural events (Baum, 1995) like the phenomenon of gravity. These behavioral events are analyzed in relation to the past and present environment (ontogenic and cultural contingencies) and evolutionary history (phylogenic contingencies). This approach sets itself apart from other behavioral philosophies (e.g., methodological behaviorism) by including overt behavior as an important variable but also acknowledging unobserved behavior (i.e., private events) ( Skinner, 1953). Moreover, internal states, intervening variables and hypothetical constructs (mentalistic explanations) are not used to understand or analyze behavior (Baum, 2011). This philosophy sets the foundation for the other three domains described below.  

The experimental analysis of behavior (EAB) is a natural science approach to the study of behavior.  The methodology includes rate of response as a basic dependent variable, repeated or continuous measurement of clearly defined response classes, within subject comparisons, visual analysis of data and an emphasis on describing functional relations between behavior and controlling variables. EAB methodology was founded by B.F. Skinner and first publicly presented in his book The Behavior of Organisms (1938/1966) (Cooper, Heron, \& Heward, 2007). EAB is often referred to as basic research. 

Applied behavior analysis (ABA) is a science that develops its technology based on the principles of behavior and applies them systematically to produce socially significant behavior change. Furthermore, experimentation is used to identify the independent variables responsible for behavior change. Lastly, the adequacy of ABA work is determined by the seven characteristics (applied, behavioral, analytic, technological, conceptually systematic effective and generalizable) set forth by Baer, Wolf and Risley (1968) (Cooper, Heron, \& Heward, 2007). 

Behavioral service delivery consists of putting ABA principles that have been experimentally validated into practice.  Behavioral service providers design, implement an evaluate behavior change procedures applied to socially significant behavior (Cooper, Heron, \& Heward, 2007). It is important that behavioral service providers apply a conceptual framework in order to offer a thorough explanation of the causes of behavior that are consistent with the established science of behavior. However it is also important that they can explain these concepts to non-behavioral service providers and families in everyday language by essentially, strengthening two verbal repertoires (Johnston, 2013). 

The four above mentioned domains have been described as an overlapping continuum that includes similarities and differences between each domain. This view emphasizes the fact that the four domains should be dependent on each other and mutually influenced by developments in each of the other domains (Cooper, Heron, \& Heward 2007; Moore \& Cooper, 2003). Moore and Cooper (2003) argue that students of behavior analysis be offered a balanced approach by incorporating all four domains into their training experience. 
%
%
%
\subsection{Assessment}
\begin{enumerate}
\item Ask the supervisee to define each of the domains and then state how each influences the other domains. 
\item Throughout the course of supervision, when reading behavior analytic articles have the supervisee describe the implications of the article in terms of its relation to each of the four domains 
\item Require the supervisee to speak in technical dialect but also provide the supervisee with opportunities to practice restating their precise understanding of behavioral concepts in everyday language that can be understood by families and non-behavioral providers while retaining the underlying philosophy of behavior. 
\item Have a discussion with the supervisee about how the four domains are interrelated and why it is important that the domains influence each other and not operate in isolation.  
\end{enumerate}
%
\subsection{Relevant Literature}
\begin{refsection}
\nocite{baer1968some,
        baum1995radical,
        baum2011what,
        cooper2007applied,
        johnston2014radical,
        moore2003some,
        skinner1953science}
\printbibliography[heading=none]
\end{refsection}
%
\subsection{Related Tasks} 
\fourbOne{}\\ 
\fourgFour{}\\
\fourgFive{}\\
\fourgSix{}\\
\fourFKOne{}\\
\fourFKTwo{}\\
\fourFKThree{}\\
\fourFKFour{}\\
\fourFKFive{}\\
\fourFKSix{}\\
\fourFKSeven{}\\
\fourFKEight{}\\
