\clearpage \section{\fourFKEleven{}}
\subsection{Definition}  
Environment - ``The conglomerate of real circumstances in which the organism or reference part of the organism exists; behavior cannot occur in the absence of environment'' (Cooper, Heron, \& Heward, 2007, p. 694).

Stimulus - ``Any physical event, combination of events, or relation among events'' (Catania, 2013, p. 466).  

Stimulus class - ``A group of stimuli that share specified common elements along formal (e.g. size, color), temporal (e.g. antecedent or consequent), and/or functional (e.g., discriminative stimulus) dimensions'' (Cooper, Heron, \& Heward, 2007, p. 705).\\

``Any group of stimuli sharing a predetermined set of common elements in one or more of these dimensions'' (Cooper, Heron, \& Heward, 2007, p. 27).  
\subsection{Examples}
A trip to the mall
\begin{enumerate}
\item  Environment:  Willis is shopping at the local mall. The local mall would be an environment.
\item Stimulus: Willis is walking through the food court.  He smells some pizza cooking from one of the establishments and suddenly his stomach starts growling.  He stops and gets some food.  All of the things in the food court, including the smells, the changes in his stomach, and the food are stimuli.
\item Stimulus class:  At the food court, Willis will buy items that will all function as reinforcers for eating behavior. In this case, the burger, the fries, and the cookie he bought are in the same stimulus class.
%
\end{enumerate}
%
\subsection{Assessment}
\begin{enumerate}
\item Ask your supervisee to define environment, stimulus, and stimulus class.   
\item Ask your supervisee to identify the environment, the stimulus (stimuli), and the stimulus class(es) from the above examples.  Use examples of stimulus classes related to the formal, temporal, and functional dimensions. 
\item Ask your supervisee to create other examples and a non-examples of his/her own. 
\item Have your supervisee to compare and contrast these terms. 
%
\end{enumerate}
%
\subsection{Relevant Literature}
\begin{refsection}
\nocite{catania2013learning,
        cooper2007applied,
        johnston1993strategies,
        michael2004concepts}
\printbibliography[heading=none]
\end{refsection}
%
\subsection{Related Tasks}
\fourFKEleven{}\\
