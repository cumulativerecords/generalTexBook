\clearpage \section[\foureEight{}]{\foureEight{}%
              \sectionmark{E-08 Use the matching law...}}
\sectionmark{E-08 Use the matching law...}
\subsection{Definition} 
Matching Law - ``When two or more concurrent-interval schedules are available, the relative rate of response matches (or equals) the relative rate of reinforcement. More generally, the matching law states that the distribution of behavior between (or among) alternative sources of reinforcement is equal to the distribution of reinforcement for these alternatives'' (Pierce \& Cheney, 2013, p. 260).

Choice - ``...the emission of one of two or more alternative and, usually, incompatible responses'' (Catania, 2007, p.431).

Organisms are constantly confronted with making choices; the allocation of responding is based upon the probability of reinforcement for that response. There are also other variables known to effect response allocation such as magnitude of reinforcement, quality of reinforcement, delay to reinforcement, and duration of reinforcement (Baum, 1974). If a variable that is affecting responding on a particular option cannot be identified, this is known as bias. An example of bias might be a right-handed person responding on an option to the right side. This variable and others are accounted for using different coefficients in the matching law.

Multiple basic and applied studies with humans and non-humans have demonstrated that behavior is allocated to response options based on reinforcement schedules available on those options (Baum, 1974; Borrero \& Vollmer, 2002; Epling \& Pierce, 1983).

There is debate about the status of the matching law as a convenient description vs. a fundamental property of behavior (c.f., Catania, 1981; Killeen, 2015, Rachlin, 1971)
%
\subsection{Assessment}
\begin{enumerate}
\item Have supervisee define choice and describe the matching law.
\item Have supervisee describe variables known to influence response allocation among alternatives.
\item Have supervisee set up and conduct an experiment involving concurrent schedules with different magnitudes/qualities/delays/duration of reinforcement on different options.
\item Have supervisee use the equation for the matching law to investigate matching for two different responses.
%
\end{enumerate}
%
\subsection{Relevant Literature}
\begin{refsection}
\nocite{baum1974two,
        baum1979matching,
        borrero2002application,
        catania2013learning,
        catania2012discussion,
        cooper2007applied,
        epling1983applied,
        herrnstein1961relative,
        killeen2015arithmetic,
        pierce2013behavior,
        rachlin1971tautology}
\printbibliography[heading=none]
\end{refsection}
%
\subsection{Related Tasks}
\fouraFourteen{}\\
