\clearpage \section[\fourFKEight{}]{\fourFKEight{}%
              \sectionmark{FK-08 Distinguish between rad...}}
\sectionmark{FK-08 Distinguish between rad...}
\subsection{Definition}
Radical behaviorism – ``the philosophy of a science of behavior treated as a subject matter in its own right apart from internal explanations, mental or physiological'' (Skinner, 1989, p. 122).

Methodological behaviorism – ``represents a formal and strategic agreement to regard the relation between publicly observable stimulus variables and publicly observable behavior as the appropriate subject matter for psychology as a science'' (Moore, 2008, p. 385).

The distinction between radical and methodological behaviorism can be summed up by the views on private events. Private events, or events observable by only the individual engaging in the response, are not included in the analysis of behavior by a methodological behaviorist position. Radical behaviorists consider private events to be no different than any other behavior, therefore, allowing it to be understood within the same conceptual framework understood for overt behavior.

\subsection{Assessment}
\begin{enumerate}
\item Ask your supervisee to describe the distinction between radical and methodological behaviorism.
\item Have your supervisee describe the advantages of the methodological behaviorist's view.
\item Have your supervisee describe the advantages of the radical behaviorist's view.
\end{enumerate}
%
\subsection{Relevant Literature}
\begin{refsection}
\nocite{baum2011what,
        cooper2007applied,
        moore2011review,
        moore2009radical,
        moore2008conceptual,
        skinner1989recent}
\printbibliography[heading=none]
\end{refsection}
%
\subsection{Related Tasks}
\fourbOne{}\\
\fourgFour{}\\
\fourgFive{}\\
\fourFKOne{}\\
\fourFKSeven{}\\
