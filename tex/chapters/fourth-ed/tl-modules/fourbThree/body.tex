\clearpage \section[\fourbThree{}]{\fourbThree{}%
              \sectionmark{B-03 Systematically... independent var}}
\sectionmark{B-03 Systematically... independent var}
              
              

%\sectionmark{B-03 Systematically arrange independent variables...}%
%\section[B-03 Systematically arrange independent variables...]{\fourbThree{}}
\subsection{Definitions} 
Independent Variable - ``The variable that is systematically manipulated by the researcher in an experiment to see whether changes in the independent variable produce reliable changes in the dependent variable. In applied behavior analysis, it is usually an environmental event or condition antecedent or consequent to the dependent variable. Sometimes called the intervention or treatment variable'' (Cooper, Heron, \& Heward, 2007, p. 697).\\

Dependent Variable - ``The variable in an experiment measured to determine if it changes as a result of the manipulations of the independent variable; in applied behavior analysis, it represents some measure of a socially significant behavior'' (Cooper et al. 2007, p. 693).\\

Dependent variables must be operationally defined to allow for consistent assessment and replication of the assessment process, measured repeatedly within and across controlled conditions, recording is assessed for consistency across the experiment using inter-observer agreement, and dependent variables must be socially significant to the individual or those around them. (Horner, Carr, Halle, McGee, Odom, \& Wolery, 2005)\\

Experimental control is achieved when predicted change in the dependent variable (i.e., the behavior) covaries with manipulations of the independent variable (i.e., the intervention) showing the effectiveness of the independent variable on the dependent variable of a participant. (Horner et al., 2005)
%
\subsection{Examples} 
\begin{enumerate}
\item   A student consistently disrupts group activities. When given visuals for appropriate behavior (i.e., quiet voice, calm body) paired with gestural redirection, disruptive behavior in group lessons decreases. The teacher then takes the visuals away for a week to see if fading these supports would be an option. The gestural redirection for inappropriate behavior is still in place. The student's disruptive behavior remains low. When the redirection is removed the following week. The student engages in increased disruptive behavior during this week, so the teacher decides to continue the gestural prompts and the disruptive behavior decreases again. 
\item Non-example: A student with attention deficits consistently disrupts group activities. His teacher occasionally uses the visuals for appropriate behavior outlined in the BSP and the disruptive behavior does not decrease.
\end{enumerate}

\subsection{Assessment}
\begin{enumerate}
\item Give supervisees article abstracts on single subject research. Have them identify the dependent variable and independent variable for the study.
\item Have supervisees identify the independent and dependent variables in the example listed above.
\item Have supervisees read Horner et al., (2005) The Use of Single-Subject Research to Identify Evidence-Based Practice in Special Education and complete a brief summary of the article and ask them to identify what compromises the integrity of a functional relationship and define the quality indicators outlined for effective single-subject research.
\end{enumerate}
%
\subsection{Relevant Literature} 
\begin{refsection}
\nocite{cooper2007applied,horner2005use}
\printbibliography[heading=none]
\end{refsection}%%

\subsection{Related Tasks}
\fourbFour{}\\
\fourbFive{}\\
\fourbSix{}\\
\fourbSeven{}\\
\fourbNine{}\\
\fourbEleven{}\\ 
\fourhFour{}\\
\fouriOne{}\\
