\clearpage \section[\fourgSeven{}]{\fourgSeven{}%
              \sectionmark{G-07 Practice within one's limits...}}
\sectionmark{G-07 Practice within one's limits...}
\subsection{Definition}
Behavior analysts follow guidelines related to boundaries of competence in 1.02 of the Behavior Analyst Certification Board professional and ethical compliance code for behavior analysts:

1.02 Boundaries of Competence.\\
(a) All behavior analysts provide services, teach, and conduct research only within the boundaries of their competence, defined as being commensurate with their education, training, and supervised experience. \\
(b) Behavior analysts provide services, teach, or conduct research in new areas (e.g., populations, techniques, behaviors) only after first undertaking appropriate study, training, supervision, and/or consultation from persons who are competent in those areas... (BACB, 2014, p.4).\\

Practicing within your area of competence, training and experience\\
If, for example, a senior therapist working within an intensive behavior intervention program for preschoolers suddenly began working with adults with phobias then they would be in violation of this ethical guideline. Cooper, Heron, and Heward (2007) go on further to say that even within one's competence area, if a situation exceeds your training or experience then a referral to another behavior analyst should be made. If there is a gap in expertise, available then workshops and conferences may be accessed.  Mentors, supervisors and colleagues can provide additional training.  

Bailey and Burch (2011) relate this ethical boundary to the Hippocratic Oath, ``Do no harm.''  The guideline addresses the responsible conduct of behavior analysts, ensures the safety of clients, and protects the integrity of the field.
%
\subsection{Assessment}
\begin{enumerate}
\item Discuss case examples from Bailey and Burch (2011).
\item Have the supervisee come up with five fictitious examples of situations where the ethical guideline 1.02 was broken and provide creative solutions to the situation.
\item Use behavior skills training to teach your supervisee how to respond to a client asking your supervisee to provide services within an area in which they had no experience.
%
\end{enumerate}
%
\subsection{Relevant Literature}
\begin{refsection}
\nocite{bailey2013ethics,
        bac2014professional,
        cooper2007applied,
        van1988right}
\printbibliography[heading=none]
\end{refsection} 
%
\subsection{Related Tasks}
\fourbTwo{}\\
\fourgTwo{}\\
\fourgSix{}\\
\fourkEight{}\\
\fourkNine{}\\
