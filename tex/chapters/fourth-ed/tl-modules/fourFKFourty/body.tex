\clearpage \section{\fourFKFourty{}}
\subsection{Definition} 
Matching law - ``a quantitative formulation stating that the relative rates of different responses tend to equal the relative reinforcement rates they produce'' (Catania, 2007, p. 449).\\

Herrnstein (1961) described pigeon's distribution of responding on concurrent schedules of reinforcement. He found the relation between absolute rate of reinforcement and the absolute rate of responding is a linear function that passes through the origin. In other words, if rate of reinforcement and rate of responding are plotted on the x and y axis the result of the data is very close to a line that passes through the origin with the slope of 1. The matching equation can be denoted as follows:\\

The term  is behavior measured as rate of response for behavior a, and  is behavior measured as rate of response for behavior b. The term  is the scheduled rate or reinforcement for response a, and  is the scheduled rate of reinforcement for response b.\\

The matching law has been demonstrated across a variety of species including pigeons, cows (Mathews \& Temple, 1979), rats (Poling, 1978), free ranging flocks of birds (Baum, 1974), and humans (Conger \& Killeen, 1974). \\

There are also multiple applied examples. For example, Mace, Neef, Shade, and Mauro (1994) found special education high school students spent time on math problems, arranged in different stacks, that was equal to the relative rate of reinforcement for completing math problems from the different stacks. Also, Borrero and Vollmer (2002), after conducting functional analyses to find maintaining variables for problem behavior, found that proportional rates of problem behavior relative to problem behavior matched the proportional rate of reinforcement for 4 individuals with disabilities. 
%
\subsection{Assessment}
\begin{enumerate}
\item Ask your supervisee to describe matching law and how it relates to choice (response allocation).
\item Have supervisee explain a situation where the matching law could be applied and describe the concurrent rates of reinforcement in place for two different behaviors.
\item Relate matching law to the selection of alternative behavior in a DRA.
%
\end{enumerate}
%
\subsection{Relevant Literature}
\begin{refsection}
\nocite{baum1974choice,
        borrero2002application,
        conger1974use,
        herrnstein1961relative,
        mace1994limited,
        matthews1979concurrent,
        poling1978performance}
\printbibliography[heading=none]
\end{refsection}
%
\subsection{Related Tasks}
\fouraFourteen{}\\
\fourbThree{}\\
\foureEight{}\\
\fourgFour{}\\
