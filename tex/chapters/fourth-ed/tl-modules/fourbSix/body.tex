\clearpage \section{\fourbSix{}}
\subsection{Definition} 
Changing criterion design - ``An experimental design in which an initial baseline phase is followed by a series of treatment phases consisting of successive and gradually changing criteria for reinforcement or punishment. Experimental control is evidenced by the extent the level of responding changes to conform to each new criterion'' (Cooper, Heron \& Heward, 2007, pp. 691-692).

``The design requires initial baseline observations on a single target behavior. This baseline phase is followed by implementation of a treatment program in each of a series of treatment phases. Each treatment phase is associated with a stepwise change in criterion rate for the target behavior. Thus, each phase of the design provides a baseline for the following phase. When the rate of the target behavior changes with each stepwise change in the criterion, therapeutic change is replicated and experimental control is demonstrated'' (Hartmann \& Hall, 1976, p. 527).

Guidelines for using the changing criterion design include:
\begin{enumerate}
\item Manipulation of the length of phases. Each phase serves as a baseline to compare responding to the next phase. Each phase must be long enough to display stable responding before moving to the next phase.
\item ``Varying the size of the criterion change enables a more convincing demonstration of experimental control'' (Cooper et al., 2007, p. 222). Criterion change magnitude must be carefully considered so that the criterion is not too large and unattainable but also not too small in magnitude which would not demonstrate sufficient experimental control.
\item Experimental control is demonstrated through replication of treatment effects. Therefore, as the number of phases increases, so does the opportunity to replicate treatment effects and enhance experimental control.
\end{enumerate}
%
\subsection{Examples}
\begin{enumerate}
\item Jim created a class wide reinforcement program to increase the vocabulary test scores of a 1st grade class. He wanted to make sure that the reinforcement program was effective so he set specific score criterion for the class, to monitor their progress. Average baseline test scores were 55\% correct for the entire class. Jim set the first criterion phase at 70\% of the test questions answered correctly for the entire class. After 4 weeks, the class met these criteria for 3 consecutive tests, so Jim set the classroom performance criterion to 80\% of the test questions answered correctly. This time the class met the criterion in 3 weeks and Jim increased the criterion to 90\%. Once again, the class met these criteria for 3 consecutive tests. Jim concluded that his intervention was likely responsible for the change in test scores, since the test score reliably increased when the criteria were altered and required a greater score.
\item (Non-example) John created a reinforcement program to increase Larry's rate of answering questions during class. After 3 weeks, the data indicated that Larry was answering more questions appropriately in class. However, there was a new teacher in the classroom and other variables that may have accounted for this change. John wanted to see if the program was increasing this behavior, so he decided to remove the reinforcement program for a week to see if Larry's rate of answering questions decreased.
\end{enumerate}
%
\subsection{Assessment}
\begin{enumerate}
\item Have supervisee describe the changing criterion design and state when it may be most appropriate, strengths of this design, as well as limitations of the changing criterion design.
\item Have supervisee create a hypothetical analysis using the changing criterion design. Have him/her state why the changing criterion design was the most effective design to display experimental control.
\item Have supervisee label the parts of a completed changing criterion design graph and describe how the graph displays experimental control. 
\end{enumerate}
%
\subsection{Relevant Literature}
\begin{refsection}
\nocite{cooper2007applied,
    hartmann1976changing,
    mcdougall2005range,
    mclaughlin1983examination,
    hall1977changing,
    allen2001exposure}
\printbibliography[heading=none]
\end{refsection}
%
\subsection{Related Tasks}
\fourbFour{}\\
\fourbFive{}\\
\fourbSeven{}\\
\fourbNine{}\\
\fourbEleven{}\\ 
\fourhFour{}\\
\fouriOne{}\\
