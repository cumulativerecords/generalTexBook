\clearpage \section[\fourdEight{}]{\fourdEight{}%
              \sectionmark{D-08 Use discrete-trial...}}
\sectionmark{D-08 Use discrete-trial...}
\subsection{Definition} 
Free operant - `` behaviors that have discrete beginning and ending points, require minimal displacement of the organism in time and space, can be emitted at nearly any time, do not require much time for completion, and can be emitted over a wide range of response rates'' (Cooper, Heron, \& Heward, 2007, p. 696).

Rate of responding is typically used to measure behavior considered to be free operant. Rate of responding is typically used because it uses count per unit of time. (i.e., a person can type 50 words per minute). However, rate of responding is not typically used to measure behavior that occurs within discrete trials. These responses can occur only within limited or restricted situations. 

Discrete trial responses include responses to flash cards, answering teacher's questions, and choosing an answer from an array. 

Discrete trial- ``Any operant whose response rate is controlled by a given opportunity to emit the response. Each discrete response occurs when an opportunity to respond exists'' (Cooper et al., 2007, p. 694).

\subsection{Assessment}
\begin{enumerate}
\item Have supervisee create a simple discrete trial program to teach a skill. Have him/her run conduct a trial using this program and identify the parts of a discrete trial.
\item Have supervisee identify the differences between a discrete trial and a free operant trial. Have him/her list pros and cons for each method.
\item Have supervisee identify what type of measurement procedures and data you would take for free operant trials as well as discrete trials. Have him/her describe why they would use that measurement system for each.
\end{enumerate}
%
\subsection{Relevant Literature}
\begin{refsection}
\nocite{cooper2007applied,
    malott2003principles,
    mazur2002learning}
\printbibliography[heading=none]
\end{refsection}
%could not locate bibliography or verify reference.
%Otto, J. (2003). Discrete trial procedures vs. free-operant procedures. Retrieved from http://old.dickmalott.com/students/undergradprogram/psy3600/discrete_vs_free.html. 
%
\subsection{Related Tasks}
\fouraOne{}\\
\fouraTwo{}\\
\fouraSeven{}\\
\fourdThree{}\\
\foureOne{}\\
\foureTwelve{}\\
\fourhOne{}\\
\fourFKTen{}\\
\fourFKEleven{}\\
\fourFKFourtySeven{}\\
