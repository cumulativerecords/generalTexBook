\clearpage \section{\fourFKFourtyOne{}}
\subsection{Definition}
Contingency shaped behavior – Behavior that is ``selected and maintained by controlled, temporally close consequences'' (Cooper, Heron, \& Heward, 2007, p. 42).  These consequences may either be reinforcing or punishing.  
%
\subsection{Examples}
\begin{enumerate}
\item Melvin puts a dollar into the soda machine and pushes the cola button. Seconds later a can of soda comes out. He opens the soda and drinks it.  He buys 3 more drinks from the same machine that week.
\item Simon's friend Ernest is a prankster.  Ernest shakes up a can of soda and offers Simon a drink.  The can sprays him in the face and soaks his clothing.  The next time Ernest offers a soda, Simon is hesitant to accept.  Although he'd like to open the soda and drink it, he hands it back expecting another explosive surprise.
\item Thirsty Floyd finds a 12 pack of old sodas in the storeroom.  He cracks a can and starts to drink. Unfortunately, the soda has gone bad. Floyd gets sick from drinking the soda. In the future Floyd avoids drinking old sodas.
\item (Non-example) Horace's mom tells him that drinking soda is bad for him. Horace avoids drinking soda in the future.
%
\end{enumerate}
%
\subsection{Assessment}
\begin{enumerate}
\item Ask your supervisee to explain contingency shaped behavior 
\item Ask your supervisee how contingency shaped is different from rule-governed behavior
\item Ask your supervisee to create another example and non-example of his/her own. 
\item Ask your supervisee to state why it might be better to use contingency shaped consequences as opposed consequences, which are more delayed or rule governed      
\end{enumerate}
%
\subsection{Relevant Literature}
\begin{refsection}
\nocite{cooper2007applied,
        malott2003principles,
        michael2004concepts}
\printbibliography[heading=none]
\end{refsection}
%
\subsection{Related Tasks}
\foureFour{}\\
\fourFKFourtyTwo{}\\
