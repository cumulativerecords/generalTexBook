\clearpage \section{\fourfFive{}}
\subsection{Definition}
Personalized system of instruction (PSI) was created by Fred S. Keller.  PSI uses self-paced modules with study guides to help direct students' learning. Proctors are used to help students with the material and lectures are not as common compared to traditional teaching formats. PSI focuses on self-paced learning by the student rather than teacher directed instruction. Specific criteria on mastery tests are required before moving on to the next module. Keller (1968) summarized his writings from 1967 describing features of this type of training: 
\begin{enumerate}
\item The go-at-your-own-pace component
\item The unit-perfection requirement for advancement
\item The use of lectures and demonstrations sparingly but not for critical information
\item The promotion of written word in teacher-student communication
\item The use of proctors for repeated testing, immediate scoring, answering questions, and personal-social part of the educational process
\end{enumerate}
%
\subsection{Assessment}
\begin{enumerate}
\item Have supervisee identify the key components of personalized Systems of Instruction (PSI).
\item Have supervisee provide benefits and drawbacks for instruction using PSI.
\item Have supervisee choose a topic and create a mock PSI curriculum on that topic. Include modules, competency exams, and guidelines for other considerations when using PSI for that topic. 
%
\end{enumerate}
%
\subsection{Relevant Literature}
\begin{refsection}
\nocite{axelrod1992disseminating,
        buskist1991life,
        keller1994fred,
        keller1968good,
        keller1982psi,
        twyman1998fred,
        eyre2007keller}
\printbibliography[heading=none]
\end{refsection}
%
\subsection{Related Tasks}
\fourfOne{}\\
\fourfFour{}\\
\fourjFourteen{}\\
