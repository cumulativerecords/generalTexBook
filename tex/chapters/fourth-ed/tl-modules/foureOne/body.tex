\clearpage \section[\foureOne{}]{\foureOne{}%
              \sectionmark{E-01 Use interv... manipulation}}
\subsection{Definition}
While it is commonly known that behaviors are maintained by consequences, including antecedent interventions within an individual's treatment package can often expedite positive behavioral change and mitigate negative effects of consequent strategies (e.g. extinction bursts). Some antecedent strategies include motivating operations, discriminative stimuli, non-contingent reinforcement and usage of high probability request sequences.

Michael (1982, p. 149) describes motivating operations as ``a stimulus change which, (1) given the momentary effectiveness of some particular type of reinforcement (2) increases the frequency of a particular type of response (3) because that stimulus change has been correlated with an increase in the frequency with which that type of response has been followed by that type of reinforcement.''

Skinner first explored this concept, describing deprivation and satiation to be motivating variables that govern behavior. Simply put, reinforcers obtain most of their reinforcing value depending on the individual's drive to obtain that reinforcer, which is a direct result of deprivation-satiation contingencies. The motivating operation for one's behavior that has run three miles in the heat is to quench their thirst increasing the value of water as a reinforcer. Having no money to put into a vending machine to get a bottle of water is the motivating operation to ask friends for loose change. Similarly, after the person drinks an entire bottle of water, water may no longer function as a reinforcer. Behavior analysts can thereby affect behavioral change by manipulating motivating operations (e.g. challenging behavior maintained by escape from non-preferred tasks may be mitigated by giving the individual frequent breaks). 

Another antecedent strategy is effectively programming for discriminative stimuli. Skinner claimed that virtually all operant behavior falls under stimulus control, asserting that ‘‘if all behavior were equally likely to occur on all occasions, the result would be chaotic'' (Skinner, 1953, p. 108).  It is therefore important for individuals to learn to discriminate between conditions in which specific responses will be reinforced and when these responses will not. Discriminative stimuli evoke behavior because they have been correlated with increased probability of accessing a reinforcer. For instance, teaching a student to mand for a break can be problematic if the student mands for a break continuously throughout the day, thereby yielding very little on-task behavior. However, this can be possibly remedied by the availability of a break is represented by the presence of a break icon (e.g. break icon is the discriminative stimulus, signaling that if the student asks for a break, a break will be granted).  

Other antecedent strategies include usage of non-contingent reinforcement and usage of high-probability request sequences. Non-contingent reinforcement is ``an antecedent intervention in which stimuli with known reinforcing properties are delivered on a fixed-time or variable-time schedule independent of the learner's behavior'' (Cooper, Heron \& Heward, 2007, p. 489). This operates on the principle of motivating operations. By satiating an individual with wants/needs, the individual is no longer motivated to engage in responses that used to generate that want/need (e.g. giving attention to a student every five minutes may abolish attention as a reinforcer, thereby reducing the need to engage in inappropriate attention-seeking behavior). 
%
\subsection{Assessment}
\begin{enumerate}
\item Have your supervisee list and describe applicable antecedent interventions.
\item In a clinical setting or during role-play, have your supervisee describe what motivating operations may be affecting the client's behavior.
\item Describe a scenario and have the supervisee lists some potential antecedent strategies that can be used and have them describe why they chose these strategies.
\end{enumerate}
%
\subsection{Relevant Literature}
\begin{refsection}
\nocite{michael1982distinguishing,
        skinner1953science,
        smith1997antecedent}
\printbibliography[heading=none]
\end{refsection}
%
\subsection{Related Tasks} 
\foureNine{}\\
\fourFKTwentySix{}\\
\fourFKTwentySeven{}\\
\fourFKTwentyNine{}\\
