\clearpage \section[\fourjEleven{}]{\fourjEleven{}%
            \sectionmark{J-11 Program for stimulus...}}
\sectionmark{J-11 Program for stimulus...}
\subsection{Definition}
Behavior analysts teach socially significant skills to help clients function where they work or live. Program interventions for the client behavior to contact naturally occurring reinforcement contingencies in their typical environments using new skills. Generalization might result in a ``a great deal of generalized behavior change; that is, after all components of an intervention have been terminated, the learner may emit the newly acquired target behavior, as well as several functionally related behaviors not observed previously in his repertoire, at every appropriate opportunity in all relevant settings, and he may do so indefinitely'' (Cooper, Heron, \& Heward, 2007, p. 621).

To increase stimulus generalization, the behavior analyst systematically varies where intervention is implemented, conditions under which it is implemented, and people who implement it in a gradual manner from acquisition through practicing stages. To develop a response class of functionally equivalent skills, the behavior analyst trains using a variety of responses that eventually may include incorrect responses and novel correct responses.
%
\subsection{Assessment}
\begin{enumerate}
\item Have supervisee read Stokes and Baer (1977).
\item Have supervisee list and describe the ways one might program for generalization.
\item Provide examples of skills that may be taught to an individual. Have supervisee describe how they can program for generalization and indicate which type of generalization (stimulus or response) they will be training.
\end{enumerate}
%
%
%
%
%\subsection{Examples}
%\begin{enumerate}
%\item 
%\end{enumerate}
%
\subsection{Relevant Literature}
\begin{refsection}
\nocite{cooper2007applied,
        stokes1977implicit}
\printbibliography[heading=none]
\end{refsection}
%
\subsection{Related Tasks}
\foureTwo{}\\
\fouriSix{}\\
\fourjTwo{}\\
\fourjTwelve{}\\
\fourFKEleven{}\\
\fourFKTwelve{}\\
\fourFKTwentyFour{}\\
\fourFKThirtySix{}\\
