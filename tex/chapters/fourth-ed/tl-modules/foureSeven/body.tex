\clearpage \section{\foureSeven{}}
\subsection{Definition}
What is behavioral contrast?
\begin{enumerate}
\item George Reynolds first presented behavioral contrast in 1961. He defined behavioral contrast as ``an increase in the rate of responding in one component of a multiple schedule when certain changes occur in the other component'' (p. 60). 
\item Cooper, Heron and Heward (2007) state that behavioral contrast ``can occur as a function of a change in reinforcement or punishment density on one component of a multiple schedule'' (p. 337).
\end{enumerate}
%
\subsection{Examples}
\begin{enumerate}
\item Cooper, Heron and Heward (2007) give a good example of behavioral contrast to illustrate the concept; ``...a pigeon pecks a backlit key, which alternates between blue and green, reinforcement is delivered on the same schedule on both keys, and the bird pecks at approximately the same rate regardless of the key's color'' (p. 337). However, this changes so that responses on one component of the schedule are punished, i.e., pecks on the blue key are punished because reinforcement is not delivered, but pecks on the other (green) key continue to produce reinforcement. As a result, rate of responding decreases on the blue key and rate of responding on the green key increases, even though no more reinforcement is delivered from the green key than before.
\end{enumerate}
%
Plan for the effects of behavioral contrast
\begin{enumerate}
\item It is important to consider prior to beginning an intervention, whether behavioral contrast may occur as a result of that planned intervention. If behavioral contrast is a possibility; then planning for its occurrence is crucial.
\item Cooper, Heron and Heward (2007, p. 338) suggest that one way to minimize or completely prevent the contrast effects of punishment is to plan the intervention so that the consequence is consistently applied to the target behavior across all relevant environments and stimulus conditions. All those involved in the client's life that may be required to deliver the consequence, will need to be thoroughly trained to ensure its consistent implementation.
\item Additionally, reinforcement will need to be minimized, or where possible, withheld, when the target behavior has occurred. Similarly, training will need to be provided to all those involved so that the client isn't receiving reinforcement when the target behavior is emitted.
\end{enumerate}
%
\subsection{Assessment}
\begin{enumerate}
\item Ask your Supervisee to define behavioral contrast.
\item Ask your Supervisee to describe what will happen in this example if behavioral contrast is in effect: 
\item A child has been playing with two musical toys, allocating an equal amount of time playing with each toy. One of the musical toys is yellow and one of the musical toys is red. The red toy's battery begins to give out so that when the child presses the button, sometimes the music is not produced. However, the yellow musical toy continues to work well and music is produced each time the child pushes the button. What will happen to the rate of responding for each of the musical toys? (Answer = the rate of responding on the red musical toy will decrease and the rate of responding on the yellow toy will increase, even though there isn't any additional reinforcement being produced from the yellow toy).
\end{enumerate}
%
\subsection{Relevant Literature}
\begin{refsection}
\nocite{cooper2007applied,
        fagen1978behavioral,
        hantula1994behavioral,
        mcsweeney1998habituation,
        reynolds1961behavioral,
        reynolds1963some,
        tarbox2005verbal,
        weatherly1996picking,
        weatherly2002rats}
\printbibliography[heading=none]
\end{refsection}
%
\subsection{Related Tasks}
\fourcTwo{}\\
\fourdFifteen{}\\
\fourdSixteen{}\\
\fourdSeventeen{}\\
\fourdNineteen{}\\
\fourjTen{}\\
\fourkTwo{}\\
\fourFKThirtyEight{}\\
\fourFKFourty{}\\
