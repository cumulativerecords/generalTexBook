\clearpage \section{\fourFKFourtyThree{}}
\subsection{Definition}  
Echoic - ``An elementary verbal operant involving a response that is evoked by a verbal discriminative stimulus that has point-to-point correspondence and formal similarity with the response'' (Cooper, Heron, \& Heward, 2007, p. 694).   
%
\subsection{Examples}
\begin{enumerate}
\item Mrs. Platypus is instructing her 3rd grade class on their math facts.  She holds up a card stating that, ``three times nine is eighteen.''  She then restates the fact asking the class to repeat.  The class says, ``three times nine is eighteen'' in unison.  Mrs. Platypus praises the students for their repetition.
\item Mr. Penguin is a kindergarten teacher.  He is working with one student on his reading skills.  He shows little Timmy the letter R.  He tells him that the letter R makes the ``rrr'' sound and asks him to repeat.  Little Timmy says, ``rrr,'' and Mr. Penguin comments, ``Nice job Timmy.''  
\item Mrs. Dodo the art teacher needs one of her students to run to the office and get some supplies. One of the children volunteers.  She tells him that she needs him to get, ``Crayons, markers, and paint.''  He repeats, ``Crayons, markers, and paint.''  ``Exactly,'' Mrs. Dodo says sending him on his way.  
\item (Non-example) Mrs. Platypus is still working on math facts with her class.  She holds up the math fact 4x9= and asks the students to give the answer.  Susie Q raises her hand and answers ``thirty-six.''  
%
\end{enumerate}
%
\subsection{Assessment}
\begin{enumerate}
\item Ask the supervisee to give the definition of an echoic
\item Ask the supervisee to give several examples of echoics
\item Ask the supervisee to give a non-example of an echoic 
%
\end{enumerate}
%
\subsection{Relevant Literature}
\begin{refsection}
\nocite{cooper2014applied,
        sundberg2008verbal,
        skinner1957verbal}
\printbibliography[heading=none]
\end{refsection} 
%
\subsection{Related Tasks}
\fourdFour{}\\
\fourdTen{}\\
