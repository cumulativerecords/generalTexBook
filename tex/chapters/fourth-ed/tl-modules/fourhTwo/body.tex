\clearpage \section[\fourhTwo{}]{\fourhTwo{}%
              \sectionmark{H-02 Select a schedule of obs...}}
\sectionmark{H-02 Select a schedule of obs...}
\subsection{Definition}
BCBAs select the most appropriate forms of measurement. The target behavior should have a clear and observable operational definition so that it can be recorded during a period of observation. Using a consistent measurement procedure and schedule of observation will help ensure that the data are truly reflective of the target behavior you want to measure. Specify a time interval (or observation period) for recording data. Additionally, time intervals used in interval recording should remain consistent across observations (e.g., if you collect data on a target behavior during a 15 minute observation using 30-second partial intervals then you should use the same procedure on subsequent observations).  If you change any aspect of the operational definition, observational period, or measurement procedures, explain the change when presenting data.
%
\subsection{Examples}
\begin{enumerate}
\item If your client's parent reported that aggression only happens when the client is asked to brush their teeth, then you want to ensure that you can observe the client during times they are asked to brush their teeth.
\item An instructor wanted to use the PLACHECK measurement procedure to record his students' involvement in an in-class group assignment.  It would be appropriate to collect these data when students separate into groups and start the assignment.  Recording data during the lecture or a test would not be an appropriate observation period.
%
\end{enumerate}
%
\subsection{Assessment}
\begin{enumerate}
\item Provide supervisee with a target behavior and ask supervisee to explain what measurement procedure should be used and observation periods should occur.
\item Provide supervisee with various scenarios of target behaviors, measurement procedure, and observation period and ask supervisee to determine the appropriateness of the measurement procedure and observation.
\item Ask supervisee to develop procedures for observing and recording a target behavior in a client's behavior intervention program.
%
\end{enumerate}
%
\subsection{Relevant Literature}
\begin{refsection}
\nocite{cooper2007applied,
        bailey201025}
\printbibliography[heading=none]
\end{refsection}
%
\subsection{Related Tasks}
\fouraOne{}\\
\fouraTwo{}\\
\fouraThree{}\\
\fouraFour{}\\
\fouraFive{}\\
\fouraSix{}\\
\fouraSeven{}\\
\fouraTwelve{}\\
\fouraThirteen{}\\
\fouraFourteen{}\\
\fourFKFourtySeven{}\\
\fourFKFourtyEight{}\\
