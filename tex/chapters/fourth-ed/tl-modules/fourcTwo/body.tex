\clearpage \section[\fourcTwo{}]{\fourcTwo{}%
              \sectionmark{C-02 State... punishment.}}
\sectionmark{C-02 State... punishment.}
\subsection{Definition}
Punishment is sometimes used to change or shape behavior and may cause unwanted side effects. 
\begin{enumerate}
\item For instance, those individuals who are being treated using punishment procedures may become aggressive (Azrin \& Holz, 1966) or may have strong emotional reactions to such measures.  
\item An adult or child may become subject to negative modeling (such as imitating scolding or hitting behavior).
\item Those treated through the use of punishment may seek out escape/avoidance of the punisher or the contingencies surrounding punishment. 
\item In extreme cases, the use of punishment can result in harm or injury to the child or adult.  
\end{enumerate}

Punishment may also have unwanted effects with regards to an individual's future learning.  It may not appropriately generalize to new situations requiring further intervention. When used as the sole intervention in a treatment package, it fails to teach an individual an alternative behavior to engage in and consequently individuals may revert back to old behaviors without a replacement strategy. These behaviors may diminish temporarily only to be subject to a recovery of responding at a later period of time. (Catania, 1998)

As a result, the majority of those in the field agree that ``punishment be limited to those situations in which other interventions have failed'' (May, Risley, Twardosz, Friedman, Bijou, Wexler et al., 1975 as cited in Iwata et al., 1994, p. 198).  Iwata et al., (1994) described that reinforcement approaches to behavior reduction were just as effective as punishment approaches and that if a functional analysis of the problem behavior was done, the need for the use of punishment procedures was greatly reduced.  

During the supervision process, be diligent in choosing interventions, which are based on reinforcement and not solely on punishment.  The function of a problem behavior should always be assessed before making decisions regarding an individuals program to ensure effective treatment.  If a team has deemed that punishment is necessary as a part of a treatment package, it is important to state any potential unwanted effects of any procedure being utilized and to attempt to plan for these.

Consider the following when planning for punishment effects:
\begin{enumerate}
\item A team should always adhere to the ``Fair Pair Rule'' when using punishment.  This states that a ``practitioner should choose one or more alternatives to increase for every behavior targeted for reduction'' (White \& Haring, 1980, p. 423).
\item Be sure to plan for continuation of the procedure to different environments, staff and stimuli (any and all that apply).
\item Avoid modeling any behavior which you do not want the adult or child to imitate
\item The team should develop a contingency plan for managing aggression or extreme emotional responses (should they occur) and have safety measures in place to avoid accidental injury to the individual.
\item The team should develop a plan to manage any escape/avoidant behaviors that may occur 
\item Be aware the effects of punishment can be difficult to predict.  Staff may need to adjust the plan over time if the affects are not therapeutic or effective.  
\end{enumerate}
%
\subsection{Assessment}
\begin{enumerate}
\item Ask the supervisee to state the unwanted effects of punishment.
\item Ask the supervisee to plan for unwanted effects of punishment.  The supervisor should provide examples of commonly used punishment procedures within agency (such as restraint, time outs, or other punitive measures) and ask the supervisee to propose solutions to these problems. 
\item Ask the supervisee to select one behavior to target for increase for each behavior targeted for decrease. 
\end{enumerate}
%
\subsection{Relevant Literature}
\begin{refsection}
\nocite{azrin1966punishment,
    catania1998learning,
    cooper2007applied,
    iwata1994toward,
    may1976guidelines,
    white1980exceptional}
\printbibliography[heading=none]
\end{refsection}  
%
\subsection{Related Tasks}
\fourcTwo{}\\
\fourdSixteen{}\\
\fourdSeventeen{}\\
\fourdNineteen{}\\
\foureSeven{}\\
\fourFKThirtyOne{}\\
