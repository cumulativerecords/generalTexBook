\clearpage \section[\fourcOne{}]{\fourcOne{}%
              \sectionmark{C-01 State... reinforcement.}}
\sectionmark{C-01 State... reinforcement.}
\subsection{Definition}
Reinforcement has been long been defined as a crucial element to behavioral change. However, there are considerations that behavioral analysts should explore before implementing reinforcement strategies. Here are some considerations for the use of positive reinforcement:
\begin{enumerate}
\item May suppress the desired response (e.g., The availability of the reinforcer elicits behavior that may compete with the target response) (Balsam \& Bondy, 1983). 
\item May not be feasible for an individual that has little or no learning history with that reinforcement contingency (e.g., An individual that is being taught to swallow solid food may not progress with a program solely using positive reinforcement due to low baseline levels of swallowing solid foods) (Riordan, Iwata, Wohl \& Finney, 1984).  
\item Increases the frequency of the target behavior, thereby reducing the frequency in other responses that may also be desirable (e.g., While teaching a student to raise their hand and wait until they are called on in class, the student no longer garners others' attention by calling their name) (Balsam \& Bondy, 1983). 
\item May evoke aggression in others, especially in conditions which there are limited quantities of the reinforcer. Aggression may be directed at individuals that are also competing for same reinforce (Balsam \& Bondy, 1983).
\item  May also evoke aggression when group contingencies are used. Individual may become aggressive towards lower-performing teammates (Balsam \& Bondy, 1983). 
\item Removal of positive reinforcer has been correlated with lower than baseline levels of responding (Balsam \& Bondy, 1983).
\end{enumerate}
%
There are also considerations for the use of negative reinforcement. Here are some:
\begin{enumerate}
\item Can result in more challenging behavior due to the continuation of aversive stimulation if target behavior is not displayed 
\item Research shows that even for escape-maintained behavior, positive reinforcement contingencies may compete with negative reinforcement contingencies, therefore, decreasing escape-maintained behavior (Lerman, Volkert \& Trosclair, 2007). 
\item Example: A child that engages in aggression to escape tasks may be more likely complete tasks without aggression if access to an iPad was given contingent on work completion. This may be more effective than reducing aggression by providing breaks contingent on appropriate asking.
\item Negative reinforcement contingencies may reinforce minimal requirements needed to avoid/escape aversive stimulus; does not focus on quality of target response (Balsam \& Bondy, 1983).
\item Requires continuous aversive stimulation and aversive stimulation often elicits aggressive responses.
\item These unwanted effects of reinforcement can be curbed taking baseline levels of the target behavior before setting criteria for reinforcement, implementing preference assessments routinely, ensuring reinforcement schedule and reinforcers chosen are as natural to the individual's environment as possible to promote generalization, systematic thinning of the reinforcement schedule, and having concurrent schedules of reinforcement for positive and negative reinforcement when negative reinforcement is utilized.
\end{enumerate}
%
\subsection{Assessment}
\begin{enumerate}
\item Ask supervisee to list considerations for the use of positive reinforcement.
\item Ask supervisee to list considerations for the use of negative reinforcement.
\item Have supervisee outline which considerations may affect a particular client and what behavioral strategies can be used to curb these unwanted effects.
\end{enumerate}
%
\subsection{Relevant Literature}
\begin{refsection}
\nocite{balsam1983negative,
    flora2004power,
    kodak2007further,
    riordan1980behavioral}
\printbibliography[heading=none]
\end{refsection}
%
\subsection{Related Tasks}
\fourdOne{}\\
\fourdTwo{}\\
\fourdTwentyOne{}\\
\foureEleven{}\\
\fouriSeven{}\\
\fourjFour{}\\
\fourjFive{}\\
\fourjSix{}\\
\fourjSeven{}\\
\fourjEleven{}\\
