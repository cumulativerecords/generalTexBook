\clearpage \section{\fourFKTwentyFour{}}
\subsection{Definition}
Stimulus control - ``A situation in which the frequency, latency, duration, or amplitude of a behavior is altered by the presence or absence of an antecedent stimulus'' (Cooper, Heron, \& Heward, 2007, p. 705).\\

Discriminated operant - ``An operant that occurs more frequently under some antecedent conditions than under others'' (Cooper, Heron, \& Heward, 2007, p.694).\\

Discriminative stimulus (SD) - ``A stimulus in the presence of which responses of some type have been reinforced and in the absence of which the same type of responses have occurred and not been reinforced''  (Cooper, Heron, \& Heward, 2007, p. 694).

\subsection{Examples}
\begin{enumerate}
\item When the telephone rings, George picks up the receiver.  Picking up the receiver is under the stimulus control of a ringing phone. 
\item In the above example, picking up the receiver is the discriminated operant. 
\item In the above example, the telephone's ring is the discriminative stimulus.
%
\end{enumerate}
%
\subsection{Assessment}
\begin{enumerate}
\item Label the following:  The traffic light turns red, John steps on the break.  What is under stimulus control? What is the discriminated operant?  What is the discriminative stimulus? 
\item Ask the supervisee to explain how he can bring a student's behavior of saying ``dog'' under the stimulus control of the picture of a dog.
\item Ask supervisee how they could use stimulus control to reduce the jumping behavior of a man who jumps up and down so much that he is damaging his feet.
%
\end{enumerate}
%
\subsection{Relevant Literature}
\begin{refsection}
\nocite{cooper2007applied}
\printbibliography[heading=none]
\end{refsection} 
%
\subsection{Related Tasks}
\fourdNineteen{}\\
\fourFKTwentyTwo{}\\
\fourFKTwentyNine{}\\
