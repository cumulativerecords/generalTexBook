\clearpage \section[\fourFKFourtySeven{}]{\fourFKFourtySeven{}%
              \sectionmark{FK-47 Identify the measur...}}
\sectionmark{FK-47 Identify the measur...}
\subsection{Definition}
According to Johnston and Pennypacker (1993), behavior has 3 fundamental dimensional quantities (properties) that can be measured.
\begin{enumerate}
\item Repeatability – Refers to the fact that a behavior can occur repeatedly through time (i.e., behavior can be counted) (e.g., count, frequency, rate)
\item Temporal extent – Refers to the fact that every instance of behavior occurs during some amount of time (i.e., when behavior occurs it can be measured in time.) (e.g., duration)
\item Temporal locus – Refers to the fact that every instance of behavior occurs at a certain point in time with respect to other events. (i.e., occurrences of behavior can be measured at points in time.) (e.g., latency, interresponse time) (As cited in Cooper, Heron, \& Heward, 2007, p. 26)
\end{enumerate}
%
\subsection{Assessment}
\begin{enumerate}
\item Ask Supervisee to measure their own duration related to a task (eg. give them a timer and crossword puzzle to complete)
\item Ask Supervisee to measure and calculate the rate of someone tapping their pen (or another discrete behavior) during a 10 minute meeting
\item Ask Supervisee to observe a conversation between colleagues and measure latency regarding question asking-answering.  Have Supervisee use a timer/stop watch to record latency
\item Ask Supervisee to record their own latency during a supervision meeting when asked to define a task list item, vs. a concept 
\item Ask Supervisee to measure interresponse time (IRT) by observing someone eating a meal for 5 minutes; have Supervisee record time between swallowing one bit of food and next bite and report the average IRT.
%
\end{enumerate}
%
\subsection{Relevant Literature}
\begin{refsection}
\nocite{cooper2007applied,
        johnston2010strategies,
        thomason2011response,
        worsdell2002duration}
\printbibliography[heading=none]
\end{refsection}
%
\subsection{Related Tasks}
\fouraOne{}\\
\fouraTwo{}\\
\fouraThree{}\\
\fouraFour{}\\
\fouraFive{}\\
\fouraNine{}\\
\fourdTwentyOne{}\\
