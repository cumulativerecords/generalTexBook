\clearpage \section{\fourFKThirtySix{}}
\subsection{Definition} 
Response generalization - ``The extent to which a learner emits untrained responses that are functionally equivalent to the trained target behavior'' (Cooper, Heron, \& Heward, 2007, p. 620).\\
%
``Improvements in behavior are most beneficial when they are long lasting, appear in other appropriate environments, and spill over to other related behaviors... When evaluating applied behavior analysis research, consumers should consider the maintenance and generalization of behavior change in their evaluation of a study'' (Cooper et al., 2007, p. 250).\\
%
\subsection{Examples}
\begin{enumerate}
\item A young child learns to open a door at their house by turning a door knob. One day, while at a friend's house, they encounter a door that has a handle rather than a knob. The child is able to turn the handle and open the unfamiliar door. This is an example of response generalization because functional both responses are equal (they open result in the door being opened) but the response topographies are different.
%
\end{enumerate}
%
\subsection{Assessment}
\begin{enumerate}
\item Have supervisee provide examples of response generalization.
\item Provide examples of both response generalization and stimulus generalization and have supervisee indicate which type of generalization the example is referring to and describe why.
\item Have supervisee describe why response generalization is important when assessing behavior change and skill acquisition.
%
\end{enumerate}
%
\subsection{Relevant Literature}
\begin{refsection}
\nocite{baer1968some,
        cooper2007applied,
        fantuzzo1981generalization,
        goetz1973social,
        sidman1994equivalence}
\printbibliography[heading=none]
\end{refsection}
%
\subsection{Related Tasks}
\fourbOne{}\\
\foureSix{}\\
\foureEleven{}\\
\fouriOne{}\\
\fouriTwo{}\\
\fourjEleven{}\\
\fourjTwelve{}\\
\fourjFourteen{}\\
\fourkNine{}\\
\fourFKTen{}\\
\fourFKEleven{}\\
\fourFKTwelve{}\\
\fourFKThirtySeven{}\\
