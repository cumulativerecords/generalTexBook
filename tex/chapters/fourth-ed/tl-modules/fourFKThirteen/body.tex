\clearpage \section{\fourFKThirteen{}}
\subsection{Definition} 
Unconditioned stimulus (US) - is a ``stimulus change that elicits respondent behavior (i.e., unconditioned response) in the absence of prior learning.  The UR is typically regarded as a built-in bodily mechanism that exists through natural evolution'' (Cooper, Heron, \& Heward, 2007, pp. 30, 39).   The US-UR relation is an unconditioned reflex.
%
\subsection{Examples}
\begin{enumerate}
\item Air blowing in eye (US) $\rightarrow$ blinking (UR)
\item Cold/low temperature (US) $\rightarrow$ shivering (UR)
\item Hot/high temperature (US) $\rightarrow$ sweating (UR)
\item Food in mouth (US) $\rightarrow$ salivation (UR)
\item Hot surface (US) $\rightarrow$ move hand away (UR)
%
\end{enumerate}
%
\subsection{Assessment}
\begin{enumerate}
\item Ask supervisee to provide examples of US-UR relations.
\item Ask supervisee to discriminate between respondent behavior and operant behavior.
%
\end{enumerate}
%
\subsection{Relevant Literature}
\begin{refsection}
\nocite{bijou1961child,
        cooper2007applied,
        pavlov1927conditional}
\printbibliography[heading=none]
\end{refsection}

%
\subsection{Related Tasks}
\fourFKFourteen{}\\
\fourFKFifteen{}\\
\fourFKSixteen{}\\
