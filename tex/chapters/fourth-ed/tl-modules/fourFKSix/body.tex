\clearpage \section{\fourFKSix{}}
\subsection{Definition}
Pragmatism – ``a reasonable and logical way of doing things or of thinking about problems that is based on dealing with specific situations instead of on ideas and theories'' (Merriam-Webster.com, 2015).

ABA is an inclusive approach that is easily replicable for socially significant effects by a variety of individuals that may benefit from its methodology. Jon Bailey (2000, p. 477) stated that ``It seems to me that applied behavior analysis is more relevant than ever before and that it offers our citizens, parents, teachers, and corporate and government leaders advantages that cannot be matched by any other psychological approach...'' 

``Classroom teachers, parents, coaches, workplace supervisors, and sometimes the participants themselves implemented the interventions found effective in many ABA studies. This demonstrates the pragmatic element of ABA. Although doing ABA requires far more than learning to administer a few simple procedures, it is not prohibitively complicated or arduous'' (Cooper et al., 2007, p. 19). 

In other words, the pragmatism of ABA is in its practicality and justification of methods that give it appeal to a wider audience compared to other sciences searching for ``truth.''

\subsection{Examples}
\begin{enumerate}
\item Gloria was looking for a reinforcement program for her classroom because her students were not turning their homework in on time. She consulted with the district BCBA and was able to come up with an effective and simple class-wide reinforcement program that helped her students to turn their homework on time.
\item (Non-example) Richard was a first grade teacher and wanted to represent his student's data using a scatterplot graph. However, he did not have previous training in this area and was unable to accomplish this task. He felt that this method for graphical display was too difficult to figure out.
%
\end{enumerate}
%
\subsection{Assessment}
\begin{enumerate}
\item Have supervisee explain reinforcement, punishment, mand, and tact in simple, pragmatic terms that a layperson could apply. 
\item Have supervisee identify and describe some common ABA practices and techniques that are used by professionals who have not been directly trained in ABA. Have him/her describe why these approaches represent the pragmatic nature of applied behavior analysis.  
%
\end{enumerate}
%
\subsection{Relevant Literature}
\begin{refsection}
\nocite{bailey2000futurist,
        cooper2007applied,
        heward2005focus,}
\printbibliography[heading=none]
\end{refsection}
%
\subsection{Related Tasks}
\fourbOne{}\\
\fourgFour{}\\
\fourgSix{}\\
\fourkEight{}\\
\fourkNine{}\\
