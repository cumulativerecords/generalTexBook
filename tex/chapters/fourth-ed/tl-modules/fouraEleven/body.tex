\clearpage \section[\fouraEleven{}]{\fouraEleven{}%
              \sectionmark{A-11 Design... cumulative record}}
\sectionmark{A-11 Design... cumulative record}
\subsection{Definition}
Cumulative record – recording method that involves ``the number of responses recorded during each observation period is added to the total number of responses recorded during previous observation periods'' (Cooper, Heron \& Heward, 2007, p. 138). The value on the y-axis represents the cumulative number of responses recorded and the value on the x-axis represents time (i.e., observation periods).  Once the response rate exceeds the maximum value on the y-axis, the curve resets to zero and begins again.  Cumulative records display the overall response rate and visually depict the learner's rate of acquisition for a series of behavior targets (e.g., total number of skills mastered throughout services, number of sight words learned).  Data are interpreted on a cumulative record by analyzing the slope in which the steeper the slope, the higher the response rate.
%
\subsection{Examples}
\begin{enumerate}
\item The cumulative record below indicates the number of attributes learned by a first grade student.  The overall response rate is 13 attributes across 181 sessions.  In general, data in this graph suggests that there was a fairly slow rate of acquisition. However, the slope is much steeper between sessions 1 and 61, indicating that the rate of acquisition was quicker during the first part of the intervention. 
\end{enumerate}
%
\subsection{Assessment}
\begin{enumerate}
\item Ask supervisee to identify behavioral targets that would be appropriate to graph in a cumulative record
\item Ask supervisee to create a cumulative record graph.
\item Show supervisee various examples of cumulative records and ask supervisee to interpret the data.
\end{enumerate}
%
\subsection{Relevant Literature}
\begin{refsection}
\nocite{cooper2007applied,ferster1957schedules}
\printbibliography[heading=none]
\end{refsection}
%Cooper, J. O., Heron, T. E., \& Heward, W. L. (2007). Constructing and interpreting graphic displays of behavioral data. Applied Behavior Analysis (pp. 126-157). Upper Saddle River, NJ: Pearson Prentice Hall.
%Ferster, C. B., \& Skinner, B. F. (1957). Schedules of reinforcement. New York, NY: Appleton-Century-Crofts.
% 
\subsection{Related Tasks}
\fouraTen{}\\
\fourhOne{}\\
\fourhTwo{}\\
\fourhThree{}\\
\fourhFour{}\\
\fourhFive{}\\
