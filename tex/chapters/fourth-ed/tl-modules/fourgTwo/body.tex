\clearpage \section[\fourgTwo{}]{\fourgTwo{}%
              \sectionmark{G-02 Consider biological/med...}}
\sectionmark{G-02 Consider biological/med...}
\subsection{Definition}
The Behavior Analyst Certification Board (BACB) instructs that a ``behavior analyst recommends seeking a medical consultation if there is any reasonable possibility that a referred behavior influenced by medical or biological variables'' in section 3.02 Medical Consultation, of the BACB professional and ethical compliance code for behavior analysts (2014)

This is relevant both in research and practice. Therefore, the first step in the assessment process should be to determine whether the problem may be due to a medical/biological issue and whether a medical evaluation has been completed (Cooper, Heron \& Heward, 2007). Failure to rule out medical needs would be unethical as it would delay potentially necessary medical treatment that may even prove life threatening dependent on the medical concerns or the severity of the challenging behavior.

Possible pain related disorders or other medical/biological disorders that restrict an individual's ability to engage in appropriate behavior should be investigated.  Some relevant behavioral topics correlated with a high likelihood of medical and biological causes are feeding disorders, toileting challenges (e.g., encopresis and incontinence), sleep problems and self-injury. Take self-injury for example; studies have shown that self-injurious behavior (SIB) has been maintained by pain attenuation which, can be categorized as automatic negative reinforcement behavior (Carr \& Smith, 1995; O'Reilly, 1997).  In detail, an increase in painful stimulation is an establishing operation (EO), thereby increasing behavior that has been reinforced by pain reduction.

Aside from the common examples presented above, it is possible that any form of challenging behavior could be a result of an underlying medical or biological issue. For example, aggressive behavior may also be related to pain related disorders which act as an EO (Carr et al., 2003; Skinner, 1953). The argument has also been made that aggressive behavior in response to painful stimulation may be respondent behavior (Ulrich \& Azrin, 1962). Furthermore, Kennedy and Meyer (1996) found that the occurrence of allergy symptoms and sleep deprivation were correlated with an increase in escape maintained challenging behavior. 

\subsection{Assessment}
\begin{enumerate}
\item Ask the supervisee what the first part of assessment should be regarding specific situations that would require collaboration with medical professionals to rule out any underlying medical issues. Have them give a rationale as to why this is important. 
\item Ask the supervisee to list a few possible medical/biological considerations that should be ruled out when treating a feeding disorder, toileting issue or sleep problem.
\item Provide supervisee with examples such as this one: A student in a school setting is engaging in severe tooth picking and the classroom teacher is calling you for advice on what to do. Then ask the supervisee what the first step of assessment should be to treating this challenging behavior. 
%
\end{enumerate}
%
\subsection{Relevant Literature}
\begin{refsection}
\nocite{bac2014professional,
        carr2003menstrual,
        carr1995biological,
        cooper2007applied,
        kennedy1996sleep,
        o1997functional,
        skinner1953science,
        ulrich1962reflexive}
\printbibliography[heading=none]
\end{refsection}
%
\subsection{Related Tasks} 
\fourgOne{}\\
\fourgThree{}\\
\fourgSix{}\\
\fourgSeven{}\\
\fourFKThirteen{}\\
\fourFKTwentySix{}\\
