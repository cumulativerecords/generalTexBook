\clearpage \section{\fourbNine{}}
\subsection{Definition}
When designing an experiment it is sometimes useful to combine experimental design elements to strengthen the demonstration of experimental control. For instance it may be valuable to combine a multiple baseline design with a reversal design. 

For example, Colón et al. (2012) used a nonconcurrent multiple baseline design across participants to analyze the effects of verbal operant training on appropriate vocalizations and vocal stereotypy. RIRD was implemented and examined using a reversal design for each participant exposed to their procedure.

In 1985, Alexander used a multiple baseline across students with reversal design to evaluate the effects of a study skill training procedure.

Johnston and Pennypacker (2009) point out that experimenters often combine and intermingle many different types of designs as necessary. Categorizing types of designs is really a more valuable thing for the student than it is for the researcher. 

Murray Sidman in Tactics of Scientific Research says that it is not valuable to say that there are rules to follow when designing an experiment. He says ``this would be disastrous'' (Sidman, 1960/1988, p. 214). Simply put, he says, ``The fact is that there are no rules of experimental design'' (Sidman, 1960/1988, p. 214).

The most important thing is that the experiment is designed to answer some question we have about the natural world. Sidman says, ``We conduct experiments to find out something we do not know'' (Sidman, 1960/1988, p.214).
%
\subsection{Assessment}
\begin{enumerate}
\item Have your supervisee design an experiment that would be best suited to use a combination of design elements.
\item Ask your supervisee to point to the sections of the graphs in Colon et al. (2012) and Alexander (1985) that reflect the types of experimental designs used. 
\item Ask your supervisee to describe what Sidman meant when he said that having rules for designing an experiment would be ``disastrous.''
\end{enumerate}
%
\subsection{Relevant Literature}
\begin{refsection}
\nocite{alexander1985effect,colon2012effects,iwata2000skill,johnston2010strategies}
\printbibliography[heading=none]
\end{refsection}
%                                                                   
\subsection{Related Tasks}
\fourbThree{}\\
\fourbSeven{}\\
\fourhOne{}\\
\fouriFive{}\\
