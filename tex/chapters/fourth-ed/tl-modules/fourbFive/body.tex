\clearpage \section[\fourbFive{}]{\fourbFive{}%
              \sectionmark{B-05 Use alternating treatments...}}
\sectionmark{B-05 Use alternating treatments...}
\subsection{Definition} 
Alternating treatments design - ``An experimental design in which two or more conditions (one of which may be a no treatment control condition) are presented in rapidly alternating succession (e.g., on alternating sessions or days) independent of the level of responding; differences in responding between or among conditions are attributed to the effects of the conditions (also called concurrent schedule design, multielement design, multiple schedule design)'' (Cooper, Heron, \& Heward, 2007, p. 689).
%
\subsection{Examples}
\begin{enumerate}
\item An experiment that entails conducting DRA and no programmed treatment during alternating sessions to compare treatment to the no treatment. 
\item An experiment that entails conducting DRI, DRO and DRA on alternating days to compare all treatments to each other. 
\end{enumerate}
%
\subsection{Assessment}
\begin{enumerate}
\item Ask the supervisee to either describe an alternating treatment design they have used in the past or have them describe an alternating treatments design.
\item Have supervisee look at various figures  from the articles (such as some of those below) as well as articles that did not use an alternating treatment design and have them determine which figures depict the use of an alternating treatment design. 
\item Have the supervisee look at figures in the articles below and describe what characteristics make it an alternating treatment design
\item Have the supervisee describe the pros of using an alternating treatment design and the condition in which the use of an alternating treatment design would not be desirable
\end{enumerate}
%
\subsection{Relevant Literature}
\begin{refsection}
\nocite{barbetta1993effects,
    barlow1979alternating,
    cooper2007applied,
    iwata1994toward,
    martens1992effects,
    singh1985comparison,
    ulman1975multielement}
\printbibliography[heading=none]
\end{refsection}
%
\subsection{Related Tasks} 
\fourbThree{}\\ 
\fourjNine{}\\
