\clearpage \section[\foureFive{}]{\foureFive{}%
              \sectionmark{E-05 Use... group contingencies.}}
\sectionmark{E-05 Use... group contingencies.}
\subsection{Definition}
The three group contingencies as defined by Cooper, Heron, \& Heward (2007):\\
 ``An independent group contingency is an arrangement in which a contingency is presented to all members of a group, but reinforcement is delivered only to those group members who meet the criterion outlined in the contingency'' (Cooper et al., 2007, p. 568).

``An interdependent group contingency is one in which all members of a group must meet the criterion of the contingency (individually and as a group) before any member earns the reward'' (Cooper et al., 2007, p. 569).

``Under a dependent group contingency the reward for the whole group is dependent on the performance of an individual student or small group'' (Cooper et al., 2007, p. 568).

These three contingencies use the principles of reinforcement to change the behavior of a group of individuals. They involve ``a common consequence (usually, but not necessarily, a reward intended to function as reinforcement) contingent on the behavior of one member of the group, the behavior of part of the group, or the behavior of everyone in the group'' (Cooper et al., 2007, p. 567).
%
\subsection{Examples}
\begin{enumerate}
\item Independent Group Contingency: Each student was given a math worksheet and received a special sticker if they completed the work without engaging in disruptive behavior. Billy, Johnny, and Sam finished their work quietly and earned stickers, but Danny was disruptive and only finished half his worksheet so he did not earn a sticker.
\item Interdependent Group Contingency: Each student in Mrs. Kelly's class had to complete their math worksheets before they were allowed to go outside for recess. The students who finished first were allowed to help any of the struggling students. Sam was the last one working and Roger came over to help him complete his work. Once Sam was finished the whole class earned recess time.
\item Dependent Group Contingency: At the end of football practice, Mr. Bill told the team that they could stop running wind sprints if Roger caught a long pass from the coach. Roger caught that pass and the team cheered as they went to shower and practice ended.
\end{enumerate}
%
\subsection{Assessment}
\begin{enumerate}
\item Have supervisee identify and describe group contingencies he/she has encountered in his/her professional career.
\item Give examples of various contingencies and ask supervisees to identify which of the three group contingencies is exhibited in each example.
\item Have supervisee choose a group contingency and create guidelines for a program to change the behavior of a group using the designated contingency.
\end{enumerate}
%
\subsection{Relevant Literature}
\begin{refsection}
\nocite{cooper2007applied,
        kamps2011class,
        litoe1975brief,
        nevin1982effects,
        theodore2001randomization}
\printbibliography[heading=none]
\end{refsection}
%
%
\subsection{Related Tasks}
\fourcOne{}\\
\fourdOne{}\\
\fourdTwo{}\\
\fourdNineteen{}\\
\fourdTwentyOne{}\\
\foureFour{}\\
\fourfTwo{}\\
