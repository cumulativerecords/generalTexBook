\clearpage \section[\fourhOne{}]{\fourhOne{}%
              \sectionmark{H-01 Select a measurement sys...}}
\sectionmark{H-01 Select a measurement sys...}
\subsection{Definition}
Assessment and treatment decisions of behavior analysts rely on data. A behavior analyst can design and implement effective treatments only when the data accurately represent the behavior of interest (validity) and have been reliably recorded as they are observed. To accomplish this goal, behavior analysts choose behaviors and dimensions of those behaviors to facilitate accurate and consistent recording within a given context for each client. Behavior analysts measure ``three fundamental properties, or dimensional quantities'' (Cooper et al., 2007, p. 75) of behavior: 
\begin{enumerate}
\item Repeatability--Behavior can be counted in the same way each time it occurs.
\item Temporal extent--Behavior can be measured in relation to time.
\item Temporal locus—Behavior occurs in relation to other behaviors.
\end{enumerate}

First, identify which of these three properties will provide the most accurate method for quantifying behavior. Then, decides what dimension of behavior to measure, such as a count of occurrences, frequency of behavior per unit of time, duration, latency, or other. Last, decide who will record the behavior and in what context. If a preferred measurement system is unlikely to be effectively implemented, then the analyst has to reconsider definitions or recording circumstances in order to obtain adequate and accurate data for decision-making. 
%
\subsection{Examples}
\begin{enumerate}
\item A teacher is asked to record the number of times a student yells out inappropriate language in class each day. Since the time the student spends in class varies widely during the week, the behavior analyst designed a measurement system to record rate (number of occurrences divided by time in class) of yelling out inappropriate language. With these data, the behavior analyst can compare the student's behavior across time of unequal durations. 
\item A parent is committed to accurately recording data to show duration of a child's tantrums, but found that she was not always in close proximity with the child when the behavior begins. The behavior analyst identifies two conditions during the day when the mother can accurately record duration of each tantrum (the first 30 minutes after school and the last 30 minutes before bedtime). The mother records duration of total tantruming behavior during each 30 mimute-observation twice a day.
\end{enumerate}
%
\subsection{Assessment}
\begin{enumerate}
\item Have supervisee list and describe each measurable unit of behavior.
\item Have supervisee list and describe each of the three fundamental properties of behavior.
\item Provide the supervisee with a number of scenarios. Have the supervisee design/describe a data collection procedure to measure target behavior.
%
\end{enumerate}
%
\subsection{Relevant Literature}
\begin{refsection}
\nocite{cooper2007applied,
        ledford2009single}
\printbibliography[heading=none]
\end{refsection}
%
\subsection{Related Tasks}
\fouraOne{}\\
\fouraTwo{}\\
\fouraThree{}\\
\fouraFour{}\\
\fouraNine{}\\ 
\fouraTen{}\\
\fourdFive{}\\
\fourhTwo{}\\
\fourhThree{}\\
\fourhFour{}\\
\fourhFive{}\\
\fouriOne{}\\
\fouriFive{}\\
\fourkSeven{}\\
\fourFKThirtyThree{}\\
\fourFKFourtyOne{}\\
\fourFKFourtySeven{}\\
\fourFKFourtyEight{}\\
