\clearpage \section{\fourFKThirtyThree{}}
\subsection{Definition}
Functional relation - ``An experimentally determined relation that shows that the dependent variable depends on or is a function of the independent variable and nothing else'' (Johnston \& Pennypacker, 2009, p. 358).\\

``A `cause' becomes a ‘change in an independent variable' and an `effect' a `change in a dependent variable.' The old ‘cause-and-effect connection' becomes a ‘functional relation.' The new terms do not suggest how a cause causes its effect; they merely assert that different events tend to occur together in a certain order'' (Skinner, 1953, p.23).\\

For every response are a number of factors that influence the likelihood that it occurs. Each one of these factors can be used as an independent measure in an experiment. If an experiment shows that there is a different between a context in which this variable is present vs when this variable is absent, we consider it to be a functional relation (Cooper, Heron, \& Heward, 2007).\\

For behavior analysts, functional relations are important to discover. When we understand how a behavior is related to the environment, we can then decide what treatment to use.\\
%
\subsection{Examples}
\begin{enumerate}
\item Sonny has been engaging in some eloping behavior within a school building.  He has been known to leave his classroom area and to run to other rooms within the building. His teachers have started delivering small pieces of candy for on-task behavior, as he works on his schoolwork. The teachers have started seeing an increase in on-task behavior and a decrease in elopement. A substitute teacher came into the classroom for a week but did not know about the on-task candy delivery the first couple of days.   Sonny started eloping again. When the aides told the substitute teacher about the contingency for on-task behavior, the substitute started delivering the candy on the same schedule as the other teacher. Sonny again started sitting down, remaining on task, and elopement decreased. It can be said that there is likely a functional relation between the schedule of candy delivery and the elopement and/or on-task behavior.
\item Three semi-busy 3-way intersections in the small town of Passamaquoddy has had a series of accidents over the past few years.  These intersections have had yield signs up but there have been several accidents at each location.  The town decides to replace the yield signs with 3 stop signs instead. They use a multiple baseline design across locations. After seeing that when, and only when, the new stop signs are implemented, accidents have decreased in that location. It can be said that there is a functional relation between the placement of the stop signs and the change in accidents reported there.  
%
\item (Non-example) A child with autism has been engaging in some eloping behavior within a school building.  He has been known to leave his classroom area and to run to other rooms within the building. His teachers have explained to him that the other rooms are off limits but this has not had an impact on his behavior nor has simply ensuring that the doors are closed. His teacher decides to put up a green light on rooms that it is o.k. to enter.  There has been no change in behavior from the previously recorded levels of entering the off-limits classrooms.  It can be said that there is no functional relation between the presence of green light signs and the off-limits classroom entering behavior.
%
\end{enumerate}
%
\subsection{Assessment}
\begin{enumerate}
\item Ask your Supervisee to give a definition for functional relation.
\item Ask your supervisee to create other examples and a non-example of his/her own. 
\item Ask the Supervisee why it is important to only manipulate one variable at a time
\item Ask the Supervisee to state how you would know if a functional relation exists between the independent variable and the dependent variable. 
%
\end{enumerate}
%
\subsection{Relevant Literature}
\begin{refsection}
\nocite{cooper2007applied,
        johnston2010strategies,
        skinner1953science}
\printbibliography[heading=none]
\end{refsection}
%
\subsection{Related Tasks}
\fourbThree{}\\
\fourFKThirtyThree{}\\
\fourhThree{}\\
\fourhFive{}\\
\fouriFive{}\\
