\clearpage \section{\fourkSix{}}
Smith, Parker, Taubman, and Lovaas (1992) found that knowledge transfer from a staff training workshop did not generalize to the group home.  Research suggests that training is most effective if there is training and ongoing supervision in the environment where the behavior change program is occurring (Parsons \& Reed, 1995; Sarokoff \& Sturmey, 2004; Miles \& Wilder, 2009).

Codding, Feinberg, Dunn and Pace (2004) found that treatment integrity increased following a one hour performance feedback session every other week.  Social validity ratings provided favorable feedback for the frequent supervisions. 

Cooper, Heron and Heward (2007) suggest that supervision of data takers and booster training is necessary to avoid observer drift. 

Observing and graphing data will provide immediate feedback on the participant's performance.  This can lead to quick decisions, modifications if necessary, or the termination of ineffective programs.  This supervision is necessary to create the most effective interventions and troubleshoot areas for improvement. It is also important to observe the intervention in the natural environment in order to determine if the intervention is realistic and practical in the natural environment.

Summary of Rationale for Supervision of Behavior Change Agents
\begin{enumerate}
\item Provides effective knowledge transfer for individuals implementing the intervention
\item Increases motivation and treatment integrity
\item Reduces error associated with data collection
\item Allows for quick clinical decisions to modify or terminate programs which ensures the most effective treatment
\end{enumerate}
%
\subsection{Assessment}
\begin{enumerate}
\item Ask supervisee to describe why it is important to provide supervision of behavior change agents.
\item Look at data with supervisee and ask them to discuss what modification they may consider
\item Have supervisee observe intervention in the home environment and ask them to write down suggestions to increase the effectiveness of the plan.
%
\end{enumerate}
%
\subsection{Relevant Literature}
\begin{refsection}
\nocite{codding2005effects,
        cooper2007applied,
        miles2009effects,
        parsons1995training,
        sarokoff2004effects,
        smith1992transfer}
\printbibliography[heading=none]
\end{refsection} 
% 
\subsection{Related Tasks}
\fourhThree{}\\
\fourhFour{}\\
\fourkTwo{}\\
\fourkThree{}\\
\fourkFour{}\\
