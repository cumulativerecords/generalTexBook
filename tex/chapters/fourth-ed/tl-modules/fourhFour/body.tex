\clearpage \section[\fourhFour{}]{\fourhFour{}%
              \sectionmark{H-04 Evaluate changes in...}}
\sectionmark{H-04 Evaluate changes in...}
\subsection{Definition}
Level - ``The value on the vertical axis scale around which a set of behavioral measures converge is called level. In the visual analysis of behavioral data, level is examined within a condition in terms of its absolute value (mean, median, and/or range) on the y-axis scale, the degree of stability or variability, and the extent of change from one level to another... The mean level of a series of behavioral measures within a condition can be graphically illustrated by the addition of a mean level line: a horizontal line drawn through a series of data points within a condition at that point on the vertical axis equaling the average value of the series of measures'' (Cooper, Heron \& Heward, 2007, pp. 150-151). 

Trend - ``The overall direction taken by a data path is its trend'' (Cooper, Heron \& Heward, 2007, p. 151). Trends can be described in terms of their direction, i.e., ascending, descending or zero/no trend. They can also be described in terms of their degree or magnitude, and the extent of variability, the data points around the trend have. The direction and degree of trend shown in a series of data points can be visually represented on a graph by drawing a straight line through the data. This is called a trend line or line of progress (Cooper, Heron \& Heward, 2007).

Variability - ``How often and the extent to which multiple measures of behavior yield different outcomes is called variability'' (Cooper, Heron \& Heward, 2007, p. 150).
%
\subsection{Assessment}
\begin{enumerate}
\item Ask your Supervisees to define level/trend/variability.
\item Ask your Supervisee to graph a set of data and describe the level/trend/variability shown in the data.
\end{enumerate}
%
\subsection{Relevant Literature}
\begin{refsection}
\nocite{cooper2007applied,
        keohane2005teachers,
        lindsley1985quantified,
        mccain1979statistical,
        white2005trend}
\printbibliography[heading=none]
\end{refsection}
%
\subsection{Related Tasks}
\fouraTen{}\\
\fouraEleven{}\\
\fourhThree{}\\
\fourhFour{}\\
\fouriOne{}\\
\fouriFive{}\\
\fourFKFourtySeven{}\\%
