\clearpage \section{\foureThirteen{}}
\subsection{Definition} 
Matching-to-sample - ``A procedure for investigating conditional relations and stimulus equivalence. A matching-to-sample trial begins with the participant making a response that presents or reveals the sample stimulus; next, the sample stimulus may or may not be removed, and two or more comparison stimuli are presented. The participant then selects one the comparison stimuli. Responses that select a comparison stimulus that matches the sample stimulus are reinforced, and no reinforcement is provided for responses selecting the nonmatching comparison stimuli'' (Cooper, Heron, \& Heward, 2007, p. 699).

\subsection{Examples}
\begin{enumerate}
\item A teacher presents a student with a picture of an apple. The teacher then lays out three other picture cards away from the original picture of an apple. One picture card depicts an apple, the second picture card depicts a banana, and the final picture card depicts an orange. The teacher holds up the initial picture card depicting an apple and states, ``match.''  The student takes the picture of an apple and places it on top of the corresponding picture of an apple. The teacher says, ``Great job'' and gives the student a high five.
\end{enumerate}
%
\subsection{Assessment}
\begin{enumerate}
\item Have supervisee create a mock lesson displaying match-to-sample task. He/she must create a match-to-sample program, a data sheet to record responses, and task materials needed to complete the task. Have him/her role-play this task scenario with supervisor. Supervisor will play the role of the client and supervisee will play the role of the teacher or therapist.
\item Have supervisee create a lesson that demonstrates stimulus equivalence. He/she must start with match-to-sample task and then use other topographies of sample stimulus to display stimulus equivalence. 
\item Have supervisee describe the match-to-sample procedure and stimulus equivalence. Have him/her discuss how match-to-sample procedures can be implemented to test for stimulus equivalence. 
\end{enumerate}
%
\subsection{Relevant Literature}
\begin{refsection}
\nocite{cooper2007applied,
        cumming1965complex,
        fields2010varieties,
        sidman1986matching}
\printbibliography[heading=none]
\end{refsection}
%
\subsection{Related Tasks}
\fourdThree{}\\
\fourdEight{}\\
\foureTwo{}\\
\foureSix{}\\
\foureTwelve{}\\
