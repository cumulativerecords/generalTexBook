\clearpage \section[\fourhThree{}]{\fourhThree{}%
              \sectionmark{H-03 Select a data display...}}
\sectionmark{H-03 Select a data display...}
\subsection{Definition}
Behavior change is an ongoing process that must be continuously evaluated.  This evaluation occurs through an analysis of data that reflects the quantifiable form of the behavior of interest.  However, understanding the extent of behavior change can be difficult if one is looking at raw data alone.  As such, behavior analysts use graphic displays to analyze, interpret, and communicate the results of behavior interventions (Cooper, Heron, \& Heward, 2007).  The most common graphic displays include line graphs, bar graphs, cumulative records, Standard Celeration charts, and scatterplots.  The clinical utility of each graphic display varies so it is important to select the graphic display that will most accurately illustrate what the behavior analyst wants to understand. 

Cooper et al. (2007) outlines the following purposes for the different graphic displays.  

Line graphs are the most common form of graphic display and can be used to (1) show multiple dimensions of one behavior, (2) two or more different behaviors, (3) a behavior under different conditions, (4) changes in the target behavior relative to the manipulation of an independent variable, (5) and the behavior of multiple learners. 

Bar graphs are typically used to (1) display discrete data that cannot be captured by an underlying dimension reflected on a horizontal axis and (2) provide an easy comparison of variables during different conditions. 

Cumulative records are useful when the behavior analyst wants to (1) illustrate the total number of responses made over time, (2) the graph is used as a means to provide feedback to the learner, (3) the behavior of interest can only occur once during the specified measurement period, and (4) an analysis of a specific instance during an experiment is warranted.

The Standard Celeration Chart is a semilogarithmic chart that is used to reflect a linear measure of change across time.  Lastly, scatterplots illustrate the comparative distribution of discrete measures in a data set and can be useful to uncover relationships across different subsets of data.
%
\subsection{Examples}
\begin{enumerate}
\item If you want to see data paths across three behaviors and different intervention conditions, then a line graph is the most appropriate graphic display.
\item Your client has rapidly acquired several language targets.  A cumulative record can illustrate acquisition and is more efficient than creating a line graph for each acquired language target.
%
\end{enumerate}
%
\subsection{Assessment}
\begin{enumerate}
\item Ask supervisee to explain the utility of each type of graphic display.
\item Provide supervisee with examples of graphs and ask supervisee to identify which type of display is used and interpret the data presented in the display.
\item Ask supervisee to create at least one of each type of graphic display.
\item Provide examples of different measures of behavior and what information is desired from each set of data and ask supervisee to identify which graphic display would be most appropriate to communicate the results.
%
\end{enumerate}
%
\subsection{Relevant Literature}
\begin{refsection}
\nocite{cooper2007applied,
        parsonson1978analysis}
\printbibliography[heading=none]
\end{refsection}
%
\subsection{Related Tasks}
\fouraTen{}\\
\fouraEleven{}\\
\fouraTwelve{}\\
\fourFKFourtySeven{}\\
\fourbThree{}\\
\fourhFour{}\\
\fourhFive{}\\
