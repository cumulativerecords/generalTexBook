\clearpage \section{\foureTen{}}
\subsection{Definition} 
The Premack principle - ''...a principle of reinforcement which states that an opportunity to engage in more probable behaviors (or activities) will reinforce less probable behaviors (or activities)'' (Volkmar, 2013, p. 2345).

``For example, if a child often plays computer games (more probable) and avoids completing math problems (less probable), we might allow her to play the computer after (contingent upon) completing 15 math problems. Prior to the introduction of the Premack principle, systems of reinforcement were viewed as the contingency between a stimulus and behavior. The Premack principle expanded the existing reinforcement contingency of stimulus behavior to include contingencies between two behaviors. This principle is often referred to as ‘grandma's rule' because grandmothers (or any caregivers) often apply this principle: ‘you have to eat your vegetables (less probable) before you can have dessert (more probable)''' (Volkmar, 2013, p. 2345).

In education, the Premack principle is the basis for ``first/then'' strategies. ``First/then'' strategies consist of a teacher telling a student ``First X, then Y'' with X being a less preferred activity or task demand and Y being a more preferred activity contingent on the completion of X.

Premack principles use preferred activities as reinforcers to help increase engagement in less preferred activities or demands. 

\subsection{Examples}
\begin{enumerate}
\item A father tells his teenage son, ``When you have finished washing the dishes, you can watch TV.''
\end{enumerate}
%
\subsection{Assessment}
\begin{enumerate}
\item Have supervisee give examples of Premack's principle in his/her daily life.
\item Have supervisee create a role play scenario in which he/she demonstrates the use of the Premack principle.
\item Have supervisee find an article on the use of the Premack principle, summarize, and discuss benefits and limitations of use.
\end{enumerate}
%
\subsection{Relevant Literature}
\begin{refsection}
\nocite{azrin2007physical,
        volkmar2013encyclopedia,
        cooper2007applied,
        mazur1975matching,
        sigafoos2005premack,
        welsh1993application}
\printbibliography[heading=none]
\end{refsection}
%
\subsection{Related Tasks}
\fourdOne{}\\
\fourdTwo{}\\
\fouriSeven{}\\
