\clearpage \section[\fourbTen{}]{\fourbTen{}%
              \sectionmark{B-10 Conduct a component...}}
\sectionmark{B-10 Conduct a component...}
\subsection{Definition}
Component analysis - ``An experiment designed to identify the active elements of a treatment condition, the relative contributions of different variables in a treatment package, and/or the necessary and sufficient components of an intervention. Component analyses take many forms, but the basic strategy is to compare levels of responding across successive phases in which the intervention is implemented with one or more of the components left out'' (Cooper, Heron, \& Heward, 2007, p. 692).
%
\subsection{Examples}
\begin{enumerate}
\item An experiment that compares response blocking with and without redirection. 
\item An experiment that compares differential reinforcement of alternative behavior with and without extinction
\item An experiment that removes one procedure at a time from a treatment package consisting of three procedures (e.g., Token system, response cost and extinction) and compares responding under each condition to responding during implementation of the full treatment package. 
\end{enumerate}
%
\subsection{Assessment}
\begin{enumerate}
\item Ask Supervisee to explain why component analysis research is important to clinical practice (e.g. efficiency)
\item Ask the supervisee to either describe a treatment package they use in their fieldwork or practicum that could that could be tested via a component analysis or have them describe a hypothetical treatment package that could be tested via a component analysis 
\item Have supervisee look at the figures in the articles listed below and explain the findings of the analysis by determining the effective components of the intervention package.
\end{enumerate}
%
\subsection{Relevant Literature}
\begin{refsection}
\nocite{cooper2007applied,hardesty2014effects,ward2010component,ward2012component}
\printbibliography[heading=none]
\end{refsection}
%
\subsection{Related Tasks} 
\fourbThree{}\\
