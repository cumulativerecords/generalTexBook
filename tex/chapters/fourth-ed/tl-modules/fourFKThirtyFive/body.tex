\clearpage \section{\fourFKThirtyFive{}}
\subsection{Definition} 
Stimulus discrimination – ``is when a response consistently occurs in the presence of a specific or controlling antecedent stimulus and not in the presence of new or related stimuli.  This is in direct contrast to stimulus generalization in which related antecedent stimuli may evoke the same response'' (Cooper, Heron, \& Heward, 2007, pp. 395-396).    
%
\subsection{Examples}
\begin{enumerate}
\item During a tooth brushing routine, a child selects their tooth brush from the cup that holds a number of toothbrushes.
\item You own a small white car and can walk directly to your car after leaving the mall even though there are several small white cars parked around your car.
\item You are taking three graduate level ABA classes.  Two of the professors keep track of attendance and incorporate that into your final grade and one professor does not track attendance or incorporate that into your final grade.  As a result, you periodically miss this class since that your grade will not be lowered due to attendance and you regularly attend the other two classes. 
%
\end{enumerate}
%
\subsection{Assessment}
\begin{enumerate}
\item Ask supervisee to identify natural examples of stimulus discrimination.
\item Ask supervisee to compare and contrast stimulus discrimination and stimulus generalization.
\item Observe supervisee use stimulus discrimination procedures with clients.
%
\end{enumerate}
%
\subsection{Relevant Literature}
\begin{refsection}
\nocite{cooper2007applied,
        green2001behavior}
\printbibliography[heading=none]
\end{refsection}
%
\subsection{Related Tasks}
\foureSix{}\\
\fourFKEleven{}\\
\fourFKTwentyFour{}\\
\fourFKThirtyFour{}\\
\fourFKThirtySeven{}\\
