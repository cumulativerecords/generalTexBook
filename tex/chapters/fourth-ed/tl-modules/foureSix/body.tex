\clearpage \section{\foureSix{}}
\subsection{Definition}
In the field of applied behavior analysis, a number of procedures have been used to teach new concepts.  One of these procedures is known as stimulus equivalence.  In 1971, Murray Sidman discovered that a previously untaught, unreinforced stimuli could come under stimulus control through its pairing with other stimuli which were explicitly taught (Sidman, 1971). This concept revolutionized the field as it demonstrated a new way of teaching that could potentially reduce the amount of time needed to teach a new class of stimuli.   ``Behavior analysts define stimulus equivalence by testing stimulus-stimulus relations.  A positive demonstration of all three behavioral tests (i.e. reflexivity, symmetry, and transitivity) is necessary to meet the definition of an equivalence relation among a set of arbitrary stimuli'' (Cooper, Heron, \& Heward, 2007, p. 398). 
\begin{enumerate}
\item Reflexivity describes the action of selecting a stimulus that is matched to itself in the absence of training and reinforcement (A=A).  For instance an individual is shown three pictures; a penny, a nickel, and a dime.  When given an identical picture of a penny, he matches it to the identical picture of a penny in the array (Sidman, 1994).
\item Symmetry describes the reversibility of the sample stimulus and a comparison stimulus (A=B and B=A).  For instance an individual who is taught to select the picture of a penny (out of an array of 3), when the word penny is given, would also be able to choose the comparison spoken word penny shown the picture of the penny without being previously taught this correlation (Sidman, 1994).  
\item Transitivity is the most crucial test for demonstrating stimulus equivalence.  A third, untrained relation emerges as a result of being taught the first two relations.  (A=C and C=A) ``...emerges as a product of training two other stimulus-stimulus relations'' (Cooper, Heron, \& Heward, 2007, p. 399).  
\end{enumerate}

The following equation demonstrates the basic principals of stimulus equivalence:
\begin{enumerate}
\item If A = B, and
\item B = C, then
\item  A = C (Sidman and Tailby, 1982).
\end{enumerate}

When using stimulus equivalence, decide what relations are to be taught (i.e. spoken word to picture, picture to written word, drawing to real-life picture, etc.).  Decide which order the conditional relations are to be taught.  Teach the relations A=B and B=C to mastery criteria.  Once mastery criteria are met for the first two relations, test for reflexivity, symmetry, and transitivity using the same criteria.  If the participant demonstrates these relations without having been previously been taught them, they will have acquired the third relation C=A that demonstrates the most important test for stimulus equivalence.   

\subsection{Assessment}
\begin{enumerate}
\item Ask the supervisee to explain the concept of stimulus equivalence.
\item Ask the supervisee to name the three tests that demonstrate the basic principles of stimulus equivalence and describe each of these.
\item Ask the supervisee to give examples of some new concepts that might be taught through stimulus equivalence
\end{enumerate}
%
\subsection{Relevant Literature}
\begin{refsection}
\nocite{cooper2007applied,
        sidman1971reading,
        sidman1994equivalence,
        sidman1982conditional}
\printbibliography[heading=none]
\end{refsection}
%
\subsection{Related Tasks}
\foureSix{}\\
\foureThirteen{}\\
\fourFKEleven{}\\
\fourFKTwelve{}\\
\fourFKThirteen{}\\
\fourFKTwentyFour{}\\
\fourFKTwentyEight{}\\
\fourFKThirtyFive{}\\
