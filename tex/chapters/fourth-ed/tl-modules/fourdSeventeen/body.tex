\clearpage \section[\fourdSeventeen{}]{\fourdSeventeen{}%
              \sectionmark{D-17 Use appr... punishment.}}
\subsection{Definition}
Punishment - ``Occurs when stimulus change immediately follows a response and decreases the future frequency of that type of behavior in similar conditions'' (Cooper, Heron, \& Heward, 2007, p. 702).*

Legislation and agency policies limit the use of punishment. Lerman and Vorndran (2002) suggested that punishment may be considered if:
\begin{enumerate}
\item The challenging behavior produces serious physical harm and has to be suppressed quickly
\item Reinforcement based treatments have not reduced the problem behavior to socially acceptable levels or 
\item The reinforcer maintaining the challenging behavior cannot be identified or withheld 
\end{enumerate}

\noindent BACB Labels specific considerations regarding punishment in the ethical guideline 4.08:\\
4.08 Considerations Regarding Punishment Procedures:
\begin{itemize}
\item (a) Behavior analysts recommend reinforcement rather than punishment whenever possible. 
\item (b) If punishment procedures are necessary, behavior analysts always include reinforcement procedures for alternative behavior in the behavior-change program. 
\item (c) Before implementing punishment-based procedures, behavior analysts ensure that appropriate steps  
have been taken to implement reinforcement-based procedures unless the severity or dangerousness of the behavior necessitates immediate use of aversive procedures.
\item (d) Behavior analysts ensure that aversive procedures are accompanied by an increased level of training, supervision, and oversight. Behavior analysts must evaluate the effectiveness of aversive procedures in a timely manner and modify the behavior-change program if it is ineffective. Behavior analysts always include a plan to discontinue the use of aversive procedures when no longer needed. (BACB, 2014, pp.12-13)
\end{itemize}
%
Ethical Considerations Related to Punishment as outlined by Cooper, Heron, and Heward (2007):
\begin{enumerate}
\item The right to safe and humane treatment
\item Least restrictive alternative
\item Right to effective treatment
\end{enumerate}

Appropriate Use of Punishment as outlined by Cooper, Heron, and Heward (2007):
\begin{enumerate}
\item Conduct a functional assessment
\item Attempt reinforcement based strategies (behavior as above does not reach socially acceptable levels)
\item Conduct punisher assessment
\item Ensure informed consent is given
\item Include reinforcement based strategies with punishment procedures
\item Ensure all staff are trained in the procedure and monitored closely
\item Use punishers of sufficient quality and magnitude
\item Use varied punishers
\item Deliver punisher at the beginning of a behavioral sequence
\item Punish each instance of the behavior Initially
\item Shift to intermittent schedule gradually
\item If delay in punishment use mediation
\item Supplement punishment with complementary interventions
\item Be prepared for negative side effects
\item Collect data, graph and evaluate daily
\item Discontinue procedure if a decrease in behavior is not observed
\end{enumerate}
%
\subsection{Assessment}
\begin{enumerate}
\item Provide scenarios in which clients would not qualify for a punishment procedure (e.g., behavior does not cause physical harm, reinforcement based strategies have not been attempted, or consent was not obtained.)
\item Ask Supervisee to list the four considerations the BACB lists when considering punishment
\item Ask the Supervisee to list the side effects of punishment
\item Ask the Supervisee to outline the recommendation for a client who had been receiving a punishment procedure for 2 months and head hitting remained consistent at 10 times a day. (should discontinue)
\item Have the supervisee list all of the things that must happen prior to a punishment procedure beginning (functional assessment, reinforcement based program ineffective, consent obtained, staff trained)
\end{enumerate}
%
\subsection{Relevant Literature}
\begin{refsection}
\nocite{bailey2013ethics,
        bac2014professional,
        cooper2007applied,
        foxx1982decreasing,
        van1988right,
        lerman2002status,
        iwata1988development}
\printbibliography[heading=none]
\end{refsection} 
%
\subsection{Related Tasks}
\fourdSixteen{}\\
\fourdFifteen{}\\
\fourdNineteen{}\\
\foureEleven{}\\
\fourgSeven{}\\
\fourjTen{}\\
\fourFKNineteen{}\\
\fourFKTwenty{}\\
\fourFKTwentyOne{}\\
\fourFKThirtyEight{}\\
%
Footnotes\\
Positive punishment may also be described as a type of aversive control. Negative side effects include: emotional or aggressive reactions, behavioral contrast, escape and avoidance of the punisher, modeling of inappropriate behavior and the overuse associated with negative reinforcement of the person presenting the punisher (Cooper, Heron, \& Heward, 2007)
