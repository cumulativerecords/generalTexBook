\clearpage \section{\fourFKSeventeen{}}
\subsection{Definition}
Unconditioned reinforcer - ``A stimulus change that increases the frequency of any behavior that immediately precedes it irrespective of the organism's learning history with the stimulus. Unconditioned reinforcers are the product of the evolutionary development of the species (phylogeny) Also called primary or unlearned reinforcer'' (Cooper, Heron, \& Heward, 2007 p.707).

``...momentary effectiveness of an unconditioned reinforcer is a function of current motivating operations'' (Cooper et al., 2007, p. 39).
%
\subsection{Examples}
\begin{enumerate}
\item Food, water, oxygen, warmth, and sexual stimulation are some examples of unconditioned reinforcers.
\item A teacher gives a child a pretzel after the child does a task. The child's engagement in the task increases in the future. 
\item This is an example of unconditioned reinforcement.
%
\end{enumerate}
%
\subsection{Assessment}
\begin{enumerate}
\item Have supervisee create a list of unconditioned reinforcers. Have him/her define and describe the role satiation and deprivation plays in unconditioned reinforcement. 
\item Have supervisee give examples of unconditioned reinforcers. Have him/her describe the difference between conditioned and unconditioned reinforcement.
\item Have supervisee explain the relationship between conditioned and unconditioned reinforcers and the role unconditioned reinforcers may play in creating conditioned reinforcers.
\end{enumerate}
%
\subsection{Relevant Literature}
\begin{refsection}
\nocite{bijou1965child,
        cooper2007applied,
        gewirtz2000infant,
        malott1978behavior,
        pelaez1996infants,
        skinner1953science}
\printbibliography[heading=none]
\end{refsection}
%
\subsection{Related Tasks}
\fourcOne{}\\
\fourdOne{}\\\
\fourdTwo{}\\\
\fourdNineteen{}\\
\fourFKTwo{}\\
\fourFKThirteen{}\\
\fourFKSixteen{}\\
\fourFKNineteen{}\\
\fourFKTwentyOne{}\\
\fourFKTwentySix{}\\
\fourFKThirty{}\\
