\clearpage \section{\fourdThree{}}
\subsection{Definition}
Prompts – ``...antecedent stimuli that increase the probability of a desired response'' (Piazza, \& Roane, 2014, p. 256)

Prompt fading – ``...transfer stimulus control from therapist delivered prompts to stimuli in the natural environment that should evoke appropriate responses'' (Walker, 2008 as cited in Fisher, Piazza, \& Roane, 2014, p. 412).

Prompts are used when teaching skills. Prompts can be used when teaching in task analysis, discrete trial, incidental teaching, etc. Prompt fading is important as the learner begins to show competence with the skill being taught. Fading allows the learner to become independent and meet naturalistic reinforcers for his/her behavior.

Prompts are generally divided into two categories: stimulus prompts and response prompts.

Stimulus prompts – ``...those in which some property of the criterion stimulus is altered, or other stimuli are added to or removed from the criterion stimulus'' (Etzel \& LeBlanc, 1979 cited in Fisher, Piazza, \& Roane, 2014, p. 256)
\subsection{Examples}
\begin{enumerate}
\item Response prompts – ``...addition of some behavior on the part of an instructor to evoke the desired learner behavior'' (Fisher, Piazza, \& Roane, 2014, p. 256).
\item Most-to-least prompting
\item Least-to-most prompting
\item Time delay prompts
\end{enumerate}
%
\subsection{Assessment}
\begin{enumerate}
\item Ask your supervisee to role-play several types of prompt strategies.
\item Ask your supervisee to role-play several types of prompt fading procedures.
\item Ask your supervisee to describe the transfer of stimulus control when a prompt is faded out e.g., what stimulus is controlling behavior while prompting vs what stimulus is controlling behavior after the prompt has been faded out.
\end{enumerate}
%
\subsection{Relevant Literature}
\begin{refsection}
\nocite{etzel1979simplest,
    fisher2014handbook,
    walker2008constant}
\printbibliography[heading=none]
\end{refsection}
%
\subsection{Related Tasks}
\fourdFour{}\\
\fourdFive{}\\
\fourdSix{}\\
\fourdSeven{}\\
\fourdEight{}\\
\foureOne{}\\
\foureTwo{}\\
\foureThirteen{}\\
\fourFKTwentyFour{}\\
