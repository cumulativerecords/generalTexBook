\clearpage \section{\fourbFour{}}
\subsection{Definition}
Reversal design - ``Any experimental design in which the researcher attempts to verify the effect of the independent variable by ``reversing'' responding to a level obtained in a previous condition; encompasses experimental designs in which the independent variable is withdrawn (A-B-A-B) or reversed in its focus (e.g., DRI/DRA)'' (Cooper, Heron, \& Heward, 2007, p. 703).

Withdrawal design - ``A term used by some researchers as a synonym for an A-B-A-B design; also used to describe experiments in which an effective treatment is sequentially or partially withdrawn to promote the maintenance of behavior changes'' (Cooper, Heron, \& Heward, 2007, p. 708).
%
\subsection{Examples}
\begin{enumerate}
\item An experiment that entails exposing a participant to a condition of no programmed reinforcement for a work task (baseline) until steady state is achieved, then exposes a participate to a condition in which they earn stickers contingent on a work task (intervention) and then repeats these two conditions respectively. 
\item An experiment in which baseline consists of the reinforcement of challenging behavior and the treatment consists of differential reinforcement of an alternative/replacement behavior and both conditions are replicated at least twice. 
\end{enumerate}
%
\subsection{Assessment}
\begin{enumerate}
\item  Ask the supervisee to either describe a time that they used of a withdrawal or reversal design or have them describe a hypothetical experiment using a withdrawal or reversal design. 
\item Have supervisee look at the figures in the articles (such as those listed below) as well as other articles and determine which ones are reversal/withdrawal designs. 
\item Have the supervisee look at figures in the articles below and describe what characteristics make it a reversal or withdrawal design
\item Have the supervisee describe the pros of using a reversal design and the condition in which the use of a reversal design would not be desirable
\end{enumerate}
%
\subsection{Relevant Literature}
\begin{refsection}
\nocite{cooper2007applied,
    anderson2002use,
    baer1970recent,
    falcomata2004evaluation,
    lerman2002reinforcement}
\printbibliography[heading=none]
\end{refsection}
%
\subsection{Related Tasks} 
\fourbThree{}\\ 
\fourjNine{}\\
%
\subsection{Footnotes}
Some authors exclusively use the term reversal design for studies in which the contingency is reversed  (or switched to another behavior) as in DRO and DRA/DRI reversal techniques and the term withdrawal design for studies that employ an A-B-A-B approach where the A signifies baseline condition and B the treatment condition (Cooper, Heron, \& Heward, 2007). 

A multiple treatment reversal design can also be used to compare the effects of two or more treatment conditions to baseline and/or to the other treatments (e.g., ABABACAC, ABABCBCB) (Cooper, Heron, \& Heward, 2007).
