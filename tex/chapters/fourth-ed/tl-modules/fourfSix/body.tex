\clearpage \section{\fourfSix{}}
\subsection{Definition}
Incidental teaching - When the ``instructor assesses the child's ongoing interests, follows the child's lead, restricts access to high interest items, and constructs a lesson within the natural context, with a presumably more motivated child'' (Anderson \& Romanczyk, 1999, p. 169).

Incidental teaching requires an instructor to use moments in the natural environment as teaching opportunities. It can be used to teach language based skills, social skills, play skills, or other skills as well.

\subsection{Examples}
\begin{enumerate}
\item Todd often struggles to initiate play with his peers during recess. His teacher decided to go over and prompt Todd to introduce himself to Mike and ask him to play a game. 
\item (Non-example) Rich wanted to help teach Todd multiplication, so he gave him a worksheet and sat down with him to go through the problems one by one.
\end{enumerate}
%
\subsection{Assessment}
\begin{enumerate}
\item Have supervisee define and describe incidental teaching.
\item Have supervisee describe how they use incidental teaching at work to teach various skills.
\item Have supervisee describe the pros and cons of incidental teaching.
%
\end{enumerate}
%
\subsection{Relevant Literature}
\begin{refsection}
\nocite{anderson1999early,
        cooper2007applied,
        hart1975incidental,
        mcgee2007incidental,
        mcgee1999incidental,
        mcgee1983modified}
\printbibliography[heading=none]
\end{refsection}
%
\subsection{Related Tasks}
\fourbThree{}\\ 
\fourdFour{}\\
\fourdFive{}\\
\fourdEleven{}\\
\fourjSix{}\\
\fourjEleven{}\\
\fourFKFourtyFour{}\\
