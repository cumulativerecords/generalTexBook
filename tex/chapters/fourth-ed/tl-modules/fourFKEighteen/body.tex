\clearpage \section{\fourFKEighteen{}}
\subsection{Definition}
Conditioned reinforcer - ``A stimulus change that functions as a reinforcer because of prior pairing with one or more other reinforcers; sometimes called secondary or learned reinforcers'' (Cooper et al., 2007, p. 692). 

Conditioned reinforcement – ``the operation, or process, of a response producing a conditioned reinforcer that increases the likelihood that response occurs in the future'' (Cooper et al., 2007, p. 40).
%
\subsection{Examples}
\begin{enumerate}
\item  Money, tokens, stickers.
\item A teacher says ``good job'' after a student returns their homework. The student continues to return their homework in the future. 
%
\end{enumerate}
%
\subsection{Assessment}
\begin{enumerate}
\item Have supervisee explain the differences between conditioned and unconditioned reinforcers.
\item Have supervisee explain the process of producing a conditioned reinforcer (i.e., token systems). Have him/her give an example from their professional experience.
\item Have supervisee read and summarize a journal article on the topic of conditioned reinforcement. 
\end{enumerate}
%
\subsection{Relevant Literature}
\begin{refsection}
\nocite{alessi1992models,
        cooper2007applied,
        higgins2001effects,
        michael2004concepts,
        morse1977determinants}
\printbibliography[heading=none]
\end{refsection}
%
\subsection{Related Tasks}
\fourcOne{}\\
\fourdOne{}\\
\fourdTwo{}\\
\fourdTwenty{}\\
\fourdTwentyOne{}\\
\fourfTwo{}\\
\fourjFour{}\\
\fourkFour{}\\
\fourFKTwo{}\\
\fourFKFourteen{}\\
\fourFKFifteen{}\\
\fourFKSixteen{}\\
\fourFKSeventeen{}\\
\fourFKTwentyOne{}\\
\fourFKTwentySix{}\\
\fourFKTwentySeven{}\\
