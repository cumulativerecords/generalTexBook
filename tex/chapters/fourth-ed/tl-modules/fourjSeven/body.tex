\clearpage \section[\fourjSeven{}]{\fourjSeven{}%
              \sectionmark{J-07 Select... constraints.}}
\sectionmark{J-07 Select... constraints.}
\subsection{Definition}
``The independent variable should be evaluated not only in terms of its effects on the dependent variable, but also in terms of its social acceptability, complexity, practicality, and cost'' (Cooper, Heron, \& Heward, 2007, p. 250).

One method for determining the feasibility of an intervention is by asking consumers (parents, teachers, administrators) to rate the social validity of the client's performance. Questions that are typically posed to consumers before interventions are implemented include asking the consumer how reasonable they feel the intervention is, asking the consumer's willingness to implement the intervention strategies, asking if the consumer would be willing to change the environment to implement the intervention, asking how disruptive the intervention may be to the natural environment, asking how costly it would be to implement the intervention, asking if there will be any discomfort in the client when implementing these procedures, and asking if carrying out the intervention will fit with the classroom or setting routines (Reimers \& Wacker, 1988 cited from Cooper et al., 2007, pp. 238-239).
%
\subsection{Examples}
\begin{enumerate}
\item Rob has decided to implement a reinforcement program based on appropriate responses rather than a fixed time DRO program. He understands that there is no paraprofessional in the classroom to help run the program and the teacher has other educational duties so she cannot run a timer and deliver reinforcement consistently enough for a rigorous DRO. 
%
\end{enumerate}
%
\subsection{Assessment}
\begin{enumerate}
\item Have supervisee come up create a list of appropriate questions to ask consumers when determining an interventions appropriateness and acceptability. Have each supervisee create his/her own treatment acceptability rating form.
\item Have supervisee list and describe various extraneous factors that must be taken into consideration before implementing an intervention. Have supervisee explain why it is important to have consumer satisfaction with an intervention program.
\end{enumerate}
%
\subsection{Relevant Literature}
\begin{refsection}
\nocite{cooper2007applied,
        hawkins1984meaningful,
        reimers1988parents,
        wolf1978social}
\printbibliography[heading=none]
\end{refsection}
%
\subsection{Related Tasks}
\fourcOne{}\\
\fourgSix{}\\
\fourgEight{}\\
\fouriOne{}\\
\fouriTwo{}\\
\fourjSix{}\\
\fourjEight{}\\
\fourkSeven{}\\
\fourkNine{}\\
\fourFKEleven{}\\
