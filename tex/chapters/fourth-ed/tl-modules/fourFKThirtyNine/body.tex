\clearpage \section{\fourFKThirtyNine{}}
\subsection{Definition}
Behavioral momentum - ``A metaphor to describe a rate of responding and its resistance to change following an alteration in reinforcement conditions'' (Cooper, Heron, \& Heward, 2007, p. 691).\\

``In classical physics, momentum is defined as the product of velocity and mass. Translating metaphorically, behavioral momentum is the product of response rate and resistance to change'' (Nevin, 1992, p. 302).\\

Response rate had been used as the measure of response rate for many years. With the introduction of behavioral momentum, Nevin challenges this and describes resistance to change as a better way to measure response strength (Nevin, 1974).\\

Behavioral momentum is not synonymous with the high-p request sequence. Use caution when describing behavioral momentum this way (Nevin, 1996).\\
%
\subsection{Examples}
\begin{enumerate}
\item ``If you are working at the computer, and you keep working even though you are called to dinner, that is an example of behavioral momentum'' (Pierce \& Cheney, 2013, p. 134).
\item A student is coloring at his desk. The teacher asks him to come down to the rug to listen to a story. He continues to color for a few more seconds.
\item James is watching TV. His remote stops working. He continues to push the button despite the battery being dead.
%
\end{enumerate}
%
\subsection{Assessment}
\begin{enumerate}
\item Ask your supervisee to describe behavioral momentum
\item Ask your supervisee to give an example of behavioral momentum.
%
\end{enumerate}
%
\subsection{Relevant Literature}
\begin{refsection}
\nocite{brandon1997applying,
        cooper2007applied,
        nevin1974response,
        nevin1992integrative,
        nevin1996momentum,
        nevin1983analysis}
\printbibliography[heading=none]
\end{refsection}
%
\subsection{Related Tasks}
\foureNine{}\\
\fourFKTen{}\\
