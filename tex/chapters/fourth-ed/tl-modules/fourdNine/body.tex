\clearpage \section[\fourdNine{}]{\fourdNine{}%
              \sectionmark{D-09 Use the verbal operants...}}
\subsection{Definition}
In the field of applied behavior analysis extensive research has been done on the development of verbal behavior.  

``Verbal behavior involves social interactions between speakers and listeners, whereby speakers gain access to reinforcement and control their environment through the behavior of listeners'' (Sundberg as cited in Cooper, Heron, \& Heward, 2007, p. 529). Verbal operants are the basic units of this exchange.  

In 1957 B.F. Skinner identified six elementary verbal operants in his book on Verbal Behavior.  These included mands, tacts, intraverbals, echoics, textuals, and transcription.  ``Skinner's analysis suggests that a complete verbal repertoire is composed of each of the different elementary operants, and separate speaker and listener repertoires'' (Sundberg as cited in Cooper, et al., 2007, p. 541).   

Since Skinner described these operants, those in the field have applied these concepts to both language assessment and training.  In order to evaluate whether or not specific language training is necessary, a variety of standardized tools have been used to test an individual's receptive and expressive language abilities.  These include but are not limited to: the Peabody Picture Vocabulary Test III (Dunn \& Dunn, 1997), the Comprehensive Receptive and Expressive Vocabulary Test (Hammill \& Newcomer, 1997), the Assessment of Basic Language and Learning Skills (ABLLS) (Partington \& Sundberg, 1998), the Verbal Behavior Milestones Assessment and Placement Program (VB-MAPP) and the CELF-4 Semel, Wiig, \& Secord, 2003).  

Not all of these tests will identify deficits in one or more of the verbal operants. Some children who may be proficient in tacting (such as labeling things in their environment such as letters and numbers) may fail to make appropriate mands for desired items (Cooper, et al., 2007).  In this case it is important for behavior analysts to use a combination of approaches or less standardized methods to assess these needs.  It may be helpful to observe the individual in their natural environment and take data on their verbal interactions.  It will be important to ask questions such as:
\begin{enumerate}
\item What is the frequency of and complexity of mands?
\item What is the frequency and complexity of tacting behavior?
\item Will the child or individual demonstrate echoic behavior when prompted?
\item Does the child or individual engage in intraverbal behavior with known caregivers?
\item Can or will the child or individual read words that are written down for him?
\item Can or will the child or individual write words that are said to him? 
\end{enumerate}
%
\subsection{Assessment}
\begin{enumerate}
\item Ask the supervisee to name the basic unit of language
\item Ask the supervisee to name all 6 of the elementary verbal operants
\item Ask the supervisee to name some of the standardized tests often used to assess language
\item Ask the supervisee to explain why these standardized tests may not provide adequate information
\item Ask the supervisee to describe how one might assess an individual's use of verbal operants if testing fails to yield enough information.
\end{enumerate}
%
\subsection{Relevant Literature}
\begin{refsection}
\nocite{cooper2007applied,
    partington1998assessment,
    semel2003clinical,
    skinner1957verbal,
    sundberg2008verbal,
    sundberg1998teaching}
\printbibliography[heading=none]
\end{refsection}
%cannot verify reference.
%Hammill, D., \& Newcomer, P.L. (1997).  Test of language development-3.  Austin, TX: Pro-Ed.
%
\subsection{Related Tasks}
\fourdTen{}\\
\fourdEleven{}\\
\fourdTwelve{}\\
\fourdThirteen{}\\
\fourdFourteen{}\\
\fourFKFourtyThree{}\\
\fourFKFourtyFour{}\\
\fourFKFourtyFive{}\\
\fourFKFourtySix{}\\
