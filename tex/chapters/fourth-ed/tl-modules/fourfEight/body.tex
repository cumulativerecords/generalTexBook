\clearpage \section{\fourfEight{}}
\subsection{Definition}
``A set of procedures and processes by which an individual's communication skills (i.e., production as well as comprehension) can be maximized for functional and effective communication. It involves supplementing or replacing natural speech and/or writing with aided (e.g. picture communication symbols, line drawing, Blissymbols, and tangible objects) and/or unaided symbols (e.g. manual signs, gestures, and finger spelling)'' (American Speech-Language-Hearing Association, 2002).

Augmentative communication includes any modality that supplements a person with difficulties engaging in spoken language. This can include gestures, sign language, PECS, electronic devices, picture books, etc.

The use of augmentative communication should be considered as a way to allow the user to access reinforcement from the natural environment. Often functionally equivalent responses can be taught to replace problematic behavior therefore leading to a decrease in that behavior (Durand, 1999).

Teaching of functional communication using augmentative communication devices should be taught using the same strategies to teach other skills (prompt fading, reinforcement considerations, generalization considerations, etc.)

Augmentative communication systems should not be confused with the long-discredited ``facilitated communication'' which is a pseudoscientific attempt at getting people with developmental disabilities to communicate.

\subsection{Examples}
\begin{enumerate}
\item Bill uses pictures to tell his teachers when he wants.
\item Ralph is eating snack and signs ``more'' to his teacher after running out of crackers to eat. She gladly hands him more and praises him for using his words.
\item Hillary uses an app on a tablet device that generates speech sounds so others respond to her.
\item (Non-example) Bill wants more crackers so tells his teacher, ``I want some more, please.'' using spoken word.
\item (Non-example) Bill wants more crackers so he hits his fist on the table and screams. His teacher says, ``oh, you're still hungry? Here are a few more crackers.''
\end{enumerate}
%
\subsection{Assessment}
\begin{enumerate}
\item Have supervisee research various augmentative communication systems. Have him/her choose a system and describe it in detail.
\item Have supervisee identify the benefits of using alternative communication systems for non-vocal-verbal students.
\item Have supervisee describe situations where they would seek the advice of speech-language professionals for more information regarding the pros and cons of each system.
\end{enumerate}
%
\subsection{Relevant Literature}
\begin{refsection}
\nocite{am2002aug,
        charlop2002using,
        dattilo1991facilitating,
        jacobson1995history,
        durand1999functional,
        mirenda1990communication}
\printbibliography[heading=none]
\end{refsection}

%
\subsection{Related Tasks}
\fourdThree{}\\
\fourdFour{}\\
\fourdFive{}\\
\fourfSeven{}\\
