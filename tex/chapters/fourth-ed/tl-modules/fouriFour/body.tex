\clearpage \section[\fouriFour{}]{\fouriFour{}%
              \sectionmark{I-04 Design and implement...}}
\sectionmark{I-04 Design and implement...}
\subsection{Definition}
Functional behavior assessment (FBA) - ``A systematic method of assessment for obtaining information about the purposes (functions) a problem behavior serves for a person; results are used to guide the design of an intervention for decreasing the problem behavior and increasing appropriate behavior'' (Cooper, Heron, \& Heward, 2007, p. 696).

Functional analysis (FA) - ``An analysis of the purposes of problem behavior, wherein antecedents and consequences representing those in the person's natural routines are arranged within an experimental design so that their separate effects on problem behavior can be observed and measured'' (Cooper, Heron, \& Heward, 2007, p. 696).* The FA is considered to be best practice standard in conducting a functional assessment (Hanley, Iwata, \& McCord 2003).  

Descriptive assessment - ``Direct observation of problem behavior and the antecedent and consequent events under naturally occurring conditions'' (Cooper, Heron, \& Heward, 2007, p. 693).* The advantages are that the information yields what happens in the individual's natural environment, does not disrupt the individual's routine, and provides information for designing a functional analysis. The disadvantages of these assessments are false positives due to behavior maintained by intermittent reinforcement or the presence of antecedent and consequent events which are often present but have no functional relation, the time required in taking data, and inaccurate data collection. Also, there is little correspondence between descriptive analysis outcomes being compared to functional analysis outcomes (Pence, Roscoe, Bourret, and Ahearn, 2009)

Indirect assessment - ``Structured interviews, checklists, rating scales, or questionnaires used to obtain information from people who are familiar with the person exhibiting the problem behavior'' (Cooper, Heron, \& Heward, 2007, p. 697).  The advantages of indirect assessments are that the forms can yield valuable information and are convenient.  The disadvantage is the lack of research supporting the reliability of these measurements.  

The ethical guidelines of BACB requires BCBAs to conduct a functional assessment according to 3.01 Behavior-Analytic Assessment. \\
(a) Behavior analysts conduct current assessments prior to making recommendations or developing behavior-change programs. The type of assessment used is determined by clients' needs and consent, environmental parameters, and other contextual variables. When behavior analysts are developing a behavior-reduction program, they must first conduct a functional assessment.\\
(b) Behavior analysts have an obligation to collect and graphically display data, using behavior-analytic conventions, in a manner that allows for decisions and recommendations for behavior-change program development (BACB, 2014, p.8).

Role of Functional Behavior Assessment as outlined by Cooper, Heron, and Heward (2007).
\begin{enumerate}
\item Identifies antecedent variables that may be altered to prevent problem behaviors.
\item Identifies reinforcement contingencies which can be altered so problem behavior no longer receives reinforcement.
\item Identifies reinforcers for alternative replacement behavior.
\item Reduces the reliance on default technologies such as punishment.
\end{enumerate}

Conducting a functional behavior assessment as outlined by Cooper, Heron, and Heward (2007).
\begin{enumerate}
\item Conduct indirect assessments
\item Conduct a descriptive assessment
\item Analyze data  from the indirect and descriptive assessment and create a hypotheses 
\item Test conditions the data suggest may be contributing to the behavior using a functional analysis
\item Develop intervention strategies based on the results.
\end{enumerate}
%
\subsection{Assessment}
\begin{enumerate}
\item Have supervisee list in order the process of completing a functional assessment.
\item Have supervisee describe the benefits and limitations of each of the assessment procedures.
\item Use behavior skills training techniques to teach your supervisee how to conduct an indirect assessment.  Accompany the supervisee in completing their first indirect assessment and provide reinforcement and feedback following the session.
\item Use behavior skills training techniques to teach your supervisee how to collect ABC data.  Explain the process, model and then practice using you tube videos of challenging behaviors.  Provide reinforcement and feedback and continue practicing until the supervisee clearly demonstrates skills in collecting ABC data.  Accompany the supervisee in completing live ABC data, take inter-rater reliability and compare scores providing reinforcement and feedback.
\item Use behavior skills training techniques to teach your supervisee how to complete an FA. Practice until the supervisee clearly demonstrates skills in the control condition and some test conditions of an FA.  Accompany the supervisee in completing live FAs, prompting and providing reinforcement and feedback.  Continue to monitor supervisee in this process until they have demonstrated the completion of multiple FA's accurately and are able to set up individualized FAs based on the descriptive data.
\end{enumerate}
%
\subsection{Relevant Literature}
\begin{refsection}
\nocite{baer1987some,
        bac2014professional,
        cooper2007applied,
        hanley2003functional,
        iwata1994functions,
        pence2009relative,
        sasso1992use}
\printbibliography[heading=none]
\end{refsection}
%
\subsection{Related Tasks}
\fourbThree{}\\
\fourbFive{}\\
\foureOne{}\\
\fourhOne{}\\
\fourhThree{}\\
\fourhFive{}\\
\fouriThree{}\\
%
\subsection{Footnotes}
*Functional analysis may also be called an analog analysis or experimental analysis.
*Descriptive analysis may also be called direct assessment.
