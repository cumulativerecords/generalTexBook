\clearpage \section[\fourbEleven{}]{\fourbEleven{}%
              \sectionmark{B-11 Conduct a parametric... }}
\sectionmark{B-11 Conduct a parametric...}
\subsection{Definition}
Parametric analysis - ``An experiment designed to compare the differential effects of a range of values of the independent variable'' (Cooper, Heron, \& Heward, 2007, p. 701).
%
\subsection{Examples}
\begin{enumerate}
\item An experiment designed to analyze different magnitudes of a punishment procedure to determine the least intrusive magnitude of a stimulus to decrease behavior.
\item An experiment designed to analyze the optimal quality of attention necessary to reinforce appropriate behavior 
\end{enumerate}
%
\subsection{Assessment}
\begin{enumerate}
\item Ask Supervisee to list some parameters (schedule, immediacy, quality, quantity) of an independent variable that can be manipulated experimentally
\item Have supervisee describe a hypothetical parametric analysis inclusive of the independent variable parameter to be manipulated and the functional relation to be tested 
\item Ask the supervisee how they could use a parametric analysis to test for the optimal level of treatment integrity. Then discuss what parameter they would measure (i.e., schedule, immediacy) and which values they would select to answer this experimental question.  
\end{enumerate}
%
\subsection{Relevant Literature}
\begin{refsection}
\nocite{cooper2007applied,lerman1996methodology,lerman2002reinforcement}
\printbibliography[heading=none]
\end{refsection}
%
\subsection{Related Tasks} 
\fourbThree{}\\
\fourFKThirtyThree{}\\
