\clearpage \section{\fourdEleven{}}
\subsection{Definition}
Mands are important in the development of language in children.  The development of a mands allows an individual to communicate their wants and needs, including basic needs such as food and water, to those around them.  When an early learner fails to develop a mand repertoire, they may not be able to effectively communicate with others and may not be able to access these reinforcers.  This can lead to frustration, learned helplessness, or a variety of other maladaptive behaviors such aggression, self-injurious behavior social withdrawal or tantrums (Cooper, Heron, \& Heward, 2007).

When mands fail to develop typically, it is crucial begin language training.   Cooper, Heron and Heward (2007) suggest teaching mands before all other types of verbal behavior as manding allows an individual to gain access to their wants and needs. During early training a variety of mands should be taught so that the child learns to differentiate their response based on their current needs and MO. Instructors should focus on teaching bids for edibles and tangibles before making other requests.   Sundberg \& Partington (1998) suggest that teachers should use a combination of ``prompting, fading and differential reinforcement to transfer control from stimulus variables to motivative variables'' (as cited in Cooper, Heron, \& Heward, 2007, p. 541).  It is important that both the echoic prompt and the non-verbal stimulus be faded out for mand training to be effective.  

When using mand training, the trainer should:
\begin{enumerate}
\item Establish an MO (motivating operation).  This may be done formally through preference assessment procedures or more informally through observations or caregiver report.  It is important that a child be motivated to make a request for mand training to be effective.  Note: It may be helpful to ensure that a reinforcer has been withheld prior to training to ensure that it is potent.  For instance a child who has just recently eaten is not as likely to be motivated to request food.  
\item Enrich the environment with potential reinforcers (things that the child generally seems to prefer such as foods and toys).
\item Wait for the child to initiate or show interest in the non-verbal stimulus (the child reaches for an item, emits some sort of vocalization, points to it, etc.). 
\item Use an echoic prompt to label the non-verbal stimulus.  Successively reinforce closer and closer approximations to the target verbal response and follow with specific reinforcement (the requested item).  
\item Once the client is able to imitate the verbal model in the presence of the stimulus, gradually fade out the echoic prompt to establish the response ``under the multiple control of the MO and the nonverbal stimulus'' (Sundberg, as cited in Cooper, Heron, \& Heward, 2007, pp. 541-542).  
\item Finally the presentation of the non-verbal stimulus should also be faded out so that the response is only under the control of the MO.  This helps to ensure that the individual can make the request regardless of whether or not the item is physically present within the environment. 
\item Gradually increase the verbal requirement over time so that the child is making more complex and specific requests (``I want the chocolate cookie.'').
\end{enumerate}
%
\subsection{Assessment}
\begin{enumerate}
\item Ask the supervisee to state what mand training can be used to teach.
\item Ask the supervisee to give examples of mand training.
\item Ask the supervisee to state why it is important to fade the non-verbal stimulus and the echoic prompt.  
\end{enumerate}
%
\subsection{Relevant Literature}
\begin{refsection}
\nocite{cooper2007applied,
    drash1999using,
    skinner1957verbal,
    sundberg2001contriving,
    sundberg1998teaching}
\printbibliography[heading=none]
\end{refsection}
%
\subsection{Related Tasks}
\fourFKFourtyFour{}\\
