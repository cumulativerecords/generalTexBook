\clearpage \section[\fouraThirteen{}]{\fouraThirteen{}%
              \sectionmark{A-13 Design... discontinuous}}
\sectionmark{A-13 Design... discontinuous}
\subsection{Definition}
Time sampling – ``...refers to a variety of methods for observing and recording behavior during intervals or at specific moments in time. The basic procedure involves dividing the observation period into time intervals and then recording the presence or absence of behavior within or at the end of interval... Three forms of time sampling used often by applied behavior analysts are whole-interval recording, partial-interval recording and momentary time sampling'' (Cooper, Heron, \& Heward, 2007, p. 90). 

Whole-Interval Recording\\
Once the interval has ended, the observer records whether the behavior has occurred throughout the entire interval. Whole-interval recording tends to underestimate how much a behavior is occurring because the behavior has to be emitted for the entire interval in order to get recorded (Cooper, Heron, \& Heward, 2007).

Partial-Interval Recording\\
With the partial-interval recording method, the time of observation is again divided into intervals and a behavior is recorded as having occurred if it has occurred at some point during the interval. Data are usually reported as percentage of intervals (Cooper, Heron, \& Heward, 2007).

Momentary Time Sampling\\
With this type of measurement, a period of time is divided up into intervals and the observer records whether the behavior is occurring at the precise moment the interval ends (Cooper, Heron and Heward, 2007).

\subsection{Examples}
Whole-Interval Recording
\begin{enumerate}
\item Student moves one or both hands repeatedly and rapidly by bending at the wrist, such that fingers move more than 2 inches. 
\item (Non-example) Waving hand to say ``hello'' or ``goodbye.''
\item An observer divides a 5-minute observation period into intervals of 5 seconds. A student flaps his hands for the entire 5-second interval. At the end of the 5-second interval, the observer records the behavior as having occurred.
\end{enumerate}
%
Partial-Interval Recording\\
Student displays \textit{palilalia}, or repeating a word, phrase, or sentence with no direct, observable relationship with the immediate environment. An observer divides a 5-minute observation period up into intervals of 10 seconds. A student emits palilalia for the entire 10-second interval. At the end of the 10-second interval, the observer records the behavior as having occurred.
\begin{enumerate}
\item Calling-out behavior in a pupil: student raises voice above conversation level when not called on by teacher.
\item A 30-minute observation period is divided into one-minute intervals.  At the end of the one minute interval, the behavior is recorded as having occurred because the pupil called-out after the first 30 seconds of the interval.
\item A child touches or manipulates toys. A 5-minute observation period is divided up into 10-second intervals. The observer records the behavior as having occurred in the last 10-second interval because the child had engaged with the toy at some point during the interval. 
\end{enumerate}

Momentary Time Sampling\\
\begin{enumerate}
\item Movie watching: client is seated, and head and eyes are oriented toward screen.
\item An observer is measuring a client's engagement with a movie across a 2-minute period. The 2-minute period is divided up into 5-second intervals. The observer records the behavior as being present at the end of the first 5-second interval as the client was watching the movie appropriately at that specific point in time. 
\end{enumerate}
%
\subsection{Assessment}
\begin{enumerate}
\item Ask your Supervisee to describe each of the 3 types of time sampling methods listed above.
\item Have your Supervisee practice each type of data collection method in his/her job role or through role-playing.
\item Ask them to tell you which time sampling method is being employed in each of the following examples:
\item An observer is interested in a client's interactions with his/her peers. They observe him/her across a 10-minute period; at the end of each 10-second interval, they record the behavior as being present if the client has had any interaction with his/her peers at all during the interval. (Answer = Partial-interval recording).
\item An observer is examining a client's on-task behavior in class. They observe him/her for a 60-minute period and divide the hour up into 5-minute intervals. If the client is on-task at the end of the 5-minute interval, on-task behavior is scored as having been observed. (Answer = Momentary time sampling). 
\item A client's humming behavior is being observed; the observer divides a 15-minute observation time up into 30-second intervals. If humming was observed throughout the entire 30-second interval, the behavior is scored as having occurred in that interval. (Answer = Whole-interval recording). 
\end{enumerate}
%
\subsection{Relevant Literature}
\begin{refsection}
\nocite{cooper2007applied,
    meany2007comparison,
    powell1975evaluation,
    suen1991reappraisal}
\printbibliography[heading=none]
\end{refsection}
%
\subsection{Related Tasks}
\fouraTwelve{}\\
\fourhOne{}\\
\fouriOne{}\\
\fourFKFourtySeven{}\\
\fourFKFourtyEight{}\\
