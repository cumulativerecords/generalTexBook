\clearpage \section{\fouraTwo{}}
%
%
\subsection{Definition}
Rate - ``A ratio of count per observation time; often expressed as count per standard unit of time and calculated by dividing the number of responses recorded by the number of number of standard units of time in which observations were conducted'' (Cooper, Heron, \& Heward, 2007, p. 85).*
%
\subsection{Examples}
\begin{enumerate}
\item    Eating Chips: A young child is sitting at a table where there is a bag of potato chips. They eat 8 chips, stand up, and walk to the living room to watch TV for the rest of the hour. Rate of chip eating is 8 per hour.
\item Basketball Dribbles: Child is playing basketball for 30 minutes. Dribbles 7 times and then practices foul shots. He shoots 15 times and between each shot he dribbles 3 times. Frequency of dribbling is 52 dribbles per 30 minutes.
\end{enumerate}

\subsection{Assessment} 
\begin{enumerate}
\item Ask your supervisee to identify the frequency of chip eating or basketball dribbles in examples.
\item Have supervisee measure a frequency of a behavior on the job or in a role play.
\end{enumerate}
%
\subsection{Relevant Literature}
\begin{refsection}
\nocite{catania2013learning,cooper2007applied}
\printbibliography[heading=none]
\end{refsection}

\subsection{Related Tasks}
\fouriOne{}\\
\fourhOne{}\\
\fourFKFourtySeven{}\\

\subsection{Footnotes}
* Alternatively, rate is not always defined synonymously with frequency throughout the discipline of behavior analysis. Catania (2013) defines rate as ``responses per unit time'' (p. 458) but frequency as ``total responses over a fixed time, over a session of variable duration or, in trial procedure, over a fixed number of trials'' (p. 443) Cooper, Heron, \& Heward (2007) functionally defines ``count'' whereas Catania defines ``frequency.''
