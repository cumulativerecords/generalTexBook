\clearpage \section{\fourFKFourteen{}}
\subsection{Definition} 
Reflex - ``The reliable production of a response by a stimulus'' (Catania, 1998, p. 8).

Respondent behavior - ``behavior that is elicited by antecedent stimuli. Respondent behavior is induced, or brought out, by a stimulus that precedes the behavior; nothing else is required for the response to occur'' (Cooper et al., 2007, p. 29).

``New stimuli can acquire the ability to elicit respondents. Called respondent conditioning, this type of learning is associated with Russian physiologist Ivan Petrovich Pavlolv...'' (Cooper et al., 2007, p. 30).

Pavlov's experiments consisted of a group of dogs who were trained to salivate at the sound of a metronome started just prior to feeding them. Before initial training, the presence of food (US) elicited salivation (UR), but the metronome (NS) was not paired with this response. After numerous trials of food being paired with the sound of the metronome, the dogs began salivating whenever they heard the metronome. After being paired with the presentation of food for several trials, the metronome became a conditioned stimulus (CS) and a conditioned reflex (CR) was elicited. 

\subsection{Examples}
\begin{enumerate}
\item Roger usually drinks soda every day for lunch. When drinking soda, the sugar (US) inside his blood elicits the release of insulin from his pancreas (UR). Now, when he opens the soda, the snap of the can (CS) starts to elicit the release of insulin (CR) before he takes a drink. 
%
\end{enumerate}
%
\subsection{Assessment}
\begin{enumerate}
\item Have supervisee list various examples of respondent behavior. Have him/her explain respondent conditioning and define stimulus-stimulus pairing, unconditioned stimulus, neutral stimulus, conditioned stimulus, and conditioned reflex.
\item Have supervisee identify and describe an example of respondent conditioning (not an example from the Cooper et al. 2007 text or Pavlov's experiments). 
\item Have supervisee create an abstract for an experiment involving respondent conditioning. Have him/her describe how they would conduct the experiment to achieve respondent conditioning.
\item Have supervisee compare and contrast respondent conditioning and operant conditioning.
\end{enumerate}
%
\subsection{Relevant Literature}
\begin{refsection}
\nocite{catania1998learning,
        cooper2007applied,
        skinner1984evolution,
        poling2001principles,
        skinner1938behavior,
        pavlov1928lectures}
\printbibliography[heading=none]
\end{refsection}
%Subreference not explicitly mentioned in above text.
%Poling, A., \& Braatz, D. (2001). Principles of learning: Respondent and operant conditioning and human behavior. Handbook of organizational performance: Behavior analysis and management, 23-49.
%
\subsection{Related Tasks}
\fourFKTen{}\\
\fourFKThirteen{}\\
\fourFKFifteen{}\\
\fourFKSixteen{}\\
\fourFKSeventeen{}\\
\fourFKTwentyFour{}\\
\fourFKTwentySix{}\\
\fourFKThirtyFive{}\\
