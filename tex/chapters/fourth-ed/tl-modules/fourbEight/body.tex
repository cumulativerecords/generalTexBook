\clearpage \section{\fourbEight{}}
Like multiple baseline designs (MBD), multiple probe designs (MPD) are ``rigorous in their evaluation of threats to interval validity; and are practical for teachers and clinicians who want their research efforts to be wholly compatible with their instructional or therapy activities'' (Gast, 2009, p. 277). Multiple probe designs have an additional advantage in applied settings in that intermittent measures of baseline conditions streamline data collection and still maintain the requirement that responding does not change until intervention is applied (baseline logic). Either one probe is taken periodically in baseline conditions and at least three days immediately before applying intervention, MPD (days), or probes occur in brief sessions of a few baseline measurements taken at least three consecutive days before intervention, MPD (conditions). Experimental control is demonstrated if probe evidence across each tier of similar, but functionally different behaviors, participants, or conditions remains relatively stable until intervention is implemented. 
%
\subsection{Examples}
\begin{enumerate}
\item Multiple probe designs are particularly useful for researchers and teachers in educational settings to efficiently demonstrate results of instructional interventions when teaching across functionally different new skills (behaviors), across multiple students (participants), or across different sets of skills (conditions).
\item Multiple probe designs might not be appropriate if assessing the effects of intervention on severe behaviors that result in injury or property destruction because of the requirement that intervention be delayed across each tier while a person continues to engage in severe behaviors with lasting consequences.
\end{enumerate}
%
\subsection{Assessment}
\begin{enumerate}
\item Ask the supervisee to explain the meaning of ``baseline logic'' and the reason it enables measurement of experimental control.
\item Ask the supervisee why MPD across behaviors requires similar, but functionally different behaviors in each tier of a MPD.
\item Give the supervisee three articles in which researchers based their conclusions on MPD line graph data. Ask the supervisee to interpret results of the study based on the graphic data. Compare to the conclusions written by study authors.
\end{enumerate}
%
\subsection{Relevant Literature}
\begin{refsection}
\nocite{ledford2009single,horner1978multiple,thompson1982training}
\printbibliography[heading=none]
\end{refsection}
%
\subsection{Related Tasks}
\fourbThree{}\\
\fourbSeven{}\\
\fourhOne{}\\
\fourhTwo{}\\
\fourhFour{}\\
\fouriFive{}\\
\fourFKThirtySix{}\\
