\clearpage \section[\fourFKTwentyEight{}]{\fourFKTwentyEight{}%
              \sectionmark{FK-28 Transitive, reflexive...}}
\sectionmark{FK-28 Transitive, reflexive...}
\subsection{Definition}
Conditioned motivating operations consist of ``...motivating variables that alter the reinforcing effectiveness of other stimuli, objects, or events, but only as a result of an organism's learning history...'' (Cooper, Heron, \& Heward, 2007, p. 384).\\

The 3 types of conditioned motivating operations are surrogate (CMO-S), reflexive (CMO-R), and transitive (CMO-T).\\

``Any stimulus that systematically precedes the onset of painful stimulation becomes a CMO-R (reflexive- CMO), in that its occurrence will evoke any behavior that has been followed by such reinforcement'' (Cooper et al., 2007, p. 385).\\

``When an environmental variable is related to the relation between another stimulus and some form of improvement, the presence of that variable functions as a transitive CMO, or CMO-T, to establish the second condition's reinforcing effectiveness and to evoke the behavior that has been followed by that reinforcer'' (Cooper et al., 2007, p. 387).\\

Surrogate CMO's are stimuli that have been paired with another motivating operation and ``acquired a form of behavioral effectiveness by being paired with a behaviorally effective stimulus'' (Cooper et al., 2007, p. 384). There is not strong evidence for this type of CMO.
%
\subsection{Examples}
\begin{enumerate}
\item (CMO-R) A child engages in escape-maintained problem behavior during matching to sample instruction. In time, the child engages in escape-maintained problem behavior when the materials for matching to sample are brought out, before instruction has begun. 
\item (CMO-T) You walk up to your front door and turn the knob, but the door is locked. You reach into your pocket and grab your keys and unlock the door. 
\item (CMO-S) In the presence of a stimulus that has been paired with a cold environment, the value of stimuli that produce warmth increases.
%
\end{enumerate}
%
\subsection{Assessment}
\begin{enumerate}
\item Have supervisee list and define the 3 types of conditioned motivating operations. 
\item Have him/her give examples of each type of conditioned motivating operation.
\item Have supervisee explain the difference between conditioned and unconditioned motivating operations.
%
\end{enumerate}
%
\subsection{Relevant Literature}
\begin{refsection}
\nocite{carbone2007role,
        cooper2007applied,
        laraway2003motivating,
        mcgill1999establishing,
        michael2004concepts,
        michael1993establishing,
        mineka1975some,
        rosales2007contriving}
\printbibliography[heading=none]
\end{refsection}
%
\subsection{Related Tasks}
\fourdOne{}\\
\foureOne{}\\
\fourFKThirteen{}\\
\fourFKFourteen{}\\
\fourFKFifteen{}\\
\fourFKSixteen{}\\
\fourFKEighteen{}\\
\fourFKTwenty{}\\
\fourFKTwentySeven{}\\
\fourFKTwentyNine{}\\
\fourFKThirty{}\\
