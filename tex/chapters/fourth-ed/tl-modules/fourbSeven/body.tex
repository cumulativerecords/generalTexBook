\clearpage \section{\fourbSeven{}}
\subsection{Definition}
Multiple baseline design – ``An experimental design that begins with the concurrent measurement of two or more behaviors in a baseline condition, followed by the application of the treatment variable to one of the behaviors while baseline conditions remain in effect for the other behavior(s) After maximum change has been noted in the first behavior, the treatment variable is applied in sequential fashion to each of the other behaviors in the design'' (Cooper, Heron \& Heward, 2007, p. 699).

Multiple baselines are useful when the target behavior is likely to be irreversible, for example, in skill acquisition. And are also useful when it may be impractical or undesirable to implement a reversal design. For example, in decreasing aggression toward peers. One drawback of the multiple baseline design is potentially the length of time that treatment or intervention is withheld for the last behavior or setting being targeted. 

In the delayed baseline design, collection of baseline data for other target behaviors is taken after baseline measurements for the previous behaviors. This design may be effective when a reversal design is not possible, when resources are limited, or when a new behavior or subject becomes available. Behaviors must be measured at the same time and the independent variable cannot be applied to the next behavior until the previous behavior change has been established. There should be a significant difference in the length of baseline conditions between the different behaviors and the independent variable should first be applied to the behavior demonstrating the greatest level of stable responding in baseline.

Other variations in multiple baselines designs are concurrent and nonconcurrent uses of the design. In concurrent multiple baseline designs the data are collected in the same time period. In nonconcurrent multiple baseline designs data can be collected at different times, and different lengths of baselines are collected, following which implementation of the treatment or intervention is conducted—creating multiple A-B experiments. The experiments are then arranged by length of baseline to create a multiple baseline design.  ``According to single-case design logic, the nonconcurrent MB design demonstrates only prediction and replication, and not the critical verification of the intervention's effects'' (Carr, 2005, p. 220).

\subsection{Examples}
\begin{enumerate}
\item Rod conducted an FBA on Billy's aggression and property destruction. Both behaviors were determined to be maintained by escape from demands. Rod decided to implement the same intervention for each behavior using a multiple baseline design because they both served the same function and a reversal would possibly reestablish the dangerous behavior after therapeutic effects were observed. 
\item (Non-example) Bob wanted to determine the effects of response blocking and redirection on hand flapping with one of his students. He implemented this procedure and once it proved effective, decided to eliminate the intervention to determine if this procedure was the likely cause of the behavior decrease. 
\end{enumerate}
%
\subsection{Assessment}
\begin{enumerate}
\item Have supervisee describe the multiple baseline design and state when it may be most appropriate, strengths of this design, as well as limitations of the multiple baseline design.
\item Have supervisee create a hypothetical analysis using the multiple baseline design. Have him/her state why the multiple baseline design was the most effective design to display experimental control.
\item Have supervisee label the parts of a completed multiple baseline design graph and describe how the graph displays experimental control. 
\end{enumerate}
%
\subsection{Relevant Literature}
\begin{refsection}
\nocite{cooper2007applied,
  barger2004multiple,
  carr2005recommendations,
  harris1985comparisons,
  harvey2004nonconcurrent,
  watson1981non,
  zhan2001single}
\printbibliography[heading=none]
\end{refsection}
%
\subsection{Related Tasks}
\fourbThree{}\\
\fourbFour{}\\
\fourbEight{}\\
\fourbNine{}\\
\fourbTen{}\\
\fourbEleven{}\\
\foureOne{}\\
\fourhThree{}\\
\fouriFive{}\\
\fourFKThirtyThree{}\\
