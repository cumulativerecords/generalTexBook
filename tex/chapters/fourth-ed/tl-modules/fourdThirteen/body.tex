\clearpage \section{\fourdThirteen{}}
\subsection{Definition}
``Many children with autism, developmental disabilities, or other language delays suffer from defective or nonexistent intraverbal repertoires, even though some can emit hundreds of mands, tacts, and receptive responses'' (Sundberg, as cited in Cooper, Heron, \& Heward, 2007, p. 545).  

Although typically developing children generally acquire this type of language on their own, some learners may not acquire this type of language without specific training in the skill.  In such cases, intraverbal training may be recommended.  Prior to starting intraverbal training, the learner must have a acquired a variety of pre-requisite skills such as being able to mand, tact, engage in echoic behavior or imitation, to receptively identify stimuli, and to do match to sample tasks (Sundberg \& Partington, 1998). The goal is not to teach new language, but to bring words or phrases that are currently under nonverbal stimulus control entirely under verbal stimulus control (Cooper, Heron, \& Heward, 2007).  For instance, a child who has previously learned to tact or echo the word ``cow'' when they see a picture of a cow, may learn to then say ``cow'' when his teacher has asked, ``What says moo?''

It is recommended that simple intraverbal interactions that are appropriate to the child's developmental age be taught before more complicated responses.  Fill in the blank relations are often the easiest to teach first (Cooper et al., 2007).  For instance, a learner may be taught to fill in the word ``star'' after someone has delivered the line ``Twinkle, twinkle little...''    A teacher may start by using visual stimuli and then gradually fade out these prompts as the child is successful so that only the verbal stimulus is presented.  

Since intraverbal behavior is reinforced by generalized conditioned reinforcement (i.e., social reinforcement via praise, eye contact, body language, etc.) it may be challenging to motivate some students initially to engage in the desired response.  Trainers may need to initially pair specific reinforcement (such as a crayon after the child has responded ``crayon'' when asked, ``What do you color with?'') initially and then fade this over time (Cooper et al., 2007). 

Varying both the verbal stimuli and the verbal responses over time will help to strengthen these responses (Cooper et al., 2007).   For instance a child who has learned to respond ``bear'' when asked, ``What is your favorite toy?'' may then learn to respond with more complexity such as ``blue bear'' or ``my blue bear with the purple hat.''  The teacher can also vary the verbal prompt such as asking, ``What toy do you like the most?'' which is simply another way of phrasing ``what is your favorite toy,'' and is a part of the same stimulus class.    

When using intraverbal training, the trainer should:
\begin{enumerate}
\item Be sure that the desired verbal responses are already in the child's or individual's repertoire
\item Ensure that the listener is attending.  Make sure that they are looking in your direction, are making eye contact and that the environment isn't too noisy or distracting.
\item Deliver the verbal stimulus (i.e. ``The itsy bitsy...'') and pause.  Initially pair the verbal response with a nonverbal stimulus that can be faded out over time (such as a picture of a spider or spider puppet).
\item Provide reinforcement for correct responding. Specific reinforcement such as providing edibles, access to the spider puppet, etc. should be faded over time so that social reinforcement becomes the reinforcing consequence.  
\item Once simple intraverbal relations have been established, teach the child to respond to variations to the verbal stimulus (``Who went up the water spout?'') or respond with more complexity.    
\end{enumerate}
%
\subsection{Assessment}
\begin{enumerate}
\item Ask the supervisee to state what intraverbal training can be used to teach.
\item Ask the supervisee to state the prerequisite skills that are needed to teach intraverbal behavior.  
\item Ask the supervisee to give examples of intraverbal training.
\item Ask the supervisee to discuss how intraverbal training should be delivered.
\item Ask the supervisee to state why nonverbal stimuli should be faded out.
\end{enumerate}
%
\subsection{Relevant Literature}
\begin{refsection}
\nocite{cooper2007applied,
    partington1993teaching,
    skinner1957verbal,
    sundberg1998teaching,
    vedora2009teaching}
\printbibliography[heading=none]
\end{refsection}
%
\subsection{Related Tasks}
\fourdThirteen{}\\
\fourFKFourtySix{}\\
