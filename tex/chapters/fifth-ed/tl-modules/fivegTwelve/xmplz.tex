%
\subsection{Examples}
Related definitions and examples are presented below.
\begin{enumerate}
\item Reflexivity - ``A type of stimulus-to-stimulus relation in which the student, without any prior training or reinforcement for doing so, selects a comparison stimulus that is the same as the same stimulus'' (Cooper, Heron \& Heward, 2007, p. 702).\\
 Example: Without prior reinforcement or training, when shown a picture of a dog and given a picture of the same dog, a rat, and a cow, student matches the picture of the two dogs (e.g. A=A).
\item Symmetry - ``A type of stimulus-to-stimulus relationship in which the learner, without prior training or reinforcement for doing so, demonstrates the reversibility of matched sample and comparison stimuli'' (Cooper, Heron \& Heward, 2007, p. 705).\\
 Example: Student is taught that when given the written word dog to select the picture of a dog. Without further reinforcement or training, when given the picture of the dog, student selects the written word dog (e.g. If A=B, then B=A). 
\item Transitivity - ``A derived stimulus-stimulus relation that emerges as a product of training two other stimulus-stimulus relations'' (Cooper, Heron \& Heward, 2007, p. 706).\\
 Example: Student is taught that when given the written word dog to select the picture of the dog (e.g. A=B). Student is also taught to select the picture of the dog when hearing the spoken word dog (e.g. B=C). Without further reinforcement or training, student selects the written word dog after hearing the spoken word dog (e.g. C=A). 

\item Example of Stimulus Equivalence\\
When learner responds without prior reinforcement and training that A=A (exhibiting reflexivity) and if A=B, then B must also = A (exhibiting symmetry) and finally that if A=B and B=C, then C must also equal A (exhibiting transitivity). 

\item Non-Example of Stimulus Equivalence:\\
 When learner responds without prior reinforcement and training that A=A (exhibiting reflexivity) and if A=B, then B must also equal A (exhibiting symmetry) but cannot show that if A=B and B=C, then C must also equal A (failure to exhibit transitivity). 
%
\end{enumerate}
%
