\subsection{Assessment}
\begin{enumerate}
\item Ask the supervisee to explain the concept of stimulus equivalence.
\item Ask the supervisee to name the three tests that demonstrate the basic principles of stimulus equivalence and describe each of these.
\item Ask the supervisee to give examples of some new concepts that might be taught through stimulus equivalence
\end{enumerate}
%
\subsection{Assessment}
\begin{enumerate}
\item Have supervisees display equivalence with respect to the words ``reflexivity,'' ``transitivity,'' and ``symmetry'' in the spoken form, written form, and written definitions.
\item Have supervisee assess for stimulus equivalence on the job or during role-play 
\item Have supervisee demonstrate an example of stimulus equivalence during role-play
%
\end{enumerate}
%
\subsection{Assessment}
\begin{enumerate}
\item Have supervisee demonstrate the difference between trial-and-error learning and errorless learning on the job or during role-play.
\item Have supervisee describe how and when prompts will be faded to promote independent responding. 
%
\end{enumerate}
%
\subsection{Assessment}
\begin{enumerate}
\item Ask supervisee to discriminate between direct-acting contingencies and rule-governed behavior.
\item Ask supervisee to provide examples of rules.
\item Ask supervisee to identify rules that may be governing a client's behavior.
\item If a rule exists, ask supervisee to describe how a direct-acting contingency can be used instead and vice versa.
\end{enumerate}
%
