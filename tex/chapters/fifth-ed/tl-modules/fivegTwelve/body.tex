%\clearpage \section{\foureSix{}}
\subsection{Definition}
In the field of applied behavior analysis, a number of procedures have been used to teach new concepts.  One of these procedures is known as stimulus equivalence.  In 1971, Murray Sidman discovered that a previously untaught, unreinforced stimuli could come under stimulus control through its pairing with other stimuli which were explicitly taught (Sidman, 1971). This concept revolutionized the field as it demonstrated a new way of teaching that could potentially reduce the amount of time needed to teach a new class of stimuli.   ``Behavior analysts define stimulus equivalence by testing stimulus-stimulus relations.  A positive demonstration of all three behavioral tests (i.e. reflexivity, symmetry, and transitivity) is necessary to meet the definition of an equivalence relation among a set of arbitrary stimuli'' (Cooper, Heron, \& Heward, 2007, p. 398). 
\begin{enumerate}
\item Reflexivity describes the action of selecting a stimulus that is matched to itself in the absence of training and reinforcement (A=A).  For instance an individual is shown three pictures; a penny, a nickel, and a dime.  When given an identical picture of a penny, he matches it to the identical picture of a penny in the array (Sidman, 1994).
\item Symmetry describes the reversibility of the sample stimulus and a comparison stimulus (A=B and B=A).  For instance an individual who is taught to select the picture of a penny (out of an array of 3), when the word penny is given, would also be able to choose the comparison spoken word penny shown the picture of the penny without being previously taught this correlation (Sidman, 1994).  
\item Transitivity is the most crucial test for demonstrating stimulus equivalence.  A third, untrained relation emerges as a result of being taught the first two relations.  (A=C and C=A) ``...emerges as a product of training two other stimulus-stimulus relations'' (Cooper, Heron, \& Heward, 2007, p. 399).  
\end{enumerate}

The following equation demonstrates the basic principals of stimulus equivalence:
\begin{enumerate}
\item If A = B, and
\item B = C, then
\item  A = C (Sidman and Tailby, 1982).
\end{enumerate}

When using stimulus equivalence, decide what relations are to be taught (i.e. spoken word to picture, picture to written word, drawing to real-life picture, etc.).  Decide which order the conditional relations are to be taught.  Teach the relations A=B and B=C to mastery criteria.  Once mastery criteria are met for the first two relations, test for reflexivity, symmetry, and transitivity using the same criteria.  If the participant demonstrates these relations without having been previously been taught them, they will have acquired the third relation C=A that demonstrates the most important test for stimulus equivalence.   

\subsection{Assessment}
\begin{enumerate}
\item Ask the supervisee to explain the concept of stimulus equivalence.
\item Ask the supervisee to name the three tests that demonstrate the basic principles of stimulus equivalence and describe each of these.
\item Ask the supervisee to give examples of some new concepts that might be taught through stimulus equivalence
\end{enumerate}
%
\subsection{Relevant Literature}
\begin{refsection}
\nocite{cooper2007applied,
        sidman1971reading,
        sidman1994equivalence,
        sidman1982conditional}
\printbibliography[heading=none]
\end{refsection}
%
\subsection{Related Lessons}
\foureSix{}\\
\foureThirteen{}\\
\fourFKEleven{}\\
\fourFKTwelve{}\\
\fourFKThirteen{}\\
\fourFKTwentyFour{}\\
\fourFKTwentyEight{}\\
\fourFKThirtyFive{}\\
%
%\clearpage \section{\fourFKTwelve{}}
\subsection{Definition}
Stimulus equivalence - ``The emergence of accurate responding to untrained and nonreinforced stimulus-stimulus relations following the reinforcement of responses to some stimulus-stimulus relations. A positive demonstration of reflexivity, symmetry and transitivity is necessary to meet the definition of equivalence'' (Cooper, Heron \& Heward, 2007, p. 705).'' 

\subsection{Examples}
Related definitions and examples are presented below.
\begin{enumerate}
\item Reflexivity - ``A type of stimulus-to-stimulus relation in which the student, without any prior training or reinforcement for doing so, selects a comparison stimulus that is the same as the same stimulus'' (Cooper, Heron \& Heward, 2007, p. 702).\\
 Example: Without prior reinforcement or training, when shown a picture of a dog and given a picture of the same dog, a rat, and a cow, student matches the picture of the two dogs (e.g. A=A).
\item Symmetry - ``A type of stimulus-to-stimulus relationship in which the learner, without prior training or reinforcement for doing so, demonstrates the reversibility of matched sample and comparison stimuli'' (Cooper, Heron \& Heward, 2007, p. 705).\\
 Example: Student is taught that when given the written word dog to select the picture of a dog. Without further reinforcement or training, when given the picture of the dog, student selects the written word dog (e.g. If A=B, then B=A). 
\item Transitivity - ``A derived stimulus-stimulus relation that emerges as a product of training two other stimulus-stimulus relations'' (Cooper, Heron \& Heward, 2007, p. 706).\\
 Example: Student is taught that when given the written word dog to select the picture of the dog (e.g. A=B). Student is also taught to select the picture of the dog when hearing the spoken word dog (e.g. B=C). Without further reinforcement or training, student selects the written word dog after hearing the spoken word dog (e.g. C=A). 

\item Example of Stimulus Equivalence\\
When learner responds without prior reinforcement and training that A=A (exhibiting reflexivity) and if A=B, then B must also = A (exhibiting symmetry) and finally that if A=B and B=C, then C must also equal A (exhibiting transitivity). 

\item Non-Example of Stimulus Equivalence:\\
 When learner responds without prior reinforcement and training that A=A (exhibiting reflexivity) and if A=B, then B must also equal A (exhibiting symmetry) but cannot show that if A=B and B=C, then C must also equal A (failure to exhibit transitivity). 
%
\end{enumerate}
%
\subsection{Assessment}
\begin{enumerate}
\item Have supervisees display equivalence with respect to the words ``reflexivity,'' ``transitivity,'' and ``symmetry'' in the spoken form, written form, and written definitions.
\item Have supervisee assess for stimulus equivalence on the job or during role-play 
\item Have supervisee demonstrate an example of stimulus equivalence during role-play
%
\end{enumerate}
%
\subsection{Relevant Literature}
\begin{refsection}
\nocite{cooper2007applied,
        sidman1997equivalence,
        sidman2009equivalence}
\printbibliography[heading=none]
\end{refsection}
%
%\clearpage \section{\foureTwelve{}}
\subsection{Definition}
Errorless learning - an ``approach whereby the task is manipulated to eliminate/reduce errors. Tasks are executed in such a way that the subject is unlikely to make errors'' (Fillingham, Hodgson, Sage, \& Ralph, 2003, p. 339). 

Errorless learning techniques include most-to-least prompt fading or stimulus shaping/fading techniques. Prompts are removed gradually as the individual becomes more adept with the skill, thereby reducing the likelihood of errors.  To apply errorless learning, behavioral strategies utilized may include: response prevention (e.g. only S+ is presented allowing for only correct responding or physical guidance is provided with instruction so incorrect responses are not possible); verbal prompt fading; modeling; stimulus fading (e.g. emphasizing a physical dimension of the stimuli to evoke a correct response such as by illuminating the correct selection, S+, and presenting the incorrect selection, S-, in a dimmer format); or stimulus shaping (e.g. increasing likelihood of correct responding by gradually changing the shape of the stimulus to maintain correct responding). 

The advantages of errorless learning include that it removes negative side effects involved with trial-and-error learning and that it is proven particularly effective among individuals that suffer from brain damage or have a developmental disorder. The disadvantages include cost, time-intensity, and maybe considered less natural than trial-and-error learning (Mueller, Palkovic \& Maynard, 2007).

Trial-and-error learning, being presented with stimuli in which both the correct selection (S+) and incorrect selection (S-) are available, can lead to adverse side effects due to the possibility of incorrect responding and failure to access reinforcers. Research has shown that this can result in aggression, negative emotional responses and stimulus overselectivity (Mueller et al., 2007).
%
\subsection{Assessment}
\begin{enumerate}
\item Have supervisee demonstrate the difference between trial-and-error learning and errorless learning on the job or during role-play.
\item Have supervisee describe how and when prompts will be faded to promote independent responding. 
%
\end{enumerate}
%
\subsection{Relevant Literature}
\begin{refsection}
\nocite{fillingham2003application,
        mueller2007errorless,
        terrace1963errorless}
\printbibliography[heading=none]
\end{refsection}
%
\subsection{Related Lessons} 
\fourdThree{}\\
\fourdFour{}\\
\fourFKTwentyFour{}\\
%
%\clearpage \section{\foureThree{}}
\subsection{Definition}
Rules are descriptions of behavioral contingencies (e.g., ``Putting a sweater on when it is cold outside will help you stay warm'').  When rules are followed, behavior can come under the control of delayed or indirect consequences therefore resulting in rule-governed behavior.  Malott and Trojan-Suarez (2003) suggest that all instructions involve rules.  For example, incomplete rules (e.g., ``Stop it'') provide minimal instruction (e.g., stop) and imply an outcome (e.g., you might get in trouble).   It is argued that rules function as reinforcement-based or punishment-based discriminative stimuli (Malott \& Trojan-Suarez, 2003). Skinner (1969) referred to rules, instructions, advice, and laws as contingency-specifying stimuli, describing the  relations of everyday life.

Evidence that behavior is the result of instructional control or rule following is provided if: (1) there is no obvious or immediate consequence of the behavior; (2) the delivery of the consequence following the behavior exceeds 30 seconds; (3) behavior changes without reinforcement; (4) a substantial increase in the rate of behavior occurs following one instance of direct contact with reinforcement; and (5) the rule exists but no consequence (including automatic reinforcement) exists following the behavior (Cooper, Heron, \& Heward, 2007).
%
\subsection{Assessment}
\begin{enumerate}
\item Ask supervisee to discriminate between direct-acting contingencies and rule-governed behavior.
\item Ask supervisee to provide examples of rules.
\item Ask supervisee to identify rules that may be governing a client's behavior.
\item If a rule exists, ask supervisee to describe how a direct-acting contingency can be used instead and vice versa.
\end{enumerate}
%
\subsection{Relevant Literature}
\begin{refsection}
\nocite{cooper2007applied,
        hayes2004rule,
        malott2003principles,
        skinner1969contingencies}
\printbibliography[heading=none]
\end{refsection}
%
\subsection{Related Lessons}
\fourdOne{}\\
\fourdSixteen{}\\
\fourkTwo{}\\
\fourFKThirty{}\\
\fourFKThirtyOne{}\\
\fourFKThirtyThree{}\\
\fourFKFourtyOne{}\\
\fourFKFourtyTwo{}\\
