\clearpage \section{\fourdFour{}}
\subsection{Definition}
Imitation - ``A behavior controlled by any physical movement that serves as a novel model excluding vocal-verbal behavior, has formal similarity with the model, and immediately follows the occurrence of the model (e.g. within seconds of the model presentation).  An imitative behavior is a new behavior emitted following a novel antecedent event (i.e., the model)'' (Cooper, Heron, \& Heward, 2007, p. 697).

Model - ``An antecedent stimulus that evokes the imitative behavior'' (Cooper, Heron, \& Heward, 2007, p. 413)

Imitation training is a method often used to teach learners new imitation skills.  During imitation training the learner learns to imitate the behavior of the person providing the model (Cooper, Heron, \& Heward, 2007). Reinforcement is typically delivered for imitation of the model. Prompt fading is often used.  

\subsection{Examples} Modeling a new behavior
\begin{enumerate}
\item Example: A therapist sits across from his student Zeke in the preschool classroom.  The therapist prompts the student to ``do this,'' and claps her hands.  Zeke responds within two seconds of the therapist's prompt and claps.  The therapist responds by adding a token to Zeke's token board.
\item Example:  During job training, the job coach, Flora, demonstrates to her new employee, Fauna, how to turn the copy machine off at the end of the workday.  Fauna then immediately flicks the switch, demonstrating that she understands the expectation.  Flora praises Fauna for paying attention.  
\item Non-example: A therapist sits across from his student Zeke in the preschool classroom.  The therapist prompts the student to ``do this,'' and claps her hands.  Zeke does not imitate the model and reinforcement is not delivered.  Later during the day Zeke excitedly claps his hands when his favorite song is playing on the t.v.
\end{enumerate}
%
\subsection{Assessment}
\begin{enumerate}
\item Ask your Supervisee to define both ``imitation'' and ``model.''
\item Ask your Supervisee to state the purpose of imitation training.
\item Ask your supervisee to give several examples of some new skills that one might want to select for imitation training (other than the ones listed above).
\end{enumerate}
%
\subsection{Relevant Literature}
\begin{refsection}
\nocite{baer1967development,
    cooper2007applied,
    striefel1974transfer}
\printbibliography[heading=none]
\end{refsection}
%
\subsection{Related Tasks}
\fourdFour{}\\
