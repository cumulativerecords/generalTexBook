\clearpage \section{\fourFKTen{}}
\subsection{Definition}  
Behavior - ``The activity of living organisms; human behavior includes everything that people do'' (Cooper, Heron, \& Heward, 2007, p. 25).

Response - A single instance or occurrence of a specific class or type of behavior'' (Cooper, Heron, \& Heward, 2007, p. 703).  

Response Class - ``A group of responses of varying topography, all of which produce the same effect on the environment'' (Cooper, Heron, \& Heward, 2007, p. 703).  
%
\subsection{Examples}
Opening a door
\begin{enumerate}
\item Example (behavior):  Beezus stands up from her chair, heads towards a closed door, and pushes on it.  The door is now open.  
\item Example (response): When encountering a closed door, Beezus extends an open palm and pushes on it.  The door is now open.   
\item Example (response class): Beezus encounters many doors during her day. Sometimes she opens them with an open palm and sometimes with a closed palm. Some times she opens them with her left hand and at other times with her right hand.  No matter which way she chooses to open a door, the result is always the same.  
%
\end{enumerate}
%
\subsection{Assessment}
\begin{enumerate}
\item Ask your supervisee to define behavior, response, and response class. 
\item Ask your supervisee to identify the behavior(s), responses, and response class from the above examples.  
\item Ask your supervisee to create other examples and non-examples of his/her own. 
\item Have your supervisee describe how these three terms are related. 
%
\end{enumerate}
%
\subsection{Relevant Literature}
\begin{refsection}
\nocite{cooper2007applied,
        johnston1993strategies,
        michael2004concepts,
        skinner1969contingencies}
\printbibliography[heading=none]
\end{refsection}
%
\subsection{Related Lessons}
\fourFKThirtySix{}\\
\fourgFive{}\\
\fouriOne{}\\
%
\clearpage \section{\fourFKFourtyFour{}}
\subsection{Definition}
Mand - ``An elementary verbal operant that is evoked by an MO and followed by specific reinforcement'' (Cooper, Heron, \& Heward, 2007, p. 699).\\

The form of the response is specific and under control of motivating operations. Response topography can vary: vocal, sign language, augmentative communication, pushing, reaching, hitting, etc.
%
\subsection{Examples}
\begin{enumerate}
\item ``I want a cookie.'' (This is a mand for an item. Mands can include verbs, use of adjectives, prepositions, pronouns etc.)
\item A child says ``watch me'' after learning how to ride a bike independently (mand for attention)
\item Asking questions like ``what's your name? or ``where's the phone?'' (mand for information)
\item Child says, ``No!'' when parent is about to use blender (mand for avoidance of an aversive)
%

%
\end{enumerate}
%
\subsection{Assessment}
\begin{enumerate}
\item Ask your Supervisee to recall how they asked for supervision
\item Ask your Supervisee to list the types of mands they would emit if they were lost in a foreign county and needed directions to a local gas station
\item Ask you supervisee to list 5 ways they use mands in an inappropriate way (eg. complain about work to get attention)
%
\end{enumerate}
%
\subsection{Relevant Literature}
\begin{refsection}
\nocite{cooper2007applied,
        laraway2003motivating,
        michael1988establishing,
        sundberg2001benefits,
        sweeney2007transferring}
\printbibliography[heading=none]
\end{refsection}
%
\subsection{Related Lessons}
\fourdNine{}\\
\fourdEleven{}\\
\fourFKTwentySeven{}\\
\fourFKTwentyEight{}\\
