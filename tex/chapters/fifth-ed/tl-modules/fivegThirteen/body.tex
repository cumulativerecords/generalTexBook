%\clearpage \section{\fourFKThirtyNine{}}
\subsection{Definition}
Behavioral momentum - ``A metaphor to describe a rate of responding and its resistance to change following an alteration in reinforcement conditions'' (Cooper, Heron, \& Heward, 2007, p. 691).\\

``In classical physics, momentum is defined as the product of velocity and mass. Translating metaphorically, behavioral momentum is the product of response rate and resistance to change'' (Nevin, 1992, p. 302).\\

Response rate had been used as the measure of response rate for many years. With the introduction of behavioral momentum, Nevin challenges this and describes resistance to change as a better way to measure response strength (Nevin, 1974).\\

Behavioral momentum is not synonymous with the high-p request sequence. Use caution when describing behavioral momentum this way (Nevin, 1996).\\
%
\subsection{Examples}
\begin{enumerate}
\item ``If you are working at the computer, and you keep working even though you are called to dinner, that is an example of behavioral momentum'' (Pierce \& Cheney, 2013, p. 134).
\item A student is coloring at his desk. The teacher asks him to come down to the rug to listen to a story. He continues to color for a few more seconds.
\item James is watching TV. His remote stops working. He continues to push the button despite the battery being dead.
%
\end{enumerate}
%
\subsection{Assessment}
\begin{enumerate}
\item Ask your supervisee to describe behavioral momentum
\item Ask your supervisee to give an example of behavioral momentum.
%
\end{enumerate}
%
\subsection{Relevant Literature}
\begin{refsection}
\nocite{brandon1997applying,
        cooper2007applied,
        nevin1974response,
        nevin1992integrative,
        nevin1996momentum,
        nevin1983analysis}
\printbibliography[heading=none]
\end{refsection}
%
\subsection{Related Lessons}
\foureNine{}\\
\fourFKTen{}\\
%
%\clearpage \section{\foureNine{}}
\subsection{Definition}
Using a high-probability (high-p) request sequence involves presenting a series of requests that the individual has a history of following before presenting a target request (low-p).  The following is recommended in order to maximize the effectiveness of the high-p request sequence: (1) tasks/requests should already be in the learner's repertoire (i.e., the skill is considered mastered); (2) high-p requests should be presented rapidly; (3) the first low-p request should be presented immediately after reinforcement for high-p compliance; and (4) and salient reinforcers should be used for low-p requests (Cooper, Heron, \& Heward, 2007).  It is also recommended to avoid using low-difficulty tasks immediately after a maladaptive behavior that was triggered by a high-difficulty task (Sailors, Guess, Rutherford, \& Baer, 1968).  In the initial stages of acquisition of a low-p request, increasing the number of high-p requests increases the effectiveness of the high-p request sequence (Mace, 1996).  High-p request sequences may be helpful in reducing the reinforcing value of escape from requests and the maladaptive behaviors that often occur when low-p requests are presented. (Cooper, Heron, \& Heward, 2007)

There is a common misconception that the effects of the high-p sequence are related exclusively to the repeated delivery of the high-p demands. The dense schedule of reinforcement is also a necessary component for this procedure. Zuluaga \& Normand (2008) tested the effects of the high-p sequence without reinforcement versus the high-p sequence with reinforcement. Compliance with low-p demands increased only when reinforcers were delivered after the high-p requests.

There are some discussions in the field about whether the high-p request sequence and behavioral momentum should be considered synonymous. In 1996, Nevin describes his concerns by suggesting that ``translating the terms of the metaphor into the high-p procedure, or indeed any other application, encounters some uncertainties and entails a fair amount of speculation; thus, alternative accounts are surely possible'' (p.554).
\subsection{Assessment}
\begin{enumerate}
\item Ask supervisee to demonstrate implementing a high-p request sequence with a client.  Supervisor should provide modeling and feedback as necessary.
\item Require that the supervisee reads and summarizes Zululaga \& Normand (2008).
\item Ask supervisee to describe why Nevin suggests that the high-p request sequence and behavioral momentum should not be considered synonymous.
\end{enumerate}
%
\subsection{Relevant Literature}
\begin{refsection}
\nocite{cooper2007applied,
        davis1996variant,
        mace1996pursuit,
        mace1990behavioral,
        nevin1996momentum,
        sailor1968control}
\printbibliography[heading=none]
\end{refsection}
%
\subsection{Related Lessons}
\fourdOne{}\\
\fourdTwo{}\\
\foureOne{}\\
\foureTen{}\\
