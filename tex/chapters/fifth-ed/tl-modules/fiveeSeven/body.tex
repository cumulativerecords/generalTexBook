\clearpage \section[\fourgSix{}]{\fourgSix{}%
              \sectionmark{G-06 Provide behavior-analyt...}}
\sectionmark{G-06 Provide behavior-analyt...}
\subsection{Definition}
Content area 2.03 (a) (Behavior Analysts' Responsibility to Clients) of the professional and ethical compliance code for behavior analysts states that, ``When indicated and professionally appropriate, behavior analysts cooperate with other professionals, in a manner that is consistent with the philosophical assumptions and principles of behavior analysis, in order to effectively and appropriately serve their clients.'' In other words, it is our ethical responsibility to collaborate and communicate with all service providers and other individual stakeholders if it will best service our clients.
%
\subsection{Examples}
\begin{enumerate}
\item Mary has been exhibiting aggression during sessions with her occupational therapist whenever a novel task is presented. The BCBA has come up with a general behavior support plan for all staff working with Mary to decrease these behaviors. The plan involves multiple options that all are relevant to the addressing the function of the aggression. The BCBA sets up a training to discuss and explain the proper implementation of these intervention strategies, and get feedback from other service providers such as the occupational therapist to increase the social validity of the treatment and increase the likelihood of treatment fidelity.
%
\end{enumerate}
%
\subsection{Assessment}
\begin{enumerate}
\item Have supervisee identify and list all caregivers and professionals who may provide services to the individuals on their case load. Have supervisee list ways that they can increase communication with these individuals to ensure the most effective interventions and treatment of their clients.
\item Have supervisee discuss how to effectively disseminate intervention information and train others on behavior analytic techniques, while using non-technical language that other service providers can comprehend and will be able to implement. 
\item Discuss behavioral skills training and how this could be an effective tool to train others in the implementation of behavior analytic procedures. 
%
\end{enumerate}
%
\subsection{Relevant Literature}
\begin{refsection}
\nocite{cooper2007applied,
        kelly2013collaborative,
        bac2014professional}
\printbibliography[heading=none]
\end{refsection}
%
\subsection{Related Lessons}
\fourgFour{}\\
\fourgSeven{}\\
\fourgEight{}\\
\fouriSix{}\\
\fourjEleven{}\\
\fourjFourteen{}\\
\fourkThree{}\\
\fourkEight{}\\
\fourkNine{}\\
