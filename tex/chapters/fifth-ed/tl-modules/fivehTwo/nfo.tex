%\clearpage \section[\fouriSix{}]{\fouriSix{}%
\subsection{Definition}
Hawkins (1984, p. 284) (cited from Cooper, Heron \& Heward, 2007, p. 56) defined habilitation as ``the degree to which the person's repertoire maximizes short and long term reinforcers for that individual and for others, and minimizes short and long term punishers.''

When determining what behaviors to target, one can use the relevance of behavior rule (Ayllon and Azrin, 1968) as a guide. This rule states that a target behavior should only be selected if it is likely to produce reinforcement for the client in their natural environment. Another key factor is deciding if the behavior will generalize to other settings and be sustainable once the behavior change program has ended. 

Cooper et al. (2007) provide some considerations when choosing a target behavior to increase, decrease, or maintain. These include:
\begin{enumerate}
\item Does this behavior pose any danger to the client or others?
\item How many opportunities will the person have to use this new behavior? Or how often does this problem behavior occur?
\item How long-standing is the problem or skill deficit?
\item Will changing the behavior produce higher rates of reinforcement for the person?
\item What will be the relative importance of this target behavior to the future skill development and independent functioning?
\item Will changing this behavior reduce negative attention from others?
\item Will the new behavior produce reinforcement for significant others?
\item How likely is success in changing this target behavior?
\item How much will it cost to change this behavior?
\end{enumerate}
%
