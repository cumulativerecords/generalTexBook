%\clearpage \section{\fourFKThirtyFour{}}
\subsection{Definition}
Conditional discrimination – ``refer to a concept related to stimulus equivalence and can be created using matching-to-sample procedures.  Conditional discriminations operate within a four-term contingency that accounts for the environmental context, such that the contingency appears as the following: contextual stimulus $\rightarrow$ SD $\rightarrow$ response $\rightarrow$ reinforcement'' (Cooper, Heron, \& Heward, 2007, p. 400).\\

The contextual events operating within this four-term contingency become conditional discriminations (Sidman, 1994).   Moreover, a specific conditional discrimination implies that there is a specific conditional relation, referring to the direct outcome of the reinforcement contingency (Carrigan \& Sidman, 1992).  Different conditional relations are reflected in the properties of stimulus equivalence (e.g., reflexivity, symmetry, and transitivity).  In order for new conditional discriminations to emerge, at least one conditional discrimination must be directly trained.  This initial discrimination is often arbitrary (e.g., training A > B implies that B $<$ A). 
%
%\subsection%\clearpage \section{\fourFKThirtyFive{}}
%\subsection{Definition} 
Stimulus discrimination – ``is when a response consistently occurs in the presence of a specific or controlling antecedent stimulus and not in the presence of new or related stimuli.  This is in direct contrast to stimulus generalization in which related antecedent stimuli may evoke the same response'' (Cooper, Heron, \& Heward, 2007, pp. 395-396).    
%
%\clearpage \section{\fourFKThirtySix{}}
%\subsection{Definition} 
Response generalization - ``The extent to which a learner emits untrained responses that are functionally equivalent to the trained target behavior'' (Cooper, Heron, \& Heward, 2007, p. 620).\\
%
``Improvements in behavior are most beneficial when they are long lasting, appear in other appropriate environments, and spill over to other related behaviors... When evaluating applied behavior analysis research, consumers should consider the maintenance and generalization of behavior change in their evaluation of a study'' (Cooper et al., 2007, p. 250).\\
%
%\clearpage \section{\fourFKThirtySeven{}}
\subsection{Definition} 
Stimulus generalization - ``When an antecedent stimulus has a history of evoking a response that has been reinforced in its presence, the same type of behavior tends to be evoked by stimuli that share similar physical properties with the controlling antecedent stimulus'' (Cooper, Heron, \& Heward, 2007, p. 705).
%
