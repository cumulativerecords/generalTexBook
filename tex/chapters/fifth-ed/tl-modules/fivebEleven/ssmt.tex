\subsection{Assessment}
\begin{enumerate}
\item Conditional discriminations can be a difficult concept to understand without having adequate knowledge of stimulus control, stimulus equivalence, and matching-to-sample procedures.  Therefore, in order to assess for learning, the supervisee should be asked to define and provide examples of conditional discriminations within the context of explaining stimulus control, stimulus equivalence, and matching-to-sample procedures.
\item Once the supervisee can explain conditional discriminations, applied knowledge can be assessed by asking supervisee to demonstrate direct training of conditional discriminations by using matching-to-sample and stimulus equivalence procedures.  The supervisor will model these procedures, have supervisee demonstrate the skill and provide feedback based on their performance.  
%
\end{enumerate}
%
%\subsection{Assessment}
\begin{enumerate}
\item Ask supervisee to identify natural examples of stimulus discrimination.
\item Ask supervisee to compare and contrast stimulus discrimination and stimulus generalization.
\item Observe supervisee use stimulus discrimination procedures with clients.
\end{enumerate}
%
%\subsection{Assessment}
\begin{enumerate}
\item Have supervisee provide examples of response generalization.
\item Provide examples of both response generalization and stimulus generalization and have supervisee indicate which type of generalization the example is referring to and describe why.
\item Have supervisee describe why response generalization is important when assessing behavior change and skill acquisition.
%
\end{enumerate}
%
%\subsection{Assessment}
Using the example above:
\begin{enumerate}
\item Have intern identify the difference between response generalization and stimulus generalization.
\item Have intern identify the difference between stimulus generalization and stimulus discrimination. 
\item Have intern identify the qualities of each and give examples.
\item Have intern give examples from their workplace of stimulus generalization.
\end{enumerate}