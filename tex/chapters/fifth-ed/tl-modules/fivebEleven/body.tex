\clearpage \section{\fourFKThirtyFour{}}
\subsection{Definition}
Conditional discrimination – ``refer to a concept related to stimulus equivalence and can be created using matching-to-sample procedures.  Conditional discriminations operate within a four-term contingency that accounts for the environmental context, such that the contingency appears as the following: contextual stimulus $\rightarrow$ SD $\rightarrow$ response $\rightarrow$ reinforcement'' (Cooper, Heron, \& Heward, 2007, p. 400).\\

The contextual events operating within this four-term contingency become conditional discriminations (Sidman, 1994).   Moreover, a specific conditional discrimination implies that there is a specific conditional relation, referring to the direct outcome of the reinforcement contingency (Carrigan \& Sidman, 1992).  Different conditional relations are reflected in the properties of stimulus equivalence (e.g., reflexivity, symmetry, and transitivity).  In order for new conditional discriminations to emerge, at least one conditional discrimination must be directly trained.  This initial discrimination is often arbitrary (e.g., training A > B implies that B $<$ A). 
%
\subsection{Examples}
\begin{enumerate}
\item In a matching-to-sample trial, ``A'' is given as the conditional sample and is then presented in a series of five letters ``z,'' ``k,'' ``a,'' ``t,'' and ``b.''  By correctly matching ``A'' to ``a'' the learner is discriminating stimuli and selecting the correct comparison that will be reinforced.  The learner's response will not be reinforced if a stimulus other than ``a'' is selected.  In this example, the conditional sample ``A'' reflects the contextual stimulus as it creates a context for which stimulus to discriminate.  
%
\end{enumerate}
%
\subsection{Assessment}
\begin{enumerate}
\item Conditional discriminations can be a difficult concept to understand without having adequate knowledge of stimulus control, stimulus equivalence, and matching-to-sample procedures.  Therefore, in order to assess for learning, the supervisee should be asked to define and provide examples of conditional discriminations within the context of explaining stimulus control, stimulus equivalence, and matching-to-sample procedures.
\item Once the supervisee can explain conditional discriminations, applied knowledge can be assessed by asking supervisee to demonstrate direct training of conditional discriminations by using matching-to-sample and stimulus equivalence procedures.  The supervisor will model these procedures, have supervisee demonstrate the skill and provide feedback based on their performance.  
%
\end{enumerate}
%
\subsection{Relevant Literature}
\begin{refsection}
\nocite{bush1989contextual,
        carrigan1992conditional,
        cooper2007applied,
        johnson1993conditional,
        sidman1994equivalence}
\printbibliography[heading=none]
\end{refsection}
%
\subsection{Related Tasks}
\foureSix{}\\
\fourFKEleven{}\\
\fourFKTwentyFour{}\\
\fourFKThirtyFive{}\\
\fourFKThirtySeven{}\\
\fourjFourteen{}\\
%
\clearpage \section{\fourFKThirtyFive{}}
\subsection{Definition} 
Stimulus discrimination – ``is when a response consistently occurs in the presence of a specific or controlling antecedent stimulus and not in the presence of new or related stimuli.  This is in direct contrast to stimulus generalization in which related antecedent stimuli may evoke the same response'' (Cooper, Heron, \& Heward, 2007, pp. 395-396).    
%
\subsection{Examples}
\begin{enumerate}
\item During a tooth brushing routine, a child selects their tooth brush from the cup that holds a number of toothbrushes.
\item You own a small white car and can walk directly to your car after leaving the mall even though there are several small white cars parked around your car.
\item You are taking three graduate level ABA classes.  Two of the professors keep track of attendance and incorporate that into your final grade and one professor does not track attendance or incorporate that into your final grade.  As a result, you periodically miss this class since that your grade will not be lowered due to attendance and you regularly attend the other two classes. 
%
\end{enumerate}
%
\subsection{Assessment}
\begin{enumerate}
\item Ask supervisee to identify natural examples of stimulus discrimination.
\item Ask supervisee to compare and contrast stimulus discrimination and stimulus generalization.
\item Observe supervisee use stimulus discrimination procedures with clients.
%
\end{enumerate}
%
\subsection{Relevant Literature}
\begin{refsection}
\nocite{cooper2007applied,
        green2001behavior}
\printbibliography[heading=none]
\end{refsection}
%
\subsection{Related Tasks}
\foureSix{}\\
\fourFKEleven{}\\
\fourFKTwentyFour{}\\
\fourFKThirtyFour{}\\
\fourFKThirtySeven{}\\
%
\clearpage \section{\fourFKThirtySix{}}
\subsection{Definition} 
Response generalization - ``The extent to which a learner emits untrained responses that are functionally equivalent to the trained target behavior'' (Cooper, Heron, \& Heward, 2007, p. 620).\\
%
``Improvements in behavior are most beneficial when they are long lasting, appear in other appropriate environments, and spill over to other related behaviors... When evaluating applied behavior analysis research, consumers should consider the maintenance and generalization of behavior change in their evaluation of a study'' (Cooper et al., 2007, p. 250).\\
%
\subsection{Examples}
\begin{enumerate}
\item A young child learns to open a door at their house by turning a door knob. One day, while at a friend's house, they encounter a door that has a handle rather than a knob. The child is able to turn the handle and open the unfamiliar door. This is an example of response generalization because functional both responses are equal (they open result in the door being opened) but the response topographies are different.
%
\end{enumerate}
%
\subsection{Assessment}
\begin{enumerate}
\item Have supervisee provide examples of response generalization.
\item Provide examples of both response generalization and stimulus generalization and have supervisee indicate which type of generalization the example is referring to and describe why.
\item Have supervisee describe why response generalization is important when assessing behavior change and skill acquisition.
%
\end{enumerate}
%
\subsection{Relevant Literature}
\begin{refsection}
\nocite{baer1968some,
        cooper2007applied,
        fantuzzo1981generalization,
        goetz1973social,
        sidman1994equivalence}
\printbibliography[heading=none]
\end{refsection}
%
\subsection{Related Tasks}
\fourbOne{}\\
\foureSix{}\\
\foureEleven{}\\
\fouriOne{}\\
\fouriTwo{}\\
\fourjEleven{}\\
\fourjTwelve{}\\
\fourjFourteen{}\\
\fourkNine{}\\
\fourFKTen{}\\
\fourFKEleven{}\\
\fourFKTwelve{}\\
\fourFKThirtySeven{}\\
%
\clearpage \section{\fourFKThirtySeven{}}
\subsection{Definition} 
Stimulus generalization - ``When an antecedent stimulus has a history of evoking a response that has been reinforced in its presence, the same type of behavior tends to be evoked by stimuli that share similar physical properties with the controlling antecedent stimulus'' (Cooper, Heron, \& Heward, 2007, p. 705).
%
\subsection{Examples}
\begin{enumerate}
\item  A child says, ``mommy'' in the presence of her mother, but also says ``mommy'' when she sees her grandmother or daycare provider.
%
\end{enumerate}
%
\subsection{Assessment}
Using the example above:
\begin{enumerate}
\item Have intern identify the difference between response generalization and stimulus generalization.
\item Have intern identify the difference between stimulus generalization and stimulus discrimination. 
\item Have intern identify the qualities of each and give examples.
\item Have intern give examples from their workplace of stimulus generalization.
\end{enumerate}
%
\subsection{Relevant Literature}
\begin{refsection}
\nocite{baer1968some,
        cooper2007applied,
        cuvo2003stimulus,
        guttman1956discriminability,
        johnston1979relation,
        stokes1977implicit}
\printbibliography[heading=none]
\end{refsection}
%
\subsection{Related Tasks}
\fourbOne{}\\
\foureSix{}\\
\fourjEleven{}\\
\fourjFourteen{}\\
\fourFKTen{}\\
\fourFKEleven{}\\
\fourFKTwelve{}\\
\fourFKThirtySix{}\\
