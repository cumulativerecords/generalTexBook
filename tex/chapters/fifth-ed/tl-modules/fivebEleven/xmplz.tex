\subsection{Examples}
\begin{enumerate}
\item In a matching-to-sample trial, ``A'' is given as the conditional sample and is then presented in a series of five letters ``z,'' ``k,'' ``a,'' ``t,'' and ``b.''  By correctly matching ``A'' to ``a'' the learner is discriminating stimuli and selecting the correct comparison that will be reinforced.  The learner's response will not be reinforced if a stimulus other than ``a'' is selected.  In this example, the conditional sample ``A'' reflects the contextual stimulus as it creates a context for which stimulus to discriminate.  
\end{enumerate}
%
%\subsection{Examples}
\begin{enumerate}
\item During a tooth brushing routine, a child selects their tooth brush from the cup that holds a number of toothbrushes.
\item You own a small white car and can walk directly to your car after leaving the mall even though there are several small white cars parked around your car.
\item You are taking three graduate level ABA classes.  Two of the professors keep track of attendance and incorporate that into your final grade and one professor does not track attendance or incorporate that into your final grade.  As a result, you periodically miss this class since that your grade will not be lowered due to attendance and you regularly attend the other two classes. 
\end{enumerate}
%
%\subsection{Examples}
\begin{enumerate}
\item A young child learns to open a door at their house by turning a door knob. One day, while at a friend's house, they encounter a door that has a handle rather than a knob. The child is able to turn the handle and open the unfamiliar door. This is an example of response generalization because functional both responses are equal (they open result in the door being opened) but the response topographies are different.
%
\end{enumerate}
%
%\subsection{Examples}
\begin{enumerate}
\item  A child says, ``mommy'' in the presence of her mother, but also says ``mommy'' when she sees her grandmother or daycare provider.
\end{enumerate}