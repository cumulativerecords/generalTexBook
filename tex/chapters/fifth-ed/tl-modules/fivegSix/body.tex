%\clearpage \section{\foureThree{}}
\subsection{Definition}
Rules are descriptions of behavioral contingencies (e.g., ``Putting a sweater on when it is cold outside will help you stay warm'').  When rules are followed, behavior can come under the control of delayed or indirect consequences therefore resulting in rule-governed behavior.  Malott and Trojan-Suarez (2003) suggest that all instructions involve rules.  For example, incomplete rules (e.g., ``Stop it'') provide minimal instruction (e.g., stop) and imply an outcome (e.g., you might get in trouble).   It is argued that rules function as reinforcement-based or punishment-based discriminative stimuli (Malott \& Trojan-Suarez, 2003). Skinner (1969) referred to rules, instructions, advice, and laws as contingency-specifying stimuli, describing the  relations of everyday life.

Evidence that behavior is the result of instructional control or rule following is provided if: (1) there is no obvious or immediate consequence of the behavior; (2) the delivery of the consequence following the behavior exceeds 30 seconds; (3) behavior changes without reinforcement; (4) a substantial increase in the rate of behavior occurs following one instance of direct contact with reinforcement; and (5) the rule exists but no consequence (including automatic reinforcement) exists following the behavior (Cooper, Heron, \& Heward, 2007).
%
\subsection{Assessment}
\begin{enumerate}
\item Ask supervisee to discriminate between direct-acting contingencies and rule-governed behavior.
\item Ask supervisee to provide examples of rules.
\item Ask supervisee to identify rules that may be governing a client's behavior.
\item If a rule exists, ask supervisee to describe how a direct-acting contingency can be used instead and vice versa.
\end{enumerate}
%
\subsection{Relevant Literature}
\begin{refsection}
\nocite{cooper2007applied,
        hayes2004rule,
        malott2003principles,
        skinner1969contingencies}
\printbibliography[heading=none]
\end{refsection}
%
\subsection{Related Lessons}
\fourdOne{}\\
\fourdSixteen{}\\
\fourkTwo{}\\
\fourFKThirty{}\\
\fourFKThirtyOne{}\\
\fourFKThirtyThree{}\\
\fourFKFourtyOne{}\\
\fourFKFourtyTwo{}\\
