\clearpage \section{\fourdEighteen{}}
\subsection{Definition}
Extinction ``involves eliminating the reinforcement contingency maintaining a response'' and has ``been used successfully to reduce the frequency of a variety of behavioral disorders'' (Lerman \& Iwata, 1995, p. 93). To use it effectively, determine the reinforcer and prevent the behavior from contacting reinforcement. 

Extinction occurs when the behavior no longer produces reinforcement such as:
\begin{itemize}
\item Social Positive
\item Social Negative
\item Automatic reinforcement 
\end{itemize}

Extinction may have unwanted side effects:
\begin{enumerate}
\item Extinction bursts
\item Initial increases in target behavior as the individual engages in an increased rate of behavior to access previous reinforcers
\item Extinction-induced aggression
\item Increased unwanted novel behavior
\end{enumerate}

To use extinction effectively, withhold all reinforcers maintaining the problem behavior and provide frequent opportunities for the individual's behavior to contact new reinforcement contingencies (e.g. problem behavior no longer produces reinforcement while replacement behavior produces does produce reinforcement). It may not work if the client aggresses for attention (e.g., response blocking may reinforce aggression).

Extinction is not advised in environments where peers are likely to imitate problem behaviors (e.g., classroom settings). Use of extinction for severe aggression or self-injury behaviors may result in harm to client or peers. Extinction is not selected to manage severely disruptive high-frequency behavior.

Do not use extinction as a singular intervention. Unwanted effects of extinction are reduced when coupled with differential reinforcement strategies (Athens \& Vollmer, 2010).
%
\subsection{Examples}
\begin{enumerate}
\item Teach client to raise hand for attention (differential reinforcement strategy) while putting spitting for attention on extinction. Give client access to previous reinforcers to minimize aversive effects of extinction.
\end{enumerate}
%
\subsection{Assessment}
\begin{enumerate}
\item Provide hypothetical scenarios and have supervisee determine what type of extinction procedure should be utilized and what appropriate replacement behavior should be reinforced. 
\item Have supervisee explain 3 scenarios in which extinction should not be utilized. 
\item Have supervisee describe additional treatment components which can increase the efficacy of treatment.
\end{enumerate}
%
\subsection{Relevant Literature}
\begin{refsection}
\nocite{lerman1995prevalence,
    athens2010investigation}
\printbibliography[heading=none]
\end{refsection}
%
\subsection{Related Tasks} 
\fourcThree{}\\
\fourdEighteen{}\\
\fourdNineteen{}\\
\fourjSix{}\\
\fourjSeven{}\\
\fourjNine{}\\
\fourjTen{}\\
