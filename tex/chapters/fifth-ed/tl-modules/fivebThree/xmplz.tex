\subsection{Examples}
\begin{enumerate}
\item Air blowing in eye (US) $\rightarrow$ blinking (UR)
\item Cold/low temperature (US) $\rightarrow$ shivering (UR)
\item Hot/high temperature (US) $\rightarrow$ sweating (UR)
\item Food in mouth (US) $\rightarrow$ salivation (UR)
\item Hot surface (US) $\rightarrow$ move hand away (UR)
%
\end{enumerate}
%
\subsection{Examples}
\begin{enumerate}
\item Roger usually drinks soda every day for lunch. When drinking soda, the sugar (US) inside his blood elicits the release of insulin from his pancreas (UR). Now, when he opens the soda, the snap of the can (CS) starts to elicit the release of insulin (CR) before he takes a drink. 
%
\end{enumerate}
%
\subsection{Examples}
\begin{enumerate}
\item   A rat is deprived of food. The rat walks near a specific part of their cage and receives food. As a result, the future probability of the rat walking toward that area of the cage increases.
\item A child hits their sibling when fighting over a toy. The child is sent to timeout. As a result, the future probability of hitting their sister decreases.
%
\end{enumerate}
%
\subsection{Examples}
\begin{enumerate}
\item  After a traumatic event involving physical abuse, every time a male walks into the room, your client ``freezes'' and does not follow instructions. This could be due to elicited behavior (``freezing'' in the presence of conditioned aversive stimuli) in competition with operant behavior (following instructions).
\item A medication, when consumed, will elicit respondent behavior that makes certain things more or less aversive. Consider if your client starts taking a medication to decrease aggression maintained by access to toys. The effect of the medication may decrease the likelihood that toys function as a reinforcer in effect decreasing the amount of aggression. It may increase the likelihood that food functions as a reinforcer.
\end{enumerate}
%
