\clearpage \section[\fourdTwo{}]{\fourdTwo{}%
              \sectionmark{D-02 Use appr... reinforcement}}
\sectionmark{D-02 Use appr... reinforcement}
\subsection{Definition}
Schedule of reinforcement - ``A rule specifying the environmental arrangements and response requirements for reinforcement; a description of a contingency of reinforcement'' (Cooper, Heron \& Heward, 2007, p. 703).

There are two basic schedules of reinforcement: continuous (CRF) and intermittent (INT). CRF is useful when teaching a new response. With CRF, reinforcement is provided each time the target behavior occurs. As a result, the desired behavior is strengthened. INT is used for thinning schedules of reinforcement and transitioning to naturally occurring reinforcement contingencies. 

INT may be defined as having a fixed ratio schedule, variable ratio schedule, fixed interval schedule, or variable interval schedule. When learning basic schedules of reinforcement, it is not only important to understand how the schedules are defined, but also understand the effects of each type of INT.  Ferster and Skinner (1957) and Cooper et al. (2007) thoroughly discuss these concepts and it is strongly encouraged that supervisors and supervisees read this material. The table below serves as a reference and briefly illustrates the key points of basic INT.
%
\subsection{Assessment}
\begin{enumerate}
\item Ask supervisee to define basic schedules of reinforcement.
\item Ask supervisee to provide examples of basic schedules of reinforcement.
\item Supervisor can graph response rates of INT schedules and ask supervisee to analyze graphs and identify which INT schedule is in place.
\end{enumerate}
%
\subsection{Relevant Literature}
\begin{refsection}
\nocite{cooper2007applied,ferster1957schedules}
\printbibliography[heading=none]
\end{refsection}
%
%
\subsection{Related Tasks}
\fourdNineteen{}\\
\fourdTwenty{}\\
\fourdTwentyOne{}\\
\fourFKTwentyOne{}\\
\fourFKFourty{}\\
\fourFKFourtyOne{}\\
%
\clearpage \section{\fourFKTwentyOne{}}
\subsection{Definition}
Schedule of reinforcement/punishment – ``rule that specifies the environmental arrangements and response requirements that will result in reinforcement or punishment'' (Cooper, Heron \& Heward, 2007, p. 703).

Related Definitions:\\

A continuous schedule of reinforcement: reinforcement is given for each occurrence of behavior (also known as a 1:1 schedule of reinforcement or CRF).

A continuous schedule of punishment: punishment is given after each occurrence of behavior (also known as 1:1 schedule of punishment).
 
An intermittent schedule of reinforcement: reinforcement is given after some, but not all occurrences of behavior.

\subsection{Examples}
Types of intermittent schedules of reinforcement are listed with examples below.
\begin{enumerate}
\item A fixed ratio schedule of reinforcement: Requires a completion of a specified number of responses to gain access to reinforcement. Example: A student may have to complete 5 correct math problems on a computer game before progressing to the next level (this is a FR-5 schedule of reinforcement).
\item A variable ratio schedule of reinforcement: Requires the completion of a varied number of responses to gain access to reinforcement. Example: A young girl may be called on when she raises her hand quietly in class on average once every 5 times. Sometimes, the teacher calls on her every 4 times she raises her hand. Other times the teacher calls on her every 6 times she raises her hand. The teacher provides attention on average every 5 times (this is a VR-5 schedule of reinforcement).
\item A fixed interval schedule of reinforcement: provides reinforcement for the first response after a fixed duration of time. Example: An alarm clock is set for 7:00 am every morning. If an individual presses snooze, it will allow the individual to sleep in again for 10 minutes. The individual cannot press snooze before the alarm rings (this is a FI-10 minute schedule of reinforcement).
\item A variable interval schedule of reinforcement: produces reinforcement for the first response after a variable duration of time. Example: A person goes to a fast food restaurant. Sometimes he has to stand in line, while other times, he may order immediately upon entering. This interval varies each time he goes to the restaurant. 
\end{enumerate}
%
\subsection{Assessment}
\begin{enumerate}
\item During a job or during role-play, have your supervisee determine what schedule of reinforcement or punishment is being used
\item Use SAFMEDS or flashcards to practice definitions related to the various types of schedules.
%
\end{enumerate}
%
\subsection{Relevant Literature}
\begin{refsection}
\nocite{cooper2007applied}
\printbibliography[heading=none]
\end{refsection}
%
\subsection{Related Tasks} 
\fourdTwo{}\\
\fourdSeventeen{}\\
