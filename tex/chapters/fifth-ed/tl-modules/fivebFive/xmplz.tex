%
%\subsection{Examples}
Types of intermittent schedules of reinforcement are listed with examples below.
\begin{enumerate}
\item A fixed ratio schedule of reinforcement: Requires a completion of a specified number of responses to gain access to reinforcement. Example: A student may have to complete 5 correct math problems on a computer game before progressing to the next level (this is a FR-5 schedule of reinforcement).
\item A variable ratio schedule of reinforcement: Requires the completion of a varied number of responses to gain access to reinforcement. Example: A young girl may be called on when she raises her hand quietly in class on average once every 5 times. Sometimes, the teacher calls on her every 4 times she raises her hand. Other times the teacher calls on her every 6 times she raises her hand. The teacher provides attention on average every 5 times (this is a VR-5 schedule of reinforcement).
\item A fixed interval schedule of reinforcement: provides reinforcement for the first response after a fixed duration of time. Example: An alarm clock is set for 7:00 am every morning. If an individual presses snooze, it will allow the individual to sleep in again for 10 minutes. The individual cannot press snooze before the alarm rings (this is a FI-10 minute schedule of reinforcement).
\item A variable interval schedule of reinforcement: produces reinforcement for the first response after a variable duration of time. Example: A person goes to a fast food restaurant. Sometimes he has to stand in line, while other times, he may order immediately upon entering. This interval varies each time he goes to the restaurant. 
\end{enumerate}
%
