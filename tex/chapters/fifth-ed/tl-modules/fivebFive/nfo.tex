%\clearpage \section[\fourdTwo{}]{\fourdTwo{}%
              \sectionmark{D-02 Use appr... reinforcement}}
\sectionmark{D-02 Use appr... reinforcement}
\subsection{Definition}
Schedule of reinforcement - ``A rule specifying the environmental arrangements and response requirements for reinforcement; a description of a contingency of reinforcement'' (Cooper, Heron \& Heward, 2007, p. 703).

There are two basic schedules of reinforcement: continuous (CRF) and intermittent (INT). CRF is useful when teaching a new response. With CRF, reinforcement is provided each time the target behavior occurs. As a result, the desired behavior is strengthened. INT is used for thinning schedules of reinforcement and transitioning to naturally occurring reinforcement contingencies. 

INT may be defined as having a fixed ratio schedule, variable ratio schedule, fixed interval schedule, or variable interval schedule. When learning basic schedules of reinforcement, it is not only important to understand how the schedules are defined, but also understand the effects of each type of INT.  Ferster and Skinner (1957) and Cooper et al. (2007) thoroughly discuss these concepts and it is strongly encouraged that supervisors and supervisees read this material. The table below serves as a reference and briefly illustrates the key points of basic INT.
%
%\clearpage \section{\fourFKTwentyOne{}}
%\subsection{Definition}
Schedule of reinforcement/punishment – ``rule that specifies the environmental arrangements and response requirements that will result in reinforcement or punishment'' (Cooper, Heron \& Heward, 2007, p. 703).

Related Definitions:\\

A continuous schedule of reinforcement: reinforcement is given for each occurrence of behavior (also known as a 1:1 schedule of reinforcement or CRF).

A continuous schedule of punishment: punishment is given after each occurrence of behavior (also known as 1:1 schedule of punishment).
 
An intermittent schedule of reinforcement: reinforcement is given after some, but not all occurrences of behavior.

