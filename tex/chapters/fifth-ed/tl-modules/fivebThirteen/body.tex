\clearpage \section{\fourFKFourtyTwo{}}
\subsection{Definition} 
Rule governed behavior - ``Behavior governed by a rule (i.e., a verbal statement of an antecedent-behavior-consequence contingency), enables human behavior (e.g., fastening a seatbelt) to come under the indirect control of temporarily remote or improbable but potentially significant consequences (e.g., avoiding injury in auto accident). Often used in contrast to contingency-shaped behavior; a term used to indicate behavior selected and maintained by controlled, temporally close consequences'' (Cooper, Heron, \& Heward, 2007, p. 703).
%
\subsection{Examples}
\begin{enumerate}
\item Matilda's mother told her that if she gets all of her homework done this week that she will take her out for ice cream on Saturday. It is only Wednesday and Matilda still hasn't finished all of the work. This is a new contingency for Matilda, but when Saturday arrived, all of her homework was done.
\item Felicia has never put a metal fork in the toaster before but her dad told her that she could be electrocuted if she tried to get her bagel out that way. She always gets a wooden spoon to get the bagel out.
\item Non-example: Matilda's mother told her that if she gets all of her homework done this week that she will take her out for ice cream on Saturday. It is only Wednesday and Matilda still hasn't finished all of the work. Her mother brings her to get ice cream every week that she gets her homework done. When Saturday arrived, all of her homework was done.
%
\end{enumerate}
%
\subsection{Assessment}
\begin{enumerate}
\item Ask your supervisee to explain rule-governed behavior 
\item Ask your supervisee how rule-governed behavior is different from contingency-shaped behavior
\item Ask your supervisee to create another example and non-example of his/her own.  
\end{enumerate}
%
\subsection{Relevant Literature}
\begin{refsection}
\nocite{cooper2007applied,
        malott1988rule,
        malott1991role,
        schlinger1990reply,
        skinner1969contingencies}
\printbibliography[heading=none]
\end{refsection}
%
\subsection{Related Tasks}
\foureFour{}\\
\fourFKFourtyOne{}\\
%
\clearpage \section{\fourFKThirtyOne{}}
\subsection{Definition}
``The AB because of C formulation is a general statement that the relation between an event (B) and its context (A) is because of consequences (C)... Applied to Skinner's three-term contingency, the relation between (A) the setting and (B) behavior exists because of (C) consequences that occurred for previous AB (setting-behavior) relations. The idea [is] that reinforcement strengthens the setting-behavior relation rather than simply strengthening behavior'' (Moxley, 2004, p. 111).
``The three term contingency- antecedent, behavior, and consequence- is sometimes called the ABC's of behavior analysis... The term contingency has several meanings signifying various types of temporal and functional relations between behavior and antecedent and consequent variables... When a reinforcer (or punisher) is said to be contingent on a particular behavior, the behavior must be emitted for the consequence to occur'' (Cooper, Heron, \& Heward, 2007, pp. 41-42).
%
\subsection{Examples}
\begin{enumerate}
\item When John has not eaten in a while he asks his caregiver for a snack. When John asks he's given a snack. In this case the antecedents are food deprivation and the presence of someone who can provide food. The behavior would be the request for a snack and the consequence is being provided with a snack.
%
\end{enumerate}
%
\subsection{Assessment}
\begin{enumerate}
\item Have supervisee identify and describe the ABC three term contingency.
\item Have supervisee give specific examples of the ABC three term contingency.
\item Have supervisee identify and describe other principles and terms related to the three term contingency (i.e., motivating operations, setting events, establishing operations, discriminative stimulus, stimulus control, etc.)
\item Have supervisee state how the ABC three term contingency related to both punishment and reinforcement.
\end{enumerate}
%
\subsection{Relevant Literature}
\begin{refsection}
\nocite{azrin1966punishment,
        cooper2007applied,
        glenn1992revolutionary,
        michael2004concepts,
        moxley2004pragmatic,
        sulzer1977applying,
        vollmer2002punishment,
        vollmer1991establishing}
\printbibliography[heading=none]
\end{refsection}
%
\subsection{Related Tasks}
\fourbOne{}\\
\foureOne{}\\
\fourgFour{}\\
\fouriOne{}\\
\fouriTwo{}\\
\fourFKTen{}\\
\fourFKEleven{}\\
\fourFKFifteen{}\\
\fourFKTwentyOne{}\\
\fourFKTwentySeven{}\\
\fourFKThirty{}\\
\fourFKThirtyThree{}\\
\fourFKThirtyFour{}\\
\fourFKThirtyFive{}\\
\fourFKFourtyOne{}\\
