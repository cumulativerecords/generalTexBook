%%\clearpage \section{\fourFKFourtyTwo{}}
\subsection{Definition} 
Rule governed behavior - ``Behavior governed by a rule (i.e., a verbal statement of an antecedent-behavior-consequence contingency), enables human behavior (e.g., fastening a seatbelt) to come under the indirect control of temporarily remote or improbable but potentially significant consequences (e.g., avoiding injury in auto accident). Often used in contrast to contingency-shaped behavior; a term used to indicate behavior selected and maintained by controlled, temporally close consequences'' (Cooper, Heron, \& Heward, 2007, p. 703).
%
%\clearpage \section{\fourFKThirtyOne{}}
\subsection{Definition}
``The AB because of C formulation is a general statement that the relation between an event (B) and its context (A) is because of consequences (C)... Applied to Skinner's three-term contingency, the relation between (A) the setting and (B) behavior exists because of (C) consequences that occurred for previous AB (setting-behavior) relations. The idea [is] that reinforcement strengthens the setting-behavior relation rather than simply strengthening behavior'' (Moxley, 2004, p. 111).
``The three term contingency- antecedent, behavior, and consequence- is sometimes called the ABC's of behavior analysis... The term contingency has several meanings signifying various types of temporal and functional relations between behavior and antecedent and consequent variables... When a reinforcer (or punisher) is said to be contingent on a particular behavior, the behavior must be emitted for the consequence to occur'' (Cooper, Heron, \& Heward, 2007, pp. 41-42).
%
