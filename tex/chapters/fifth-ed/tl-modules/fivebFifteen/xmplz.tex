\subsection{Examples}
Related definitions and examples are presented below.
\begin{enumerate}
\item Reflexivity - ``A type of stimulus-to-stimulus relation in which the student, without any prior training or reinforcement for doing so, selects a comparison stimulus that is the same as the same stimulus'' (Cooper, Heron \& Heward, 2007, p. 702).\\
 Example: Without prior reinforcement or training, when shown a picture of a dog and given a picture of the same dog, a rat, and a cow, student matches the picture of the two dogs (e.g. A=A).
\item Symmetry - ``A type of stimulus-to-stimulus relationship in which the learner, without prior training or reinforcement for doing so, demonstrates the reversibility of matched sample and comparison stimuli'' (Cooper, Heron \& Heward, 2007, p. 705).\\
 Example: Student is taught that when given the written word dog to select the picture of a dog. Without further reinforcement or training, when given the picture of the dog, student selects the written word dog (e.g. If A=B, then B=A). 
\item Transitivity - ``A derived stimulus-stimulus relation that emerges as a product of training two other stimulus-stimulus relations'' (Cooper, Heron \& Heward, 2007, p. 706).\\
 Example: Student is taught that when given the written word dog to select the picture of the dog (e.g. A=B). Student is also taught to select the picture of the dog when hearing the spoken word dog (e.g. B=C). Without further reinforcement or training, student selects the written word dog after hearing the spoken word dog (e.g. C=A). 

\item Example of Stimulus Equivalence\\
When learner responds without prior reinforcement and training that A=A (exhibiting reflexivity) and if A=B, then B must also = A (exhibiting symmetry) and finally that if A=B and B=C, then C must also equal A (exhibiting transitivity). 

\item Non-Example of Stimulus Equivalence:\\
 When learner responds without prior reinforcement and training that A=A (exhibiting reflexivity) and if A=B, then B must also equal A (exhibiting symmetry) but cannot show that if A=B and B=C, then C must also equal A (failure to exhibit transitivity). 
%
\end{enumerate}
%
\subsection{Examples}
\begin{enumerate}
\item Air blowing in eye (US) $\rightarrow$ blinking (UR)
\item Cold/low temperature (US) $\rightarrow$ shivering (UR)
\item Hot/high temperature (US) $\rightarrow$ sweating (UR)
\item Food in mouth (US) $\rightarrow$ salivation (UR)
\item Hot surface (US) $\rightarrow$ move hand away (UR)
%
\end{enumerate}
\subsection{Examples}
\begin{enumerate}
\item Roger usually drinks soda every day for lunch. When drinking soda, the sugar (US) inside his blood elicits the release of insulin from his pancreas (UR). Now, when he opens the soda, the snap of the can (CS) starts to elicit the release of insulin (CR) before he takes a drink. 
%
\end{enumerate}
%
\subsection{Examples}
\begin{enumerate}
\item   A rat is deprived of food. The rat walks near a specific part of their cage and receives food. As a result, the future probability of the rat walking toward that area of the cage increases.
\item A child hits their sibling when fighting over a toy. The child is sent to timeout. As a result, the future probability of hitting their sister decreases.
%
\end{enumerate}
%
\subsection{Examples}
\begin{enumerate}
\item  After a traumatic event involving physical abuse, every time a male walks into the room, your client ``freezes'' and does not follow instructions. This could be due to elicited behavior (``freezing'' in the presence of conditioned aversive stimuli) in competition with operant behavior (following instructions).
\item A medication, when consumed, will elicit respondent behavior that makes certain things more or less aversive. Consider if your client starts taking a medication to decrease aggression maintained by access to toys. The effect of the medication may decrease the likelihood that toys function as a reinforcer in effect decreasing the amount of aggression. It may increase the likelihood that food functions as a reinforcer.
\end{enumerate}
%
