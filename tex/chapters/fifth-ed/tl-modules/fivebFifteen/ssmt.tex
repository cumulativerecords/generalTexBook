\subsection{Assessment}
\begin{enumerate}
\item Have supervisees display equivalence with respect to the words ``reflexivity,'' ``transitivity,'' and ``symmetry'' in the spoken form, written form, and written definitions.
\item Have supervisee assess for stimulus equivalence on the job or during role-play 
\item Have supervisee demonstrate an example of stimulus equivalence during role-play
%
\end{enumerate}
\subsection{Assessment}
\begin{enumerate}
\item Ask supervisee to provide examples of US-UR relations.
\item Ask supervisee to discriminate between respondent behavior and operant behavior.
%
\end{enumerate}
\subsection{Assessment}
\begin{enumerate}
\item Have supervisee list various examples of respondent behavior. Have him/her explain respondent conditioning and define stimulus-stimulus pairing, unconditioned stimulus, neutral stimulus, conditioned stimulus, and conditioned reflex.
\item Have supervisee identify and describe an example of respondent conditioning (not an example from the Cooper et al. 2007 text or Pavlov's experiments). 
\item Have supervisee create an abstract for an experiment involving respondent conditioning. Have him/her describe how they would conduct the experiment to achieve respondent conditioning.
\item Have supervisee compare and contrast respondent conditioning and operant conditioning.
\end{enumerate}
%
\subsection{Assessment}
\begin{enumerate}
\item Have supervisee provide examples of operant conditioning.
\item Have supervisee describe ways of determining if operant conditioning is occurring (detect a reinforcing or punishing effect on behavior).
%
\end{enumerate}
%
\subsection{Assessment}
\begin{enumerate}
\item Ask your supervisee to describe how respondent behavior can interact with operant behavior.
\item Ask your supervisee to give an example of when this might occur with one of his/her clients during a specific treatment procedure.
\end{enumerate}
%
