\clearpage \section{\fourFKTwelve{}}
\subsection{Definition}
Stimulus equivalence - ``The emergence of accurate responding to untrained and nonreinforced stimulus-stimulus relations following the reinforcement of responses to some stimulus-stimulus relations. A positive demonstration of reflexivity, symmetry and transitivity is necessary to meet the definition of equivalence'' (Cooper, Heron \& Heward, 2007, p. 705).'' 

\subsection{Examples}
Related definitions and examples are presented below.
\begin{enumerate}
\item Reflexivity - ``A type of stimulus-to-stimulus relation in which the student, without any prior training or reinforcement for doing so, selects a comparison stimulus that is the same as the same stimulus'' (Cooper, Heron \& Heward, 2007, p. 702).\\
 Example: Without prior reinforcement or training, when shown a picture of a dog and given a picture of the same dog, a rat, and a cow, student matches the picture of the two dogs (e.g. A=A).
\item Symmetry - ``A type of stimulus-to-stimulus relationship in which the learner, without prior training or reinforcement for doing so, demonstrates the reversibility of matched sample and comparison stimuli'' (Cooper, Heron \& Heward, 2007, p. 705).\\
 Example: Student is taught that when given the written word dog to select the picture of a dog. Without further reinforcement or training, when given the picture of the dog, student selects the written word dog (e.g. If A=B, then B=A). 
\item Transitivity - ``A derived stimulus-stimulus relation that emerges as a product of training two other stimulus-stimulus relations'' (Cooper, Heron \& Heward, 2007, p. 706).\\
 Example: Student is taught that when given the written word dog to select the picture of the dog (e.g. A=B). Student is also taught to select the picture of the dog when hearing the spoken word dog (e.g. B=C). Without further reinforcement or training, student selects the written word dog after hearing the spoken word dog (e.g. C=A). 

\item Example of Stimulus Equivalence\\
When learner responds without prior reinforcement and training that A=A (exhibiting reflexivity) and if A=B, then B must also = A (exhibiting symmetry) and finally that if A=B and B=C, then C must also equal A (exhibiting transitivity). 

\item Non-Example of Stimulus Equivalence:\\
 When learner responds without prior reinforcement and training that A=A (exhibiting reflexivity) and if A=B, then B must also equal A (exhibiting symmetry) but cannot show that if A=B and B=C, then C must also equal A (failure to exhibit transitivity). 
%
\end{enumerate}
%
\subsection{Assessment}
\begin{enumerate}
\item Have supervisees display equivalence with respect to the words ``reflexivity,'' ``transitivity,'' and ``symmetry'' in the spoken form, written form, and written definitions.
\item Have supervisee assess for stimulus equivalence on the job or during role-play 
\item Have supervisee demonstrate an example of stimulus equivalence during role-play
%
\end{enumerate}
%
\subsection{Relevant Literature}
\begin{refsection}
\nocite{cooper2007applied,
        sidman1997equivalence,
        sidman2009equivalence}
\printbibliography[heading=none]
\end{refsection}
%
\clearpage \section{\fourFKThirteen{}}
\subsection{Definition} 
Unconditioned stimulus (US) - is a ``stimulus change that elicits respondent behavior (i.e., unconditioned response) in the absence of prior learning.  The UR is typically regarded as a built-in bodily mechanism that exists through natural evolution'' (Cooper, Heron, \& Heward, 2007, pp. 30, 39).   The US-UR relation is an unconditioned reflex.
%
\subsection{Examples}
\begin{enumerate}
\item Air blowing in eye (US) $\rightarrow$ blinking (UR)
\item Cold/low temperature (US) $\rightarrow$ shivering (UR)
\item Hot/high temperature (US) $\rightarrow$ sweating (UR)
\item Food in mouth (US) $\rightarrow$ salivation (UR)
\item Hot surface (US) $\rightarrow$ move hand away (UR)
%
\end{enumerate}
%
\subsection{Assessment}
\begin{enumerate}
\item Ask supervisee to provide examples of US-UR relations.
\item Ask supervisee to discriminate between respondent behavior and operant behavior.
%
\end{enumerate}
%
\subsection{Relevant Literature}
\begin{refsection}
\nocite{bijou1961child,
        cooper2007applied,
        pavlov1927conditional}
\printbibliography[heading=none]
\end{refsection}

%
\subsection{Related Lessons}
\fourFKFourteen{}\\
\fourFKFifteen{}\\
\fourFKSixteen{}\\
%
\clearpage \section{\fourFKFourteen{}}
\subsection{Definition} 
Reflex - ``The reliable production of a response by a stimulus'' (Catania, 1998, p. 8).

Respondent behavior - ``behavior that is elicited by antecedent stimuli. Respondent behavior is induced, or brought out, by a stimulus that precedes the behavior; nothing else is required for the response to occur'' (Cooper et al., 2007, p. 29).

``New stimuli can acquire the ability to elicit respondents. Called respondent conditioning, this type of learning is associated with Russian physiologist Ivan Petrovich Pavlolv...'' (Cooper et al., 2007, p. 30).

Pavlov's experiments consisted of a group of dogs who were trained to salivate at the sound of a metronome started just prior to feeding them. Before initial training, the presence of food (US) elicited salivation (UR), but the metronome (NS) was not paired with this response. After numerous trials of food being paired with the sound of the metronome, the dogs began salivating whenever they heard the metronome. After being paired with the presentation of food for several trials, the metronome became a conditioned stimulus (CS) and a conditioned reflex (CR) was elicited. 

\subsection{Examples}
\begin{enumerate}
\item Roger usually drinks soda every day for lunch. When drinking soda, the sugar (US) inside his blood elicits the release of insulin from his pancreas (UR). Now, when he opens the soda, the snap of the can (CS) starts to elicit the release of insulin (CR) before he takes a drink. 
%
\end{enumerate}
%
\subsection{Assessment}
\begin{enumerate}
\item Have supervisee list various examples of respondent behavior. Have him/her explain respondent conditioning and define stimulus-stimulus pairing, unconditioned stimulus, neutral stimulus, conditioned stimulus, and conditioned reflex.
\item Have supervisee identify and describe an example of respondent conditioning (not an example from the Cooper et al. 2007 text or Pavlov's experiments). 
\item Have supervisee create an abstract for an experiment involving respondent conditioning. Have him/her describe how they would conduct the experiment to achieve respondent conditioning.
\item Have supervisee compare and contrast respondent conditioning and operant conditioning.
\end{enumerate}
%
\subsection{Relevant Literature}
\begin{refsection}
\nocite{catania1998learning,
        cooper2007applied,
        skinner1984evolution,
        poling2001principles,
        skinner1938behavior,
        pavlov1928lectures}
\printbibliography[heading=none]
\end{refsection}
%Subreference not explicitly mentioned in above text.
%Poling, A., \& Braatz, D. (2001). Principles of learning: Respondent and operant conditioning and human behavior. Handbook of organizational performance: Behavior analysis and management, 23-49.
%
\subsection{Related Lessons}
\fourFKTen{}\\
\fourFKThirteen{}\\
\fourFKFifteen{}\\
\fourFKSixteen{}\\
\fourFKSeventeen{}\\
\fourFKTwentyFour{}\\
\fourFKTwentySix{}\\
\fourFKThirtyFive{}\\
%
\clearpage \section{\fourFKFifteen{}}
\subsection{Definition}
Operant Conditioning - ``The basic process by which operant learning occurs; consequences (stimulus changes immediately following responses) result in an increased (reinforcement) or decreased (punishment) frequency of the same type of behavior under similar motivational and environmental conditions in the future'' (Cooper, Heron \& Heward, 2007, pp. 700-701).
%
\subsection{Examples}
\begin{enumerate}
\item   A rat is deprived of food. The rat walks near a specific part of their cage and receives food. As a result, the future probability of the rat walking toward that area of the cage increases.
\item A child hits their sibling when fighting over a toy. The child is sent to timeout. As a result, the future probability of hitting their sister decreases.
%
\end{enumerate}
%
\subsection{Assessment}
\begin{enumerate}
\item Have supervisee provide examples of operant conditioning.
\item Have supervisee describe ways of determining if operant conditioning is occurring (detect a reinforcing or punishing effect on behavior).
%
\end{enumerate}
%
\subsection{Relevant Literature}
\begin{refsection}
\nocite{cooper2007applied,
        mcallister1969application}
\printbibliography[heading=none]
\end{refsection}
%
\subsection{Related Lessons}
\fourFKFifteen{}\\
\fourFKThirtyOne{}\\
\fourFKThirtyThree{}\\
%
\clearpage \section{\fourFKSixteen{}}
\subsection{Definition}
Respondent behavior - ``A response component of a reflex; behavior that is elicited, or induced, by antecedent stimuli'' (Cooper, Heron, \& Heward, 2007, p. 703).

Operant behavior – ``Behavior that is selected, maintained, and brought under stimulus control as a function of its consequences'' (Cooper, Heron, \& Heward, 2007, p. 701).

Operant and respondent behavior interact very commonly. They may occur concurrently when a stimulus both evokes an operant response while at the same time elicits a respondent response on the part of the organism. The procedures involved with what we call operant or respondent conditioning are names of procedures for the ease of use of our field. There are respondent and operant interactions occurring whenever an organism behaves. 

Pierce and Cheney (2013) describe it this way: ``When biologically relevant stimuli such as food are contingent on an organism's operant behavior, species-characteristic, innate behavior is occasionally elicited at the same time'' (p. 194). The presence of stimuli that have been paired with aversive or appetitive stimulation will elicit respondent behavior at the same time operant behavior is occurring to access or avoid those stimuli.

``The neural capacity for operant conditioning arose on the basis of species history; organisms that changed their behavior as result of life experience had an advantage over animals that did not do so'' (Pierce \& Cheney, 2013, p. 194).

Certain respondent behavior interacts with operant behavior. The effects are often described as motivating operations. For instance, behavior changes before and after meal times, with or without medications, after traumatic events, or disruptions in family life. 

\subsection{Examples}
\begin{enumerate}
\item  After a traumatic event involving physical abuse, every time a male walks into the room, your client ``freezes'' and does not follow instructions. This could be due to elicited behavior (``freezing'' in the presence of conditioned aversive stimuli) in competition with operant behavior (following instructions).
\item A medication, when consumed, will elicit respondent behavior that makes certain things more or less aversive. Consider if your client starts taking a medication to decrease aggression maintained by access to toys. The effect of the medication may decrease the likelihood that toys function as a reinforcer in effect decreasing the amount of aggression. It may increase the likelihood that food functions as a reinforcer.
\end{enumerate}
%
\subsection{Assessment}
\begin{enumerate}
\item Ask your supervisee to describe how respondent behavior can interact with operant behavior.
\item Ask your supervisee to give an example of when this might occur with one of his/her clients during a specific treatment procedure.
\end{enumerate}
%
\subsection{Relevant Literature}
\begin{refsection}
\nocite{cooper2007applied,
        pierce2013behavior,
        davis1977operant}
\printbibliography[heading=none]
\end{refsection}
%
\subsection{Related Lessons}
\fourgTwo{}\\
\fourgFive{}\\
\fourFKSeven{}\\
\fourFKThirteen{}\\
\fourFKFourteen{}\\
\fourFKFifteen{}\\
