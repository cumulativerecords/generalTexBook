%\clearpage \section{\fourFKTwelve{}}
\subsection{Definition}
Stimulus equivalence - ``The emergence of accurate responding to untrained and nonreinforced stimulus-stimulus relations following the reinforcement of responses to some stimulus-stimulus relations. A positive demonstration of reflexivity, symmetry and transitivity is necessary to meet the definition of equivalence'' (Cooper, Heron \& Heward, 2007, p. 705).'' 

%\clearpage \section{\fourFKThirteen{}}
\subsection{Definition} 
Unconditioned stimulus (US) - is a ``stimulus change that elicits respondent behavior (i.e., unconditioned response) in the absence of prior learning.  The UR is typically regarded as a built-in bodily mechanism that exists through natural evolution'' (Cooper, Heron, \& Heward, 2007, pp. 30, 39).   The US-UR relation is an unconditioned reflex.
%
%\clearpage \section{\fourFKFourteen{}}
\subsection{Definition} 
Reflex - ``The reliable production of a response by a stimulus'' (Catania, 1998, p. 8).

Respondent behavior - ``behavior that is elicited by antecedent stimuli. Respondent behavior is induced, or brought out, by a stimulus that precedes the behavior; nothing else is required for the response to occur'' (Cooper et al., 2007, p. 29).

``New stimuli can acquire the ability to elicit respondents. Called respondent conditioning, this type of learning is associated with Russian physiologist Ivan Petrovich Pavlolv...'' (Cooper et al., 2007, p. 30).

Pavlov's experiments consisted of a group of dogs who were trained to salivate at the sound of a metronome started just prior to feeding them. Before initial training, the presence of food (US) elicited salivation (UR), but the metronome (NS) was not paired with this response. After numerous trials of food being paired with the sound of the metronome, the dogs began salivating whenever they heard the metronome. After being paired with the presentation of food for several trials, the metronome became a conditioned stimulus (CS) and a conditioned reflex (CR) was elicited. 

%\clearpage \section{\fourFKFifteen{}}
\subsection{Definition}
Operant Conditioning - ``The basic process by which operant learning occurs; consequences (stimulus changes immediately following responses) result in an increased (reinforcement) or decreased (punishment) frequency of the same type of behavior under similar motivational and environmental conditions in the future'' (Cooper, Heron \& Heward, 2007, pp. 700-701).
%
%\clearpage \section{\fourFKSixteen{}}
\subsection{Definition}
Respondent behavior - ``A response component of a reflex; behavior that is elicited, or induced, by antecedent stimuli'' (Cooper, Heron, \& Heward, 2007, p. 703).

Operant behavior – ``Behavior that is selected, maintained, and brought under stimulus control as a function of its consequences'' (Cooper, Heron, \& Heward, 2007, p. 701).

Operant and respondent behavior interact very commonly. They may occur concurrently when a stimulus both evokes an operant response while at the same time elicits a respondent response on the part of the organism. The procedures involved with what we call operant or respondent conditioning are names of procedures for the ease of use of our field. There are respondent and operant interactions occurring whenever an organism behaves. 

Pierce and Cheney (2013) describe it this way: ``When biologically relevant stimuli such as food are contingent on an organism's operant behavior, species-characteristic, innate behavior is occasionally elicited at the same time'' (p. 194). The presence of stimuli that have been paired with aversive or appetitive stimulation will elicit respondent behavior at the same time operant behavior is occurring to access or avoid those stimuli.

``The neural capacity for operant conditioning arose on the basis of species history; organisms that changed their behavior as result of life experience had an advantage over animals that did not do so'' (Pierce \& Cheney, 2013, p. 194).

Certain respondent behavior interacts with operant behavior. The effects are often described as motivating operations. For instance, behavior changes before and after meal times, with or without medications, after traumatic events, or disruptions in family life. 

