%\clearpage \section{\fourFKSeventeen{}}
\subsection{Definition}
Unconditioned reinforcer - ``A stimulus change that increases the frequency of any behavior that immediately precedes it irrespective of the organism's learning history with the stimulus. Unconditioned reinforcers are the product of the evolutionary development of the species (phylogeny) Also called primary or unlearned reinforcer'' (Cooper, Heron, \& Heward, 2007 p.707).

``...momentary effectiveness of an unconditioned reinforcer is a function of current motivating operations'' (Cooper et al., 2007, p. 39).
%
\subsection{Examples}
\begin{enumerate}
\item Food, water, oxygen, warmth, and sexual stimulation are some examples of unconditioned reinforcers.
\item A teacher gives a child a pretzel after the child does a task. The child's engagement in the task increases in the future. 
\item This is an example of unconditioned reinforcement.
\end{enumerate}
%
\subsection{Assessment}
\begin{enumerate}
\item Have supervisee create a list of unconditioned reinforcers. Have him/her define and describe the role satiation and deprivation plays in unconditioned reinforcement. 
\item Have supervisee give examples of unconditioned reinforcers. Have him/her describe the difference between conditioned and unconditioned reinforcement.
\item Have supervisee explain the relationship between conditioned and unconditioned reinforcers and the role unconditioned reinforcers may play in creating conditioned reinforcers.
\end{enumerate}
%
\subsection{Relevant Literature}
\begin{refsection}
\nocite{bijou1965child,
        cooper2007applied,
        gewirtz2000infant,
        malott1978behavior,
        pelaez1996infants,
        skinner1953science}
\printbibliography[heading=none]
\end{refsection}
%
\subsection{Related Tasks}
\fourcOne{}\\
\fourdOne{}\\\
\fourdTwo{}\\\
\fourdNineteen{}\\
\fourFKTwo{}\\
\fourFKThirteen{}\\
\fourFKSixteen{}\\
\fourFKNineteen{}\\
\fourFKTwentyOne{}\\
\fourFKTwentySix{}\\
\fourFKThirty{}\\
%
%\clearpage \section{\fourFKEighteen{}}
\subsection{Definition}
Conditioned reinforcer - ``A stimulus change that functions as a reinforcer because of prior pairing with one or more other reinforcers; sometimes called secondary or learned reinforcers'' (Cooper et al., 2007, p. 692). 

Conditioned reinforcement – ``the operation, or process, of a response producing a conditioned reinforcer that increases the likelihood that response occurs in the future'' (Cooper et al., 2007, p. 40).
%
\subsection{Examples}
\begin{enumerate}
\item  Money, tokens, stickers.
\item A teacher says ``good job'' after a student returns their homework. The student continues to return their homework in the future. 
%
\end{enumerate}
%
\subsection{Assessment}
\begin{enumerate}
\item Have supervisee explain the differences between conditioned and unconditioned reinforcers.
\item Have supervisee explain the process of producing a conditioned reinforcer (i.e., token systems). Have him/her give an example from their professional experience.
\item Have supervisee read and summarize a journal article on the topic of conditioned reinforcement. 
\end{enumerate}
%
\subsection{Relevant Literature}
\begin{refsection}
\nocite{alessi1992models,
        cooper2007applied,
        higgins2001effects,
        michael2004concepts,
        morse1977determinants}
\printbibliography[heading=none]
\end{refsection}
%
\subsection{Related Tasks}
\fourcOne{}\\
\fourdOne{}\\
\fourdTwo{}\\
\fourdTwenty{}\\
\fourdTwentyOne{}\\
\fourfTwo{}\\
\fourjFour{}\\
\fourkFour{}\\
\fourFKTwo{}\\
\fourFKFourteen{}\\
\fourFKFifteen{}\\
\fourFKSixteen{}\\
\fourFKSeventeen{}\\
\fourFKTwentyOne{}\\
\fourFKTwentySix{}\\
\fourFKTwentySeven{}\\
%
%\clearpage \section{\fourFKFourtyOne{}}
\subsection{Definition}
Contingency shaped behavior – Behavior that is ``selected and maintained by controlled, temporally close consequences'' (Cooper, Heron, \& Heward, 2007, p. 42).  These consequences may either be reinforcing or punishing.  
%
\subsection{Examples}
\begin{enumerate}
\item Melvin puts a dollar into the soda machine and pushes the cola button. Seconds later a can of soda comes out. He opens the soda and drinks it.  He buys 3 more drinks from the same machine that week.
\item Simon's friend Ernest is a prankster.  Ernest shakes up a can of soda and offers Simon a drink.  The can sprays him in the face and soaks his clothing.  The next time Ernest offers a soda, Simon is hesitant to accept.  Although he'd like to open the soda and drink it, he hands it back expecting another explosive surprise.
\item Thirsty Floyd finds a 12 pack of old sodas in the storeroom.  He cracks a can and starts to drink. Unfortunately, the soda has gone bad. Floyd gets sick from drinking the soda. In the future Floyd avoids drinking old sodas.
\item (Non-example) Horace's mom tells him that drinking soda is bad for him. Horace avoids drinking soda in the future.
\end{enumerate}
%
\subsection{Assessment}
\begin{enumerate}
\item Ask your supervisee to explain contingency shaped behavior 
\item Ask your supervisee how contingency shaped is different from rule-governed behavior
\item Ask your supervisee to create another example and non-example of his/her own. 
\item Ask your supervisee to state why it might be better to use contingency shaped consequences as opposed consequences, which are more delayed or rule governed      
\end{enumerate}
%
\subsection{Relevant Literature}
\begin{refsection}
\nocite{cooper2007applied,
        malott2003principles,
        michael2004concepts}
\printbibliography[heading=none]
\end{refsection}
%
\subsection{Related Tasks}
\foureFour{}\\
\fourFKFourtyTwo{}\\
%
%\clearpage \section{\foureTen{}}
\subsection{Definition} 
The Premack principle - ''...a principle of reinforcement which states that an opportunity to engage in more probable behaviors (or activities) will reinforce less probable behaviors (or activities)'' (Volkmar, 2013, p. 2345).

``For example, if a child often plays computer games (more probable) and avoids completing math problems (less probable), we might allow her to play the computer after (contingent upon) completing 15 math problems. Prior to the introduction of the Premack principle, systems of reinforcement were viewed as the contingency between a stimulus and behavior. The Premack principle expanded the existing reinforcement contingency of stimulus behavior to include contingencies between two behaviors. This principle is often referred to as ‘grandma's rule' because grandmothers (or any caregivers) often apply this principle: ‘you have to eat your vegetables (less probable) before you can have dessert (more probable)''' (Volkmar, 2013, p. 2345).

In education, the Premack principle is the basis for ``first/then'' strategies. ``First/then'' strategies consist of a teacher telling a student ``First X, then Y'' with X being a less preferred activity or task demand and Y being a more preferred activity contingent on the completion of X.

Premack principles use preferred activities as reinforcers to help increase engagement in less preferred activities or demands. 

\subsection{Examples}
\begin{enumerate}
\item A father tells his teenage son, ``When you have finished washing the dishes, you can watch TV.''
\end{enumerate}
%
\subsection{Assessment}
\begin{enumerate}
\item Have supervisee give examples of Premack's principle in his/her daily life.
\item Have supervisee create a role play scenario in which he/she demonstrates the use of the Premack principle.
\item Have supervisee find an article on the use of the Premack principle, summarize, and discuss benefits and limitations of use.
\end{enumerate}
%
\subsection{Relevant Literature}
\begin{refsection}
\nocite{azrin2007physical,
        volkmar2013encyclopedia,
        cooper2007applied,
        mazur1975matching,
        sigafoos2005premack,
        welsh1993application}
\printbibliography[heading=none]
\end{refsection}
%
\subsection{Related Tasks}
\fourdOne{}\\
\fourdTwo{}\\
\fouriSeven{}\\
