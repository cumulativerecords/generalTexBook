%\clearpage \section{\fourFKSeventeen{}}
\subsection{Definition}
Unconditioned reinforcer - ``A stimulus change that increases the frequency of any behavior that immediately precedes it irrespective of the organism's learning history with the stimulus. Unconditioned reinforcers are the product of the evolutionary development of the species (phylogeny) Also called primary or unlearned reinforcer'' (Cooper, Heron, \& Heward, 2007 p.707).

``...momentary effectiveness of an unconditioned reinforcer is a function of current motivating operations'' (Cooper et al., 2007, p. 39).
%
%\clearpage \section{\fourFKEighteen{}}
%\subsection{Definition}
Conditioned reinforcer - ``A stimulus change that functions as a reinforcer because of prior pairing with one or more other reinforcers; sometimes called secondary or learned reinforcers'' (Cooper et al., 2007, p. 692). 

Conditioned reinforcement – ``the operation, or process, of a response producing a conditioned reinforcer that increases the likelihood that response occurs in the future'' (Cooper et al., 2007, p. 40).
%
%\clearpage \section{\fourFKFourtyOne{}}
%\subsection{Definition}
Contingency shaped behavior – Behavior that is ``selected and maintained by controlled, temporally close consequences'' (Cooper, Heron, \& Heward, 2007, p. 42).  These consequences may either be reinforcing or punishing.  
%
%\clearpage \section{\foureTen{}}
%\subsection{Definition} 
The Premack principle - ''...a principle of reinforcement which states that an opportunity to engage in more probable behaviors (or activities) will reinforce less probable behaviors (or activities)'' (Volkmar, 2013, p. 2345).

``For example, if a child often plays computer games (more probable) and avoids completing math problems (less probable), we might allow her to play the computer after (contingent upon) completing 15 math problems. Prior to the introduction of the Premack principle, systems of reinforcement were viewed as the contingency between a stimulus and behavior. The Premack principle expanded the existing reinforcement contingency of stimulus behavior to include contingencies between two behaviors. This principle is often referred to as ‘grandma's rule' because grandmothers (or any caregivers) often apply this principle: ‘you have to eat your vegetables (less probable) before you can have dessert (more probable)''' (Volkmar, 2013, p. 2345).

In education, the Premack principle is the basis for ``first/then'' strategies. ``First/then'' strategies consist of a teacher telling a student ``First X, then Y'' with X being a less preferred activity or task demand and Y being a more preferred activity contingent on the completion of X.

Premack principles use preferred activities as reinforcers to help increase engagement in less preferred activities or demands. 

