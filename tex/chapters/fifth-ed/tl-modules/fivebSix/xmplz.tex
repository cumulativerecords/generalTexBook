%
\subsection{Examples}
\begin{enumerate}
\item Melvin puts a dollar into the soda machine and pushes the cola button. Seconds later a can of soda comes out. He opens the soda and drinks it.  He buys 3 more drinks from the same machine that week.
\item Simon's friend Ernest is a prankster.  Ernest shakes up a can of soda and offers Simon a drink.  The can sprays him in the face and soaks his clothing.  The next time Ernest offers a soda, Simon is hesitant to accept.  Although he'd like to open the soda and drink it, he hands it back expecting another explosive surprise.
\item Thirsty Floyd finds a 12 pack of old sodas in the storeroom.  He cracks a can and starts to drink. Unfortunately, the soda has gone bad. Floyd gets sick from drinking the soda. In the future Floyd avoids drinking old sodas.
\item (Non-example) Horace's mom tells him that drinking soda is bad for him. Horace avoids drinking soda in the future.
\end{enumerate}
%
\begin{enumerate}
\item Bright lights, loud sounds, extreme temperatures, certain tastes (sour, bitter), physical restraint, loss of bodily support, extreme muscular efforts, etc.
\end{enumerate}
\subsection{Examples}
\begin{enumerate}
\item Similar to classical conditioning, a tone (neutral stimulus) is repeatedly paired with an electric shock (unconditioned punisher) whenever a dog barks, in time the tone (conditioned punisher) suppresses the bark in the absence of the electric shock.
\item A child engages in aggression. A parent responds to aggression by taking away their child's favorite video game contingent on every instance of aggression.  The parent begins to pair removal of the video games with a reprimand.  The reprimand may function as a conditioned punisher if aggression continues to decrease following the presentation of a reprimand without taking away the video games. This process illustrates conditioned punishment.
\item Conditioned punishers may be referred to as learned or secondary punishers.
\end{enumerate}
%
\subsection{Examples}
\begin{enumerate}
\item Screaming to avoid/escape washing hands
\item Example: A young woman at a group home was observed to scream loudly every time she was instructed to wash her hands. Each time she began screaming, she was allowed to avoid the task for an average of ten minutes or escape the task altogether. A behavioral analyst instructed group home staff to put this behavior on extinction.  After being instructed to wash her hands, group home staff physically guided her to comply even if she began screaming. After a week, screaming decreased to near zero levels.
\item (Non-example) A young woman at a group home was observed to scream loudly every time she was instructed to wash her hands. Each time she began screaming, she was allowed to avoid the task for an average of ten minutes or escape the task altogether. When she engaged in screaming, group home staff would tell her that if she stopped screaming and complied she would be given chips, a preferred food.
\item Extinction of behavior maintained by positive reinforcement: This occurs when behavior to access tangibles, activities, and/or attention is no longer reinforced. 
\item Extinction of behavior maintained by negative reinforcement: This occurs when behavior to avoid/escape an aversive stimulus/event is no longer reinforced. 
\item Extinction of behavior maintained by automatic reinforcement: This occurs when behavior that provides a natural and automatic sensory consequence is no longer reinforced. (e.g., a child is blocked each time he raises both hands above his mid-line to engage in hand-flapping.) 
\end{enumerate}
%
