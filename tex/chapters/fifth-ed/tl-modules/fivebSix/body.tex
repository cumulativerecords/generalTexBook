\clearpage \section[\fourdSeventeen{}]{\fourdSeventeen{}%
              \sectionmark{D-17 Use appr... punishment.}}
\subsection{Definition}
Punishment - ``Occurs when stimulus change immediately follows a response and decreases the future frequency of that type of behavior in similar conditions'' (Cooper, Heron, \& Heward, 2007, p. 702).*

Legislation and agency policies limit the use of punishment. Lerman and Vorndran (2002) suggested that punishment may be considered if:
\begin{enumerate}
\item The challenging behavior produces serious physical harm and has to be suppressed quickly
\item Reinforcement based treatments have not reduced the problem behavior to socially acceptable levels or 
\item The reinforcer maintaining the challenging behavior cannot be identified or withheld 
\end{enumerate}

\noindent BACB Labels specific considerations regarding punishment in the ethical guideline 4.08:\\
4.08 Considerations Regarding Punishment Procedures:
\begin{itemize}
\item (a) Behavior analysts recommend reinforcement rather than punishment whenever possible. 
\item (b) If punishment procedures are necessary, behavior analysts always include reinforcement procedures for alternative behavior in the behavior-change program. 
\item (c) Before implementing punishment-based procedures, behavior analysts ensure that appropriate steps  
have been taken to implement reinforcement-based procedures unless the severity or dangerousness of the behavior necessitates immediate use of aversive procedures.
\item (d) Behavior analysts ensure that aversive procedures are accompanied by an increased level of training, supervision, and oversight. Behavior analysts must evaluate the effectiveness of aversive procedures in a timely manner and modify the behavior-change program if it is ineffective. Behavior analysts always include a plan to discontinue the use of aversive procedures when no longer needed. (BACB, 2014, pp.12-13)
\end{itemize}
%
Ethical Considerations Related to Punishment as outlined by Cooper, Heron, and Heward (2007):
\begin{enumerate}
\item The right to safe and humane treatment
\item Least restrictive alternative
\item Right to effective treatment
\end{enumerate}

Appropriate Use of Punishment as outlined by Cooper, Heron, and Heward (2007):
\begin{enumerate}
\item Conduct a functional assessment
\item Attempt reinforcement based strategies (behavior as above does not reach socially acceptable levels)
\item Conduct punisher assessment
\item Ensure informed consent is given
\item Include reinforcement based strategies with punishment procedures
\item Ensure all staff are trained in the procedure and monitored closely
\item Use punishers of sufficient quality and magnitude
\item Use varied punishers
\item Deliver punisher at the beginning of a behavioral sequence
\item Punish each instance of the behavior Initially
\item Shift to intermittent schedule gradually
\item If delay in punishment use mediation
\item Supplement punishment with complementary interventions
\item Be prepared for negative side effects
\item Collect data, graph and evaluate daily
\item Discontinue procedure if a decrease in behavior is not observed
\end{enumerate}
%
\subsection{Assessment}
\begin{enumerate}
\item Provide scenarios in which clients would not qualify for a punishment procedure (e.g., behavior does not cause physical harm, reinforcement based strategies have not been attempted, or consent was not obtained.)
\item Ask Supervisee to list the four considerations the BACB lists when considering punishment
\item Ask the Supervisee to list the side effects of punishment
\item Ask the Supervisee to outline the recommendation for a client who had been receiving a punishment procedure for 2 months and head hitting remained consistent at 10 times a day. (should discontinue)
\item Have the supervisee list all of the things that must happen prior to a punishment procedure beginning (functional assessment, reinforcement based program ineffective, consent obtained, staff trained)
\end{enumerate}
%
\subsection{Relevant Literature}
\begin{refsection}
\nocite{bailey2013ethics,
        bac2014professional,
        cooper2007applied,
        foxx1982decreasing,
        van1988right,
        lerman2002status,
        iwata1988development}
\printbibliography[heading=none]
\end{refsection} 
%
\subsection{Related Tasks}
\fourdSixteen{}\\
\fourdFifteen{}\\
\fourdNineteen{}\\
\foureEleven{}\\
\fourgSeven{}\\
\fourjTen{}\\
\fourFKNineteen{}\\
\fourFKTwenty{}\\
\fourFKTwentyOne{}\\
\fourFKThirtyEight{}\\
%
Footnotes\\
Positive punishment may also be described as a type of aversive control. Negative side effects include: emotional or aggressive reactions, behavioral contrast, escape and avoidance of the punisher, modeling of inappropriate behavior and the overuse associated with negative reinforcement of the person presenting the punisher (Cooper, Heron, \& Heward, 2007)
%
\clearpage \section{\fourFKFourtyOne{}}
\subsection{Definition}
Contingency shaped behavior – Behavior that is ``selected and maintained by controlled, temporally close consequences'' (Cooper, Heron, \& Heward, 2007, p. 42).  These consequences may either be reinforcing or punishing.  
%
\subsection{Examples}
\begin{enumerate}
\item Melvin puts a dollar into the soda machine and pushes the cola button. Seconds later a can of soda comes out. He opens the soda and drinks it.  He buys 3 more drinks from the same machine that week.
\item Simon's friend Ernest is a prankster.  Ernest shakes up a can of soda and offers Simon a drink.  The can sprays him in the face and soaks his clothing.  The next time Ernest offers a soda, Simon is hesitant to accept.  Although he'd like to open the soda and drink it, he hands it back expecting another explosive surprise.
\item Thirsty Floyd finds a 12 pack of old sodas in the storeroom.  He cracks a can and starts to drink. Unfortunately, the soda has gone bad. Floyd gets sick from drinking the soda. In the future Floyd avoids drinking old sodas.
\item (Non-example) Horace's mom tells him that drinking soda is bad for him. Horace avoids drinking soda in the future.
%
\end{enumerate}
%
\subsection{Assessment}
\begin{enumerate}
\item Ask your supervisee to explain contingency shaped behavior 
\item Ask your supervisee how contingency shaped is different from rule-governed behavior
\item Ask your supervisee to create another example and non-example of his/her own. 
\item Ask your supervisee to state why it might be better to use contingency shaped consequences as opposed consequences, which are more delayed or rule governed      
\end{enumerate}
%
\subsection{Relevant Literature}
\begin{refsection}
\nocite{cooper2007applied,
        malott2003principles,
        michael2004concepts}
\printbibliography[heading=none]
\end{refsection}
%
\subsection{Related Tasks}
\foureFour{}\\
\fourFKFourtyTwo{}\\
%
\clearpage \section{\fourFKNineteen{}}
\subsection{Definition}
Unconditioned punisher – ``A stimulus change that decreases the frequency of any behavior that immediately precedes it irrespective of the organism's learning history with the stimulus'' (Cooper, Heron, \& Heward, 2007, p. 707).*
%
\subsection{Examples}
\begin{enumerate}
\item Bright lights, loud sounds, extreme temperatures, certain tastes (sour, bitter), physical restraint, loss of bodily support, extreme muscular efforts, etc.
%
\end{enumerate}
%
\subsection{Assessment}
\begin{enumerate}
\item Ask the supervisee to describe an example of unconditioned punishment.
\item Use the supervisee to describe the difference between unconditioned punishment and an unconditioned punisher.
\item Ask the supervisee to list as many unconditioned punishers as possible in one minute.
%
\end{enumerate}
%
\subsection{Relevant Literature}
\begin{refsection}
\nocite{cooper2007applied,
        herman1964punishment}
\printbibliography[heading=none]
\end{refsection} 
%    
%
\subsection{Related Tasks}
\fourdSeventeen{}\\
\fourdSixteen{}\\
\fourdNineteen{}\\
\foureEleven{}\\
\fourgSeven{}\\
\fourjTen{}\\
\fourFKTwenty{}\\
%
\subsection{Footnotes}
*Conditioned punishers are products of the evolutionary development of the species (Cooper, Heron, \& Heward, 2007).\\
*Conditioned punishers are also called primary or unlearned punishers (Cooper, Heron, \& Heward, 2007).\\
%
\clearpage \section{\fourFKTwenty{}}
\subsection{Definition}
Conditioned punisher – ``a stimulus that functions as a punisher as the result of being paired with unconditioned or conditioned punishers'' (Cooper, Heron, \& Heward, 2007, p. 40).

Conditioned punishment as defined by Hake and Azrin (1965) is a process that ``results when it can be shown (1) there is little or no punishment effect before the stimulus is paired with an unconditioned punisher, but (2) a punishment effect occurs after (3) the stimulus has been paired, or is being paired, with an unconditioned punisher'' (p. 279).  Evidence of conditioned punishment was suggested in early research when a reduction in a response was observed following the process of pairing a stimulus with an electric shock followed by discontinuing the shock and making the stimulus contingent upon a selected response (Hake \& Azrin, 1965).

\subsection{Examples}
\begin{enumerate}
\item Similar to classical conditioning, a tone (neutral stimulus) is repeatedly paired with an electric shock (unconditioned punisher) whenever a dog barks, in time the tone (conditioned punisher) suppresses the bark in the absence of the electric shock.
\item A child engages in aggression. A parent responds to aggression by taking away their child's favorite video game contingent on every instance of aggression.  The parent begins to pair removal of the video games with a reprimand.  The reprimand may function as a conditioned punisher if aggression continues to decrease following the presentation of a reprimand without taking away the video games. This process illustrates conditioned punishment.
\item Conditioned punishers may be referred to as learned or secondary punishers.
%
\end{enumerate}
%
\subsection{Assessment}
\begin{enumerate}
\item Ask supervisee to explain the process of conditioned punishment.
\item Ask supervisee to define a conditioned punisher.
\item Ask supervisee to provide examples of conditioned punishment and a conditioned punisher.
\item Ask supervisee to identify examples of conditioned punishment in a client's environment.
%
\end{enumerate}
%
\subsection{Relevant Literature}
\begin{refsection}
\nocite{bailey2013ethics,
        cooper2007applied,
        hake1965conditioned,
        iwata1988development}
\printbibliography[heading=none]
\end{refsection}
%
\subsection{Related Tasks}
\fourcTwo{}\\
\fourdFifteen{}\\
\fourdSixteen{}\\
\fourdSeventeen{}\\
\fourdEighteen{}\\
\fourdNineteen{}\\
\fourFKFourteen{}\\
\fourFKSeventeen{}\\
\fourFKEighteen{}\\
\fourFKNineteen{}\\
\fourFKTwentyOne{}\\
%
\clearpage \section{\fourFKTwentyTwo{}}
\subsection{Definition}
Extinction - ``The discontinuing of reinforcement of a previously reinforced behavior; the primary effect is a decrease in the frequency of the behavior until it reaches a pre-reinforced level or ultimately ceases to occur'' (Cooper, Heron \& Heward, 2007, p. 695).
%
\subsection{Examples}
\begin{enumerate}
\item Screaming to avoid/escape washing hands
\item Example: A young woman at a group home was observed to scream loudly every time she was instructed to wash her hands. Each time she began screaming, she was allowed to avoid the task for an average of ten minutes or escape the task altogether. A behavioral analyst instructed group home staff to put this behavior on extinction.  After being instructed to wash her hands, group home staff physically guided her to comply even if she began screaming. After a week, screaming decreased to near zero levels.
\item (Non-example) A young woman at a group home was observed to scream loudly every time she was instructed to wash her hands. Each time she began screaming, she was allowed to avoid the task for an average of ten minutes or escape the task altogether. When she engaged in screaming, group home staff would tell her that if she stopped screaming and complied she would be given chips, a preferred food.
\item Extinction of behavior maintained by positive reinforcement: This occurs when behavior to access tangibles, activities, and/or attention is no longer reinforced. 
\item Extinction of behavior maintained by negative reinforcement: This occurs when behavior to avoid/escape an aversive stimulus/event is no longer reinforced. 
\item Extinction of behavior maintained by automatic reinforcement: This occurs when behavior that provides a natural and automatic sensory consequence is no longer reinforced. (e.g., a child is blocked each time he raises both hands above his mid-line to engage in hand-flapping.) 
%
\end{enumerate}
%
\subsection{Assessment}
\begin{enumerate}
\item Provide a hypothetical scenario and have your supervisee determine if an extinction procedure is in place. If so, have your supervisee define which type of extinction.
\item Have your supervisee give you 3 various examples of extinction. 
\item Describe the pros and cons of extinction procedures based on readings assigned.
%
\end{enumerate}
%
\subsection{Relevant Literature}
\begin{refsection}
\nocite{cooper2007applied,
        lerman1999side,
        magee2000extinction}
\printbibliography[heading=none]
\end{refsection}
%
\subsection{Related Tasks} 
\fourcThree{}\\
\fourdEighteen{}\\
\fourdNineteen{}\\
