%\clearpage \section[\fourkFour{}]{\fourkFour{}%
\subsection{Definition}
Rationale\\
Staff performance in the application of behavioral strategies is critical to the success of the behavior intervention plan.  Developing a system to monitor staff performance and motivate staff performance is just as important as developing an effective intervention plan.  

Staff performance models:

Richman, Riordan, Reiss, Pyles, and Bailey (1988) found that self-monitoring and supervision feedback increased staff performance.

Arco (2008) found that defining the process before training, providing on-the job supervisory feedback, and having staff provide self-generated outcome feedback before and after training was effective at increasing and maintaining the performance in behavioral treatment programs.  

Codding, Feinberg, Dunn, and Pace (2004) found that treatment integrity increased following a one hour performance feedback session every other week.  Social validity ratings provided favorable feedback for the frequent supervisions.  

Iwata, Bailey, Brown, Foshee, and Alpern (1976) found that performance-based lottery improved the performance of institutional staff.
%
Elements of providing effective performance based monitoring and reinforcement systems
\begin{enumerate}
\item Provide clear instructions and objectives in observable measurable terms
\item Develop treatment integrity checklist
\item Train the supervisor to provide supervision and frequent on the job feedback (both corrective and positive)
\item Train staff to collect a self-monitoring system for their performance 
\item Train the supervisor to provide social or tangible reinforcement based on performance of the staff
\item Teach the supervisor to graph and monitor staff performance while looking for trend lines
\end{enumerate}
%
\subsection{Assessment}
