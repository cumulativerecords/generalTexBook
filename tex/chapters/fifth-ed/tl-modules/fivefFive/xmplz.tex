% 
\subsection{Examples}
\begin{enumerate}
\item Marvin is working to teach a student to make requests for preferred edibles using a picture exchange communication system.  Unfortunately, the child is not demonstrating a preference and is equally likely to choose either of the two cards presented.  He is equally likely to consume any edible that is associated with one of two cards.  Marvin decides to do a preference assessment to see if he can identify an edible that the boy does not like.  He conducts a paired stimulus preference assessment with eight items he thinks the student will potentially not prefer.   He notices that initially the boy chooses the black licorice but then spits it out.  Over several presentations the child's selection avoids this item.  It is not consumed on any of the presentations scoring it as 0\% or ``not preferred.''  
\item The Zippadeedooda Perfume Company is testing a new line of products.  They hire ten women to rate their new line of fragrances.  They ask the ladies to smell each of the 8 samples provided and use a Likert type scale to have them rate each of the fragrances; really like, somewhat like, neutral, somewhat dislike, and really dislike.  Those that score the overall lowest ratings according to the test group are not marketed, as it is unlikely that their client base will find them appealing either.  
\item (Non-example) Marvin's student has worked hard on his schoolwork and has earned a choice of an edible.  Marvin takes a box of candies out of his desk containing eight different types of sweets and tells his student that he can pick one.  The boy chooses a chocolate candy over all the other options and eats it.  He cannot assume that the child dislikes the rest of the candies because he did not observe the boy eating any of the others.  
\end{enumerate}
%
