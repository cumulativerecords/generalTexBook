%\clearpage \section{\fouraFourteen{}}
Types of Choice Measures:
\begin{enumerate}
\item Stimulus preference assessment – ``a variety of procedures used to determine the stimuli that a person prefers, the relative preference values (high versus low) of those stimuli, the conditions under which those preference values remain in effect, and their presumed value as reinforcers'' (Cooper, Heron, \& Heward, 2007, p. 705).
\item Single-stimulus preference assessment – also called a ``successive choice'' method. A stimulus is presented one at a time. Approaches to the items are recorded. Preference is based on whether or not individual approached item. (Pace, Ivancic, Edwards, Iwata, \& Page, 1985)
\item Paired choice preference assessment –also called a ``forced choice'' method. Consists of the simultaneous presentation of two stimuli. The observer records which of the two stimuli the learner chooses. Presentations continue until all stimuli are paired with each other stimulus. A hierarchy can then be formed using the choices (Fisher, Piazza, Bowman, Hagopian, Owens, \& Slevin, 1992).
\item Multiple stimulus assessment – an extension of the paired-stimulus procedure developed by Fisher and colleagues (1992). The person chooses a preferred stimulus from an array of three or more stimuli. (Cooper, Heron, \& Heward, 2007)
\item Multiple stimulus without replacement assessment – an extension of the procedures described by Windsor and colleagues (1994). Once an item was selected, DeLeon and Iwata (1996) did not replace previously chosen stimuli. Each choice was among the remaining stimuli. 
\item Free-operant assessment – Developed by Roane and colleagues (1998), participants had free and continuous access to the entire array of stimuli for 5 minutes. Duration of item manipulation is recorded. 
\item Response restriction assessment– Developed by Hanley and colleagues (2003), a free operant arrangement was used to measure preference. Stimuli with high interaction relative to the other stimuli during a session were restricted for the remaining sessions.
\item Duration assessment – Developed by Hagopian and colleagues (2001), items were presented one at a time. Duration of engagement was measured for each item.
\item Concurrent-chain assessment (Concurrent-chain schedules) – ``concurrent schedules in which the reinforcers are themselves schedules that operate separately and in the presence of different stimuli'' (Catania, 2013, p. 433). Completion of the initial link schedule of reinforcement gives access to the terminal link schedule of reinforcement. Preference for particular schedules of reinforcement, or other environmental arrangements can be measured by responding in the initial links (Hanley, Iwata, \& Lindberg, 1999).
\end{enumerate}
% 
%\clearpage \section[\fouriSeven{}]{\fouriSeven{}%
\subsection{Definition}  
Stimulus preference assessment-``a variety of procedures used to determine (a) the stimuli that the person prefers, (b) the relative preference values of those stimuli, (c) the conditions under which those preference values change when task demands, deprivation states, or schedules of reinforcement are modified''  (Cooper, Heron, \& Heward, 2007, pp. 275-276).

Preference assessments allow for one to evaluate a large number of stimuli in a brief period of time (Hagopian, Long, \& Rush, 2004). These stimuli are likely to function as reinforcers for a client/participant. Preference assessments also take into account care provider options (Green et al. 1988).   
%
%\clearpage \section[\fourjFour{}]{\fourjFour{}%
\subsection{Definition}
The importance and ethical necessity of basing intervention strategies on client preferences:
\begin{enumerate}
\item As behavior analysts, it is our ethical responsibility to continually put our client's needs first, and this includes, considering which type of intervention may be more preferred by the clients we serve. As Bailey and Burch (2011) state, one of our core ethical principles is treating others with care and compassion and this encompasses giving our clients choices (Bailey \& Burch, 2011).
%
\item Historically, consideration of client preferences is an area that within behavior analysis, perhaps has not been given as much attention as it deserves. In one area of study, Hanley, Piazza, Fisher, Contrucci \& Maglieri (1997) reported that, ``few if any studies have examined the social acceptability of or consumer preferences'' for the relevant treatment options but had instead given more weight to the opinions of the caregivers as opposed to those of the client (Hanley et al., 1997, p. 460). Another interesting train of thought has been that ``choice making is often not taught'' (Bannerman, Sheldon, Sherman, \& Harchik, 1990, p. 81).
%
\item Another reason for considering clients' preferences over treatment options is it may make the intervention more successful. Data from Miltenberger, Suda, Lennox and Lindeman (1991) indicated it was very important for successful treatment, to consider client preferences when selecting interventions. Findings from many other studies have also supported this premise (e.g., Berk, 1976; Hanley, Piazza, Fisher, \& Maglieri, 2005; Mendonca \& Brehm, 1983; Perlmuter \& Montry, 1973).
\end{enumerate}

Selecting interventions based on client preferences 
\begin{enumerate}
\item There are methods reported in the literature for determining which treatment method is more preferred by a client*. As such, once it has been established that an intervention is necessary to treat a behavior, it is imperative then to consider assessing a client's preference for one treatment option over others to assist with the behavior change program. In this way, treatment is more likely to be successful, will likely have more social validity (Schwartz \& Baer, 1991) and will be meeting more of our ethical standards as behavior analysts.
\end{enumerate}
%
%
Footnotes\\
*1 See Hanley, Piazza, Fisher, Contrucci \& Maglieri (1997) and Miltenberger, Suda, Lennox \& Lindeman (1991) for more information about how to test clients' preferences for different interventions.\\
%
%\clearpage \section[\foureEight{}]{\foureEight{}%
\subsection{Definition} 
Matching Law - ``When two or more concurrent-interval schedules are available, the relative rate of response matches (or equals) the relative rate of reinforcement. More generally, the matching law states that the distribution of behavior between (or among) alternative sources of reinforcement is equal to the distribution of reinforcement for these alternatives'' (Pierce \& Cheney, 2013, p. 260).

Choice - ``...the emission of one of two or more alternative and, usually, incompatible responses'' (Catania, 2007, p.431).

Organisms are constantly confronted with making choices; the allocation of responding is based upon the probability of reinforcement for that response. There are also other variables known to effect response allocation such as magnitude of reinforcement, quality of reinforcement, delay to reinforcement, and duration of reinforcement (Baum, 1974). If a variable that is affecting responding on a particular option cannot be identified, this is known as bias. An example of bias might be a right-handed person responding on an option to the right side. This variable and others are accounted for using different coefficients in the matching law.

Multiple basic and applied studies with humans and non-humans have demonstrated that behavior is allocated to response options based on reinforcement schedules available on those options (Baum, 1974; Borrero \& Vollmer, 2002; Epling \& Pierce, 1983).

There is debate about the status of the matching law as a convenient description vs. a fundamental property of behavior (c.f., Catania, 1981; Killeen, 2015, Rachlin, 1971)
%
%\clearpage \section[\foureFour{}]{\foureFour{}%
\subsection{Definition}
Contingency contract -  ``...also called a behavioral contract, is a document that specifies a contingent relationship between the completion of a specified behavior and access to, or delivery of, a specified reward such as free time, a letter grade or access to a preferred activity'' (Cooper, Heron \& Heward, 2007, p. 551). 

Contingency contracts have several components: 
\begin{enumerate}
\item Outlines the task to be completed- includes an objective definition of the task, who must complete the task and when the task must be completed.
\item Specifies the reward contingent on task completion- includes description of the reward, who will deliver the reward, who will measure, whether the task has been completed to criterion, and when the reward will be received.
\item Outlines how performance will be measured and what data will be taken. 
\end{enumerate}

Contingency contracts can be highly effective if used properly because the individual whose behavior is to be changed is involved in the process from the start. Studies have shown that contingency contracting ``...has been identified as an important step toward self-management of behavior'' (Miller \& Kelley, 1994, p. 74) because by helping to determine the parameters of the task and outlining what and when rewards should be given, reinforcer assessments have already been identified and the individual is already motivated to engage in the target behavior, which can greatly increase compliance. However, Cooper, Heron and Heward (2007) caution against using contingency contracts with all populations. There must be set criteria that the individual must already possess in order for contingency contracts to be effective. The target behavior must already be in the individual's repertoire and the individual must already be able to discriminate when and which environments are appropriate for the response to occur. Additionally, the individual's behavior must be able to ``come under the control of the visual or oral statements (rules) of the contract'' (Cooper, Heron, \& Heward, 2007, p. 558).  The individual does not need to be proficient in reading so long as the contract is asdapted using symbols, icons, photographs, etc. and the individual thoroughly understands the reinforcement contingency. 
%
%\clearpage \section{\fourFKFourty{}}
\subsection{Definition} 
Matching law - ``a quantitative formulation stating that the relative rates of different responses tend to equal the relative reinforcement rates they produce'' (Catania, 2007, p. 449).\\

Herrnstein (1961) described pigeon's distribution of responding on concurrent schedules of reinforcement. He found the relation between absolute rate of reinforcement and the absolute rate of responding is a linear function that passes through the origin. In other words, if rate of reinforcement and rate of responding are plotted on the x and y axis the result of the data is very close to a line that passes through the origin with the slope of 1. The matching equation can be denoted as follows:\\

The term  is behavior measured as rate of response for behavior a, and  is behavior measured as rate of response for behavior b. The term  is the scheduled rate or reinforcement for response a, and  is the scheduled rate of reinforcement for response b.\\

The matching law has been demonstrated across a variety of species including pigeons, cows (Mathews \& Temple, 1979), rats (Poling, 1978), free ranging flocks of birds (Baum, 1974), and humans (Conger \& Killeen, 1974). \\

There are also multiple applied examples. For example, Mace, Neef, Shade, and Mauro (1994) found special education high school students spent time on math problems, arranged in different stacks, that was equal to the relative rate of reinforcement for completing math problems from the different stacks. Also, Borrero and Vollmer (2002), after conducting functional analyses to find maintaining variables for problem behavior, found that proportional rates of problem behavior relative to problem behavior matched the proportional rate of reinforcement for 4 individuals with disabilities. 
%
