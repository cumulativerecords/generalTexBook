\clearpage \section[\fourbThree{}]{\fourbThree{}%
              \sectionmark{B-03 Systematically... independent var}}
\sectionmark{B-03 Systematically... independent var}
              
              

%\sectionmark{B-03 Systematically arrange independent variables...}%
%\section[B-03 Systematically arrange independent variables...]{\fourbThree{}}
\subsection{Definitions} 
Independent Variable - ``The variable that is systematically manipulated by the researcher in an experiment to see whether changes in the independent variable produce reliable changes in the dependent variable. In applied behavior analysis, it is usually an environmental event or condition antecedent or consequent to the dependent variable. Sometimes called the intervention or treatment variable'' (Cooper, Heron, \& Heward, 2007, p. 697).\\

Dependent Variable - ``The variable in an experiment measured to determine if it changes as a result of the manipulations of the independent variable; in applied behavior analysis, it represents some measure of a socially significant behavior'' (Cooper et al. 2007, p. 693).\\

Dependent variables must be operationally defined to allow for consistent assessment and replication of the assessment process, measured repeatedly within and across controlled conditions, recording is assessed for consistency across the experiment using inter-observer agreement, and dependent variables must be socially significant to the individual or those around them. (Horner, Carr, Halle, McGee, Odom, \& Wolery, 2005)\\

Experimental control is achieved when predicted change in the dependent variable (i.e., the behavior) covaries with manipulations of the independent variable (i.e., the intervention) showing the effectiveness of the independent variable on the dependent variable of a participant. (Horner et al., 2005)
%
\subsection{Examples} 
\begin{enumerate}
\item   A student consistently disrupts group activities. When given visuals for appropriate behavior (i.e., quiet voice, calm body) paired with gestural redirection, disruptive behavior in group lessons decreases. The teacher then takes the visuals away for a week to see if fading these supports would be an option. The gestural redirection for inappropriate behavior is still in place. The student's disruptive behavior remains low. When the redirection is removed the following week. The student engages in increased disruptive behavior during this week, so the teacher decides to continue the gestural prompts and the disruptive behavior decreases again. 
\item Non-example: A student with attention deficits consistently disrupts group activities. His teacher occasionally uses the visuals for appropriate behavior outlined in the BSP and the disruptive behavior does not decrease.
\end{enumerate}

\subsection{Assessment}
\begin{enumerate}
\item Give supervisees article abstracts on single subject research. Have them identify the dependent variable and independent variable for the study.
\item Have supervisees identify the independent and dependent variables in the example listed above.
\item Have supervisees read Horner et al., (2005) The Use of Single-Subject Research to Identify Evidence-Based Practice in Special Education and complete a brief summary of the article and ask them to identify what compromises the integrity of a functional relationship and define the quality indicators outlined for effective single-subject research.
\end{enumerate}
%
\subsection{Relevant Literature} 
\begin{refsection}
\nocite{cooper2007applied,horner2005use}
\printbibliography[heading=none]
\end{refsection}%%

\subsection{Related Lessons}
\fourbFour{}\\
\fourbFive{}\\
\fourbSix{}\\
\fourbSeven{}\\
\fourbNine{}\\
\fourbEleven{}\\ 
\fourhFour{}\\
\fouriOne{}\\
%
\clearpage \section[\fourbEleven{}]{\fourbEleven{}%
              \sectionmark{B-11 Conduct a parametric... }}
\sectionmark{B-11 Conduct a parametric...}
\subsection{Definition}
Parametric analysis - ``An experiment designed to compare the differential effects of a range of values of the independent variable'' (Cooper, Heron, \& Heward, 2007, p. 701).
%
\subsection{Examples}
\begin{enumerate}
\item An experiment designed to analyze different magnitudes of a punishment procedure to determine the least intrusive magnitude of a stimulus to decrease behavior.
\item An experiment designed to analyze the optimal quality of attention necessary to reinforce appropriate behavior 
\end{enumerate}
%
\subsection{Assessment}
\begin{enumerate}
\item Ask Supervisee to list some parameters (schedule, immediacy, quality, quantity) of an independent variable that can be manipulated experimentally
\item Have supervisee describe a hypothetical parametric analysis inclusive of the independent variable parameter to be manipulated and the functional relation to be tested 
\item Ask the supervisee how they could use a parametric analysis to test for the optimal level of treatment integrity. Then discuss what parameter they would measure (i.e., schedule, immediacy) and which values they would select to answer this experimental question.  
\end{enumerate}
%
\subsection{Relevant Literature}
\begin{refsection}
\nocite{cooper2007applied,lerman1996methodology,lerman2002reinforcement}
\printbibliography[heading=none]
\end{refsection}
%
\subsection{Related Lessons} 
\fourbThree{}\\
\fourFKThirtyThree{}\\
%
\clearpage \section{\fourFKThirtyThree{}}
\subsection{Definition}
Functional relation - ``An experimentally determined relation that shows that the dependent variable depends on or is a function of the independent variable and nothing else'' (Johnston \& Pennypacker, 2009, p. 358).\\

``A `cause' becomes a ‘change in an independent variable' and an `effect' a `change in a dependent variable.' The old ‘cause-and-effect connection' becomes a ‘functional relation.' The new terms do not suggest how a cause causes its effect; they merely assert that different events tend to occur together in a certain order'' (Skinner, 1953, p.23).\\

For every response are a number of factors that influence the likelihood that it occurs. Each one of these factors can be used as an independent measure in an experiment. If an experiment shows that there is a different between a context in which this variable is present vs when this variable is absent, we consider it to be a functional relation (Cooper, Heron, \& Heward, 2007).\\

For behavior analysts, functional relations are important to discover. When we understand how a behavior is related to the environment, we can then decide what treatment to use.\\
%
\subsection{Examples}
\begin{enumerate}
\item Sonny has been engaging in some eloping behavior within a school building.  He has been known to leave his classroom area and to run to other rooms within the building. His teachers have started delivering small pieces of candy for on-task behavior, as he works on his schoolwork. The teachers have started seeing an increase in on-task behavior and a decrease in elopement. A substitute teacher came into the classroom for a week but did not know about the on-task candy delivery the first couple of days.   Sonny started eloping again. When the aides told the substitute teacher about the contingency for on-task behavior, the substitute started delivering the candy on the same schedule as the other teacher. Sonny again started sitting down, remaining on task, and elopement decreased. It can be said that there is likely a functional relation between the schedule of candy delivery and the elopement and/or on-task behavior.
\item Three semi-busy 3-way intersections in the small town of Passamaquoddy has had a series of accidents over the past few years.  These intersections have had yield signs up but there have been several accidents at each location.  The town decides to replace the yield signs with 3 stop signs instead. They use a multiple baseline design across locations. After seeing that when, and only when, the new stop signs are implemented, accidents have decreased in that location. It can be said that there is a functional relation between the placement of the stop signs and the change in accidents reported there.  
%
\item (Non-example) A child with autism has been engaging in some eloping behavior within a school building.  He has been known to leave his classroom area and to run to other rooms within the building. His teachers have explained to him that the other rooms are off limits but this has not had an impact on his behavior nor has simply ensuring that the doors are closed. His teacher decides to put up a green light on rooms that it is o.k. to enter.  There has been no change in behavior from the previously recorded levels of entering the off-limits classrooms.  It can be said that there is no functional relation between the presence of green light signs and the off-limits classroom entering behavior.
%
\end{enumerate}
%
\subsection{Assessment}
\begin{enumerate}
\item Ask your Supervisee to give a definition for functional relation.
\item Ask your supervisee to create other examples and a non-example of his/her own. 
\item Ask the Supervisee why it is important to only manipulate one variable at a time
\item Ask the Supervisee to state how you would know if a functional relation exists between the independent variable and the dependent variable. 
%
\end{enumerate}
%
\subsection{Relevant Literature}
\begin{refsection}
\nocite{cooper2007applied,
        johnston2010strategies,
        skinner1953science}
\printbibliography[heading=none]
\end{refsection}
%
\subsection{Related Lessons}
\fourbThree{}\\
\fourFKThirtyThree{}\\
\fourhThree{}\\
\fourhFive{}\\
\fouriFive{}\\
%
\clearpage \section[\fourdTwenty{}]{\fourdTwenty{}%
              \sectionmark{D-20 Use response-independent...}}
\subsection{Definition}
Noncontingent reinforcement (NCR) – ``A procedure in which stimuli with known reinforcing properties are presented on fixed-time (FT) or variable time (VT) schedules completely independent of behavior; often used as an antecedent intervention to reduce problem behavior'' (Cooper, Heron \& Heward, 2007, p. 700).

Noncontingent reinforcement is sometimes used in applied research in an experimental design called the NCR reversal technique. This design involves a baseline phase, NCR phase (where a potential reinforcer is delivered on a fixed or variable time schedule independent of the target behavior), and a phase where the reinforcer is delivered contingent on a set behavioral criteria. The phases are repeated as necessary to indicate experimental control over the dependent variable. The NCR and baseline conditions function as a comparative measure to validate the independent variable in these studies.

Differential reinforcement procedures may limit access to reinforcement if appropriate behavior occurs at low rates. NCR gives consistent access to reinforcement. 
%
\subsection{Examples}
\begin{enumerate}
\item A DRO program was trialed for 3 weeks to decrease Jimmy's verbal protesting during group activities. Based on data collected, the DRO program was deemed ineffective for decreasing Jimmy's verbal protesting. Mr. Jones took data and found that Jimmy could quietly engage in group activities for 3 minutes before starting to protest. Mr. Jones decided to implement a 2 minute NCR program in which he would give Jimmy a sticker every 2 minutes regardless of the presence of interfering behaviors.
\item (Non-Example) Mr. Michael was concerned with Barry's aggressive behavior during group activities. He decided to give him a sticker for every 2 minutes that he did not engage in aggressive behavior.
\end{enumerate}
%
\subsection{Assessment}
\begin{enumerate}
\item Have supervisee create an NCR program and explain procedures to other supervisees.
\item Have Supervisee identify the difference between differential reinforcement and noncontingent reinforcement and give examples of both.
\item Have supervisee give examples of NCR used in his/her professional and nonprofessional life.
\end{enumerate}
%
\subsection{Relevant Literature}
\begin{refsection}
\nocite{cautela1984general,
       cooper2007applied,
       hagopian1994schedule,
       ingvarsson2008some,
       wilder2005noncontingent}
\printbibliography[heading=none]
\end{refsection}
%
\subsection{Related Lessons}
\fourbFour{}\\
\fourcOne{}\\
\fourdTwo{}\\
\fourdTwentyOne{}\\
\fourjTwo{}\\
%
\clearpage \section[\fouriTwo{}]{\fouriTwo{}%
              \sectionmark{I-02 Define environmental var...}}
\sectionmark{I-02 Define environmental var...}
\subsection{Definition}
This task relates to the importance of defining environmental variables in observable and measurable terms.
\begin{enumerate}
\item As Cooper, Heron, and Heward (2007) state, in order to achieve a high level of treatment integrity in an experiment, it is of utmost importance to ``develop complete and precise operational definitions of the treatment procedures'' (Cooper, Heron, \& Heward, 2007, p. 235). In the same way that it is critical to define target behavior in observable and measurable terms, so is the case with defining environmental variables. 
\item Baer, Wolf, and Risley (1968) stress that the ``technological'' dimension of Applied Behavior Analysis refers simply to the fact that ``the techniques making up a particular behavioral application are completely identified and described''(Baer, Wolf, \& Risley, 1968, p. 95). As such, the techniques, or environmental variables being manipulated, must be defined in observable and measurable terms to meet the technological dimension of applied behavior analysis (Cooper, Heron, \& Heward, 2007). 
\item However, historically, operationally defining independent variables has not been conducted to the standard required for a science of behavior that seeks to achieve the technological dimension of applied behavior analysis*. It has also not been done to the same standard as that of the dependent variables (Johnston \& Pennypacker, 1980; Peterson, Homer \& Wonderlich, 1982; Gresham, Gansle \& Noell, 1993). In 1982 Peterson, Homer and Wonderlich called for researchers to measure the independent variables in a more stringent manner. Unfortunately, an assessment of this area later on by Gresham, Gansle and Noell (1993) found that this had not been accomplished.
\item Defining environmental variables in observable and measurable terms
\item It is believed that environmental variable definitions should be written to meet the same standards as those required to be met by target behavior definitions (Gresham, Gansle \& Noell, 1993). They should be ``clear, concise, unambiguous, and objective'' (Cooper, Heron \& Heward, 2007, p. 235).
\item Gresham et al. (1993, p. 261) suggest that independent variable definitions can be made along four dimensions: spatial, verbal, physical and temporal. 
\end{enumerate}

\subsection{Examples}
Gresham et al. (1993, pp. 261-262) give an example of an adequate definition of an independent variable, a time-out procedure, provided by Mace, Page, Ivancic and O'Brien (1986).
\begin{enumerate}
\item Immediately following the occurrence of a target behavior (temporal dimension), (b) the therapist said, ``No, go to time-out'' (verbal dimension), (c) led the child by the arm to a prepositioned time-out chair (physical dimension), and (d) seated the child facing the corner (spatial dimension). (e) If the child's buttocks were raised from the time-out chair or if the child's head was turned more than 45° (spatial dimension), the therapist used the least amount of force necessary to guide compliance with the time-out procedure (physical dimension). (f) At the end of 2 min (temporal dimension), the therapist turned the time-out chair 45° from the corner (physical and spatial dimensions) and walked away (physical dimension).
\item Gresham et al., (1993, p. 262) argued that a failure to define operational variables along these four dimensions, as done so by Mace, Page, Ivancic and O'Brien (1986), makes ``replication and external validation of behavior-analytic investigations difficult.''
\end{enumerate}
%
\subsection{Assessment}
\begin{enumerate}
\item Ask your Supervisee to explain why it's important as a behavior analyst to define environmental variable in observable and measurable terms. 
\item Ask your Supervisee to write an operational definition for the following independent variable:
\item Verbal Praise (Answer = (a) Immediately following the occurrence of a target behavior (temporal dimension), (b) the therapist delivered verbal praise, for example, ``great job/nice work/well done'' (verbal dimension), (c) but did not provide any physical contact such as a hi-five or pat on the back (physical dimension). The therapist was within 2 to 20 feet of the client at all times during the intervention (spatial dimension).
\end{enumerate}
%
\subsection{Relevant Literature}
\begin{refsection}
\nocite{baer1968some,
        cooper2007applied,
        gresham1993treatment,
        johnston2010strategies,
        peterson1982integrity}
\printbibliography[heading=none]
\end{refsection}
%
\subsection{Related Lessons}
\fourbEleven{}\\
\fouriOne{}\\
\fouriFour{}\\
\fourjOne{}\\
\fourFKSeven{}\\
\fourFKEleven{}\\
\fourFKThirtyThree{}\\
%
Footnotes\\
*See Baer, Wolf \& Risley (1968) for more information on the seven dimensions of Applied Behavior Analysis.
