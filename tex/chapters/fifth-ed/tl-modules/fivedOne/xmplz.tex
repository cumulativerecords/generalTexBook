\subsection{Examples} 
\begin{enumerate}
\item   A student consistently disrupts group activities. When given visuals for appropriate behavior (i.e., quiet voice, calm body) paired with gestural redirection, disruptive behavior in group lessons decreases. The teacher then takes the visuals away for a week to see if fading these supports would be an option. The gestural redirection for inappropriate behavior is still in place. The student's disruptive behavior remains low. When the redirection is removed the following week. The student engages in increased disruptive behavior during this week, so the teacher decides to continue the gestural prompts and the disruptive behavior decreases again. 
\item Non-example: A student with attention deficits consistently disrupts group activities. His teacher occasionally uses the visuals for appropriate behavior outlined in the BSP and the disruptive behavior does not decrease.
\end{enumerate}

\subsection{Examples}
\begin{enumerate}
\item An experiment designed to analyze different magnitudes of a punishment procedure to determine the least intrusive magnitude of a stimulus to decrease behavior.
\item An experiment designed to analyze the optimal quality of attention necessary to reinforce appropriate behavior 
\end{enumerate}
%
\subsection{Examples}
\begin{enumerate}
\item Sonny has been engaging in some eloping behavior within a school building.  He has been known to leave his classroom area and to run to other rooms within the building. His teachers have started delivering small pieces of candy for on-task behavior, as he works on his schoolwork. The teachers have started seeing an increase in on-task behavior and a decrease in elopement. A substitute teacher came into the classroom for a week but did not know about the on-task candy delivery the first couple of days.   Sonny started eloping again. When the aides told the substitute teacher about the contingency for on-task behavior, the substitute started delivering the candy on the same schedule as the other teacher. Sonny again started sitting down, remaining on task, and elopement decreased. It can be said that there is likely a functional relation between the schedule of candy delivery and the elopement and/or on-task behavior.
\item Three semi-busy 3-way intersections in the small town of Passamaquoddy has had a series of accidents over the past few years.  These intersections have had yield signs up but there have been several accidents at each location.  The town decides to replace the yield signs with 3 stop signs instead. They use a multiple baseline design across locations. After seeing that when, and only when, the new stop signs are implemented, accidents have decreased in that location. It can be said that there is a functional relation between the placement of the stop signs and the change in accidents reported there.  
%
\item (Non-example) A child with autism has been engaging in some eloping behavior within a school building.  He has been known to leave his classroom area and to run to other rooms within the building. His teachers have explained to him that the other rooms are off limits but this has not had an impact on his behavior nor has simply ensuring that the doors are closed. His teacher decides to put up a green light on rooms that it is o.k. to enter.  There has been no change in behavior from the previously recorded levels of entering the off-limits classrooms.  It can be said that there is no functional relation between the presence of green light signs and the off-limits classroom entering behavior.
%
\end{enumerate}
%
\subsection{Examples}
\begin{enumerate}
\item A DRO program was trialed for 3 weeks to decrease Jimmy's verbal protesting during group activities. Based on data collected, the DRO program was deemed ineffective for decreasing Jimmy's verbal protesting. Mr. Jones took data and found that Jimmy could quietly engage in group activities for 3 minutes before starting to protest. Mr. Jones decided to implement a 2 minute NCR program in which he would give Jimmy a sticker every 2 minutes regardless of the presence of interfering behaviors.
\item (Non-Example) Mr. Michael was concerned with Barry's aggressive behavior during group activities. He decided to give him a sticker for every 2 minutes that he did not engage in aggressive behavior.
\end{enumerate}
%
\subsection{Examples}
Gresham et al. (1993, pp. 261-262) give an example of an adequate definition of an independent variable, a time-out procedure, provided by Mace, Page, Ivancic and O'Brien (1986).
\begin{enumerate}
\item Immediately following the occurrence of a target behavior (temporal dimension), (b) the therapist said, ``No, go to time-out'' (verbal dimension), (c) led the child by the arm to a prepositioned time-out chair (physical dimension), and (d) seated the child facing the corner (spatial dimension). (e) If the child's buttocks were raised from the time-out chair or if the child's head was turned more than 45° (spatial dimension), the therapist used the least amount of force necessary to guide compliance with the time-out procedure (physical dimension). (f) At the end of 2 min (temporal dimension), the therapist turned the time-out chair 45° from the corner (physical and spatial dimensions) and walked away (physical dimension).
\item Gresham et al., (1993, p. 262) argued that a failure to define operational variables along these four dimensions, as done so by Mace, Page, Ivancic and O'Brien (1986), makes ``replication and external validation of behavior-analytic investigations difficult.''
\end{enumerate}
%
