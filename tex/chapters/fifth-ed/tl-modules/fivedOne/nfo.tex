%\clearpage \section[\fourbThree{}]{\fourbThree{}%              
\subsection{Definitions} 
Independent Variable - ``The variable that is systematically manipulated by the researcher in an experiment to see whether changes in the independent variable produce reliable changes in the dependent variable. In applied behavior analysis, it is usually an environmental event or condition antecedent or consequent to the dependent variable. Sometimes called the intervention or treatment variable'' (Cooper, Heron, \& Heward, 2007, p. 697).\\

Dependent Variable - ``The variable in an experiment measured to determine if it changes as a result of the manipulations of the independent variable; in applied behavior analysis, it represents some measure of a socially significant behavior'' (Cooper et al. 2007, p. 693).\\

Dependent variables must be operationally defined to allow for consistent assessment and replication of the assessment process, measured repeatedly within and across controlled conditions, recording is assessed for consistency across the experiment using inter-observer agreement, and dependent variables must be socially significant to the individual or those around them. (Horner, Carr, Halle, McGee, Odom, \& Wolery, 2005)\\

Experimental control is achieved when predicted change in the dependent variable (i.e., the behavior) covaries with manipulations of the independent variable (i.e., the intervention) showing the effectiveness of the independent variable on the dependent variable of a participant. (Horner et al., 2005)
%
%\clearpage \section[\fourbEleven{}]{\fourbEleven{}%
\subsection{Definition}
Parametric analysis - ``An experiment designed to compare the differential effects of a range of values of the independent variable'' (Cooper, Heron, \& Heward, 2007, p. 701).
%
%\clearpage \section{\fourFKThirtyThree{}}
\subsection{Definition}
Functional relation - ``An experimentally determined relation that shows that the dependent variable depends on or is a function of the independent variable and nothing else'' (Johnston \& Pennypacker, 2009, p. 358).\\

``A `cause' becomes a ‘change in an independent variable' and an `effect' a `change in a dependent variable.' The old ‘cause-and-effect connection' becomes a ‘functional relation.' The new terms do not suggest how a cause causes its effect; they merely assert that different events tend to occur together in a certain order'' (Skinner, 1953, p.23).\\

For every response are a number of factors that influence the likelihood that it occurs. Each one of these factors can be used as an independent measure in an experiment. If an experiment shows that there is a different between a context in which this variable is present vs when this variable is absent, we consider it to be a functional relation (Cooper, Heron, \& Heward, 2007).\\

For behavior analysts, functional relations are important to discover. When we understand how a behavior is related to the environment, we can then decide what treatment to use.\\
%
%\clearpage \section[\fourdTwenty{}]{\fourdTwenty{}%
\subsection{Definition}
Noncontingent reinforcement (NCR) – ``A procedure in which stimuli with known reinforcing properties are presented on fixed-time (FT) or variable time (VT) schedules completely independent of behavior; often used as an antecedent intervention to reduce problem behavior'' (Cooper, Heron \& Heward, 2007, p. 700).

Noncontingent reinforcement is sometimes used in applied research in an experimental design called the NCR reversal technique. This design involves a baseline phase, NCR phase (where a potential reinforcer is delivered on a fixed or variable time schedule independent of the target behavior), and a phase where the reinforcer is delivered contingent on a set behavioral criteria. The phases are repeated as necessary to indicate experimental control over the dependent variable. The NCR and baseline conditions function as a comparative measure to validate the independent variable in these studies.

Differential reinforcement procedures may limit access to reinforcement if appropriate behavior occurs at low rates. NCR gives consistent access to reinforcement. 
%
%\clearpage \section[\fouriTwo{}]{\fouriTwo{}%
\subsection{Definition}
This task relates to the importance of defining environmental variables in observable and measurable terms.
\begin{enumerate}
\item As Cooper, Heron, and Heward (2007) state, in order to achieve a high level of treatment integrity in an experiment, it is of utmost importance to ``develop complete and precise operational definitions of the treatment procedures'' (Cooper, Heron, \& Heward, 2007, p. 235). In the same way that it is critical to define target behavior in observable and measurable terms, so is the case with defining environmental variables. 
\item Baer, Wolf, and Risley (1968) stress that the ``technological'' dimension of Applied Behavior Analysis refers simply to the fact that ``the techniques making up a particular behavioral application are completely identified and described''(Baer, Wolf, \& Risley, 1968, p. 95). As such, the techniques, or environmental variables being manipulated, must be defined in observable and measurable terms to meet the technological dimension of applied behavior analysis (Cooper, Heron, \& Heward, 2007). 
\item However, historically, operationally defining independent variables has not been conducted to the standard required for a science of behavior that seeks to achieve the technological dimension of applied behavior analysis*. It has also not been done to the same standard as that of the dependent variables (Johnston \& Pennypacker, 1980; Peterson, Homer \& Wonderlich, 1982; Gresham, Gansle \& Noell, 1993). In 1982 Peterson, Homer and Wonderlich called for researchers to measure the independent variables in a more stringent manner. Unfortunately, an assessment of this area later on by Gresham, Gansle and Noell (1993) found that this had not been accomplished.
\item Defining environmental variables in observable and measurable terms
\item It is believed that environmental variable definitions should be written to meet the same standards as those required to be met by target behavior definitions (Gresham, Gansle \& Noell, 1993). They should be ``clear, concise, unambiguous, and objective'' (Cooper, Heron \& Heward, 2007, p. 235).
\item Gresham et al. (1993, p. 261) suggest that independent variable definitions can be made along four dimensions: spatial, verbal, physical and temporal. 
\end{enumerate}
%
Footnotes\\
*See Baer, Wolf \& Risley (1968) for more information on the seven dimensions of Applied Behavior Analysis.

