%%\clearpage \section[\fourkTen{}]{\fourkTen{}%
\subsection{Definition}
Behavior analysts follow guidelines related to arranging for termination of services in 2.15 of the Behavior Analyst Certification Board professional and ethical compliance code for behavior analysts:

2.15 Interrupting or Discontinuing Services. \\
 (d) Discontinuation only occurs after efforts to transition have been made. Behavior analysts discontinue a professional relationship in a timely manner when the client: (1) no longer needs the service, (2) is not benefiting from the service, (3) is being harmed by continued service, or (4) when the client requests discontinuation. \\
(e) Behavior analysts do not abandon clients. Prior to discontinuation, for whatever reason, behavior analysts: discuss the client's views and needs, provide appropriate pre-termination services, suggest alternative service providers as appropriate, and take other reasonable steps to facilitate timely transfer of responsibility to another provider if the client needs one immediately, upon client consent. (BACB, 2014, P. 10).

Before services are terminated, behavior analysts must discuss the client's needs with all pertinent parties (e.g. client's parents, legal guardians, school administrators). The client's welfare should be prioritized above all else and a transition plan should be put in place well before services are discontinued. Referrals to other professionals should be given if appropriate. (Bailey \& Burch, 2005)
%
%\clearpage \section[\fourkFive{}]{\fourkFive{}%
\subsection{Definition}
Procedural Integrity - ``The extent to which the independent variable is applied exactly as planned and described and no other unplanned variables are administered inadvertently along with the planned treatment'' (Cooper Heron, \& Heward, 2007, pp. 706-707).

Treatment Drift - ``An undesirable situation in which the independent variable of an experiment is applied differently during later stages than it was at the outset of the study'' (Cooper Heron, \& Heward, 2007, p. 706).

Low treatment integrity is not only bad for research (confounding, cannot interpret the results) but can also lead to inconsistencies and poor outcomes in treatments.  These can be related to many factors such as experimenter bias (unfair advantage to see positive results), staff implementing only procedures they favor, treatment too difficult to implement, poor staff training, or staff turnover.

Systems to Avoid Treatment Drift
\begin{enumerate}
\item Precise operational definition
\item Make behavioral plan simple and easy to administer
\item Provide competency-based training (use behavior skills training)
\item Assess treatment integrity
\end{enumerate}

Assessing Treatment Integrity
\begin{enumerate}
\item Provide a brief checklist of each of the components in the treatment plan
\item May be self-monitored but important to conduct inter-rater reliability
\item Establish a schedule to complete the treatment integrity
\item Graph the percent of treatment integrity and monitor to ensure treatment drift does not occur
\end{enumerate}
%
\subsection{Footnotes}
Also called procedural fidelity, treatment integrity, procedural reliability, or treatment adherence.
