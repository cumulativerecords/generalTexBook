\clearpage \section[\fouriFour{}]{\fouriFour{}%
              \sectionmark{I-04 Design and implement...}}
\sectionmark{I-04 Design and implement...}
\subsection{Definition}
Functional behavior assessment (FBA) - ``A systematic method of assessment for obtaining information about the purposes (functions) a problem behavior serves for a person; results are used to guide the design of an intervention for decreasing the problem behavior and increasing appropriate behavior'' (Cooper, Heron, \& Heward, 2007, p. 696).

Functional analysis (FA) - ``An analysis of the purposes of problem behavior, wherein antecedents and consequences representing those in the person's natural routines are arranged within an experimental design so that their separate effects on problem behavior can be observed and measured'' (Cooper, Heron, \& Heward, 2007, p. 696).* The FA is considered to be best practice standard in conducting a functional assessment (Hanley, Iwata, \& McCord 2003).  

Descriptive assessment - ``Direct observation of problem behavior and the antecedent and consequent events under naturally occurring conditions'' (Cooper, Heron, \& Heward, 2007, p. 693).* The advantages are that the information yields what happens in the individual's natural environment, does not disrupt the individual's routine, and provides information for designing a functional analysis. The disadvantages of these assessments are false positives due to behavior maintained by intermittent reinforcement or the presence of antecedent and consequent events which are often present but have no functional relation, the time required in taking data, and inaccurate data collection. Also, there is little correspondence between descriptive analysis outcomes being compared to functional analysis outcomes (Pence, Roscoe, Bourret, and Ahearn, 2009)

Indirect assessment - ``Structured interviews, checklists, rating scales, or questionnaires used to obtain information from people who are familiar with the person exhibiting the problem behavior'' (Cooper, Heron, \& Heward, 2007, p. 697).  The advantages of indirect assessments are that the forms can yield valuable information and are convenient.  The disadvantage is the lack of research supporting the reliability of these measurements.  

The ethical guidelines of BACB requires BCBAs to conduct a functional assessment according to 3.01 Behavior-Analytic Assessment. \\
(a) Behavior analysts conduct current assessments prior to making recommendations or developing behavior-change programs. The type of assessment used is determined by clients' needs and consent, environmental parameters, and other contextual variables. When behavior analysts are developing a behavior-reduction program, they must first conduct a functional assessment.\\
(b) Behavior analysts have an obligation to collect and graphically display data, using behavior-analytic conventions, in a manner that allows for decisions and recommendations for behavior-change program development (BACB, 2014, p.8).

Role of Functional Behavior Assessment as outlined by Cooper, Heron, and Heward (2007).
\begin{enumerate}
\item Identifies antecedent variables that may be altered to prevent problem behaviors.
\item Identifies reinforcement contingencies which can be altered so problem behavior no longer receives reinforcement.
\item Identifies reinforcers for alternative replacement behavior.
\item Reduces the reliance on default technologies such as punishment.
\end{enumerate}

Conducting a functional behavior assessment as outlined by Cooper, Heron, and Heward (2007).
\begin{enumerate}
\item Conduct indirect assessments
\item Conduct a descriptive assessment
\item Analyze data  from the indirect and descriptive assessment and create a hypotheses 
\item Test conditions the data suggest may be contributing to the behavior using a functional analysis
\item Develop intervention strategies based on the results.
\end{enumerate}
%
\subsection{Assessment}
\begin{enumerate}
\item Have supervisee list in order the process of completing a functional assessment.
\item Have supervisee describe the benefits and limitations of each of the assessment procedures.
\item Use behavior skills training techniques to teach your supervisee how to conduct an indirect assessment.  Accompany the supervisee in completing their first indirect assessment and provide reinforcement and feedback following the session.
\item Use behavior skills training techniques to teach your supervisee how to collect ABC data.  Explain the process, model and then practice using you tube videos of challenging behaviors.  Provide reinforcement and feedback and continue practicing until the supervisee clearly demonstrates skills in collecting ABC data.  Accompany the supervisee in completing live ABC data, take inter-rater reliability and compare scores providing reinforcement and feedback.
\item Use behavior skills training techniques to teach your supervisee how to complete an FA. Practice until the supervisee clearly demonstrates skills in the control condition and some test conditions of an FA.  Accompany the supervisee in completing live FAs, prompting and providing reinforcement and feedback.  Continue to monitor supervisee in this process until they have demonstrated the completion of multiple FA's accurately and are able to set up individualized FAs based on the descriptive data.
\end{enumerate}
%
\subsection{Relevant Literature}
\begin{refsection}
\nocite{baer1987some,
        bac2014professional,
        cooper2007applied,
        hanley2003functional,
        iwata1994functions,
        pence2009relative,
        sasso1992use}
\printbibliography[heading=none]
\end{refsection}
%
\subsection{Related Tasks}
\fourbThree{}\\
\fourbFive{}\\
\foureOne{}\\
\fourhOne{}\\
\fourhThree{}\\
\fourhFive{}\\
\fouriThree{}\\
%
\subsection{Footnotes}
*Functional analysis may also be called an analog analysis or experimental analysis.
*Descriptive analysis may also be called direct assessment.
%
\clearpage \section[\fouriThree{}]{\fouriThree{}%
              \sectionmark{I-03 Design and implement indiv...}}
\sectionmark{I-03 Design and implement indiv...}
\subsection{Definition}
``Behavioral assessment involves a variety of methods including direct observations, interviews, checklists, and tests to identify and define targets for behavior change'' (Cooper, Heron, \& Heward, 2007, p. 49).

``Applied behavior analysis uses the methods of FBA to identify antecedent and consequent events and to use this information in designing interventions to change socially significant behaviors'' (Gresham, Watson, \& Skinner, 2001, p. 157).

``FBA is designed to obtain information about  the purpose (function) a behavior serves for a person... FBA is used to identify the type and source of reinforcement for challenging behaviors as the basis for intervention efforts...'' (Cooper et al., 2007, p. 501).

``Once the function of behavior is determined, this information is used to design interventions to reduce problem behaviors and to facilitate positive behaviors'' (Gresham et al., 2001, p. 158).

The first step in the process is to define the target behaviors that the assessment will focus on. These behaviors are typically identified by teachers, therapists, or caregivers due to their interference with learning, adaptive functioning, and overall quality of life. In following the principles of ABA, the behaviors targeted for assessment must be socially significant.

Another direct method of determining the function of a behavior is to conduct a functional analysis. This involves systematic manipulation of the environment, while controlling variables to evoke the target behavior under conditions representing each possible function. The typical functions of behavior include access to attention, access to tangible items, escape from demands, and automatic reinforcement. During a functional analysis, each function is assessed to determine if they are maintaining the target behavior. FA is considered to be the most reliable source for determining the function of a behavior, but may not be feasible in some settings due to the time it takes to conduct, safety implications (depending on the severity of the target behavior), and resources needed to conduct each experimental phase.

Other methods used to gather information about the function of target behaviors in a behavioral assessment include a thorough review of the client's previous records (academic reports, past evaluations, behavior support plans, IEP's, etc.), the use of behavioral rating scales such as the FAST (functional analysis screening tool), MAS (Motivational Assessment Scale), or PBQ (Problem Behavior Questionnaire)), structured interviews with caregivers (i.e., functional assessment interview form), direct observation in the target environment (i.e., home, school, community), behavior data collection and analysis, and A-B-C data collection and analysis. These indirect assessments should be used to inform an experimental functional analysis. They are not designed to determine the function of a response on their own.

Once the direct and indirect assessments are completed, this information is analyzed and the BCBA makes recommendations for intervention based on the results of assessments. 
%
\subsection{Examples}
\begin{enumerate}
\item Rich is completing a Functional Behavior Assessment on the aggression of one of his students. After teacher interviews, the completion of rating scales, and several observations in various settings, Rich hypothesizes that the function of aggression is escape from demands. He uses this information to create an intervention plan to decrease aggression at school. 
%
\end{enumerate}
%
\subsection{Assessment}
\begin{enumerate}
\item Have supervisee complete a behavior rating scale on an individual based on one of their behaviors. This can include the FAST, MAS, PBQ, or another common rating scale used in behavioral assessments. Once the supervisee has completed the rating scale, have them score the form and present the results. 
\item Supervisor will create role plays in which each supervisee will collect ABC data on specific topography of problem behavior. After a number of instances have been recorded, supervisee will analyze the data and formulate a hypothesized function of the problem behavior.
\item Have supervisee read Iwata, Dorsey, Slifer, Bauman, \& Richman (1982/1994) and describe the experimental conditions of a functional analysis.
%
\end{enumerate}
%
\subsection{Relevant Literature}
\begin{refsection}
\nocite{carr1993behavior,
        cooper2007applied,
        watson2001functional,
        iwata1994toward,
        o1997functional,
        sprague1998antisocial,
        witt2000functional}
\printbibliography[heading=none]
\end{refsection}
%
\subsection{Related Tasks}
\fourbThree{}\\
\foureOne{}\\
\fourgOne{}\\
\fourgFour{}\\
\fourgFive{}\\
\fourgEight{}\\
\fourhTwo{}\\
\fouriOne{}\\
\fouriTwo{}\\
\fouriFour{}\\
\fouriSix{}\\
\fourjOne{}\\
\fourjTwo{}\\
\fourjTen{}\\
