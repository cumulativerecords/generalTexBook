\clearpage \section[\fourcOne{}]{\fourcOne{}%
              \sectionmark{C-01 State... reinforcement.}}
\sectionmark{C-01 State... reinforcement.}
\subsection{Definition}
Reinforcement has been long been defined as a crucial element to behavioral change. However, there are considerations that behavioral analysts should explore before implementing reinforcement strategies. Here are some considerations for the use of positive reinforcement:
\begin{enumerate}
\item May suppress the desired response (e.g., The availability of the reinforcer elicits behavior that may compete with the target response) (Balsam \& Bondy, 1983). 
\item May not be feasible for an individual that has little or no learning history with that reinforcement contingency (e.g., An individual that is being taught to swallow solid food may not progress with a program solely using positive reinforcement due to low baseline levels of swallowing solid foods) (Riordan, Iwata, Wohl \& Finney, 1984).  
\item Increases the frequency of the target behavior, thereby reducing the frequency in other responses that may also be desirable (e.g., While teaching a student to raise their hand and wait until they are called on in class, the student no longer garners others' attention by calling their name) (Balsam \& Bondy, 1983). 
\item May evoke aggression in others, especially in conditions which there are limited quantities of the reinforcer. Aggression may be directed at individuals that are also competing for same reinforce (Balsam \& Bondy, 1983).
\item  May also evoke aggression when group contingencies are used. Individual may become aggressive towards lower-performing teammates (Balsam \& Bondy, 1983). 
\item Removal of positive reinforcer has been correlated with lower than baseline levels of responding (Balsam \& Bondy, 1983).
\end{enumerate}
%
There are also considerations for the use of negative reinforcement. Here are some:
\begin{enumerate}
\item Can result in more challenging behavior due to the continuation of aversive stimulation if target behavior is not displayed 
\item Research shows that even for escape-maintained behavior, positive reinforcement contingencies may compete with negative reinforcement contingencies, therefore, decreasing escape-maintained behavior (Lerman, Volkert \& Trosclair, 2007). 
\item Example: A child that engages in aggression to escape tasks may be more likely complete tasks without aggression if access to an iPad was given contingent on work completion. This may be more effective than reducing aggression by providing breaks contingent on appropriate asking.
\item Negative reinforcement contingencies may reinforce minimal requirements needed to avoid/escape aversive stimulus; does not focus on quality of target response (Balsam \& Bondy, 1983).
\item Requires continuous aversive stimulation and aversive stimulation often elicits aggressive responses.
\item These unwanted effects of reinforcement can be curbed taking baseline levels of the target behavior before setting criteria for reinforcement, implementing preference assessments routinely, ensuring reinforcement schedule and reinforcers chosen are as natural to the individual's environment as possible to promote generalization, systematic thinning of the reinforcement schedule, and having concurrent schedules of reinforcement for positive and negative reinforcement when negative reinforcement is utilized.
\end{enumerate}
%
\subsection{Assessment}
\begin{enumerate}
\item Ask supervisee to list considerations for the use of positive reinforcement.
\item Ask supervisee to list considerations for the use of negative reinforcement.
\item Have supervisee outline which considerations may affect a particular client and what behavioral strategies can be used to curb these unwanted effects.
\end{enumerate}
%
\subsection{Relevant Literature}
\begin{refsection}
\nocite{balsam1983negative,
    flora2004power,
    kodak2007further,
    riordan1980behavioral}
\printbibliography[heading=none]
\end{refsection}
%
\subsection{Related Lessons}
\fourdOne{}\\
\fourdTwo{}\\
\fourdTwentyOne{}\\
\foureEleven{}\\
\fouriSeven{}\\
\fourjFour{}\\
\fourjFive{}\\
\fourjSix{}\\
\fourjSeven{}\\
\fourjEleven{}\\
%
\clearpage \section[\fourcTwo{}]{\fourcTwo{}%
              \sectionmark{C-02 State... punishment.}}
\sectionmark{C-02 State... punishment.}
\subsection{Definition}
Punishment is sometimes used to change or shape behavior and may cause unwanted side effects. 
\begin{enumerate}
\item For instance, those individuals who are being treated using punishment procedures may become aggressive (Azrin \& Holz, 1966) or may have strong emotional reactions to such measures.  
\item An adult or child may become subject to negative modeling (such as imitating scolding or hitting behavior).
\item Those treated through the use of punishment may seek out escape/avoidance of the punisher or the contingencies surrounding punishment. 
\item In extreme cases, the use of punishment can result in harm or injury to the child or adult.  
\end{enumerate}

Punishment may also have unwanted effects with regards to an individual's future learning.  It may not appropriately generalize to new situations requiring further intervention. When used as the sole intervention in a treatment package, it fails to teach an individual an alternative behavior to engage in and consequently individuals may revert back to old behaviors without a replacement strategy. These behaviors may diminish temporarily only to be subject to a recovery of responding at a later period of time. (Catania, 1998)

As a result, the majority of those in the field agree that ``punishment be limited to those situations in which other interventions have failed'' (May, Risley, Twardosz, Friedman, Bijou, Wexler et al., 1975 as cited in Iwata et al., 1994, p. 198).  Iwata et al., (1994) described that reinforcement approaches to behavior reduction were just as effective as punishment approaches and that if a functional analysis of the problem behavior was done, the need for the use of punishment procedures was greatly reduced.  

During the supervision process, be diligent in choosing interventions, which are based on reinforcement and not solely on punishment.  The function of a problem behavior should always be assessed before making decisions regarding an individuals program to ensure effective treatment.  If a team has deemed that punishment is necessary as a part of a treatment package, it is important to state any potential unwanted effects of any procedure being utilized and to attempt to plan for these.

Consider the following when planning for punishment effects:
\begin{enumerate}
\item A team should always adhere to the ``Fair Pair Rule'' when using punishment.  This states that a ``practitioner should choose one or more alternatives to increase for every behavior targeted for reduction'' (White \& Haring, 1980, p. 423).
\item Be sure to plan for continuation of the procedure to different environments, staff and stimuli (any and all that apply).
\item Avoid modeling any behavior which you do not want the adult or child to imitate
\item The team should develop a contingency plan for managing aggression or extreme emotional responses (should they occur) and have safety measures in place to avoid accidental injury to the individual.
\item The team should develop a plan to manage any escape/avoidant behaviors that may occur 
\item Be aware the effects of punishment can be difficult to predict.  Staff may need to adjust the plan over time if the affects are not therapeutic or effective.  
\end{enumerate}
%
\subsection{Assessment}
\begin{enumerate}
\item Ask the supervisee to state the unwanted effects of punishment.
\item Ask the supervisee to plan for unwanted effects of punishment.  The supervisor should provide examples of commonly used punishment procedures within agency (such as restraint, time outs, or other punitive measures) and ask the supervisee to propose solutions to these problems. 
\item Ask the supervisee to select one behavior to target for increase for each behavior targeted for decrease. 
\end{enumerate}
%
\subsection{Relevant Literature}
\begin{refsection}
\nocite{azrin1966punishment,
    catania1998learning,
    cooper2007applied,
    iwata1994toward,
    may1976guidelines,
    white1980exceptional}
\printbibliography[heading=none]
\end{refsection}  
%
\subsection{Related Lessons}
\fourcTwo{}\\
\fourdSixteen{}\\
\fourdSeventeen{}\\
\fourdNineteen{}\\
\foureSeven{}\\
\fourFKThirtyOne{}\\
%
\clearpage \section[\fourcThree{}]{\fourcThree{}%
              \sectionmark{C-03 State... extinction.}}
\sectionmark{C-03 State... extinction.}
\subsection{Definition}
Extinction ``occurs when reinforcement of a previously reinforced behavior is discontinued; as a result the frequency of that behavior decreases in the future'' (Cooper, Heron \& Howard, 2007, p. 457). Extinction renders target behavior useless and is often a significant component contributing to the effectiveness of a behavioral program. However, extinction should be used with caution. Two common side effects may occur when extinction is utilized: an extinction burst, defined as ``an immediate increase in the frequency of the response after the removal of the positive, negative, or automatic reinforcement'' (Cooper, Heron \& Howard, 2007, p. 462), and extinction-induced aggression.

Studies have compared withdrawal of reinforcement as an aversive event and responses to extinction are similar to attack responses in laboratory subjects exposed to aversive stimulation, such as heat, shocks and physical blows (Lerman et al., 1999). Extinction should not be used as a singular intervention when self-injury or aggression is severe and cannot be prevented and appropriate safe-guards cannot be put in place. Other considerations include extinction being inappropriate in settings where maladaptive behaviors are likely to be imitated by others (e.g., classroom setting) and when extinction is not feasible (e.g. if an individual engages in physical aggression for attention, response-blocking may be enough to reinforce the individual's behavior). 

Research has shown that there are other behavioral strategies that can be utilized to mitigate the unwanted effects of extinction. These include using differential reinforcement of alternative behavior in conjunction with extinction procedures. In situations in which using extinction is not possible, research has shown that by manipulating reinforcement schedules and reinforcement parameters (e.g. quality, duration, immediacy of reinforcement) to favoring appropriate behavior rather than problem behavior, problem behavior has also been shown to decrease (Athens \& Vollmer, 2010).
%
\subsection{Assessment}
\begin{enumerate}
\item Ask your supervisee to list the possible unwanted effects of extinction.
\item Ask your supervisee in which situations should extinction not be utilized.
\item Ask your supervisee what behavioral strategies can mitigate the unwanted effects of extinction.
\item Have your supervisee determine which of his/her clients could benefit from extinction and which clients should avoid the use of extinction. 
%
\end{enumerate}
%
\subsection{Relevant Literature}
\begin{refsection}
\nocite{athens2010investigation,
    cooper2007applied,
    lerman1995prevalence,
    lerman1999side}
\printbibliography[heading=none]
\end{refsection}
%
\subsection{Related Lessons}
\fourdTwo{}\\
\fourdEighteen{}\\
\fourdNineteen{}\\
\foureOne{}\\
\foureEight{}\\
%
\clearpage \section{\foureSeven{}}
\subsection{Definition}
What is behavioral contrast?
\begin{enumerate}
\item George Reynolds first presented behavioral contrast in 1961. He defined behavioral contrast as ``an increase in the rate of responding in one component of a multiple schedule when certain changes occur in the other component'' (p. 60). 
\item Cooper, Heron and Heward (2007) state that behavioral contrast ``can occur as a function of a change in reinforcement or punishment density on one component of a multiple schedule'' (p. 337).
\end{enumerate}
%
\subsection{Examples}
\begin{enumerate}
\item Cooper, Heron and Heward (2007) give a good example of behavioral contrast to illustrate the concept; ``...a pigeon pecks a backlit key, which alternates between blue and green, reinforcement is delivered on the same schedule on both keys, and the bird pecks at approximately the same rate regardless of the key's color'' (p. 337). However, this changes so that responses on one component of the schedule are punished, i.e., pecks on the blue key are punished because reinforcement is not delivered, but pecks on the other (green) key continue to produce reinforcement. As a result, rate of responding decreases on the blue key and rate of responding on the green key increases, even though no more reinforcement is delivered from the green key than before.
\end{enumerate}
%
Plan for the effects of behavioral contrast
\begin{enumerate}
\item It is important to consider prior to beginning an intervention, whether behavioral contrast may occur as a result of that planned intervention. If behavioral contrast is a possibility; then planning for its occurrence is crucial.
\item Cooper, Heron and Heward (2007, p. 338) suggest that one way to minimize or completely prevent the contrast effects of punishment is to plan the intervention so that the consequence is consistently applied to the target behavior across all relevant environments and stimulus conditions. All those involved in the client's life that may be required to deliver the consequence, will need to be thoroughly trained to ensure its consistent implementation.
\item Additionally, reinforcement will need to be minimized, or where possible, withheld, when the target behavior has occurred. Similarly, training will need to be provided to all those involved so that the client isn't receiving reinforcement when the target behavior is emitted.
\end{enumerate}
%
\subsection{Assessment}
\begin{enumerate}
\item Ask your Supervisee to define behavioral contrast.
\item Ask your Supervisee to describe what will happen in this example if behavioral contrast is in effect: 
\item A child has been playing with two musical toys, allocating an equal amount of time playing with each toy. One of the musical toys is yellow and one of the musical toys is red. The red toy's battery begins to give out so that when the child presses the button, sometimes the music is not produced. However, the yellow musical toy continues to work well and music is produced each time the child pushes the button. What will happen to the rate of responding for each of the musical toys? (Answer = the rate of responding on the red musical toy will decrease and the rate of responding on the yellow toy will increase, even though there isn't any additional reinforcement being produced from the yellow toy).
\end{enumerate}
%
\subsection{Relevant Literature}
\begin{refsection}
\nocite{cooper2007applied,
        fagen1978behavioral,
        hantula1994behavioral,
        mcsweeney1998habituation,
        reynolds1961behavioral,
        reynolds1963some,
        tarbox2005verbal,
        weatherly1996picking,
        weatherly2002rats}
\printbibliography[heading=none]
\end{refsection}
%
\subsection{Related Lessons}
\fourcTwo{}\\
\fourdFifteen{}\\
\fourdSixteen{}\\
\fourdSeventeen{}\\
\fourdNineteen{}\\
\fourjTen{}\\
\fourkTwo{}\\
\fourFKThirtyEight{}\\
\fourFKFourty{}\\
%
\clearpage \section{\fourFKThirtyEight{}}
\subsection{Definition} 
Behavioral contrast - ``The  phenomenon in which a change in one component of a multiple schedule that increases or decreases the rate of responding on that component is accompanied by a change in the response rate in the opposite direction on the other, unaltered component of the schedule; behavior punished in one situation may increase in other situations where it's not punished... Contrast effects of punishment can be minimized, or prevented altogether, by consistently punishing occurrences of the target behavior in all relevant settings and stimulus conditions, withholding or at least minimizing the person's access to reinforcement for the target behavior, and providing alternative desirable behaviors'' (Cooper, Heron, \& Heward, 2007, p. 337)
%
\subsection{Examples}
\begin{enumerate}
\item Rich sneaks candy from home and eats it in class. His teacher catches him one day and he stops eating candy in the classroom. However, he now asks to go to the bathroom every morning and eats candy in the bathroom where the teacher cannot see him. 
%
\end{enumerate}
%
\subsection{Assessment}
\begin{enumerate}
\item Have supervisee describe the principles of punishment and reinforcement and how behavioral contrast relates to each.
\item Have supervisee explain why rates may fluctuate based on punishment and reinforcement contingencies in multiple schedules.
\item Have supervisee give examples of behavioral contrast from their experiences and what they have done in those circumstances.
%
\end{enumerate}
%
\subsection{Relevant Literature}
\begin{refsection}
\nocite{cooper2007applied,
        gross1981behavioral,
        koegel1980behavioral,
        nevin1992behavioral,
        reynolds1961behavioral}
\printbibliography[heading=none]
\end{refsection}
%
\subsection{Related Lessons}
\fourcOne{}\\
\fourcTwo{}\\
\fourdOne{}\\\
\fourdTwo{}\\\
\fourdFifteen{}\\
\fourdSixteen{}\\
\fourdSeventeen{}\\
\fourdNineteen{}\\
\foureSeven{}\\
\fourFKEighteen{}\\
\fourFKTwenty{}\\
\fourFKTwentyOne{}\\
