\clearpage \section[\fourdEight{}]{\fourdEight{}%
              \sectionmark{D-08 Use discrete-trial...}}
\sectionmark{D-08 Use discrete-trial...}
\subsection{Definition} 
Free operant - `` behaviors that have discrete beginning and ending points, require minimal displacement of the organism in time and space, can be emitted at nearly any time, do not require much time for completion, and can be emitted over a wide range of response rates'' (Cooper, Heron, \& Heward, 2007, p. 696).

Rate of responding is typically used to measure behavior considered to be free operant. Rate of responding is typically used because it uses count per unit of time. (i.e., a person can type 50 words per minute). However, rate of responding is not typically used to measure behavior that occurs within discrete trials. These responses can occur only within limited or restricted situations. 

Discrete trial responses include responses to flash cards, answering teacher's questions, and choosing an answer from an array. 

Discrete trial- ``Any operant whose response rate is controlled by a given opportunity to emit the response. Each discrete response occurs when an opportunity to respond exists'' (Cooper et al., 2007, p. 694).

\subsection{Assessment}
\begin{enumerate}
\item Have supervisee create a simple discrete trial program to teach a skill. Have him/her run conduct a trial using this program and identify the parts of a discrete trial.
\item Have supervisee identify the differences between a discrete trial and a free operant trial. Have him/her list pros and cons for each method.
\item Have supervisee identify what type of measurement procedures and data you would take for free operant trials as well as discrete trials. Have him/her describe why they would use that measurement system for each.
\end{enumerate}
%
\subsection{Relevant Literature}
\begin{refsection}
\nocite{cooper2007applied,
    malott2003principles,
    mazur2002learning}
\printbibliography[heading=none]
\end{refsection}
%could not locate bibliography or verify reference.
%Otto, J. (2003). Discrete trial procedures vs. free-operant procedures. Retrieved from http://old.dickmalott.com/students/undergradprogram/psy3600/discrete_vs_free.html. 
%
\subsection{Related Lessons}
\fouraOne{}\\
\fouraTwo{}\\
\fouraSeven{}\\
\fourdThree{}\\
\foureOne{}\\
\foureTwelve{}\\
\fourhOne{}\\
\fourFKTen{}\\
\fourFKEleven{}\\
\fourFKFourtySeven{}\\
%
\clearpage \section[\fourkNine{}]{\fourkNine{}%
              \sectionmark{K-09 Secure the support...}}
\sectionmark{K-09 Secure the support...}
\subsection{Definition}
Foxx (1996, p. 230) stated that in programming successful behavior change interventions, ``10\% is knowing what to do; 90\% is getting people to do it... Many programs are unsuccessful because these percentages have been reversed'' (Cooper et al., 2007, p. 652). 

Being explicit yet simplistic in describing programs and protocols will help secure support from other individuals in a client's environment. If a behavior change procedure or program is too difficult, technical, or places unreasonable demands on the other individuals involved, they are less likely to implement these programs. In addition, adequate training of behavior procedures should be provided to ensure proper implementation by those interacting with the client in the natural environment. Specifically, training pertaining to the delivery of reinforcers, which maintain the individual's newly acquired behavioral repertoires.

Jarmolowicz et al. (2008) compared the effectiveness of conversational language instructions and technical language instructions when explaining how to implement a treatment to caregivers. They found that the caregivers that were given conversational language instruction implemented the treatment more accurately.
%
\subsection{Examples}
\begin{enumerate}
\item Richard is trying to generalize skills learned in the special education classroom for one of his students. He went to each teacher to explain how this will help the student in their class and answered any questions they may have about the programs. In addition, he conducted a training on the specific program and offered to consult with each teacher in order to make sure generalization was successful and the repertoire was maintained.
%
\end{enumerate}
%
\subsection{Assessment}
\begin{enumerate}
\item Have supervisee identify ways they can build a rapport with other service providers and help support them when they have a student who needs to generalize and maintain skills in new settings.
\item Have supervisee choose a particular behavior change program or strategy. Have him/her describe and explain this program to an individual who does not have a background in applied behavior analysis. 
\item Have supervisee choose a specific behavior change program. Have him/her practice explaining the benefits of this program to others in order to get them on board with implementing this program.
\item Give supervisee a complex behavior change program. Have him/her simplify this program and create guidelines and instructions that they could give to an individual who does not have knowledge of applied behavior analytic strategies and techniques. 
\end{enumerate}
%
\subsection{Relevant Literature}
\begin{refsection}
\nocite{cooper2007applied,
        jarmolowicz2008effects,
        stokes1974programming}
\printbibliography[heading=none]
\end{refsection}
%
\subsection{Related Lessons}
\fourhOne{}\\
\fourjOne{}\\
\fourkThree{}\\
\fourkFour{}\\
\fourkSix{}\\
\fourkEight{}\\
%
\clearpage \section{\fourdThree{}}
\subsection{Definition}
Prompts – ``...antecedent stimuli that increase the probability of a desired response'' (Piazza, \& Roane, 2014, p. 256)

Prompt fading – ``...transfer stimulus control from therapist delivered prompts to stimuli in the natural environment that should evoke appropriate responses'' (Walker, 2008 as cited in Fisher, Piazza, \& Roane, 2014, p. 412).

Prompts are used when teaching skills. Prompts can be used when teaching in task analysis, discrete trial, incidental teaching, etc. Prompt fading is important as the learner begins to show competence with the skill being taught. Fading allows the learner to become independent and meet naturalistic reinforcers for his/her behavior.

Prompts are generally divided into two categories: stimulus prompts and response prompts.

Stimulus prompts – ``...those in which some property of the criterion stimulus is altered, or other stimuli are added to or removed from the criterion stimulus'' (Etzel \& LeBlanc, 1979 cited in Fisher, Piazza, \& Roane, 2014, p. 256)
\subsection{Examples}
\begin{enumerate}
\item Response prompts – ``...addition of some behavior on the part of an instructor to evoke the desired learner behavior'' (Fisher, Piazza, \& Roane, 2014, p. 256).
\item Most-to-least prompting
\item Least-to-most prompting
\item Time delay prompts
\end{enumerate}
%
\subsection{Assessment}
\begin{enumerate}
\item Ask your supervisee to role-play several types of prompt strategies.
\item Ask your supervisee to role-play several types of prompt fading procedures.
\item Ask your supervisee to describe the transfer of stimulus control when a prompt is faded out e.g., what stimulus is controlling behavior while prompting vs what stimulus is controlling behavior after the prompt has been faded out.
\end{enumerate}
%
\subsection{Relevant Literature}
\begin{refsection}
\nocite{etzel1979simplest,
    fisher2014handbook,
    walker2008constant}
\printbibliography[heading=none]
\end{refsection}
%
\subsection{Related Lessons}
\fourdFour{}\\
\fourdFive{}\\
\fourdSix{}\\
\fourdSeven{}\\
\fourdEight{}\\
\foureOne{}\\
\foureTwo{}\\
\foureThirteen{}\\
\fourFKTwentyFour{}\\
%
\clearpage \section[\fourjSix{}]{\fourjSix{}%
              \sectionmark{J-06 Select... environments.}}
\sectionmark{J-06 Select... environments.}
\subsection{Definition}
``Achieving optimal generalized outcomes requires thoughtful, systematic planning. This planning begins with two major steps: (1) selecting target behaviors that will meet natural contingencies of reinforcement, and (2) specifying all desired variations of the target behavior and the settings/situations in which it should (and should not) occur after instruction has ended'' (Cooper, Heron, \& Heward, 2007, p. 623). In other words, an intervention must be selected that will allow the client to access reinforcement in a specific environment. If that is not possible, then alternative interventions should be explored. 

Ayllon and Azrin (1968) state that an important rule of thumb is to choose interventions that will help produce reinforcement after the intervention is discontinued. The intervention should support the student until they can access naturally existing contingencies (i.e., verbal praise from a teacher) and then more intensive, contrived contingencies should be systematically faded. The goal of most intervention programs is to teach a skill and then fade support so the client can implement that skill across settings. 

Cooper, Heron, \& Heward (2007, p. 626) identify 5 strategic approaches to promote generalized behavior change.
\begin{enumerate}
\item Teach the full range of relevant stimulus conditions and response requirements (i.e., teaching sufficient stimulus and response examples based on the setting)
\item Make the instructional setting similar to the generalization setting. (i.e., program common stimuli and teach loosely)
\item Maximize the target behavior's contact with reinforcement in the generalization setting. (i.e., ask people in the generalization setting to reinforce the target behavior, teach the learner to recruit reinforcement, and teach the target behavior to levels of performance required by natural existing contingencies of reinforcement.)
\item Mediate generalization (i.e., teach self-management skills \& contrive mediating stimulus)
\item Train to generalize. (i.e., reinforce response variability and instruct learner to generalize)
\end{enumerate}
%
\subsection{Examples}
\begin{enumerate}
\item George set up a token economy for Bill that systematically increased the number of responses needed to earn a token. After some time, Bill was earning tokens for completing an entire worksheet rather than earning a token for each question answered. This allowed Bill to independently complete worksheets in his general education classroom without a paraprofessional by his side giving him tokens after each answer. 
%
\end{enumerate}
%
\subsection{Assessment}
\begin{enumerate}
\item Give supervisee a reinforcement program. Have him/her create a fading procedure for this program to increase the number of responses required to earn a token. 
\item Have supervisee list the five strategies for promoting generalized behavior change and have him/her give examples of each.
\item Have supervisee describe and differentiate between contrived contingencies and naturally existing contingencies and give several examples of each.
%
\end{enumerate}
%
\subsection{Relevant Literature}
\begin{refsection}
\nocite{ayllon1968token,
        baer1999plan,
        cooper2007applied,
        snell2006instruction,
        stokes1977implicit,
        stokes1989operant}
\printbibliography[heading=none]
\end{refsection}
%
\subsection{Related Lessons}
\fourgEight{}\\
\fourjSeven{}\\
\fourjEight{}\\
\fourjEleven{}\\
\fourjTwelve{}\\
\fourkSeven{}\\
\fourkNine{}\\
%
\clearpage \section{\fourjTwelve{}}
\subsection{Definition}
Response maintenance refers to ``the extent to which a learner continues to perform the target behavior after a portion or all of the intervention responsible for the behavior's initial appearance in the learner's repertoire has been terminated'' (Cooper, Heron \& Heward, 2007, p. 703). Rusch and Kazdin (1981) note that withdrawing or gradually fading components of an individual's treatment package can support response maintenance. 

Program for behavior learned in structured environments to be maintained with contingencies in the client's natural environment. Thin reinoforcement schedules so that the natural environment can support and continue to maintain similar rates in behavior (e.g., While learning to mand, a child may be given a chip every time she asks for a chip. However, as she becomes more adept with this goal, the schedule of reinforcement should move from a continuous schedule to an intermittent schedule because the child will not always be given a chip upon request in her natural environment.)

Response maintenance can often be confused with generalization across multiple exemplars. The key difference is that response maintenance is said to occur if the response can be maintained in settings and situations in which it was previously exhibited, after generalization to that setting and/or situation has already occurred at least once in the past.  For instance, if an individual was taught how to purchase items at a store and did so successfully at some point at Starbucks, McDonald's and Target but did not exhibit this skill at Starbucks a month later, a lack of response maintenance is said to occur. If, however, the individual did not exhibit the skill at Macy's where the individual has never performed the skill, a lack of generalization is said to occur (Cooper, Heron, \& Heward, 2007). 
%
\subsection{Assessment}
\begin{enumerate}
\item Have supervisee give examples of response maintenance. 
\item In a role-play or on the job, ask supervisee how he/she would program for response maintenance. 
%
\end{enumerate}
%
\subsection{Relevant Literature}
\begin{refsection}
\nocite{cooper2007applied,
        rusch1981toward}
\printbibliography[heading=none]
\end{refsection}
%
\subsection{Related Lessons}
\fourjSix{}\\
\fourjSeven{}\\
\fourjEight{}\\
\fourjEleven{}\\
%
\clearpage \section{\fourfSix{}}
\subsection{Definition}
Incidental teaching - When the ``instructor assesses the child's ongoing interests, follows the child's lead, restricts access to high interest items, and constructs a lesson within the natural context, with a presumably more motivated child'' (Anderson \& Romanczyk, 1999, p. 169).

Incidental teaching requires an instructor to use moments in the natural environment as teaching opportunities. It can be used to teach language based skills, social skills, play skills, or other skills as well.

\subsection{Examples}
\begin{enumerate}
\item Todd often struggles to initiate play with his peers during recess. His teacher decided to go over and prompt Todd to introduce himself to Mike and ask him to play a game. 
\item (Non-example) Rich wanted to help teach Todd multiplication, so he gave him a worksheet and sat down with him to go through the problems one by one.
\end{enumerate}
%
\subsection{Assessment}
\begin{enumerate}
\item Have supervisee define and describe incidental teaching.
\item Have supervisee describe how they use incidental teaching at work to teach various skills.
\item Have supervisee describe the pros and cons of incidental teaching.
%
\end{enumerate}
%
\subsection{Relevant Literature}
\begin{refsection}
\nocite{anderson1999early,
        cooper2007applied,
        hart1975incidental,
        mcgee2007incidental,
        mcgee1999incidental,
        mcgee1983modified}
\printbibliography[heading=none]
\end{refsection}
%
\subsection{Related Lessons}
\fourbThree{}\\ 
\fourdFour{}\\
\fourdFive{}\\
\fourdEleven{}\\
\fourjSix{}\\
\fourjEleven{}\\
\fourFKFourtyFour{}\\
%
\clearpage \section{\fourfEight{}}
\subsection{Definition}
``A set of procedures and processes by which an individual's communication skills (i.e., production as well as comprehension) can be maximized for functional and effective communication. It involves supplementing or replacing natural speech and/or writing with aided (e.g. picture communication symbols, line drawing, Blissymbols, and tangible objects) and/or unaided symbols (e.g. manual signs, gestures, and finger spelling)'' (American Speech-Language-Hearing Association, 2002).

Augmentative communication includes any modality that supplements a person with difficulties engaging in spoken language. This can include gestures, sign language, PECS, electronic devices, picture books, etc.

The use of augmentative communication should be considered as a way to allow the user to access reinforcement from the natural environment. Often functionally equivalent responses can be taught to replace problematic behavior therefore leading to a decrease in that behavior (Durand, 1999).

Teaching of functional communication using augmentative communication devices should be taught using the same strategies to teach other skills (prompt fading, reinforcement considerations, generalization considerations, etc.)

Augmentative communication systems should not be confused with the long-discredited ``facilitated communication'' which is a pseudoscientific attempt at getting people with developmental disabilities to communicate.

\subsection{Examples}
\begin{enumerate}
\item Bill uses pictures to tell his teachers when he wants.
\item Ralph is eating snack and signs ``more'' to his teacher after running out of crackers to eat. She gladly hands him more and praises him for using his words.
\item Hillary uses an app on a tablet device that generates speech sounds so others respond to her.
\item (Non-example) Bill wants more crackers so tells his teacher, ``I want some more, please.'' using spoken word.
\item (Non-example) Bill wants more crackers so he hits his fist on the table and screams. His teacher says, ``oh, you're still hungry? Here are a few more crackers.''
\end{enumerate}
%
\subsection{Assessment}
\begin{enumerate}
\item Have supervisee research various augmentative communication systems. Have him/her choose a system and describe it in detail.
\item Have supervisee identify the benefits of using alternative communication systems for non-vocal-verbal students.
\item Have supervisee describe situations where they would seek the advice of speech-language professionals for more information regarding the pros and cons of each system.
\end{enumerate}
%
\subsection{Relevant Literature}
\begin{refsection}
\nocite{am2002aug,
        charlop2002using,
        dattilo1991facilitating,
        jacobson1995history,
        durand1999functional,
        mirenda1990communication}
\printbibliography[heading=none]
\end{refsection}

%
\subsection{Related Lessons}
\fourdThree{}\\
\fourdFour{}\\
\fourdFive{}\\
\fourfSeven{}\\
%
\clearpage \section[\fourjTen{}]{\fourjTen{}%
              \sectionmark{J-10 When a behavior...}}
\sectionmark{J-10 When a behavior...}
\subsection{Definition}
Differential reinforcement of alternative behavior (DRA) -  ``a procedure for decreasing problem behavior in which reinforcement is delivered for a behavior that serves as desirable alternative to the behavior targeted for reduction and withheld following instances of the problem behavior'' (Cooper, Heron \& Heward, 2007, p. 693).  

When choosing a replacement behavior, look at behavior that would serve the same function or would meet the same reinforcers for the problem behavior. This response could be using vocal-verbal behavior, exchanging a symbol, using sign language, etc. Consider the pros and cons of each mode. How quickly can the response be taught? Is it likely to be less or more effortful than the problem behavior? Will the new response meet reinforcement in natural settings?

Initially, the response to be taught should be reinforced on a continuous reinforcement schedule to ensure that the individual makes steady contact with the reinforcer and that this new replacement behavior occurs often and becomes strengthened in the individual's repertoire. Once the new response is at strength, the DRA schedule should be thinned to reflect reinforcement rates that occur in the natural environment (e.g., a child learning to request for a break in lieu of eloping will not likely be granted a break every time he/she asks for one in a typical classroom environment). 

Challenging behavior may resurge when DRA schedules are thinned. There are several techniques that can be utilized to decrease this occurrence: increase the response requirement of the alternative response (e.g., if the alternate response is to ask for a break, allow a break only after completion of a set amount of work), provide a delay for reinforcement for the alternative response (e.g., provide a break after several minutes), decrease availability of alternative response materials (e.g. if break is requested utilizing a break card, limit amount of breaks or decrease presentation), and use of a multiple schedule of reinforcement such as providing more reinforcement for completing work than asking for a break (Sweeney \& Shahan, 2013). 
%
\subsection{Assessment}
\begin{enumerate}
\item At a job or during role-play, ask supervisee what would be an appropriate alternative behavior that would allow the individual to access same or similar reinforcer. Ask about alternative responses for a variety of reinforcer types.
\item Have your supervisee list considerations when selecting an appropriate alternative replacement behavior
\item At a job or during role-play, have your supervisee design criteria for when DRA schedule should be thinned and what that process should look like.
%
\end{enumerate}
%
\subsection{Relevant Literature}
\begin{refsection}
\nocite{athens2010investigation,
        cooper2007applied,
        sweeney2013effects,
        vollmer1999evaluating}
\printbibliography[heading=none]
\end{refsection}
%
\subsection{Related Lessons}
\fourdTwo{}\\
\fourdEighteen{}\\
\fourdNineteen{}\\
\fouriSix{}\\
\fourjSix{}\\
\fourjSeven{}\\
%
\clearpage \section[\fourjEight{}]{\fourjEight{}%
              \sectionmark{J-08 Select... social validity...}}
\sectionmark{J-08 Select... social validity...}
\subsection{Definition}
A distinguishing characteristic of applied behavior analysis is assessing an individual's functioning within the context of natural environments. This applied aspect focuses the behavior analyst on identifying meaningful goals and acceptable methods for intervention that will increase the individual's independence and level of functioning in natural settings. The behavior analyst sets intervention goals that comply with stated preferences of the individual client, goals of those who live and work with the individual, and consider how typical individuals function in similar environments. Analysts seek goals that are socially valid and intervention methods that are not only scientifically validated strategies for accomplishing those goals, but strategies that can be expected to be implemented consistently and with fidelity by those who will apply the strategies. Although an intervention might be effective in a clinical, controlled setting, the behavior analyst must consider intervention limitations related to ``social acceptability, complexity, practicality, and cost. Regardless of their effectiveness, treatments that are perceived by practitioners, parents, and/or clients as unacceptable or undesirable for whatever reason are unlikely to be used'' (Cooper et al, 2007, p. 250).
%
\subsection{Examples}
\begin{enumerate}
\item A child hits her six month-old sister even when their parents model and reinforce appropriate behaviors toward the baby. The parents find it difficult to avoid explaining to the child, at the same time that they block her physically, reasons her behaviors are unkind and even dangerous. The behavior analyst talks to the parents about how the parent's explanations might be reinforcing the big sister aggression. The parents and the analyst increase the opportunities they have to individually attend to the child during appropriate play throughout the day. The parents ask the analyst to help them design a structured plan to teach appropriate sibling behaviors through language, modeling, literature, role play, movies, and increased reinforcement for appropriate behaviors of the sister toward her younger sibling.
\item A man hits his head and pulls at his ears with such force that he has required emergency medical care. At the beginning of treatment, the behavior analyst recommends that the man be given access to a helmet to prevent significant injury when he is not adequately staffed to stop his behavior. His family is against the man appearing in public with a protective helmet. The behavior analyst explains the reasons such equipment might be important for protecting the man from harm when his 1:1 staff person is distracted by driving a car or interacting with clerks or others in the community. The analyst and the family agree that until interventions stops the severe self-injury, the man will participate in community activities with a helmet unless a family member accompanies staff in the community with him. 
%
\end{enumerate}
\subsection{Assessment}
\begin{enumerate}
\item A behavior analyst wants to increase a non-verbal teenager's independent functioning during daily care routines by teaching him to dress, brush teeth, and bathe independently. The boy's mother says she doesn't mind physically prompting her son through those daily care routines, but states that she hates his screaming while she does it.

\item The analyst completes a functional assessment and learns the following: The boy can complete most of the steps for dressing, brushing teeth, and bathing independently, but has not learned a consistent chain of steps for each skill. The boy screams at other times during the day when his mother uses physical prompting. Ask the supervisee to consider the social validity of the behavior analyst's goals and the preferences expressed by the boy's mother. 

\item Ask the supervisee to write at least one hypothesis to explain, based on the information above, what might be the relations between the self-care skills and the screaming behavior.

\item Ask the supervisee to explain to the mother why teaching the son chained steps for each skill is important for ending the screaming behavior in non-technical language in order to gain her support for teaching self-care routines.
\end{enumerate}
%
\subsection{Relevant Literature}
\begin{refsection}
\nocite{cooper2007applied,
        fawcett1991social,
        wolf1978social}
\printbibliography[heading=none]
\end{refsection}
%
\subsection{Related Lessons}
\fourgSix{}\\
\fourgEight{}\\
\fouriSix{}\\
\fourjFour{}\\
\fourjFive{}\\
\fourjSix{}\\
\fourjTwelve{}\\
\fourkTwo{}\\
\fourkThree{}\\
\fourkNine{}\\
%
\clearpage \section[\fourgThree{}]{\fourgThree{}%
              \sectionmark{G-03 Conduct a preliminary as...}}
\sectionmark{G-03 Conduct a preliminary as...}
\subsection{Definition}
Behavioral assessment - ``A form of assessment that involves a full range of inquiry methods (observation, interview, testing, and the systematic manipulation of antecedent or consequent variables) to identify probable antecedent and consequent controlling variables. Behavioral assessment is designed to discover resources, assets, significant others, competing contingencies, maintenance and generality factors, and possible reinforcers and/or punishers that surround the potential target behavior'' (Cooper, Heron, \& Heward, 2007, p. 691).

Five Phases of a Behavioral Assessment 
Hawkins (1979) described behavioral assessment as being funnel shaped, beginning with a broad scope and then moving to a narrow focus.  
\begin{enumerate}
\item Screening and general disposition
\item Defining and quantifying problems or goals
\item  Pinpointing the target behavior
\item Monitoring progress
\item Follow-up
\end{enumerate}

The preliminary assessment consists of the first 3 phases of this model.  It is the broad gathering of information needed in order to pinpoint the target behavior. Once the target behavior is selected, a formal functional behavioral assessment is required. 
% 
Preliminary Assessment\\
\begin{enumerate}
\item Interviews (client and significant others) 
\item Checklists  
\item Standardized Tests 
\item Direct Observations  
\end{enumerate}

Social Significance\\
Before selecting a target behavior, it is important to reflect on how important the behavior change is for the client, not to others around the client.  The rational for the behavior change must be critically analyzed.  Cooper Heron, and Heward (2007) suggest the following methods for determining the social significance of the target behavior: 
\begin{enumerate}
\item Is the behavior likely to produce reinforcement in the natural environment?
\item Is the skill useful?
\item Will it increase the individual's access to new reinforcing environments?
\item Will it allow more social interaction? 
\item Is it a pivotal behavior?  
\item Is it a behavior cusp?
\item Is it age-appropriate? 
\item If it is a behavior to be eliminated?
\item What is the replacement skill? 
\item Is the identified behavior actually problematic? 
\item Is this the identified behavior just reports or is it real? 
\end{enumerate}

Prioritizing Target Behaviors\\
If a number of target behaviors are selected which are socially significant, it is then important to prioritize the target behavior ensuring dangerous behavior is targeted first.  Other guidelines are listed by Cooper, Heron, and Heward (2007) as the following: 
\begin{enumerate}
\item Pose a danger to client or others? 
\item How often does it occur? 
\item How longstanding is the problem? 
\item Will changing the behavior produce higher rates of reinforcement? 
\item What is the importance related to overall independence? 
\item Will changing the behavior reduce negative attention? 
\item Will changing the behavior produce positive attention? 
\item How likely is success of changing the behavior? 
\item How much will it cost to change the behavior? 
\end{enumerate}
 
Defining the Target Behavior\\
Before beginning, the target behavior must be objectively and concisely defined in a clear concrete observable manner. 
 
Setting the Criteria for Behavior Change \\
Goals must be socially meaningful to the person's life.   
%
\subsection{Assessment}
\begin{enumerate}
\item Ask supervisee what assessment tools can be used to do a preliminary assessment. 
\item Provide examples of targets, which are and are not socially significant and ask the supervisee to determine if these behaviors are appropriate target behaviors.  Have supervisee explain why. 
\item Provide a list of five target responses and have the supervisee prioritize them, justifying their decisions using the guidelines provided by Cooper, Heron and Heward (2007).
\end{enumerate}
%
\subsection{Relevant Literature}
\begin{refsection}
\nocite{bailey2013ethics,
        cooper2007applied,
        hawkins1979functions,
        linehan1977issues,
        van1979social}
\printbibliography[heading=none]
\end{refsection}
%
\subsection{Related Lessons} 
\fouriOne{}\\
\fouriTwo{}\\
\fourjOne{}\\
%
\clearpage \section[\fouriSix{}]{\fouriSix{}%
              \sectionmark{I-06 Make recommendations...}}
\sectionmark{I-06 Make recommendations...}
\subsection{Definition}
Hawkins (1984, p. 284) (cited from Cooper, Heron \& Heward, 2007, p. 56) defined habilitation as ``the degree to which the person's repertoire maximizes short and long term reinforcers for that individual and for others, and minimizes short and long term punishers.''

When determining what behaviors to target, one can use the relevance of behavior rule (Ayllon and Azrin, 1968) as a guide. This rule states that a target behavior should only be selected if it is likely to produce reinforcement for the client in their natural environment. Another key factor is deciding if the behavior will generalize to other settings and be sustainable once the behavior change program has ended. 

Cooper et al. (2007) provide some considerations when choosing a target behavior to increase, decrease, or maintain. These include:
\begin{enumerate}
\item Does this behavior pose any danger to the client or others?
\item How many opportunities will the person have to use this new behavior? Or how often does this problem behavior occur?
\item How long-standing is the problem or skill deficit?
\item Will changing the behavior produce higher rates of reinforcement for the person?
\item What will be the relative importance of this target behavior to the future skill development and independent functioning?
\item Will changing this behavior reduce negative attention from others?
\item Will the new behavior produce reinforcement for significant others?
\item How likely is success in changing this target behavior?
\item How much will it cost to change this behavior?
\end{enumerate}
%
\subsection{Examples}
\begin{enumerate}
\item Dave has decided to implement an intervention to increase a student's compliance. He chose this because lack of compliance interferes with the student's ability to learn new skills and access reinforcement by completing their work and daily routines.
\end{enumerate}
%
\subsection{Assessment}
\begin{enumerate}
\item Write a list of potential target behaviors. Have supervisee rank the behaviors in order of social significance and give rationale for their decisions.
\item Have supervisee present a case study on a client they are familiar with, including the maladaptive behaviors in their repertoire. Have the supervisee choose two behaviors to target for intervention and state why they chose those behaviors.
\end{enumerate}
%
\subsection{Relevant Literature}
\begin{refsection}
\nocite{ayllon1968token,
        cooper2007applied,
        hawkins1984meaningful,
        hawkins1986selection,
        rosales1997behavioral}
\printbibliography[heading=none]
\end{refsection}
\subsection{Related Lessons}
\fourbOne{}\\ 
\fourgThree{}\\
\fourgFive{}\\
\fouriOne{}\\
\fourjOne{}\\
\fourjFive{}\\
\fourjEight{}\\
\fourjTen{}\\
\fourjThirteen{}\\
\fourFKTen{}\\
%
\clearpage \section[\fourjSeven{}]{\fourjSeven{}%
              \sectionmark{J-07 Select... constraints.}}
\sectionmark{J-07 Select... constraints.}
\subsection{Definition}
``The independent variable should be evaluated not only in terms of its effects on the dependent variable, but also in terms of its social acceptability, complexity, practicality, and cost'' (Cooper, Heron, \& Heward, 2007, p. 250).

One method for determining the feasibility of an intervention is by asking consumers (parents, teachers, administrators) to rate the social validity of the client's performance. Questions that are typically posed to consumers before interventions are implemented include asking the consumer how reasonable they feel the intervention is, asking the consumer's willingness to implement the intervention strategies, asking if the consumer would be willing to change the environment to implement the intervention, asking how disruptive the intervention may be to the natural environment, asking how costly it would be to implement the intervention, asking if there will be any discomfort in the client when implementing these procedures, and asking if carrying out the intervention will fit with the classroom or setting routines (Reimers \& Wacker, 1988 cited from Cooper et al., 2007, pp. 238-239).
%
\subsection{Examples}
\begin{enumerate}
\item Rob has decided to implement a reinforcement program based on appropriate responses rather than a fixed time DRO program. He understands that there is no paraprofessional in the classroom to help run the program and the teacher has other educational duties so she cannot run a timer and deliver reinforcement consistently enough for a rigorous DRO. 
%
\end{enumerate}
%
\subsection{Assessment}
\begin{enumerate}
\item Have supervisee come up create a list of appropriate questions to ask consumers when determining an interventions appropriateness and acceptability. Have each supervisee create his/her own treatment acceptability rating form.
\item Have supervisee list and describe various extraneous factors that must be taken into consideration before implementing an intervention. Have supervisee explain why it is important to have consumer satisfaction with an intervention program.
\end{enumerate}
%
\subsection{Relevant Literature}
\begin{refsection}
\nocite{cooper2007applied,
        hawkins1984meaningful,
        reimers1988parents,
        wolf1978social}
\printbibliography[heading=none]
\end{refsection}
%
\subsection{Related Lessons}
\fourcOne{}\\
\fourgSix{}\\
\fourgEight{}\\
\fouriOne{}\\
\fouriTwo{}\\
\fourjSix{}\\
\fourjEight{}\\
\fourkSeven{}\\
\fourkNine{}\\
\fourFKEleven{}\\
%
\clearpage \section[\fourkTwo{}]{\fourkTwo{}%
              \sectionmark{K-02 Identify the conting...}}
\sectionmark{K-02 Identify the conting...}
\subsection{Definition}
Behavior is understood to be a product of the environment in which it occurs. This is the same for all organisms, including the client, the caretakers, the professionals working with the client, and ourselves. A well trained behavior analyst accounts for the environmental arrangement for all of the individuals involved in the behavior change process. 

For instance, if a procedure is very effortful and will not produce an effect for several weeks, what will reinforce the behavior of the family member/teacher who will be implementing it? If a procedure produces lots of aggression or screaming, it should be considered that these are often aversive stimuli to the people implementing it.

``Treatment drift occurs when the application of the independent variable during later phases of an experiment differs from the way it was applied at the outset of the study'' (Cooper et al., 2007, p. 235).This is often the result of a practitioner's behavior meeting competing contingencies after having followed the plan for a period of time.

High treatment integrity can be achieved by creating a thorough and precise definition for the independent variables, simplifying the treatment procedures, providing ample training and practice to all individuals responsible for treatment, and assessing the contingencies each person's behavior will meet while following through with these interventions. 

Other factors that can help improve treatment integrity and regulate the behavior of those involved in the experiment include using less expensive and less intrusive procedures, seeking help and input from the participants family members and other people close to them, setting socially significant but easy-to-meet criterion for reinforcement, eliminating reinforcement gained outside of performing the target response, and contrive contingencies that will compete with natural contingencies. 
%
\subsection{Examples}
\begin{enumerate}
\item Don has been asked to help deliver a new differential reinforcement program to decrease his student's self-injury. However, he has been short of staffing lately and cannot do this consistently throughout the day. This has caused the program to be run without integrity and the self-injury to remain at the same rates. Rick, the BCBA who designed the program, started observing the classroom to see why the program was not working. After noticing that Don was only delivering the reinforcers intermittently and missing opportunities for reinforcement, Rick decided to retrain Don and ask the principal for extra staff when someone calls out sick. 
\item (Non-example) Roger is implementing a new response cost program to decrease verbal protesting for one of his students. However, even though the program has been run as designed for several weeks, there has been no effect on the verbal protesting of the student. The supervisor collected integrity data and found that the plan had been run as prescribed. The problem is likely related to the procedure itself and not its implementation.
%
\end{enumerate}
%
\subsection{Assessment}
\begin{enumerate}
\item Make sure the supervisee makes considerations about the effort and practicality of the treatment they attempt to get other people to implement.
\item Have supervisee identify several strategies for increasing treatment integrity. Have him/her describe how they would use these strategies in an applied setting.
\item Have supervisee read several articles in which an intervention was implemented to decrease or increase a target behavior. Have him/her identify the strategies the researcher used to increase treatment integrity.
\item Have supervisee define and describe treatment drift. Have him/her describe how they would account for the occurrence of treatment drift and adjust an intervention or experiment accordingly.
%
\end{enumerate}
%
\subsection{Relevant Literature}
\begin{refsection}
\nocite{cooper2007applied,
        fryling2012impact,
        reed2011parametric,
        wheeler2006treatment,
        wilder2006effects,
        mcintyre2007treatment,
        peterson1982integrity,
        pipkin2010effects}
\printbibliography[heading=none]
\end{refsection}
%
\subsection{Related Lessons}
\fourjOne{}\\
\fourkThree{}\\
\fourkFour{}\\
\fourkFive{}\\
\fourkSix{}\\
\fourkEight{}\\
%
\clearpage \section[\fourjNine{}]{\fourjNine{}%
              \sectionmark{J-09 Identify and address practical...}}
\sectionmark{J-09 Identify and address practical...}
\subsection{Definition}
The general goal of behavioral research is ``to demonstrate that measured changes in the target behavior occur because of experimentally manipulated changes in the environment'' (Cooper, 2007, p. 160). Without a controlled research design, practitioners cannot claim a causal relation between intervention and behavior change. Practical and ethical concerns limit practitioner's use of experimental designs in most settings. Typical risks associated with controlled research include the need to delay treatment while collecting baseline data or to withdraw interventions that are successful. Each research design has specific risks to participants associated with its application. In applied practice, meaningful, socially valid, and lasting change is the goal. Clients, parents, staff, and teachers prefer the most efficient and effective path toward treatment goals. For such reasons, practitioners seek to show evidence of a correlation rather than a causal relation between changes in a client's behavior and an intervention by providing comparison of patterns of baseline (A) and intervention (B) responding over time. The AB design has poor experimental control but strong practical and ethical value in natural settings.
%
\subsection{Examples}
\begin{enumerate}
\item A researcher-practitioner designed a multiple baseline study for a woman who hit and scratched herself. The researcher's review of the first three days of baseline data showed that occurrences of the behavior were highly variable without an obvious pattern of responding. The researcher concluded that further delay of treatment that might decrease a dangerous behavior was not ethical. The researcher knew that the strength of the results of his research would be threatened if he began treatment before he had a clear pattern in baseline responding, but his responsibility to his client, and those close to her, was his primary concern. He regretted not anticipating this possibility by choosing a research design that could have demonstrated experimental control without depending on highly stable baseline responding.
\item A student's teacher did not want the student to participate in a graduate supervisee's study because the supervisee planned to work with the student during classroom reading instruction. Even though the research study was designed to provide a benefit to the student by increasing sight-word reading, the boy's parent refused to sign permission for his participation. The supervisee realized that social validity was threatened by his original plan and he would have to arrange to work with the student after school or exclude him from the study.
\end{enumerate}
%
\subsection{Assessment}
\begin{enumerate}
\item Ask the supervisee to explain what baseline logic is and why it is important for showing causal (experimentally controlled research) or correlational (change over time) relations between an intervention and an individual's behavior change. 
\item A graduate student supervisee was planning to conduct single subject research using a withdrawal (ABAB) design. The supervisee wanted to test for a causal relation between a gel-filled wedge pillow and the fidgety, out-of-seat behavior of a middle school student in his class. His supervisor warned that removing an effective treatment might have long-lasting results on the student's performance. What design might the supervisee recommend that would show repetitive positive effects of the gel-filled pillow intervention without requiring the supervisee to withdraw a beneficial treatment? Ask the supervisee to demonstrate his decision by drawing a rough line graph showing both designs with the gel-filled pillow as the intervention and explain why one design is a better choice ethically while still meeting practical goals for the supervisee and student. 
\end{enumerate}
%
\subsection{Relevant Literature}
\begin{refsection}
\nocite{cooper2007applied,
        ledford2009single}
\printbibliography[heading=none]
\end{refsection}
%
\subsection{Related Lessons}
\fourbThree{}\\
\fourgEight{}\\
\fourhThree{}\\
\fourjTwo{}\\
%06 Select intervention strategies based on supporting environments.
\fourjSeven{}\\
\fourjEight{}\\
\fourjTen{}\\
\fourkTwo{}\\
\fourkSeven{}\\
%
\clearpage \section{\foureTwo{}}
\subsection{Definition}
Discrimination training procedures involve ``reinforcing or punishing a response in the presence of one stimulus and extinguishing it or allowing it to recover in the presence of another stimulus'' (Malott \& Trojan Suarez, 2004, p. 485). There are typically two competing contingencies when discrimination training occurs.  The first contingency involves an S-delta (i.e., signals that a specified response will not be reinforced or punished when in the presence of a specified stimulus). The second contingency involves a discriminative stimulus (i.e., signals that a specified response will be reinforced in the presence of a specific stimulus condition). When discrimination training occurs, a specified response will no longer be reinforced in the presence of an S-delta, however, that same response will be reinforced in the presence of a discriminative stimulus. The goal of discrimination training is to reinforce responses in certain stimulus conditions so that they occur more frequently when those stimulus conditions are present and over time, the response will no longer occur in the presence of the S-delta. When discrimination training is successful, the learner can discriminate which antecedent stimulus conditions will result in greater reinforcement for a given response. 
%
\subsection{Examples}
\begin{enumerate}
\item Discrimination training can be used to teach an individual appropriate times to take breaks, when or where it is acceptable to engage in self-stimulatory behaviors, what items in the kitchen can be accessed without asking for permission, and so on. Discrimination training procedures are evident in basic instructional lessons such as teaching a child to identify colors to seemingly natural situations such as only scheduling clients on days allowed by funding sources because this results in you being paid for your services. 
\item Carl's teacher determines attention is reinforcing his speaking out in class. Carl's teacher teaches Carl to ask questions when there is a green card present on the board, and not to ask questions when there is a red card on the board. She does this by only delivering attention to Carl when the green card is present on the board, and ignoring Carl when the red card is present. 
\end{enumerate}
%
\subsection{Assessment}
\begin{enumerate}
\item Ask supervisee to provide examples of how discrimination procedures can be used with a specific client or to teach a specific skill.
\item Ask supervisee to clearly operationalize the S-delta and discriminative stimulus contingencies that will be utilized during a specific discrimination training procedure.
\item Observe supervisee describe discrimination procedures to a client, colleague, etc.
\end{enumerate}
%
\subsection{Relevant Literature}
\begin{refsection}
\nocite{cooper2007applied,
        malott2003principles,
        taylor2014discrimination}
\printbibliography[heading=none]
\end{refsection}
%
\subsection{Related Lessons}
\fourdEight{}\\
\foureOne{}\\
\foureThree{}\\
\foureThirteen{}\\
\fourjEleven{}\\
\fourFKEleven{}\\
\fourFKTwentyFour{}\\
\fourFKTwentyFive{}\\
\fourFKThirtyFive{}\\
%
\clearpage \section[\fourdNine{}]{\fourdNine{}%
              \sectionmark{D-09 Use the verbal operants...}}
\subsection{Definition}
In the field of applied behavior analysis extensive research has been done on the development of verbal behavior.  

``Verbal behavior involves social interactions between speakers and listeners, whereby speakers gain access to reinforcement and control their environment through the behavior of listeners'' (Sundberg as cited in Cooper, Heron, \& Heward, 2007, p. 529). Verbal operants are the basic units of this exchange.  

In 1957 B.F. Skinner identified six elementary verbal operants in his book on Verbal Behavior.  These included mands, tacts, intraverbals, echoics, textuals, and transcription.  ``Skinner's analysis suggests that a complete verbal repertoire is composed of each of the different elementary operants, and separate speaker and listener repertoires'' (Sundberg as cited in Cooper, et al., 2007, p. 541).   

Since Skinner described these operants, those in the field have applied these concepts to both language assessment and training.  In order to evaluate whether or not specific language training is necessary, a variety of standardized tools have been used to test an individual's receptive and expressive language abilities.  These include but are not limited to: the Peabody Picture Vocabulary Test III (Dunn \& Dunn, 1997), the Comprehensive Receptive and Expressive Vocabulary Test (Hammill \& Newcomer, 1997), the Assessment of Basic Language and Learning Skills (ABLLS) (Partington \& Sundberg, 1998), the Verbal Behavior Milestones Assessment and Placement Program (VB-MAPP) and the CELF-4 Semel, Wiig, \& Secord, 2003).  

Not all of these tests will identify deficits in one or more of the verbal operants. Some children who may be proficient in tacting (such as labeling things in their environment such as letters and numbers) may fail to make appropriate mands for desired items (Cooper, et al., 2007).  In this case it is important for behavior analysts to use a combination of approaches or less standardized methods to assess these needs.  It may be helpful to observe the individual in their natural environment and take data on their verbal interactions.  It will be important to ask questions such as:
\begin{enumerate}
\item What is the frequency of and complexity of mands?
\item What is the frequency and complexity of tacting behavior?
\item Will the child or individual demonstrate echoic behavior when prompted?
\item Does the child or individual engage in intraverbal behavior with known caregivers?
\item Can or will the child or individual read words that are written down for him?
\item Can or will the child or individual write words that are said to him? 
\end{enumerate}
%
\subsection{Assessment}
\begin{enumerate}
\item Ask the supervisee to name the basic unit of language
\item Ask the supervisee to name all 6 of the elementary verbal operants
\item Ask the supervisee to name some of the standardized tests often used to assess language
\item Ask the supervisee to explain why these standardized tests may not provide adequate information
\item Ask the supervisee to describe how one might assess an individual's use of verbal operants if testing fails to yield enough information.
\end{enumerate}
%
\subsection{Relevant Literature}
\begin{refsection}
\nocite{cooper2007applied,
    partington1998assessment,
    semel2003clinical,
    skinner1957verbal,
    sundberg2008verbal,
    sundberg1998teaching}
\printbibliography[heading=none]
\end{refsection}
%cannot verify reference.
%Hammill, D., \& Newcomer, P.L. (1997).  Test of language development-3.  Austin, TX: Pro-Ed.
%
\subsection{Related Lessons}
\fourdTen{}\\
\fourdEleven{}\\
\fourdTwelve{}\\
\fourdThirteen{}\\
\fourdFourteen{}\\
\fourFKFourtyThree{}\\
\fourFKFourtyFour{}\\
\fourFKFourtyFive{}\\
\fourFKFourtySix{}\\
%
\clearpage \section{\foureTwelve{}}
\subsection{Definition}
Errorless learning - an ``approach whereby the task is manipulated to eliminate/reduce errors. Tasks are executed in such a way that the subject is unlikely to make errors'' (Fillingham, Hodgson, Sage, \& Ralph, 2003, p. 339). 

Errorless learning techniques include most-to-least prompt fading or stimulus shaping/fading techniques. Prompts are removed gradually as the individual becomes more adept with the skill, thereby reducing the likelihood of errors.  To apply errorless learning, behavioral strategies utilized may include: response prevention (e.g. only S+ is presented allowing for only correct responding or physical guidance is provided with instruction so incorrect responses are not possible); verbal prompt fading; modeling; stimulus fading (e.g. emphasizing a physical dimension of the stimuli to evoke a correct response such as by illuminating the correct selection, S+, and presenting the incorrect selection, S-, in a dimmer format); or stimulus shaping (e.g. increasing likelihood of correct responding by gradually changing the shape of the stimulus to maintain correct responding). 

The advantages of errorless learning include that it removes negative side effects involved with trial-and-error learning and that it is proven particularly effective among individuals that suffer from brain damage or have a developmental disorder. The disadvantages include cost, time-intensity, and maybe considered less natural than trial-and-error learning (Mueller, Palkovic \& Maynard, 2007).

Trial-and-error learning, being presented with stimuli in which both the correct selection (S+) and incorrect selection (S-) are available, can lead to adverse side effects due to the possibility of incorrect responding and failure to access reinforcers. Research has shown that this can result in aggression, negative emotional responses and stimulus overselectivity (Mueller et al., 2007).
%
\subsection{Assessment}
\begin{enumerate}
\item Have supervisee demonstrate the difference between trial-and-error learning and errorless learning on the job or during role-play.
\item Have supervisee describe how and when prompts will be faded to promote independent responding. 
%
\end{enumerate}
%
\subsection{Relevant Literature}
\begin{refsection}
\nocite{fillingham2003application,
        mueller2007errorless,
        terrace1963errorless}
\printbibliography[heading=none]
\end{refsection}
%
\subsection{Related Lessons} 
\fourdThree{}\\
\fourdFour{}\\
\fourFKTwentyFour{}\\
