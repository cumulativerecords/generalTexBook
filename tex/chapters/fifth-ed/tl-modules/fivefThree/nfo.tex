%\clearpage \section[\fouriOne{}]{\fouriOne{}%
\subsection{Definition}
The importance of defining behavior in observable and measurable terms:\\
As Baer, Wolf and Risley said in 1968, ``since the behavior of an individual is composed of physical events, its scientific study requires their precise measurement'' (p. 93). In order to be scientific in our study of behavior, we must be very clear about what behavior it is we are actually studying. Therefore, the target behavior must be observable and measurable. Cooper, Heron and Heward (2007) also make the point that one of the most basic tenets of science is replication. In order for other scientists to replicate an experiment or study, the definition of the behavior under investigation and how it was measured must be transparent enough, that future replication is possible.

Technically-sound written definitions of target behaviors:\\
\begin{enumerate}
\item Cooper, Heron, and Heward, (2007), suggest that a good behavioral definition is \textit{operational}, allowing the practitioner to obtain complete information about a behavior's occurrence/non-occurrence. Operational definitions allow accurate application of the procedures.
\item Cooper, Heron and Heward (2007) also state that good definitions increase the likelihood that an accurate evaluation of the effectiveness of a study or experiment will be conducted.
\end{enumerate}

Two types of target behavior definitions:\\
\begin{enumerate}
\item Cooper, Heron, and Heward (2007, p. 65) suggest that there are two types of target behavior definitions:
\item Functional (These types of definition label responses as part of the target behavior's response class if they have the same effect upon the environment.)
\item Topographical (These types of definition look at the shape or form of the target behavior.)
\end{enumerate}

How to write behavioral definitions
\begin{enumerate}
\item Cooper, Heron, and Heward (2007) cite Hawkins and Dobes (1977) as giving three characteristics of good written target behavior definitions:
\item Objective (should refer only to observable characteristics of the behavior and environment and shouldn't utilize inferential terms, such as ``feeling angry.'')
\item Clear (the definition should be readable and unambiguous.)
\item Complete (it should outline the boundaries of what is included as an instance of a response and what is not included.)
\end{enumerate}
%
%\clearpage \section[\fouriSix{}]{\fouriSix{}%
\subsection{Definition}
Hawkins (1984, p. 284) (cited from Cooper, Heron \& Heward, 2007, p. 56) defined habilitation as ``the degree to which the person's repertoire maximizes short and long term reinforcers for that individual and for others, and minimizes short and long term punishers.''

When determining what behaviors to target, one can use the relevance of behavior rule (Ayllon and Azrin, 1968) as a guide. This rule states that a target behavior should only be selected if it is likely to produce reinforcement for the client in their natural environment. Another key factor is deciding if the behavior will generalize to other settings and be sustainable once the behavior change program has ended. 

Cooper et al. (2007) provide some considerations when choosing a target behavior to increase, decrease, or maintain. These include:
\begin{enumerate}
\item Does this behavior pose any danger to the client or others?
\item How many opportunities will the person have to use this new behavior? Or how often does this problem behavior occur?
\item How long-standing is the problem or skill deficit?
\item Will changing the behavior produce higher rates of reinforcement for the person?
\item What will be the relative importance of this target behavior to the future skill development and independent functioning?
\item Will changing this behavior reduce negative attention from others?
\item Will the new behavior produce reinforcement for significant others?
\item How likely is success in changing this target behavior?
\item How much will it cost to change this behavior?
\end{enumerate}
%
