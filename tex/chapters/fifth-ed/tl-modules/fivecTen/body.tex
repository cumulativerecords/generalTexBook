\clearpage \section[\fouraTen{}]{\fouraTen{}%
              \sectionmark{A-10 Design... equal-interval}}
\sectionmark{A-10 Design... equal-interval}
\subsection{Definition}
Line graph - ``In applied behavior analysis, each point on a line graph shows the level of some quantifiable dimension of the target behavior (i.e. the dependent variable) in relation to a specified point in time and/or environmental condition'' (Cooper, Heron, \& Heward, p. 129).*\\

The behavior analyst defines behavior in quantifiable, observable terms to measure consistently and accurately. The behavior is measured in terms of a pertinent aspect of behavior that can be counted or assessed across observers. When data are plotted, the patterns they make provide for a visual analysis of levels of the behavior (shown on the vertical, y-axis) as the behavior occurs at a specific point in time or environmental condition (shown on a horizontal, x-axis). Graphs are drawn with the y-axis in a two-thirds ratio to the x-axis in order to enable accurate comparison of intervention results across graphs. The analysis interprets levels of data points, directions (trend), and stability or variability of data paths within a single condition or viewed across different conditions. These factors help an analyst assess if an individual is responding to intervention efforts in a therapeutic or non-therapeutic direction. As a result of this systematic interpretation of results, the analyst continues treatment strategies or alters them until the line graph shows consistent behavior change in a therapeutic direction.

Designing an equal interval line graph:
\begin{enumerate}
\item Approximate ratio of y to x-axis is 2:3
\item The target behavior to measure was the percent (dimension) of math homework (varied number of problems assigned) a student completed each morning when math homework was assigned (session).
\item Lowest possible percent of homework completed was zero and highest possible percent (100\%) is shown at equal intervals of outside tic marks.
\end{enumerate}

\subsection{Examples}
Before intervention began, John showed a baseline pattern of completing between zero and 10\% of his assigned math homework.
\begin{enumerate}
\item Zero value is raised above the x-axis to see data points clearly.
\item Baseline lasted 4 sessions in a non-therapeutic pattern.
\item Conditions changed from baseline to intervention between Session 4 and 5.
\item The intervention included John's teacher praising him each day that he returned any part of his math homework assigned. 
\item In the first session of intervention, John turned in 10\% of his homework. The second day after receiving teacher praise, John increased his completed amount of homework to 40\% of the assignment. 
\item John did not always increase the percent of homework completed each day that math homework was given.
\item The intervention data path shows an increasing trend overall throughout the intervention condition of teacher praise for math homework completion. 
\item Although data showed some variability, John reached criteria of at least 90\% of math homework completed for 3 consecutive days on the 12th day of teacher praise for his efforts to return homework.
\end{enumerate}
%
(Chart pending) Figure 2    Percent of daily math homework John completed in September.

*AB designs can show relations between baseline and intervention responding but cannot be used to show cause and effect.
%
\subsection{Assessment}
\begin{enumerate}
\item Ask the supervisee to explain why the percent of homework completed was the correct dimension to measure for this intervention.
\item Ask supervisee to explain the reason the AB graph design cannot be used to demonstrate experimental control but can be used in applied settings to indicate consistent and therapeutic change.
\item Ask supervisee to find missing elements in a line graph you design.
\item Ask the supervisee to operationally define a repetitive behavior of a friend or family member, identify a dimension of that behavior that would accurately represent occurrences of the behavior, and then design a line graph to show results.
\item Give the supervisee a line graph you design and ask supervisee to interpret level, trend, variability, and data path characteristics within and across conditions. 
\end{enumerate}
%
\subsection{Relevant Literature}
\begin{refsection}
\nocite{cooper2007applied,
    alberto2013applied,
    ledford2009single,
    vanselow2012online}
\printbibliography[heading=none]
\end{refsection}
%
\subsection{Related Lessons}
\fourhOne{}\\
\fourhTwo{}\\
\fourhThree{}\\
\fourhFour{}\\
\fourhFive{}\\
\fourFKFourtySeven{}\\
\fourjFifteen{}\\
%
\clearpage \section[\fouraEleven{}]{\fouraEleven{}%
              \sectionmark{A-11 Design... cumulative record}}
\sectionmark{A-11 Design... cumulative record}
\subsection{Definition}
Cumulative record – recording method that involves ``the number of responses recorded during each observation period is added to the total number of responses recorded during previous observation periods'' (Cooper, Heron \& Heward, 2007, p. 138). The value on the y-axis represents the cumulative number of responses recorded and the value on the x-axis represents time (i.e., observation periods).  Once the response rate exceeds the maximum value on the y-axis, the curve resets to zero and begins again.  Cumulative records display the overall response rate and visually depict the learner's rate of acquisition for a series of behavior targets (e.g., total number of skills mastered throughout services, number of sight words learned).  Data are interpreted on a cumulative record by analyzing the slope in which the steeper the slope, the higher the response rate.
%
\subsection{Examples}
\begin{enumerate}
\item The cumulative record below indicates the number of attributes learned by a first grade student.  The overall response rate is 13 attributes across 181 sessions.  In general, data in this graph suggests that there was a fairly slow rate of acquisition. However, the slope is much steeper between sessions 1 and 61, indicating that the rate of acquisition was quicker during the first part of the intervention. 
\end{enumerate}
%
\subsection{Assessment}
\begin{enumerate}
\item Ask supervisee to identify behavioral targets that would be appropriate to graph in a cumulative record
\item Ask supervisee to create a cumulative record graph.
\item Show supervisee various examples of cumulative records and ask supervisee to interpret the data.
\end{enumerate}
%
\subsection{Relevant Literature}
\begin{refsection}
\nocite{cooper2007applied,ferster1957schedules}
\printbibliography[heading=none]
\end{refsection}
%Cooper, J. O., Heron, T. E., \& Heward, W. L. (2007). Constructing and interpreting graphic displays of behavioral data. Applied Behavior Analysis (pp. 126-157). Upper Saddle River, NJ: Pearson Prentice Hall.
%Ferster, C. B., \& Skinner, B. F. (1957). Schedules of reinforcement. New York, NY: Appleton-Century-Crofts.
% 
\subsection{Related Lessons}
\fouraTen{}\\
\fourhOne{}\\
\fourhTwo{}\\
\fourhThree{}\\
\fourhFour{}\\
\fourhFive{}\\
