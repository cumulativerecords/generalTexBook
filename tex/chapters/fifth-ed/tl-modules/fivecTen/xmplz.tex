\subsection{Examples}
Before intervention began, John showed a baseline pattern of completing between zero and 10\% of his assigned math homework.
\begin{enumerate}
\item Zero value is raised above the x-axis to see data points clearly.
\item Baseline lasted 4 sessions in a non-therapeutic pattern.
\item Conditions changed from baseline to intervention between Session 4 and 5.
\item The intervention included John's teacher praising him each day that he returned any part of his math homework assigned. 
\item In the first session of intervention, John turned in 10\% of his homework. The second day after receiving teacher praise, John increased his completed amount of homework to 40\% of the assignment. 
\item John did not always increase the percent of homework completed each day that math homework was given.
\item The intervention data path shows an increasing trend overall throughout the intervention condition of teacher praise for math homework completion. 
\item Although data showed some variability, John reached criteria of at least 90\% of math homework completed for 3 consecutive days on the 12th day of teacher praise for his efforts to return homework.
\end{enumerate}
%
(Chart pending) Figure 2    Percent of daily math homework John completed in September.

*AB designs can show relations between baseline and intervention responding but cannot be used to show cause and effect.
%
\subsection{Examples}
\begin{enumerate}
\item The cumulative record below indicates the number of attributes learned by a first grade student.  The overall response rate is 13 attributes across 181 sessions.  In general, data in this graph suggests that there was a fairly slow rate of acquisition. However, the slope is much steeper between sessions 1 and 61, indicating that the rate of acquisition was quicker during the first part of the intervention. 
\end{enumerate}
%
