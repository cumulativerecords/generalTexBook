%\clearpage \section[\fouraTen{}]{\fouraTen{}%
%              \sectionmark{A-10 Design... equal-interval}}
%\sectionmark{A-10 Design... equal-interval}
\subsection{Definition}
Line graph - ``In applied behavior analysis, each point on a line graph shows the level of some quantifiable dimension of the target behavior (i.e. the dependent variable) in relation to a specified point in time and/or environmental condition'' (Cooper, Heron, \& Heward, p. 129).*\\

The behavior analyst defines behavior in quantifiable, observable terms to measure consistently and accurately. The behavior is measured in terms of a pertinent aspect of behavior that can be counted or assessed across observers. When data are plotted, the patterns they make provide for a visual analysis of levels of the behavior (shown on the vertical, y-axis) as the behavior occurs at a specific point in time or environmental condition (shown on a horizontal, x-axis). Graphs are drawn with the y-axis in a two-thirds ratio to the x-axis in order to enable accurate comparison of intervention results across graphs. The analysis interprets levels of data points, directions (trend), and stability or variability of data paths within a single condition or viewed across different conditions. These factors help an analyst assess if an individual is responding to intervention efforts in a therapeutic or non-therapeutic direction. As a result of this systematic interpretation of results, the analyst continues treatment strategies or alters them until the line graph shows consistent behavior change in a therapeutic direction.

Designing an equal interval line graph:
\begin{enumerate}
\item Approximate ratio of y to x-axis is 2:3
\item The target behavior to measure was the percent (dimension) of math homework (varied number of problems assigned) a student completed each morning when math homework was assigned (session).
\item Lowest possible percent of homework completed was zero and highest possible percent (100\%) is shown at equal intervals of outside tic marks.
\end{enumerate}

