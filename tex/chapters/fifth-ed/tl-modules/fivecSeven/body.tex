\clearpage \section[\fourFKFourtyEight{}]{\fourFKFourtyEight{}%
              \sectionmark{FK-48 State the adv...}}
\sectionmark{FK-48 State the adv...}
\subsection{Definition}
\textit{Continuous measurement procedures (e.g., frequency, duration).}\\
Advantages
\begin{itemize}
\item Will capture all instances of a response.
\item Will not over or underestimate behavior.
\end{itemize}
Disadvantages
\begin{itemize}
\item Sometimes is less preferred by staff as it often requires more effort and vigilance to record.
\item May be difficult to collect more than one response at a time.
\end{itemize}
%
\textit{Discontinuous measurement procedures (e.g., partial, and whole-interval recording, momentary time sampling).}\\
Advantages
\begin{itemize}
\item Requires data collection once per interval rather than every instance.
\item May be easier to collect more than one response at a time.
\item Is sometimes preferred by those collecting the data.
\end{itemize}

Disadvantages
\begin{itemize}
\item Does not collect all instances of behavior.
\item May over or under estimate the response being measured.
\end{itemize}
%
\subsection{Assessment}
\begin{enumerate}
\item Ask your Supervisee to state the advantages and disadvantages of each type of measurement procedure. 
\item Ask your Supervisee to use one of the continuous measurement procedures and evaluate its effectiveness at representing the behavior being studied.
\end{enumerate}
%
\subsection{Relevant Literature}
\begin{refsection}
\nocite{cooper2007applied,
        meany2007comparison,
        gardenier2004comparison,
        gunter2003efficacy,
        powell1975evaluation,
        suen1991reappraisal}
\printbibliography[heading=none]
\end{refsection}
%
\subsection{Related Lessons}
\fouraNine{}\\
\fouraTwelve{}\\
\fourhOne{}\\
\fourhTwo{}\\
\fouriOne{}\\
\fourFKFourtySeven{}\\
\fourFKFourtyEight{}\\
