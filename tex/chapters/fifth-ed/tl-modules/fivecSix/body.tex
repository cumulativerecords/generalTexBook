%\clearpage \section{\fouraSeven{}}
\subsection{Definition}
Trials to criterion - ``A special form of event recording; a measure of the number of responses or practice opportunities needed for a person to achieve a pre-established level of accuracy or proficiency'' (Cooper, Heron \& Heward, 2007, p.  87).\\

To use, one must first determine what constitutes a response opportunity, and what the criteria for mastery will be. Response opportunities can vary depending on the target behavior. For example, an opportunity for spelling accuracy could be a 10-question spelling test. An opportunity for responding to one's name within five seconds might be every time someone called the individual's name. Other measures such as latency, percent occurrence, rate and duration can be used to compute trials to criterion data. For instance in the latter example to compute whether the individual is responding to his name within 5 seconds, data would have to be taken on latency to respond per opportunity (Cooper et al., 2007).\\
%
\subsection{Examples}
\begin{enumerate}
\item A behavioral interventionist is teaching a child how to brush her teeth. She teaches this using a task analysis that involves backwards chaining. She provides two opportunities per session for the child to complete this skill. Data reflect that it takes the child on average four opportunities before she is able to complete the step being taught independently and to move to the next step in the task analysis.
\end{enumerate}
%
\subsection{Assessment}
\begin{enumerate}
\item Have your supervisee list uses for trials to criterion data.
\item In a clinical setting or during role-play, have your supervisee determine which intervention is more efficient using trials to criterion data. 
\item Have your supervisee complete trials to criterion data in various scenarios. 
\end{enumerate}
%
\subsection{Relevant Literature}
\begin{refsection}
\nocite{cooper2007applied,lahey1974facilitation}
\printbibliography[heading=none]
\end{refsection}
%
\subsection{Related Lessons}
\fourhOne{}\\
\fourhThree{}\\
