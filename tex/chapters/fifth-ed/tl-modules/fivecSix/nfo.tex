%\clearpage \section{\fouraSeven{}}
\subsection{Definition}
Trials to criterion - ``A special form of event recording; a measure of the number of responses or practice opportunities needed for a person to achieve a pre-established level of accuracy or proficiency'' (Cooper, Heron \& Heward, 2007, p.  87).\\

To use, one must first determine what constitutes a response opportunity, and what the criteria for mastery will be. Response opportunities can vary depending on the target behavior. For example, an opportunity for spelling accuracy could be a 10-question spelling test. An opportunity for responding to one's name within five seconds might be every time someone called the individual's name. Other measures such as latency, percent occurrence, rate and duration can be used to compute trials to criterion data. For instance in the latter example to compute whether the individual is responding to his name within 5 seconds, data would have to be taken on latency to respond per opportunity (Cooper et al., 2007).\\
%
