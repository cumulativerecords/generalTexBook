\clearpage \section[\fourjOne{}]{\fourjOne{}%
              \sectionmark{J-01 State intervention goals...}}
\sectionmark{J-01 State intervention goals...}
\subsection{Definition}
``Target behaviors are selected for study in applied behavior analysis because of their importance to the people involved. Applied behavior analysts attempt to increase, maintain, and generalize adaptive, desirable behaviors and decrease the occurrence of maladaptive, undesirable behaviors'' (Cooper, Heron, \& Heward, 2007, p. 69).

Van Houten (1979) cited in Cooper et al. (2007, p. 69) suggests ``two basic approaches to determining socially valid goals: (a) Assess the performance of people judged to be highly competent, and (b) experimentally manipulate different levels of performance to determine empirically which produces optimal results. Regardless of the method used, specifying treatment goals before intervention begins providing a guideline for continuing or terminating treatment. Further, setting objective, predetermined goals helps to eliminate disagreements or biases among those involved in evaluating a program's effectiveness.''
%
\subsection{Examples}
\begin{enumerate}
\item The intervention goal: Client's aggression will decrease by 80\% of baseline levels across 5 consecutive school days. Staff will record every instance of hitting across the school day on the data collection sheet. 
\item The intervention goal: Client will expressively identify 10 different colors with 80\% accuracy, across 3 consecutive days. Staff will record data on correct and incorrect responses.
\end{enumerate}
%
\subsection{Assessment}
\begin{enumerate}
\item Have supervisee create a sample intervention goal that targets a response for increase. 
\item Have supervisee create a sample intervention goal that targets a response for decrease.
\item Have supervisee explain why it's important to have objective predetermined intervention goals.
\end{enumerate}
%
\subsection{Relevant Literature}
\begin{refsection}
\nocite{cooper2007applied,
        van1979social}
\printbibliography[heading=none]
\end{refsection}
%
\subsection{Related Tasks}
\fourbOne{}\\ 
\fouriOne{}\\
\fouriTwo{}\\
\fourFKSeven{}\\
