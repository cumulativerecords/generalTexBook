\clearpage \section[\fourbTwo{}]{\fourbTwo{}%
              \sectionmark{B-02 Review and interpret artic...}}
\sectionmark{B-02 Review and interpret artic...}
Applied behavior analysis is an applied science that develops its technology via the discovery of environmental variables that produce socially significant behavior change. The process of putting the science into practice begins with basic researchers discovering the principles of behavior that are then tested on socially significant behavior by applied researchers and then ultimately implemented by practitioners (Cooper, Heron, \& Heward, 2007). Whether a behavior analyst is solely a practitioner or a practitioner and a researcher, it is important that they maintain close contact with the scientific literature and its possible applications by regularly reviewing and critically interpreting articles from the behavior analytic literature.

When engaging in a review of the literature it is useful to consider the criteria that define applied behavior analysis which are outlined by Baer, Wolf, \& Risley (1987). These criteria referred to as the seven dimensions (applied, behavioral, analytic, technological, conceptually systematic, effective and generality) can not only assist in determining if an intervention  meets the standards of applied behavior analysis but it can also conclude whether a research intervention is prepared to translate into  practice or if further investigation is necessary prior to  effective clinical implementation.

Cooper, Heron and Heward (2007, p. 5), assert that ``scientific knowledge is built on above all, empiricism - the practice of objective observations of the phenomena of interest.'' Therefore, a behavior analyst should also remain objective when reviewing and interpreting articles.  For example, when reviewing an article it is important to make an unbiased interpretation regarding whether or not experimental control (internal validity) was established by acknowledging possible subject, setting, measurement and/or independent variable confounds (Cooper, Heron, \& Heward, 2007). Furthermore, it is important to remain objective when conducting visual analyses of the data to determine the extent that a functional relation is demonstrated. To assist in this process, Johnston and Pennypacker (2009) suggest that a behavior analyst have extensive experience with graphical analyses both as a designer and a reader.  They also note that the first step to analyzing behavioral data is to ask whether the data presentation is straightforward and productive toward the research question by orienting to the graph's scale, axes and legend.  Next, one should conduct a visual analysis by acknowledging the number of data points, variability, level and trend for each experimental condition followed by a visual analysis across conditions  and then across participants to draw comparisons and begin to establish conclusions (Cooper, Heron, \& Heward, 2007).

     In summary, a behavior analyst's ability to critically and empirically analyze the literature requires a thorough understanding of the science and acknowledgment of the bidirectional relationship between research and practice. 
%
\subsection{Assessment}
\begin{enumerate}
\item Provide the supervisee with an article and ask them to outline and possible threats to internal validity in the article. 
\item Provide the supervisee with an article that has the results and discussion removed. Ask Supervisee to give a precise summary of the results based upon the figures and tables. Afterward, provide the results and the discussion to compare and facilitate further supervisory discussion. 
\item Ask supervisee to determine if there is a function relation based on the data presented in an article and provide a rationale to their assertion. 
\item Ask the supervisee to evaluate an article to determine if it includes the seven dimensions of applied behavior analysis
\item Throughout the course of supervision, ask the supervisee to determine whether a literature review or research article (considering past research cited) is applied, basic or both and whether there is adequate support to use the intervention in clinical practice. 
\end{enumerate}
%
\subsection{Relevant Literature}
\begin{refsection}
\nocite{cooper2007applied,baer1968some,johnston2010strategies}
\printbibliography[heading=none]
\end{refsection}
%
\subsection{Related Lessons} 
\fourbOne{}\\
\fourhFour{}\\
\fouriFive{}\\
\fourFKFour{}\\
\fourFKNine{}\\
\fourFKThirtyThree{}\\
