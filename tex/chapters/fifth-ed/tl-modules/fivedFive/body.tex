%%\clearpage \section{\fourbFour{}}
\subsection{Definition}
Reversal design - ``Any experimental design in which the researcher attempts to verify the effect of the independent variable by ``reversing'' responding to a level obtained in a previous condition; encompasses experimental designs in which the independent variable is withdrawn (A-B-A-B) or reversed in its focus (e.g., DRI/DRA)'' (Cooper, Heron, \& Heward, 2007, p. 703).

Withdrawal design - ``A term used by some researchers as a synonym for an A-B-A-B design; also used to describe experiments in which an effective treatment is sequentially or partially withdrawn to promote the maintenance of behavior changes'' (Cooper, Heron, \& Heward, 2007, p. 708).
%
\subsection{Examples}
\begin{enumerate}
\item An experiment that entails exposing a participant to a condition of no programmed reinforcement for a work task (baseline) until steady state is achieved, then exposes a participate to a condition in which they earn stickers contingent on a work task (intervention) and then repeats these two conditions respectively. 
\item An experiment in which baseline consists of the reinforcement of challenging behavior and the treatment consists of differential reinforcement of an alternative/replacement behavior and both conditions are replicated at least twice. 
\end{enumerate}
%
\subsection{Assessment}
\begin{enumerate}
\item  Ask the supervisee to either describe a time that they used of a withdrawal or reversal design or have them describe a hypothetical experiment using a withdrawal or reversal design. 
\item Have supervisee look at the figures in the articles (such as those listed below) as well as other articles and determine which ones are reversal/withdrawal designs. 
\item Have the supervisee look at figures in the articles below and describe what characteristics make it a reversal or withdrawal design
\item Have the supervisee describe the pros of using a reversal design and the condition in which the use of a reversal design would not be desirable
\end{enumerate}
%
\subsection{Relevant Literature}
\begin{refsection}
\nocite{cooper2007applied,
    anderson2002use,
    baer1970recent,
    falcomata2004evaluation,
    lerman2002reinforcement}
\printbibliography[heading=none]
\end{refsection}
%
\subsection{Related Lessons} 
\fourbThree{}\\ 
\fourjNine{}\\
%
\subsection{Footnotes}
Some authors exclusively use the term reversal design for studies in which the contingency is reversed  (or switched to another behavior) as in DRO and DRA/DRI reversal techniques and the term withdrawal design for studies that employ an A-B-A-B approach where the A signifies baseline condition and B the treatment condition (Cooper, Heron, \& Heward, 2007). 

A multiple treatment reversal design can also be used to compare the effects of two or more treatment conditions to baseline and/or to the other treatments (e.g., ABABACAC, ABABCBCB) (Cooper, Heron, \& Heward, 2007).
%
%\clearpage \section[\fourbFive{}]{\fourbFive{}%
              \sectionmark{B-05 Use alternating treatments...}}
\sectionmark{B-05 Use alternating treatments...}
\subsection{Definition} 
Alternating treatments design - ``An experimental design in which two or more conditions (one of which may be a no treatment control condition) are presented in rapidly alternating succession (e.g., on alternating sessions or days) independent of the level of responding; differences in responding between or among conditions are attributed to the effects of the conditions (also called concurrent schedule design, multielement design, multiple schedule design)'' (Cooper, Heron, \& Heward, 2007, p. 689).
%
\subsection{Examples}
\begin{enumerate}
\item An experiment that entails conducting DRA and no programmed treatment during alternating sessions to compare treatment to the no treatment. 
\item An experiment that entails conducting DRI, DRO and DRA on alternating days to compare all treatments to each other. 
\end{enumerate}
%
\subsection{Assessment}
\begin{enumerate}
\item Ask the supervisee to either describe an alternating treatment design they have used in the past or have them describe an alternating treatments design.
\item Have supervisee look at various figures  from the articles (such as some of those below) as well as articles that did not use an alternating treatment design and have them determine which figures depict the use of an alternating treatment design. 
\item Have the supervisee look at figures in the articles below and describe what characteristics make it an alternating treatment design
\item Have the supervisee describe the pros of using an alternating treatment design and the condition in which the use of an alternating treatment design would not be desirable
\end{enumerate}
%
\subsection{Relevant Literature}
\begin{refsection}
\nocite{barbetta1993effects,
    barlow1979alternating,
    cooper2007applied,
    iwata1994toward,
    martens1992effects,
    singh1985comparison,
    ulman1975multielement}
\printbibliography[heading=none]
\end{refsection}
%
\subsection{Related Lessons} 
\fourbThree{}\\ 
\fourjNine{}\\
%
%\clearpage \section{\fourbSix{}}
\subsection{Definition} 
Changing criterion design - ``An experimental design in which an initial baseline phase is followed by a series of treatment phases consisting of successive and gradually changing criteria for reinforcement or punishment. Experimental control is evidenced by the extent the level of responding changes to conform to each new criterion'' (Cooper, Heron \& Heward, 2007, pp. 691-692).

``The design requires initial baseline observations on a single target behavior. This baseline phase is followed by implementation of a treatment program in each of a series of treatment phases. Each treatment phase is associated with a stepwise change in criterion rate for the target behavior. Thus, each phase of the design provides a baseline for the following phase. When the rate of the target behavior changes with each stepwise change in the criterion, therapeutic change is replicated and experimental control is demonstrated'' (Hartmann \& Hall, 1976, p. 527).

Guidelines for using the changing criterion design include:
\begin{enumerate}
\item Manipulation of the length of phases. Each phase serves as a baseline to compare responding to the next phase. Each phase must be long enough to display stable responding before moving to the next phase.
\item ``Varying the size of the criterion change enables a more convincing demonstration of experimental control'' (Cooper et al., 2007, p. 222). Criterion change magnitude must be carefully considered so that the criterion is not too large and unattainable but also not too small in magnitude which would not demonstrate sufficient experimental control.
\item Experimental control is demonstrated through replication of treatment effects. Therefore, as the number of phases increases, so does the opportunity to replicate treatment effects and enhance experimental control.
\end{enumerate}
%
\subsection{Examples}
\begin{enumerate}
\item Jim created a class wide reinforcement program to increase the vocabulary test scores of a 1st grade class. He wanted to make sure that the reinforcement program was effective so he set specific score criterion for the class, to monitor their progress. Average baseline test scores were 55\% correct for the entire class. Jim set the first criterion phase at 70\% of the test questions answered correctly for the entire class. After 4 weeks, the class met these criteria for 3 consecutive tests, so Jim set the classroom performance criterion to 80\% of the test questions answered correctly. This time the class met the criterion in 3 weeks and Jim increased the criterion to 90\%. Once again, the class met these criteria for 3 consecutive tests. Jim concluded that his intervention was likely responsible for the change in test scores, since the test score reliably increased when the criteria were altered and required a greater score.
\item (Non-example) John created a reinforcement program to increase Larry's rate of answering questions during class. After 3 weeks, the data indicated that Larry was answering more questions appropriately in class. However, there was a new teacher in the classroom and other variables that may have accounted for this change. John wanted to see if the program was increasing this behavior, so he decided to remove the reinforcement program for a week to see if Larry's rate of answering questions decreased.
\end{enumerate}
%
\subsection{Assessment}
\begin{enumerate}
\item Have supervisee describe the changing criterion design and state when it may be most appropriate, strengths of this design, as well as limitations of the changing criterion design.
\item Have supervisee create a hypothetical analysis using the changing criterion design. Have him/her state why the changing criterion design was the most effective design to display experimental control.
\item Have supervisee label the parts of a completed changing criterion design graph and describe how the graph displays experimental control. 
\end{enumerate}
%
\subsection{Relevant Literature}
\begin{refsection}
\nocite{cooper2007applied,
    hartmann1976changing,
    mcdougall2005range,
    mclaughlin1983examination,
    hall1977changing,
    allen2001exposure}
\printbibliography[heading=none]
\end{refsection}
%
\subsection{Related Lessons}
\fourbFour{}\\
\fourbFive{}\\
\fourbSeven{}\\
\fourbNine{}\\
\fourbEleven{}\\ 
\fourhFour{}\\
\fouriOne{}\\
%
%\clearpage \section{\fourbSeven{}}
\subsection{Definition}
Multiple baseline design – ``An experimental design that begins with the concurrent measurement of two or more behaviors in a baseline condition, followed by the application of the treatment variable to one of the behaviors while baseline conditions remain in effect for the other behavior(s) After maximum change has been noted in the first behavior, the treatment variable is applied in sequential fashion to each of the other behaviors in the design'' (Cooper, Heron \& Heward, 2007, p. 699).

Multiple baselines are useful when the target behavior is likely to be irreversible, for example, in skill acquisition. And are also useful when it may be impractical or undesirable to implement a reversal design. For example, in decreasing aggression toward peers. One drawback of the multiple baseline design is potentially the length of time that treatment or intervention is withheld for the last behavior or setting being targeted. 

In the delayed baseline design, collection of baseline data for other target behaviors is taken after baseline measurements for the previous behaviors. This design may be effective when a reversal design is not possible, when resources are limited, or when a new behavior or subject becomes available. Behaviors must be measured at the same time and the independent variable cannot be applied to the next behavior until the previous behavior change has been established. There should be a significant difference in the length of baseline conditions between the different behaviors and the independent variable should first be applied to the behavior demonstrating the greatest level of stable responding in baseline.

Other variations in multiple baselines designs are concurrent and nonconcurrent uses of the design. In concurrent multiple baseline designs the data are collected in the same time period. In nonconcurrent multiple baseline designs data can be collected at different times, and different lengths of baselines are collected, following which implementation of the treatment or intervention is conducted—creating multiple A-B experiments. The experiments are then arranged by length of baseline to create a multiple baseline design.  ``According to single-case design logic, the nonconcurrent MB design demonstrates only prediction and replication, and not the critical verification of the intervention's effects'' (Carr, 2005, p. 220).

\subsection{Examples}
\begin{enumerate}
\item Rod conducted an FBA on Billy's aggression and property destruction. Both behaviors were determined to be maintained by escape from demands. Rod decided to implement the same intervention for each behavior using a multiple baseline design because they both served the same function and a reversal would possibly reestablish the dangerous behavior after therapeutic effects were observed. 
\item (Non-example) Bob wanted to determine the effects of response blocking and redirection on hand flapping with one of his students. He implemented this procedure and once it proved effective, decided to eliminate the intervention to determine if this procedure was the likely cause of the behavior decrease. 
\end{enumerate}
%
\subsection{Assessment}
\begin{enumerate}
\item Have supervisee describe the multiple baseline design and state when it may be most appropriate, strengths of this design, as well as limitations of the multiple baseline design.
\item Have supervisee create a hypothetical analysis using the multiple baseline design. Have him/her state why the multiple baseline design was the most effective design to display experimental control.
\item Have supervisee label the parts of a completed multiple baseline design graph and describe how the graph displays experimental control. 
\end{enumerate}
%
\subsection{Relevant Literature}
\begin{refsection}
\nocite{cooper2007applied,
  barger2004multiple,
  carr2005recommendations,
  harris1985comparisons,
  harvey2004nonconcurrent,
  watson1981non,
  zhan2001single}
\printbibliography[heading=none]
\end{refsection}
%
\subsection{Related Lessons}
\fourbThree{}\\
\fourbFour{}\\
\fourbEight{}\\
\fourbNine{}\\
\fourbTen{}\\
\fourbEleven{}\\
\foureOne{}\\
\fourhThree{}\\
\fouriFive{}\\
\fourFKThirtyThree{}\\
%
%\clearpage \section{\fourbEight{}}
Like multiple baseline designs (MBD), multiple probe designs (MPD) are ``rigorous in their evaluation of threats to interval validity; and are practical for teachers and clinicians who want their research efforts to be wholly compatible with their instructional or therapy activities'' (Gast, 2009, p. 277). Multiple probe designs have an additional advantage in applied settings in that intermittent measures of baseline conditions streamline data collection and still maintain the requirement that responding does not change until intervention is applied (baseline logic). Either one probe is taken periodically in baseline conditions and at least three days immediately before applying intervention, MPD (days), or probes occur in brief sessions of a few baseline measurements taken at least three consecutive days before intervention, MPD (conditions). Experimental control is demonstrated if probe evidence across each tier of similar, but functionally different behaviors, participants, or conditions remains relatively stable until intervention is implemented. 
%
\subsection{Examples}
\begin{enumerate}
\item Multiple probe designs are particularly useful for researchers and teachers in educational settings to efficiently demonstrate results of instructional interventions when teaching across functionally different new skills (behaviors), across multiple students (participants), or across different sets of skills (conditions).
\item Multiple probe designs might not be appropriate if assessing the effects of intervention on severe behaviors that result in injury or property destruction because of the requirement that intervention be delayed across each tier while a person continues to engage in severe behaviors with lasting consequences.
\end{enumerate}
%
\subsection{Assessment}
\begin{enumerate}
\item Ask the supervisee to explain the meaning of ``baseline logic'' and the reason it enables measurement of experimental control.
\item Ask the supervisee why MPD across behaviors requires similar, but functionally different behaviors in each tier of a MPD.
\item Give the supervisee three articles in which researchers based their conclusions on MPD line graph data. Ask the supervisee to interpret results of the study based on the graphic data. Compare to the conclusions written by study authors.
\end{enumerate}
%
\subsection{Relevant Literature}
\begin{refsection}
\nocite{ledford2009single,horner1978multiple,thompson1982training}
\printbibliography[heading=none]
\end{refsection}
%
\subsection{Related Lessons}
\fourbThree{}\\
\fourbSeven{}\\
\fourhOne{}\\
\fourhTwo{}\\
\fourhFour{}\\
\fouriFive{}\\
\fourFKThirtySix{}\\
%
%\clearpage \section{\fourbNine{}}
\subsection{Definition}
When designing an experiment it is sometimes useful to combine experimental design elements to strengthen the demonstration of experimental control. For instance it may be valuable to combine a multiple baseline design with a reversal design. 

For example, Colón et al. (2012) used a nonconcurrent multiple baseline design across participants to analyze the effects of verbal operant training on appropriate vocalizations and vocal stereotypy. RIRD was implemented and examined using a reversal design for each participant exposed to their procedure.

In 1985, Alexander used a multiple baseline across students with reversal design to evaluate the effects of a study skill training procedure.

Johnston and Pennypacker (2009) point out that experimenters often combine and intermingle many different types of designs as necessary. Categorizing types of designs is really a more valuable thing for the student than it is for the researcher. 

Murray Sidman in Tactics of Scientific Research says that it is not valuable to say that there are rules to follow when designing an experiment. He says ``this would be disastrous'' (Sidman, 1960/1988, p. 214). Simply put, he says, ``The fact is that there are no rules of experimental design'' (Sidman, 1960/1988, p. 214).

The most important thing is that the experiment is designed to answer some question we have about the natural world. Sidman says, ``We conduct experiments to find out something we do not know'' (Sidman, 1960/1988, p.214).
%
\subsection{Assessment}
\begin{enumerate}
\item Have your supervisee design an experiment that would be best suited to use a combination of design elements.
\item Ask your supervisee to point to the sections of the graphs in Colon et al. (2012) and Alexander (1985) that reflect the types of experimental designs used. 
\item Ask your supervisee to describe what Sidman meant when he said that having rules for designing an experiment would be ``disastrous.''
\end{enumerate}
%
\subsection{Relevant Literature}
\begin{refsection}
\nocite{alexander1985effect,colon2012effects,iwata2000skill,johnston2010strategies}
\printbibliography[heading=none]
\end{refsection}
%                                                                   
\subsection{Related Lessons}
\fourbThree{}\\
\fourbSeven{}\\
\fourhOne{}\\
\fouriFive{}\\
