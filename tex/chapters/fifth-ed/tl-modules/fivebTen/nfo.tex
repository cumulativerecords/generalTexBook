%\clearpage \section{\fourFKTwentyFour{}}
\subsection{Definition}
Stimulus control - ``A situation in which the frequency, latency, duration, or amplitude of a behavior is altered by the presence or absence of an antecedent stimulus'' (Cooper, Heron, \& Heward, 2007, p. 705).\\

Discriminated operant - ``An operant that occurs more frequently under some antecedent conditions than under others'' (Cooper, Heron, \& Heward, 2007, p.694).\\

Discriminative stimulus (SD) - ``A stimulus in the presence of which responses of some type have been reinforced and in the absence of which the same type of responses have occurred and not been reinforced''  (Cooper, Heron, \& Heward, 2007, p. 694).

%\clearpage \section{\fourFKTwentyFive{}}
%\subsection{Definition}
The same stimulus may serve multiple functions depending on the context. For instance, an aversive stimulus can function as a positive punisher or a negative reinforcer depending on whether it is added or removed contingent on a response. An appetitive stimulus can function as a positive reinforcer or a negative punisher depending on whether it is added or removed contingent on a response.

In a behavior chain, a stimulus can function as a discriminative stimulus and a reinforcer depending on how much of the chain has been completed. Behavior chains are described by Catania as ``a sequence of discriminated operants such that responses during one stimulus are followed by other stimuli that reinforce those responses and set the occasion for the next ones'' (Catania, 2013, p. 431).

Respondent behavior interacts with operant behavior in ways that can cause a single stimulus to be an eliciting stimulus for respondent behavior as well as a discriminative stimulus for operant behavior. For example, when seeing your best friend arrive to your home for a visit, this may elicit respondent behavior one might describe as ``excitement.'' Seeing your friend may also serve as a discriminative stimulus for waving at him/her.
%
