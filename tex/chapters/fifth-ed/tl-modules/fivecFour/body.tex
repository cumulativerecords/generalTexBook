\clearpage \section{\fouraThree{}}
\subsection{Definition}
Duration – ``A measure of the total extent of time in which a behavior occurs'' (Cooper, Heron, \& Heward, 2007, p. 79).
%
\subsection{Examples}
\begin{enumerate}
\item Sarah gets a fancy new piece of furniture from one of those Swedish companies.  When it arrives, Sarah realizes that it is not assembled.  She reads the complicated set of directions and begins putting it together at 2:12pm. Armed with a screwdriver and an Allen wrench, she consistently works to put it together until 3:43pm. Phew! Maybe next time she will order the one that comes fully assembled! The duration of the project was 1 hour and 31 minutes.
\item Benny gets a new yo-yo for his birthday and plays with it for 20 minutes after eating his cake and ice cream. He puts it down to play tag with his sister. The duration of yo-yo playing is 20 minutes.
\end{enumerate}
%
\subsection{Assessment}
\begin{enumerate}
\item Ask your supervisee to identify the duration of furniture assembly from the example above
\item Ask your supervisee to create another example and non-example of his/her own
\item Have your supervisee measure the duration of another behavior on the job or in role-play.
\item Have the supervisee graph the duration of another behavior measured on the job or in a role-play.
\end{enumerate}
%
\subsection{Relevant Literature}
\begin{refsection}
\nocite{cooper2007applied,deleon1999examination}
\printbibliography[heading=none]
\end{refsection}%


\subsection{Related Tasks}
\fourhOne{}\\
\fouriOne{}\\
\fourFKFourtySeven{}\\
%
\clearpage \section{\fouraFour{}}
\subsection{Definition}  
Latency - ``A measure of temporal locus; the elapsed time from the onset of a stimulus (e.g., task direction, cue) to the initiation of a response'' (Cooper, Heron, \& Heward, 2007, p. 80).  

\subsection{Examples}
\begin{enumerate}
\item Hitting the snooze button or hitting the break
\item Example: Gertrude is not a morning person.  Her alarm goes off at precisely 5:30AM.  She hears the annoying wail but doesn't respond immediately.  After 32 seconds of beeping, she whacks the snooze button, rolls over and goes back to sleep.  Latency to turning off the alarm is 32 seconds.
\item Example:  Marty is driving down a country road.  Out of nowhere a herd of deer dart out in front of his car.  It takes Marty 5 seconds from the time he first sees the deer to hit the break.  Latency from the time the deer are spotted to applying pressure to the break is 5 seconds.    
\item Non-example: Gertrude is not a morning person.  Her alarm goes off at precisely 5:30 AM.  She does not respond to its annoying wailing and continues to sleep despite the noise.  The alarm stops on its own 1 hour later.
\end{enumerate}
%
\subsection{Assessment}
\begin{enumerate}
\item Ask your supervisee to identify the latency of a few responses of your choosing.  
\item Ask your supervisee to create another example and non-example of his/her own.
\item Have your supervisee measure the latency to another behavior on the job or in role-play.
\end{enumerate}
%
\subsection{Relevant Literature}
\begin{refsection}
\nocite{cooper2007applied,thomason2011response}
\printbibliography[heading=none]
\end{refsection}

\subsection{Related Tasks}
\fourhOne{}\\
\fouriOne{}\\
\fourFKFourtySeven{}\\
%
\clearpage \section{\fouraFive{}}
\subsection{Definition}  
Interresponse time (IRT) - ``...the elapsed time between two successive responses'' (Cooper, Heron, \& Heward, 2007, p. 80).
%  
\subsection{Examples} 
\begin{enumerate}
\item Sparky loves to bark at passing cars.  He hears a car drive by the house and barks.  Thirty-seven seconds later another car passes by and Sparky barks again.  Interresponse time between barking at the vehicles is thirty-seven seconds.
\item Doodles the cat likes to scratch the furniture.  She walks over the chair and sinks her claws in.  Eleven seconds later Doodles walks over to the couch and begins to scratch again.  Interresponse time between scratches is eleven seconds.  
\item Roger the rooster doesn't know that he's only supposed to crow at dawn. He lets out crows all day long.  He is observed to crow at 3:43 in the afternoon.  He crows again at 3: 59.  Interresponse time between crows is sixteen minutes.  
\item (Non-example) Sparky's owner accidentally steps on his tail.  Sparky yelps from the pain. 
\end{enumerate}
%
\subsection{Assessment}
\begin{enumerate}
\item Ask your supervisee to identify the interresponse time in the examples above. 
\item Ask your supervisee to create another example and non-example of his/her own. 
\item Have your supervisee measure the interresponse time of a behavior on the job or in a role-play. 
\item Have your supervisee graph the interresponse time measured on the job or in a role-play.
\end{enumerate}
%
\subsection{Relevant Literature}
\begin{refsection}
\nocite{blough1963interresponse,cooper2007applied,favell1980rapid}
\printbibliography[heading=none]
\end{refsection}
%
\subsection{Related Tasks}
\fouraFive{}\\
\fouriOne{}\\
\fourhOne{}\\
\fourFKFourtySeven{}\\
%
\clearpage \section[\fourFKFourtySeven{}]{\fourFKFourtySeven{}%
              \sectionmark{FK-47 Identify the measur...}}
\sectionmark{FK-47 Identify the measur...}
\subsection{Definition}
According to Johnston and Pennypacker (1993), behavior has 3 fundamental dimensional quantities (properties) that can be measured.
\begin{enumerate}
\item Repeatability – Refers to the fact that a behavior can occur repeatedly through time (i.e., behavior can be counted) (e.g., count, frequency, rate)
\item Temporal extent – Refers to the fact that every instance of behavior occurs during some amount of time (i.e., when behavior occurs it can be measured in time.) (e.g., duration)
\item Temporal locus – Refers to the fact that every instance of behavior occurs at a certain point in time with respect to other events. (i.e., occurrences of behavior can be measured at points in time.) (e.g., latency, interresponse time) (As cited in Cooper, Heron, \& Heward, 2007, p. 26)
\end{enumerate}
%
\subsection{Assessment}
\begin{enumerate}
\item Ask Supervisee to measure their own duration related to a task (eg. give them a timer and crossword puzzle to complete)
\item Ask Supervisee to measure and calculate the rate of someone tapping their pen (or another discrete behavior) during a 10 minute meeting
\item Ask Supervisee to observe a conversation between colleagues and measure latency regarding question asking-answering.  Have Supervisee use a timer/stop watch to record latency
\item Ask Supervisee to record their own latency during a supervision meeting when asked to define a task list item, vs. a concept 
\item Ask Supervisee to measure interresponse time (IRT) by observing someone eating a meal for 5 minutes; have Supervisee record time between swallowing one bit of food and next bite and report the average IRT.
%
\end{enumerate}
%
\subsection{Relevant Literature}
\begin{refsection}
\nocite{cooper2007applied,
        johnston2010strategies,
        thomason2011response,
        worsdell2002duration}
\printbibliography[heading=none]
\end{refsection}
%
\subsection{Related Tasks}
\fouraOne{}\\
\fouraTwo{}\\
\fouraThree{}\\
\fouraFour{}\\
\fouraFive{}\\
\fouraNine{}\\
\fourdTwentyOne{}\\
%
\clearpage \section[\fourhFive{}]{\fourhFive{}%
              \sectionmark{H-05 Evaluate temporal relations...}}
\sectionmark{H-05 Evaluate temporal relations...}
\subsection{Definition}
Behavior analysts can analyze data across several temporal relations prior to visual inspection. ``The manner in which data are aggregated before transforming them into a visual display serves an equally influential role in data analysis'' (Fahmie \& Hanley, 2008, p. 320).  Such aggregation occurs with the use of within-session, between-session in time series data. 

Between-session analysis involves plotting total number of occurrences of a dependent variable within some unit of time (i.e., sessions) and visually inspecting point-by-point (i.e., session-by-session). Another prevalent type of aggregation occurrences of behavior is within-session data (likely due to its universal application). 

Within-sessing data can be analyzed via the observation of data as it changes throughout the duration of the session or at specific times during the session. Fahmie and Hanley (2008) outlined eight conditions under which within-session data are valuable:
\begin{enumerate}
\item Description of naturally occurring behavioral relations (descriptive assessment) 
\item Determination of behavioral function (functional analyses)
\item Detection of within-session trends
\item Safeguard clients from any risks associated with prolonged session exposure
\item Creation of sufficient data for analysis following abbreviated data collection
\item Determination of observation session duration
\item Clarification of counterintuitive response patterns
\item Understanding behavioral processes
\end{enumerate}

There are several methods of within-session data analysis. In the descriptive assessment literature, within-session data are calculated via conditional probabilities to determine possible temporal relations between behavior and environmental events (e.g., occurrence/nonoccurrence of putative reinforcer delivery) (Vollmer, Borrero, Wright, Camp, \& Lalli, 2001). In the functional analysis literature, within-session data have been used to compare the utility of two types of functional analyses (e.g., trial-based versus multi-element) (Kahng \& Iwata, 1999; LaRue et al., 2010). Moreover, within-session data have been used in an effort to further analyze unclear results following an unclear analysis of full session data (Call \& Mevers, 2014; Kahng \& Iwata, 1999; Payne et al., 2014; Roane, Lerman, Kelley, \& VanCamp, 1999; Vollmer, Marcus, Ringhdahl \& Roane, 1995; Vollmer et. al., 1993). For example, Kahng \& Iwata (1999) compared full 15-minute functional analysis session data with within-session data by plotting the first session of each condition into a minute-by-minute observation period. One of their findings was that within-session data clarified unclear (absence of function) results of the full session data. 

In another example, Payne et al., (2014) analyzed within-session data in different manner by comparing data when the putative establishing operation (EO) was present versus when the putative EO was absent across the last five 10-minute sessions of each condition. The results generated from the within-session data analysis was used to develop a second experimental analysis that clarified the function of the behavior for the two participants.
%
\subsection{Assessment}
\begin{enumerate}
\item Have the supervisee read the Fahmie \& Hanley (2008) article. Then provide the supervisee with examples of different data analysis units along the continuum the authors display in Figure 1. Have them place the scenarios along the continuum and discuss. 
\item Have the supervisee describe different methods of within session data collection ( e.g. minute-by-minute, event based observation period comparisons) and their utility ( a review of the methods used in the relevant literature will assist with this task)
%
\end{enumerate}
%
\subsection{Relevant Literature}
\begin{refsection}
\nocite{call2014relative,
        fahmie2008progressing,
        hartmann1980interrupted,
        iwata2000skill,
        krause1999long,
        larue2010comparison,
        payne2014using,
        roane1999within,
        tryon1982simplified,
        vollmer2001identifying,
        vollmer1993within,
        vollmer1995progressing}
\printbibliography[heading=none]
\end{refsection}
%
\subsection{Related Tasks} 
\fourhOne{}\\
\fourhFour{}\\
\fouriFive{}\\
\fourjFifteen{}\\
\fourFKFourtySeven{}\\
