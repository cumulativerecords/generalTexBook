%%\clearpage \section{\fourkSeven{}}
\subsection{Definition}
In the field of applied behavior analysis, it is crucial to have ongoing evaluation of the effectiveness of behavior programs.  This helps to ensure that the most effective treatments are being offered to a client based on ethical practices, the most current research, and the individual's needs.  

Birnbrauer (1999) lists the following steps for evaluating the effectiveness of treatment:
\begin{enumerate}
\item Describe the exact purposes of the treatment –what is it intended to achieve? 
\item Describe exactly how the treatment is conducted –there should be no mystery or secrecy about the methods and procedures being used. 
\item Describe how treatment effects were measured –what numerical data were collected and how were they collected? 
\item Show before and after data collected by independent, unbiased evaluators
\item Show follow up data –do the persons maintain gains? Do they continue to improve? Do they regress? 
\end{enumerate}

``The data obtained throughout a behavior change program or a research study are the means for that contract; they form the empirical basis for every important decision: to continue with the present procedure, to try a different intervention, or to reinstitute a previous condition'' (Cooper, Heron, \& Heward, 2007, p. 167).  It is important that if there is evidence of behavioral regression or that the treatment package is ineffective, that the team re-evaluate, make changes or adjustments, or discontinue a behavioral program entirely.  It is unethical to continue a behavioral program that is deemed ineffective. 
%
\subsection{Assessment}
\begin{enumerate}
\item Ask the supervisee to state why ongoing evaluation of a behavioral program is important.
\item Ask the supervisee to state what all important program decisions should be driven by.
\item Ask the supervisee to state who should be evaluating data.
\item Ask the supervisee under which conditions to discontinue a behavioral program.
%
\end{enumerate}
%
\subsection{Relevant Literature}
\begin{refsection}
\nocite{bailey2013ethics,
        birnbrauer1999how,
        cooper2007applied}
\printbibliography[heading=none]
\end{refsection}
%
\subsection{Related Lessons}
\fourbOne{}\\
\fourFKFourtySeven{}\\
\fourgOne{}\\
\fourhOne{}\\
\fouriOne{}\\
\fouriFive{}\\
\fouriSix{}\\
\fourjOne{}\\
\fourkSeven{}\\
