\clearpage \section[\fourkTwo{}]{\fourkTwo{}%
              \sectionmark{K-02 Identify the conting...}}
\sectionmark{K-02 Identify the conting...}
\subsection{Definition}
Behavior is understood to be a product of the environment in which it occurs. This is the same for all organisms, including the client, the caretakers, the professionals working with the client, and ourselves. A well trained behavior analyst accounts for the environmental arrangement for all of the individuals involved in the behavior change process. 

For instance, if a procedure is very effortful and will not produce an effect for several weeks, what will reinforce the behavior of the family member/teacher who will be implementing it? If a procedure produces lots of aggression or screaming, it should be considered that these are often aversive stimuli to the people implementing it.

``Treatment drift occurs when the application of the independent variable during later phases of an experiment differs from the way it was applied at the outset of the study'' (Cooper et al., 2007, p. 235).This is often the result of a practitioner's behavior meeting competing contingencies after having followed the plan for a period of time.

High treatment integrity can be achieved by creating a thorough and precise definition for the independent variables, simplifying the treatment procedures, providing ample training and practice to all individuals responsible for treatment, and assessing the contingencies each person's behavior will meet while following through with these interventions. 

Other factors that can help improve treatment integrity and regulate the behavior of those involved in the experiment include using less expensive and less intrusive procedures, seeking help and input from the participants family members and other people close to them, setting socially significant but easy-to-meet criterion for reinforcement, eliminating reinforcement gained outside of performing the target response, and contrive contingencies that will compete with natural contingencies. 
%
\subsection{Examples}
\begin{enumerate}
\item Don has been asked to help deliver a new differential reinforcement program to decrease his student's self-injury. However, he has been short of staffing lately and cannot do this consistently throughout the day. This has caused the program to be run without integrity and the self-injury to remain at the same rates. Rick, the BCBA who designed the program, started observing the classroom to see why the program was not working. After noticing that Don was only delivering the reinforcers intermittently and missing opportunities for reinforcement, Rick decided to retrain Don and ask the principal for extra staff when someone calls out sick. 
\item (Non-example) Roger is implementing a new response cost program to decrease verbal protesting for one of his students. However, even though the program has been run as designed for several weeks, there has been no effect on the verbal protesting of the student. The supervisor collected integrity data and found that the plan had been run as prescribed. The problem is likely related to the procedure itself and not its implementation.
%
\end{enumerate}
%
\subsection{Assessment}
\begin{enumerate}
\item Make sure the supervisee makes considerations about the effort and practicality of the treatment they attempt to get other people to implement.
\item Have supervisee identify several strategies for increasing treatment integrity. Have him/her describe how they would use these strategies in an applied setting.
\item Have supervisee read several articles in which an intervention was implemented to decrease or increase a target behavior. Have him/her identify the strategies the researcher used to increase treatment integrity.
\item Have supervisee define and describe treatment drift. Have him/her describe how they would account for the occurrence of treatment drift and adjust an intervention or experiment accordingly.
%
\end{enumerate}
%
\subsection{Relevant Literature}
\begin{refsection}
\nocite{cooper2007applied,
        fryling2012impact,
        reed2011parametric,
        wheeler2006treatment,
        wilder2006effects,
        mcintyre2007treatment,
        peterson1982integrity,
        pipkin2010effects}
\printbibliography[heading=none]
\end{refsection}
%
\subsection{Related Lessons}
\fourjOne{}\\
\fourkThree{}\\
\fourkFour{}\\
\fourkFive{}\\
\fourkSix{}\\
\fourkEight{}\\
