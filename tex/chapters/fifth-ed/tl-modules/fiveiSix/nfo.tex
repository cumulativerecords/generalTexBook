%\clearpage \section[\fourkTwo{}]{\fourkTwo{}%
\subsection{Definition}
Behavior is understood to be a product of the environment in which it occurs. This is the same for all organisms, including the client, the caretakers, the professionals working with the client, and ourselves. A well trained behavior analyst accounts for the environmental arrangement for all of the individuals involved in the behavior change process. 

For instance, if a procedure is very effortful and will not produce an effect for several weeks, what will reinforce the behavior of the family member/teacher who will be implementing it? If a procedure produces lots of aggression or screaming, it should be considered that these are often aversive stimuli to the people implementing it.

``Treatment drift occurs when the application of the independent variable during later phases of an experiment differs from the way it was applied at the outset of the study'' (Cooper et al., 2007, p. 235).This is often the result of a practitioner's behavior meeting competing contingencies after having followed the plan for a period of time.

High treatment integrity can be achieved by creating a thorough and precise definition for the independent variables, simplifying the treatment procedures, providing ample training and practice to all individuals responsible for treatment, and assessing the contingencies each person's behavior will meet while following through with these interventions. 

Other factors that can help improve treatment integrity and regulate the behavior of those involved in the experiment include using less expensive and less intrusive procedures, seeking help and input from the participants family members and other people close to them, setting socially significant but easy-to-meet criterion for reinforcement, eliminating reinforcement gained outside of performing the target response, and contrive contingencies that will compete with natural contingencies. 
%
