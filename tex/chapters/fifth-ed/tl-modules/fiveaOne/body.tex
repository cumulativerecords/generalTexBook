\clearpage \section{\fiveaOne{}}
%\clearpage \section{\fourFKOne{}}
\subsection{Definition}
Lawfulness of behavior – ``behavior is the result of some condition that has caused it to happen'' (Malott, 2012, p. 168)

The lawfulness of behavior makes a science of behavior possible. ``Science is, of course, more than a set of attitudes. It is a search for order, for uniformities, for lawful relations among the events in nature'' (Skinner, 1953, p.13).

If behavior did not follow universal laws related to the environment that hosts it, it would not be possible to predict or control responding in a scientific way. Skinner describes the necessity for lawfulness of behavior in this quote: ``If we are to use the methods of science in the field of human affairs, we must assume that behavior is lawful and determined. We must expect to discover that what a man does is the result of specifiable conditions and that once these conditions have been discovered, we can anticipate and to some extent determine his actions'' (Skinner, 1953, p. 6).
%
%
%
\subsection{Assessment}
\begin{enumerate}
\item Ask your supervisee to describe why an understanding of lawfulness of behavior is important when designing treatments for their client.
\item Ask your supervisee to role-play a scenario in which he/she discusses lawfulness of behavior with a caretaker or teacher of a client.
%
\end{enumerate}
%
\subsection{Relevant Literature}
\begin{refsection}
\nocite{malott2015principles,
        skinner1953science}
\printbibliography[heading=none]
\end{refsection}
%
\subsection{Related Tasks}
\fourbThree{}\\
\fourFKTwo{}\\
\fourFKThree{}\\
\fourFKFour{}\\
\fourFKFive{}\\
\fourFKSix{}\\
%
\clearpage \section[\fourFKSeven{}]{\fourFKSeven{}%
              \sectionmark{FK-07 Environmental (as opposed...}}
\sectionmark{FK-07 Environmental (as opposed...}
\subsection{Definition}
Mentalism-``An approach to explaining behavior that assumes that a mental, or ‘inner,' dimension exists that differs from a behavioral dimension and that phenomena in this dimension either directly cause or at least mediate some forms of behavior, if not all'' (Cooper, Heron, Heward, 2007, p. 699).

An environmental explanation of behavior can be described by physical events in the phylogenetic or ontogenetic history of the organism that cause behavior to occur. A behavior analyst believes that all behavior is a result of these physical events and that there is no reason to believe that there are some causes of behavior outside of physical dimensions.

It can be difficult sometimes, as we learn behavior analysis, to describe behavior without the use of mentalistic explanations (e.g., the hit me because he's frustrated). This is because in non-behavior analytic cultures, where many behavior analysts spend most of their lives, behavior is described this way and there is reinforcement available from that verbal community to perpetuate mentalistic explanations of behavior. For instance, it is common for people to believe that we each are responsible for our own actions and that the choices we make are done so with ``free-will.''  Johnston (2014) mentions that, ``After a lifetime of explaining behavior in terms of such apparent freedom, it is understandably difficult to accept what appears to be a helpless or passive role...'' (p.5) 

Much of behavior in society is controlled by consequences. Johnston (2014) says ``...we assign the responsibility for behavior not to the individual but to sources of control in the physical environment. From this perspective, holding individuals responsible for their behavior by specifying the consequences for certain actions remains an important contingency because it helps manage those tendencies to act in one way or another'' (p.11).
%
%
%
%
\subsection{Assessment}
\begin{enumerate}
\item Provide scenarios for a supervisee describing repetitive problem behaviors that might lead to a conclusion that internal events are controlling variables for behavior. Tell the supervisee to write a mentalistic explanation that might explain the behavior and then identify a radical behaviorist approach to explaining the same response.
\item Present a scenario in which a supervisee is working with parents or staff who insist that their child is hitting them because she is angry or frustrated. Ask the supervisee to role play explaining to care givers that behavior analysts look at anger and frustration a different way. 
%
\end{enumerate}
%
\subsection{Relevant Literature}
\begin{refsection}
\nocite{cooper2007applied,
        skinner1953science}
\printbibliography[heading=none]
\end{refsection}
%
\subsection{Related Tasks}
\fourbOne{}\\
\fourgFour{}\\
\fourgFive{}\\
\fouriOne{}\\
\fouriTwo{}\\
\fourkTwo{}\\
\fourFKOne{}\\
\fourFKThree{}\\
\fourFKSeven{}\\
\clearpage \section{\fourFKOne{}}
\subsection{Definition}
Lawfulness of behavior – ``behavior is the result of some condition that has caused it to happen'' (Malott, 2012, p. 168)

The lawfulness of behavior makes a science of behavior possible. ``Science is, of course, more than a set of attitudes. It is a search for order, for uniformities, for lawful relations among the events in nature'' (Skinner, 1953, p.13).

If behavior did not follow universal laws related to the environment that hosts it, it would not be possible to predict or control responding in a scientific way. Skinner describes the necessity for lawfulness of behavior in this quote: ``If we are to use the methods of science in the field of human affairs, we must assume that behavior is lawful and determined. We must expect to discover that what a man does is the result of specifiable conditions and that once these conditions have been discovered, we can anticipate and to some extent determine his actions'' (Skinner, 1953, p. 6).
%
%
%
\subsection{Assessment}
\begin{enumerate}
\item Ask your supervisee to describe why an understanding of lawfulness of behavior is important when designing treatments for their client.
\item Ask your supervisee to role-play a scenario in which he/she discusses lawfulness of behavior with a caretaker or teacher of a client.
%
\end{enumerate}
%
\subsection{Relevant Literature}
\begin{refsection}
\nocite{malott2015principles,
        skinner1953science}
\printbibliography[heading=none]
\end{refsection}
%
\subsection{Related Tasks}
\fourbThree{}\\
\fourFKTwo{}\\
\fourFKThree{}\\
\fourFKFour{}\\
\fourFKFive{}\\
\fourFKSix{}\\
%
\clearpage \section[\fourFKSeven{}]{\fourFKSeven{}%
              \sectionmark{FK-07 Environmental (as opposed...}}
\sectionmark{FK-07 Environmental (as opposed...}
\subsection{Definition}
Mentalism-``An approach to explaining behavior that assumes that a mental, or ‘inner,' dimension exists that differs from a behavioral dimension and that phenomena in this dimension either directly cause or at least mediate some forms of behavior, if not all'' (Cooper, Heron, Heward, 2007, p. 699).

An environmental explanation of behavior can be described by physical events in the phylogenetic or ontogenetic history of the organism that cause behavior to occur. A behavior analyst believes that all behavior is a result of these physical events and that there is no reason to believe that there are some causes of behavior outside of physical dimensions.

It can be difficult sometimes, as we learn behavior analysis, to describe behavior without the use of mentalistic explanations (e.g., the hit me because he's frustrated). This is because in non-behavior analytic cultures, where many behavior analysts spend most of their lives, behavior is described this way and there is reinforcement available from that verbal community to perpetuate mentalistic explanations of behavior. For instance, it is common for people to believe that we each are responsible for our own actions and that the choices we make are done so with ``free-will.''  Johnston (2014) mentions that, ``After a lifetime of explaining behavior in terms of such apparent freedom, it is understandably difficult to accept what appears to be a helpless or passive role...'' (p.5) 

Much of behavior in society is controlled by consequences. Johnston (2014) says ``...we assign the responsibility for behavior not to the individual but to sources of control in the physical environment. From this perspective, holding individuals responsible for their behavior by specifying the consequences for certain actions remains an important contingency because it helps manage those tendencies to act in one way or another'' (p.11).
%
%
%
%
\subsection{Assessment}
\begin{enumerate}
\item Provide scenarios for a supervisee describing repetitive problem behaviors that might lead to a conclusion that internal events are controlling variables for behavior. Tell the supervisee to write a mentalistic explanation that might explain the behavior and then identify a radical behaviorist approach to explaining the same response.
\item Present a scenario in which a supervisee is working with parents or staff who insist that their child is hitting them because she is angry or frustrated. Ask the supervisee to role play explaining to care givers that behavior analysts look at anger and frustration a different way. 
%
\end{enumerate}
%
\subsection{Relevant Literature}
\begin{refsection}
\nocite{cooper2007applied,
        skinner1953science}
\printbibliography[heading=none]
\end{refsection}
%
\subsection{Related Tasks}
\fourbOne{}\\
\fourgFour{}\\
\fourgFive{}\\
\fouriOne{}\\
\fouriTwo{}\\
\fourkTwo{}\\
\fourFKOne{}\\
\fourFKThree{}\\
\fourFKSeven{}\\
