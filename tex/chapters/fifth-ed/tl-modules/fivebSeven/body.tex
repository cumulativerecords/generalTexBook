\clearpage \section{\fourFKTwentyThree{}}
\subsection{Definition}
Automatic reinforcement - ``Reinforcement that occurs independent of the social mediation of others'' (Cooper Heron, \& Heward, 2007, p. 267).\\

Automatic punishment - ``Punishment that occurs independent of the social mediation by others''(Cooper Heron, \& Heward, 2007, p. 534).
%
\subsection{Examples}  
\begin{enumerate}
\item (Automatic Reinforcement) Scratching an insect bite removes an itch; eating food when hungry removes hunger, humming may be auditory reinforcement; nonfunctional movements such as hand flapping may produce a sensation, which is automatically reinforcing; some self-injurious behavior may produce a sensation, which the individual may enjoy.
\item (Automatic Punishment) Albert bites his canker sore, causing a shocking pain.  Albert is becomes cautious as he eats with his canker sore until the canker disappears.  A dog gets a thorn in his paw.  He experiences pain when he steps down on his foot.  He begins to walk on three legs. 
\end{enumerate}
%
\subsection{Assessment}
\begin{enumerate}
\item Ask your supervisee to come up with several examples of automatic punishment and automatic reinforcement. 
\item Ask your supervisee provide examples of how extinction of the following automatically reinforced behavior may occur: child making sounds by tapping the table, child receiving kinesthetic stimulation by flapping his arms, child throws up and eats vomit for the taste, child scratching surface for tactile stimulation on fingers, child flipping light switch on and off to gain a visual sensation
\item Ask supervisee if they will likely have to treat clients with an automatic punishment function (yes-many food refusal behavior may have an automatic punishment for example).
%
\end{enumerate}
%
\subsection{Relevant Literature}
\begin{refsection}
\nocite{cooper2007applied,
        vollmer1994concept}
\printbibliography[heading=none]
\end{refsection} 
%
\subsection{Related Tasks}
\fourFKSeventeen{}\\
\fourFKNineteen{}\\
\fourFKTwentyTwo{}\\
%
\clearpage \section{\fourdFive{}}
\subsection{Definition}
Shaping – ``Using differential reinforcement to produce a series of gradually changing response classes; each response class is a successive approximation toward a terminal behavior.  Members of an existent response class are selected for differential reinforcement because they more closely resemble the terminal behavior'' (Cooper, Heron, \& Heward, 2007, p. 704).
% 
\subsection{Examples}
\begin{enumerate}
\item Bernice's infant is babbling. She has been trying to get him to say, ``mama.'' While playing with him she happens to catch him making the ``mmm'' sound. She smiles and praises him for making the vocalization. Over the next several days she continues to applaud when he makes this sound.  After a few weeks she observes the baby making a ``ma'' noise.  She praises him more enthusiastically giving him tickles. Although she still continues to commend him for making the ``mmm'' sound, the social reinforcement delivered for saying ``ma'' is differentially delivered.  Some time later she catches him babbling ``ma ma ma.''  She praises him, saying, ``You said ‘mama','' giving him big hugs and kisses. Verbal praise and affection is almost exclusively delivered for saying ``ma ma ma'' now. Eventually the baby who continually hears him mother say ``mama'' (and not ``ma ma ma'') echoes his mother when she gives the verbal model.  She demonstrates the highest level of excitement for this vocalization and the baby continues to emit this response.  
\item Petunia is pet sitting for a friend.  On her way out the door the cat, Mr. Boots, escapes outside.  Petunia tries to call the feline back indoors, but every time she gets near him, Mr. Boots runs away.  Petunia has an idea.  She places a bowl of cat food outside.  Mr. Boots goes to the bowl but only when he thinks the coast is clear.  Over the next few days, she successively moves the bowl of food closer to the front door.  On the fourth day, Petunia puts the bowl just inside of the doorway.  Mr. Boots takes the bait.  While he gobbles down the food, Petunia, who had been hiding nearby shuts the door and captures the beloved cat.  
\item (Non-example) Bernice's baby has gotten bigger.  While looking at a picture book she points out a picture of a farm animal.  She tells him that this is a cow and that the cow says, ``moo.'' The baby immediately echoes the word ``moo'' and Bernice praises him.  He continues to say ``moo'' when seeing pictures of cows in other books as well.
\end{enumerate}
% 
\subsection{Assessment}
\begin{enumerate}
\item Ask your Supervisee to identify the steps taken to shape the desired behavior of saying ``mama'' above. 
\item Ask the supervisee to identify how differential reinforcement is used to shape desired behavior.  
\item Ask your supervisee to create another example and non-example of his/her own. 
\end{enumerate}
%
\subsection{Relevant Literature}
\begin{refsection}
\nocite{cooper2007applied,
    lovaas1977autistic,
    newman2009reasonable,
    pryor1999don,
    ricciardi2006shaping,
    skinner1979shaping}
\printbibliography[heading=none]
\end{refsection}
%
\subsection{Related Tasks}
\fourdOne{}\\
\fourdFive{}\\
\fourdTwentyOne{}\\
\fourFKFourtyOne{}\\
%
\clearpage \section[\fourjSix{}]{\fourjSix{}%
              \sectionmark{J-06 Select... environments.}}
\sectionmark{J-06 Select... environments.}
\subsection{Definition}
``Achieving optimal generalized outcomes requires thoughtful, systematic planning. This planning begins with two major steps: (1) selecting target behaviors that will meet natural contingencies of reinforcement, and (2) specifying all desired variations of the target behavior and the settings/situations in which it should (and should not) occur after instruction has ended'' (Cooper, Heron, \& Heward, 2007, p. 623). In other words, an intervention must be selected that will allow the client to access reinforcement in a specific environment. If that is not possible, then alternative interventions should be explored. 

Ayllon and Azrin (1968) state that an important rule of thumb is to choose interventions that will help produce reinforcement after the intervention is discontinued. The intervention should support the student until they can access naturally existing contingencies (i.e., verbal praise from a teacher) and then more intensive, contrived contingencies should be systematically faded. The goal of most intervention programs is to teach a skill and then fade support so the client can implement that skill across settings. 

Cooper, Heron, \& Heward (2007, p. 626) identify 5 strategic approaches to promote generalized behavior change.
\begin{enumerate}
\item Teach the full range of relevant stimulus conditions and response requirements (i.e., teaching sufficient stimulus and response examples based on the setting)
\item Make the instructional setting similar to the generalization setting. (i.e., program common stimuli and teach loosely)
\item Maximize the target behavior's contact with reinforcement in the generalization setting. (i.e., ask people in the generalization setting to reinforce the target behavior, teach the learner to recruit reinforcement, and teach the target behavior to levels of performance required by natural existing contingencies of reinforcement.)
\item Mediate generalization (i.e., teach self-management skills \& contrive mediating stimulus)
\item Train to generalize. (i.e., reinforce response variability and instruct learner to generalize)
\end{enumerate}
%
\subsection{Examples}
\begin{enumerate}
\item George set up a token economy for Bill that systematically increased the number of responses needed to earn a token. After some time, Bill was earning tokens for completing an entire worksheet rather than earning a token for each question answered. This allowed Bill to independently complete worksheets in his general education classroom without a paraprofessional by his side giving him tokens after each answer. 
%
\end{enumerate}
%
\subsection{Assessment}
\begin{enumerate}
\item Give supervisee a reinforcement program. Have him/her create a fading procedure for this program to increase the number of responses required to earn a token. 
\item Have supervisee list the five strategies for promoting generalized behavior change and have him/her give examples of each.
\item Have supervisee describe and differentiate between contrived contingencies and naturally existing contingencies and give several examples of each.
%
\end{enumerate}
%
\subsection{Relevant Literature}
\begin{refsection}
\nocite{ayllon1968token,
        baer1999plan,
        cooper2007applied,
        snell2006instruction,
        stokes1977implicit,
        stokes1989operant}
\printbibliography[heading=none]
\end{refsection}
%
\subsection{Related Tasks}
\fourgEight{}\\
\fourjSeven{}\\
\fourjEight{}\\
\fourjEleven{}\\
\fourjTwelve{}\\
\fourkSeven{}\\
\fourkNine{}\\
%
\clearpage \section{\fourFKThirtyOne{}}
\subsection{Definition}
``The AB because of C formulation is a general statement that the relation between an event (B) and its context (A) is because of consequences (C)... Applied to Skinner's three-term contingency, the relation between (A) the setting and (B) behavior exists because of (C) consequences that occurred for previous AB (setting-behavior) relations. The idea [is] that reinforcement strengthens the setting-behavior relation rather than simply strengthening behavior'' (Moxley, 2004, p. 111).
``The three term contingency- antecedent, behavior, and consequence- is sometimes called the ABC's of behavior analysis... The term contingency has several meanings signifying various types of temporal and functional relations between behavior and antecedent and consequent variables... When a reinforcer (or punisher) is said to be contingent on a particular behavior, the behavior must be emitted for the consequence to occur'' (Cooper, Heron, \& Heward, 2007, pp. 41-42).
%
\subsection{Examples}
\begin{enumerate}
\item When John has not eaten in a while he asks his caregiver for a snack. When John asks he's given a snack. In this case the antecedents are food deprivation and the presence of someone who can provide food. The behavior would be the request for a snack and the consequence is being provided with a snack.
%
\end{enumerate}
%
\subsection{Assessment}
\begin{enumerate}
\item Have supervisee identify and describe the ABC three term contingency.
\item Have supervisee give specific examples of the ABC three term contingency.
\item Have supervisee identify and describe other principles and terms related to the three term contingency (i.e., motivating operations, setting events, establishing operations, discriminative stimulus, stimulus control, etc.)
\item Have supervisee state how the ABC three term contingency related to both punishment and reinforcement.
\end{enumerate}
%
\subsection{Relevant Literature}
\begin{refsection}
\nocite{azrin1966punishment,
        cooper2007applied,
        glenn1992revolutionary,
        michael2004concepts,
        moxley2004pragmatic,
        sulzer1977applying,
        vollmer2002punishment,
        vollmer1991establishing}
\printbibliography[heading=none]
\end{refsection}
%
\subsection{Related Tasks}
\fourbOne{}\\
\foureOne{}\\
\fourgFour{}\\
\fouriOne{}\\
\fouriTwo{}\\
\fourFKTen{}\\
\fourFKEleven{}\\
\fourFKFifteen{}\\
\fourFKTwentyOne{}\\
\fourFKTwentySeven{}\\
\fourFKThirty{}\\
\fourFKThirtyThree{}\\
\fourFKThirtyFour{}\\
\fourFKThirtyFive{}\\
\fourFKFourtyOne{}\\
