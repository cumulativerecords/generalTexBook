%\clearpage \section{\fourFKTwentyThree{}}
\subsection{Definition}
Automatic reinforcement - ``Reinforcement that occurs independent of the social mediation of others'' (Cooper Heron, \& Heward, 2007, p. 267).\\

Automatic punishment - ``Punishment that occurs independent of the social mediation by others''(Cooper Heron, \& Heward, 2007, p. 534).
%
%\clearpage \section{\fourFKTwentyThree{}}
\subsection{Definition}
Automatic reinforcement - ``Reinforcement that occurs independent of the social mediation of others'' (Cooper Heron, \& Heward, 2007, p. 267).\\

Automatic punishment - ``Punishment that occurs independent of the social mediation by others''(Cooper Heron, \& Heward, 2007, p. 534).
%
%\clearpage \section{\fourFKTwentyThree{}}
\subsection{Definition}
Automatic reinforcement - ``Reinforcement that occurs independent of the social mediation of others'' (Cooper Heron, \& Heward, 2007, p. 267).\\

Automatic punishment - ``Punishment that occurs independent of the social mediation by others''(Cooper Heron, \& Heward, 2007, p. 534).
%
%\clearpage \section{\fourFKTwentyThree{}}
\subsection{Definition}
Automatic reinforcement - ``Reinforcement that occurs independent of the social mediation of others'' (Cooper Heron, \& Heward, 2007, p. 267).\\

Automatic punishment - ``Punishment that occurs independent of the social mediation by others''(Cooper Heron, \& Heward, 2007, p. 534).
%
%\clearpage \section{\fourdFive{}}
\subsection{Definition}
Shaping – ``Using differential reinforcement to produce a series of gradually changing response classes; each response class is a successive approximation toward a terminal behavior.  Members of an existent response class are selected for differential reinforcement because they more closely resemble the terminal behavior'' (Cooper, Heron, \& Heward, 2007, p. 704).
% 
%\clearpage \section[\fourjSix{}]{\fourjSix{}%
  %            \sectionmark{J-06 Select... environments.}}
%\sectionmark{J-06 Select... environments.}
\subsection{Definition}
``Achieving optimal generalized outcomes requires thoughtful, systematic planning. This planning begins with two major steps: (1) selecting target behaviors that will meet natural contingencies of reinforcement, and (2) specifying all desired variations of the target behavior and the settings/situations in which it should (and should not) occur after instruction has ended'' (Cooper, Heron, \& Heward, 2007, p. 623). In other words, an intervention must be selected that will allow the client to access reinforcement in a specific environment. If that is not possible, then alternative interventions should be explored. 

Ayllon and Azrin (1968) state that an important rule of thumb is to choose interventions that will help produce reinforcement after the intervention is discontinued. The intervention should support the student until they can access naturally existing contingencies (i.e., verbal praise from a teacher) and then more intensive, contrived contingencies should be systematically faded. The goal of most intervention programs is to teach a skill and then fade support so the client can implement that skill across settings. 

Cooper, Heron, \& Heward (2007, p. 626) identify 5 strategic approaches to promote generalized behavior change.
\begin{enumerate}
\item Teach the full range of relevant stimulus conditions and response requirements (i.e., teaching sufficient stimulus and response examples based on the setting)
\item Make the instructional setting similar to the generalization setting. (i.e., program common stimuli and teach loosely)
\item Maximize the target behavior's contact with reinforcement in the generalization setting. (i.e., ask people in the generalization setting to reinforce the target behavior, teach the learner to recruit reinforcement, and teach the target behavior to levels of performance required by natural existing contingencies of reinforcement.)
\item Mediate generalization (i.e., teach self-management skills \& contrive mediating stimulus)
\item Train to generalize. (i.e., reinforce response variability and instruct learner to generalize)
\end{enumerate}
%
%\clearpage \section{\fourFKThirtyOne{}}
\subsection{Definition}
``The AB because of C formulation is a general statement that the relation between an event (B) and its context (A) is because of consequences (C)... Applied to Skinner's three-term contingency, the relation between (A) the setting and (B) behavior exists because of (C) consequences that occurred for previous AB (setting-behavior) relations. The idea [is] that reinforcement strengthens the setting-behavior relation rather than simply strengthening behavior'' (Moxley, 2004, p. 111).
``The three term contingency- antecedent, behavior, and consequence- is sometimes called the ABC's of behavior analysis... The term contingency has several meanings signifying various types of temporal and functional relations between behavior and antecedent and consequent variables... When a reinforcer (or punisher) is said to be contingent on a particular behavior, the behavior must be emitted for the consequence to occur'' (Cooper, Heron, \& Heward, 2007, pp. 41-42).
%
