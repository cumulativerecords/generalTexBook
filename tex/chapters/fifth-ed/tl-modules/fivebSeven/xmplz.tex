\subsection{Examples}  
\begin{enumerate}
\item (Automatic Reinforcement) Scratching an insect bite removes an itch; eating food when hungry removes hunger, humming may be auditory reinforcement; nonfunctional movements such as hand flapping may produce a sensation, which is automatically reinforcing; some self-injurious behavior may produce a sensation, which the individual may enjoy.
\item (Automatic Punishment) Albert bites his canker sore, causing a shocking pain.  Albert is becomes cautious as he eats with his canker sore until the canker disappears.  A dog gets a thorn in his paw.  He experiences pain when he steps down on his foot.  He begins to walk on three legs. 
\end{enumerate}
%
\begin{enumerate}
\item Bernice's infant is babbling. She has been trying to get him to say, ``mama.'' While playing with him she happens to catch him making the ``mmm'' sound. She smiles and praises him for making the vocalization. Over the next several days she continues to applaud when he makes this sound.  After a few weeks she observes the baby making a ``ma'' noise.  She praises him more enthusiastically giving him tickles. Although she still continues to commend him for making the ``mmm'' sound, the social reinforcement delivered for saying ``ma'' is differentially delivered.  Some time later she catches him babbling ``ma ma ma.''  She praises him, saying, ``You said ‘mama','' giving him big hugs and kisses. Verbal praise and affection is almost exclusively delivered for saying ``ma ma ma'' now. Eventually the baby who continually hears him mother say ``mama'' (and not ``ma ma ma'') echoes his mother when she gives the verbal model.  She demonstrates the highest level of excitement for this vocalization and the baby continues to emit this response.  
\item Petunia is pet sitting for a friend.  On her way out the door the cat, Mr. Boots, escapes outside.  Petunia tries to call the feline back indoors, but every time she gets near him, Mr. Boots runs away.  Petunia has an idea.  She places a bowl of cat food outside.  Mr. Boots goes to the bowl but only when he thinks the coast is clear.  Over the next few days, she successively moves the bowl of food closer to the front door.  On the fourth day, Petunia puts the bowl just inside of the doorway.  Mr. Boots takes the bait.  While he gobbles down the food, Petunia, who had been hiding nearby shuts the door and captures the beloved cat.  
\item (Non-example) Bernice's baby has gotten bigger.  While looking at a picture book she points out a picture of a farm animal.  She tells him that this is a cow and that the cow says, ``moo.'' The baby immediately echoes the word ``moo'' and Bernice praises him.  He continues to say ``moo'' when seeing pictures of cows in other books as well.
\end{enumerate}
% 
\begin{enumerate}
\item George set up a token economy for Bill that systematically increased the number of responses needed to earn a token. After some time, Bill was earning tokens for completing an entire worksheet rather than earning a token for each question answered. This allowed Bill to independently complete worksheets in his general education classroom without a paraprofessional by his side giving him tokens after each answer. 
\end{enumerate}
%
\subsection{Examples}
\begin{enumerate}
\item When John has not eaten in a while he asks his caregiver for a snack. When John asks he's given a snack. In this case the antecedents are food deprivation and the presence of someone who can provide food. The behavior would be the request for a snack and the consequence is being provided with a snack.
%
\end{enumerate}
%
