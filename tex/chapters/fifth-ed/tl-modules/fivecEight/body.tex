\clearpage \section[\fouraNine{}]{\fouraNine{}%
              \sectionmark{A-09 Evaluate the acc...}}
\sectionmark{A-09 Evaluate the acc...}
Evaluating the accuracy and reliability of measurement procedures involves ``measuring the measurement system'' (Cooper, Heron, \& Heward, 2007, p. 110). As human error is the biggest threat to the accuracy and reliability of data, measurements must be evaluated to determine trustworthiness.  Accuracy of measurement is determined when the observed values equal the true values.  Establishing a true value requires the use of a different measurement procedure than the one used to record the observed value.  This often makes it difficult to determine a true value for many of the behaviors of interest.  Measures of reliability should be used when a true value cannot be established.  Reliability of measurement is determined when the same value is given across repeated measures of the same event, thus reliability reflects consistency. 
%
\subsection{Examples}
\begin{enumerate}
\item Accuracy: You and a friend decide to go on a 5-mile run.  Your friend tells you that she can monitor the distance because her legs always start to hurt once she runs 5 miles.  You, being a data-driven behavior analyst, decide that your friend's measurement procedure might not be the most accurate so you use your smart phone app to track the distance.  Your measurement system will likely reveal a better estimate of the true value of the distance you ran.
\item Reliability: Using the example for accuracy, both measures can be reliable if at the end of the run your friend tells you that you must have run 5 miles because her legs hurt and your app indicates you ran 5 miles.   
\end{enumerate}
%
\subsection{Assessment}
\begin{enumerate}
\item Ask supervisee to evaluate the accuracy and reliability of a measurement procedure that is being usedmm with a client.
\item Ask supervisee to provide definitions and examples of accuracy and reliability.
\end{enumerate}
%
\subsection{Relevant Literature}
\begin{refsection}
\nocite{cooper2007applied,johnston1993strategies}
\printbibliography[heading=none]
\end{refsection}
%
\subsection{Related Lessons}
\fouraOne{}\\
\fouraTwo{}\\
\fouraThree{}\\
\fouraFour{}\\
\fouraFive{}\\
\fouraSix{}\\
\fouraSeven{}\\
\fouraEight{}\\
\fourhOne{}\\
\fourhTwo{}\\
%
\clearpage \section{\fouraEight{}}
\subsection{Definition} 
Interobserver agreement (IOA) –  ``...refers to the degree to which two or more independent observers report the same observed values after measuring the same event'' (Cooper, Heron, \& Heward, 2007, p. 113).\\

``By reporting the results of IOA assessments, researchers enable consumers to judge the relative believability of the data as trustworthy and deserving interpretation'' (Cooper et al., 2007, p. 114). It, however, should be noted that the ``...fact that two observers reported the same measure of the target behavior for a session says nothing about the accuracy or reliability of either'' (Johnston \& Pennypacker, 2009, p. 149).

There are four basic types of IOA as described by Johnston and Pennypacker (2009).
\begin{enumerate}
\item Total agreement –Usually used for recording dimensional quantaties (e.g., count, duration, or latency)\\  Formula: Smaller total ÷ Larger total X 100 = Percentage of Agreement.
\item Exact agreement – Used as attempt to make sure that observers are recording the same responses. Exact agreement is often more demanding for recorders.\\
Formula: Total agreements ÷ Total number of intervals X 100 = Percentage of Agreement.
\item Interval agreement – Used for finding agreement when collecting data using interval recording or time sampling methods.\\
Formula: Total agreements ÷ Total number of intervals X 100 = Percentage of Agreement.
\item Occurrence/nonoccurrence agreement – Conservative approach for collecting agreement when interval recording or time sampling methods. Both occurances and nonoccurrances are scored separately.\\
Formula: Total agreements ÷ Total number of intervals X 100 = Percentage of Agreement.
\end{enumerate}
%
\subsection{Examples}
\begin{enumerate}
\item John is conducting a functional analysis (FA) on aggression in one of his students. He has asked Mary to observe the behavior and record data simultaneously with him to calculate Interobserver agreement. He plans on conducting a 5-minute session with 30, 10-second intervals. He plans to use interval agreement IOA.  He records 4 instances of aggression during the FA in the 21st, 22nd, 23rd and 24th interval. Mary records 4 instances in the 21st, 22nd, 23rd, and 25th interval. He calculates IOA to be 97\% or 29/30 intervals.
\item (Non-example) John is conducting a functional analysis on aggression in one of his students. He asks Mary to come in and observe but does not provide her with a data recording sheet to take data on the behavior. At the end of the session he asks Mary if she saw any aggression during the session. 
\end{enumerate}
%
\subsection{Assessment}
\begin{enumerate}
\item Have supervisee watch a video of client exhibiting target behavior. Ask him/her to record frequency data on the behavior. Record data on the same video and compare data after completion. 
\item Provide supervisee with 2 sets of data sheets based on the observance of the same behavior. Have him/her calculate Interobserver agreement based on the data provided.
\item Assign supervisee recommended article on Interobserver agreement. Have him/her summarize the article and share this with peers 
\end{enumerate}
%
\subsection{Relevant Literature}
\begin{refsection}
\nocite{cooper2007applied,boyce2000evaluation,johnston2010strategies,watkins2000interobserver}
\printbibliography[heading=none]
\end{refsection}
%
\subsection{Related Lessons}
\fouraOne{}\\
\fouraNine{}\\
\fourbTwo{}\\
\fourgSix{}\\
\fourhTwo{}\\
\fouriOne{}\\
\fouriFive{}\\
\fourjNine{}\\
\fourkFive{}\\
