%%\clearpage \section[\fouraNine{}]{\fouraNine{}%
              \sectionmark{A-09 Evaluate the acc...}}
\sectionmark{A-09 Evaluate the acc...}
Evaluating the accuracy and reliability of measurement procedures involves ``measuring the measurement system'' (Cooper, Heron, \& Heward, 2007, p. 110). As human error is the biggest threat to the accuracy and reliability of data, measurements must be evaluated to determine trustworthiness.  Accuracy of measurement is determined when the observed values equal the true values.  Establishing a true value requires the use of a different measurement procedure than the one used to record the observed value.  This often makes it difficult to determine a true value for many of the behaviors of interest.  Measures of reliability should be used when a true value cannot be established.  Reliability of measurement is determined when the same value is given across repeated measures of the same event, thus reliability reflects consistency. 
%
%\clearpage \section{\fouraEight{}}
\subsection{Definition} 
Interobserver agreement (IOA) –  ``...refers to the degree to which two or more independent observers report the same observed values after measuring the same event'' (Cooper, Heron, \& Heward, 2007, p. 113).\\

``By reporting the results of IOA assessments, researchers enable consumers to judge the relative believability of the data as trustworthy and deserving interpretation'' (Cooper et al., 2007, p. 114). It, however, should be noted that the ``...fact that two observers reported the same measure of the target behavior for a session says nothing about the accuracy or reliability of either'' (Johnston \& Pennypacker, 2009, p. 149).

There are four basic types of IOA as described by Johnston and Pennypacker (2009).
\begin{enumerate}
\item Total agreement –Usually used for recording dimensional quantaties (e.g., count, duration, or latency)\\  Formula: Smaller total ÷ Larger total X 100 = Percentage of Agreement.
\item Exact agreement – Used as attempt to make sure that observers are recording the same responses. Exact agreement is often more demanding for recorders.\\
Formula: Total agreements ÷ Total number of intervals X 100 = Percentage of Agreement.
\item Interval agreement – Used for finding agreement when collecting data using interval recording or time sampling methods.\\
Formula: Total agreements ÷ Total number of intervals X 100 = Percentage of Agreement.
\item Occurrence/nonoccurrence agreement – Conservative approach for collecting agreement when interval recording or time sampling methods. Both occurances and nonoccurrances are scored separately.\\
Formula: Total agreements ÷ Total number of intervals X 100 = Percentage of Agreement.
\end{enumerate}
%
