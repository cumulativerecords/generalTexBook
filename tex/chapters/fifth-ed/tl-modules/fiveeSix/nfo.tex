%\clearpage \section[\fourjNine{}]{\fourjNine{}%
\subsection{Definition}
The general goal of behavioral research is ``to demonstrate that measured changes in the target behavior occur because of experimentally manipulated changes in the environment'' (Cooper, 2007, p. 160). Without a controlled research design, practitioners cannot claim a causal relation between intervention and behavior change. Practical and ethical concerns limit practitioner's use of experimental designs in most settings. Typical risks associated with controlled research include the need to delay treatment while collecting baseline data or to withdraw interventions that are successful. Each research design has specific risks to participants associated with its application. In applied practice, meaningful, socially valid, and lasting change is the goal. Clients, parents, staff, and teachers prefer the most efficient and effective path toward treatment goals. For such reasons, practitioners seek to show evidence of a correlation rather than a causal relation between changes in a client's behavior and an intervention by providing comparison of patterns of baseline (A) and intervention (B) responding over time. The AB design has poor experimental control but strong practical and ethical value in natural settings.
%
