\clearpage \section[\fourjNine{}]{\fourjNine{}%
              \sectionmark{J-09 Identify and address practical...}}
\sectionmark{J-09 Identify and address practical...}
\subsection{Definition}
The general goal of behavioral research is ``to demonstrate that measured changes in the target behavior occur because of experimentally manipulated changes in the environment'' (Cooper, 2007, p. 160). Without a controlled research design, practitioners cannot claim a causal relation between intervention and behavior change. Practical and ethical concerns limit practitioner's use of experimental designs in most settings. Typical risks associated with controlled research include the need to delay treatment while collecting baseline data or to withdraw interventions that are successful. Each research design has specific risks to participants associated with its application. In applied practice, meaningful, socially valid, and lasting change is the goal. Clients, parents, staff, and teachers prefer the most efficient and effective path toward treatment goals. For such reasons, practitioners seek to show evidence of a correlation rather than a causal relation between changes in a client's behavior and an intervention by providing comparison of patterns of baseline (A) and intervention (B) responding over time. The AB design has poor experimental control but strong practical and ethical value in natural settings.
%
\subsection{Examples}
\begin{enumerate}
\item A researcher-practitioner designed a multiple baseline study for a woman who hit and scratched herself. The researcher's review of the first three days of baseline data showed that occurrences of the behavior were highly variable without an obvious pattern of responding. The researcher concluded that further delay of treatment that might decrease a dangerous behavior was not ethical. The researcher knew that the strength of the results of his research would be threatened if he began treatment before he had a clear pattern in baseline responding, but his responsibility to his client, and those close to her, was his primary concern. He regretted not anticipating this possibility by choosing a research design that could have demonstrated experimental control without depending on highly stable baseline responding.
\item A student's teacher did not want the student to participate in a graduate supervisee's study because the supervisee planned to work with the student during classroom reading instruction. Even though the research study was designed to provide a benefit to the student by increasing sight-word reading, the boy's parent refused to sign permission for his participation. The supervisee realized that social validity was threatened by his original plan and he would have to arrange to work with the student after school or exclude him from the study.
\end{enumerate}
%
\subsection{Assessment}
\begin{enumerate}
\item Ask the supervisee to explain what baseline logic is and why it is important for showing causal (experimentally controlled research) or correlational (change over time) relations between an intervention and an individual's behavior change. 
\item A graduate student supervisee was planning to conduct single subject research using a withdrawal (ABAB) design. The supervisee wanted to test for a causal relation between a gel-filled wedge pillow and the fidgety, out-of-seat behavior of a middle school student in his class. His supervisor warned that removing an effective treatment might have long-lasting results on the student's performance. What design might the supervisee recommend that would show repetitive positive effects of the gel-filled pillow intervention without requiring the supervisee to withdraw a beneficial treatment? Ask the supervisee to demonstrate his decision by drawing a rough line graph showing both designs with the gel-filled pillow as the intervention and explain why one design is a better choice ethically while still meeting practical goals for the supervisee and student. 
\end{enumerate}
%
\subsection{Relevant Literature}
\begin{refsection}
\nocite{cooper2007applied,
        ledford2009single}
\printbibliography[heading=none]
\end{refsection}
%
\subsection{Related Lessons}
\fourbThree{}\\
\fourgEight{}\\
\fourhThree{}\\
\fourjTwo{}\\
%06 Select intervention strategies based on supporting environments.
\fourjSeven{}\\
\fourjEight{}\\
\fourjTen{}\\
\fourkTwo{}\\
\fourkSeven{}\\
% Check Compliance Code 3.01 on functional analysis and 9.02 on exp.
