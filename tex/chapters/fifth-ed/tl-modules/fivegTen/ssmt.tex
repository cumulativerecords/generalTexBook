\subsection{Assessment}
\begin{enumerate}
\item Have supervisee create a mock lesson displaying match-to-sample task. He/she must create a match-to-sample program, a data sheet to record responses, and task materials needed to complete the task. Have him/her role-play this task scenario with supervisor. Supervisor will play the role of the client and supervisee will play the role of the teacher or therapist.
\item Have supervisee create a lesson that demonstrates stimulus equivalence. He/she must start with match-to-sample task and then use other topographies of sample stimulus to display stimulus equivalence. 
\item Have supervisee describe the match-to-sample procedure and stimulus equivalence. Have him/her discuss how match-to-sample procedures can be implemented to test for stimulus equivalence. 
\end{enumerate}
%
\subsection{Assessment}
\begin{enumerate}
\item Ask the supervisee to explain the concept of stimulus equivalence.
\item Ask the supervisee to name the three tests that demonstrate the basic principles of stimulus equivalence and describe each of these.
\item Ask the supervisee to give examples of some new concepts that might be taught through stimulus equivalence
\end{enumerate}
%
\subsection{Assessment}
\begin{enumerate}
\item Ask supervisee to provide a definition of derived relations.
\item Ask supervisee to provide examples of derived relations.
\item Supervisor should ensure that supervisee thoroughly understands stimulus equivalence, matching-to-sample procedures, and conditional discriminations, in order to arrange instructional procedures to promote generative learning.  Once this foundational learning has occurred, supervisor can ask supervisee to demonstrate teaching procedures that will facilitate generative learning.
\end{enumerate}
%
