\clearpage \section{\foureThirteen{}}
\subsection{Definition} 
Matching-to-sample - ``A procedure for investigating conditional relations and stimulus equivalence. A matching-to-sample trial begins with the participant making a response that presents or reveals the sample stimulus; next, the sample stimulus may or may not be removed, and two or more comparison stimuli are presented. The participant then selects one the comparison stimuli. Responses that select a comparison stimulus that matches the sample stimulus are reinforced, and no reinforcement is provided for responses selecting the nonmatching comparison stimuli'' (Cooper, Heron, \& Heward, 2007, p. 699).

\subsection{Examples}
\begin{enumerate}
\item A teacher presents a student with a picture of an apple. The teacher then lays out three other picture cards away from the original picture of an apple. One picture card depicts an apple, the second picture card depicts a banana, and the final picture card depicts an orange. The teacher holds up the initial picture card depicting an apple and states, ``match.''  The student takes the picture of an apple and places it on top of the corresponding picture of an apple. The teacher says, ``Great job'' and gives the student a high five.
\end{enumerate}
%
\subsection{Assessment}
\begin{enumerate}
\item Have supervisee create a mock lesson displaying match-to-sample task. He/she must create a match-to-sample program, a data sheet to record responses, and task materials needed to complete the task. Have him/her role-play this task scenario with supervisor. Supervisor will play the role of the client and supervisee will play the role of the teacher or therapist.
\item Have supervisee create a lesson that demonstrates stimulus equivalence. He/she must start with match-to-sample task and then use other topographies of sample stimulus to display stimulus equivalence. 
\item Have supervisee describe the match-to-sample procedure and stimulus equivalence. Have him/her discuss how match-to-sample procedures can be implemented to test for stimulus equivalence. 
\end{enumerate}
%
\subsection{Relevant Literature}
\begin{refsection}
\nocite{cooper2007applied,
        cumming1965complex,
        fields2010varieties,
        sidman1986matching}
\printbibliography[heading=none]
\end{refsection}
%
\subsection{Related Lessons}
\fourdThree{}\\
\fourdEight{}\\
\foureTwo{}\\
\foureSix{}\\
\foureTwelve{}\\
%
\clearpage \section{\foureSix{}}
\subsection{Definition}
In the field of applied behavior analysis, a number of procedures have been used to teach new concepts.  One of these procedures is known as stimulus equivalence.  In 1971, Murray Sidman discovered that a previously untaught, unreinforced stimuli could come under stimulus control through its pairing with other stimuli which were explicitly taught (Sidman, 1971). This concept revolutionized the field as it demonstrated a new way of teaching that could potentially reduce the amount of time needed to teach a new class of stimuli.   ``Behavior analysts define stimulus equivalence by testing stimulus-stimulus relations.  A positive demonstration of all three behavioral tests (i.e. reflexivity, symmetry, and transitivity) is necessary to meet the definition of an equivalence relation among a set of arbitrary stimuli'' (Cooper, Heron, \& Heward, 2007, p. 398). 
\begin{enumerate}
\item Reflexivity describes the action of selecting a stimulus that is matched to itself in the absence of training and reinforcement (A=A).  For instance an individual is shown three pictures; a penny, a nickel, and a dime.  When given an identical picture of a penny, he matches it to the identical picture of a penny in the array (Sidman, 1994).
\item Symmetry describes the reversibility of the sample stimulus and a comparison stimulus (A=B and B=A).  For instance an individual who is taught to select the picture of a penny (out of an array of 3), when the word penny is given, would also be able to choose the comparison spoken word penny shown the picture of the penny without being previously taught this correlation (Sidman, 1994).  
\item Transitivity is the most crucial test for demonstrating stimulus equivalence.  A third, untrained relation emerges as a result of being taught the first two relations.  (A=C and C=A) ``...emerges as a product of training two other stimulus-stimulus relations'' (Cooper, Heron, \& Heward, 2007, p. 399).  
\end{enumerate}

The following equation demonstrates the basic principals of stimulus equivalence:
\begin{enumerate}
\item If A = B, and
\item B = C, then
\item  A = C (Sidman and Tailby, 1982).
\end{enumerate}

When using stimulus equivalence, decide what relations are to be taught (i.e. spoken word to picture, picture to written word, drawing to real-life picture, etc.).  Decide which order the conditional relations are to be taught.  Teach the relations A=B and B=C to mastery criteria.  Once mastery criteria are met for the first two relations, test for reflexivity, symmetry, and transitivity using the same criteria.  If the participant demonstrates these relations without having been previously been taught them, they will have acquired the third relation C=A that demonstrates the most important test for stimulus equivalence.   

\subsection{Assessment}
\begin{enumerate}
\item Ask the supervisee to explain the concept of stimulus equivalence.
\item Ask the supervisee to name the three tests that demonstrate the basic principles of stimulus equivalence and describe each of these.
\item Ask the supervisee to give examples of some new concepts that might be taught through stimulus equivalence
\end{enumerate}
%
\subsection{Relevant Literature}
\begin{refsection}
\nocite{cooper2007applied,
        sidman1971reading,
        sidman1994equivalence,
        sidman1982conditional}
\printbibliography[heading=none]
\end{refsection}
%
\subsection{Related Lessons}
\foureSix{}\\
\foureThirteen{}\\
\fourFKEleven{}\\
\fourFKTwelve{}\\
\fourFKThirteen{}\\
\fourFKTwentyFour{}\\
\fourFKTwentyEight{}\\
\fourFKThirtyFive{}\\
%
\clearpage \section[\fourjFourteen{}]{\fourjFourteen{}%
              \sectionmark{J-14 Arrange instructional proc...}}
\sectionmark{J-14 Arrange instructional proc...}
\subsection{Definition}
Generative learning involves applying learning to novel contexts without being explicitly taught and is related to language and cognition. Deriving relations is based on the stimulus equivalence paradigm and procedures (Sidman, 1971). A small number of taught relations among stimuli may generate numerous derived relations (Wulfert \& Hayes, 1988). Readers are encouraged to understand stimulus equivalence prior to arranging instructional procedures to promote generative learning. Experimental procedures often utilize matching-to-sample tasks or a computerized program called Implicit Relational Assessment Procedure (IRAP) to teach derived relations. It has been extensively studied in children with autism (Kilroe, Murphy, Barnes-Holmes, \& Barnes-Holmes, 2014). The general instructional procedure involves providing explicit reinforcement for a series of conditional discriminations, after which untrained relations (i.e., derived relations) will emerge and can be subsequently reinforced. 

Stimulus equivalence is one of several empirically supported examples of derived relations, as relations can be derived based on opposition, temporality, analogy, comparison, and distinction (Stewart, McElwee, \& Ming, 2013). Relational Frame Theory (RFT) was developed as a behavior analytic account of human language and cognition (Hayes, Barnes-Holmes, \& Roche, 2001) and addresses the need for a theoretical explanation for generative learning, Resources for learning RFT are included in the relevant literature section.  

\subsection{Examples}
\begin{enumerate}
\item  A learner is taught that spoken word ``apple'' = picture of an apple = written word ``apple'' = picture of an apple. These two relations are directly taught. However, through this explicit training, the learner can derive that the spoken word ``apple'' = written word ``apple.'' In this example, reinforcement should occur for successful matching of the two trained relations.  Once the third relation is derived, the response should be reinforced. 
\end{enumerate}
%
\subsection{Assessment}
\begin{enumerate}
\item Ask supervisee to provide a definition of derived relations.
\item Ask supervisee to provide examples of derived relations.
\item Supervisor should ensure that supervisee thoroughly understands stimulus equivalence, matching-to-sample procedures, and conditional discriminations, in order to arrange instructional procedures to promote generative learning.  Once this foundational learning has occurred, supervisor can ask supervisee to demonstrate teaching procedures that will facilitate generative learning.
\end{enumerate}
%
\subsection{Relevant Literature}
\begin{refsection}
\nocite{hayes2001relational,
        sidman1994equivalence,
        stewart2013language,
        torneke2010learning,
        wulfert1988transfer}
\printbibliography[heading=none]
\end{refsection}
%
\subsection{Related Lessons}
\foureSix{}\\
\fourFKEleven{}\\
\fourFKTwentyFour{}\\
\fourFKThirtyFour{}\\
