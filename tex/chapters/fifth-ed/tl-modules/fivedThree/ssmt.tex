\subsection{Assessment}
\begin{enumerate}
\item  Ask the supervisee to either describe a time that they used of a withdrawal or reversal design or have them describe a hypothetical experiment using a withdrawal or reversal design. 
\item Have supervisee look at the figures in the articles (such as those listed below) as well as other articles and determine which ones are reversal/withdrawal designs. 
\item Have the supervisee look at figures in the articles below and describe what characteristics make it a reversal or withdrawal design
\item Have the supervisee describe the pros of using a reversal design and the condition in which the use of a reversal design would not be desirable
\end{enumerate}
%
\subsection{Assessment}
\begin{enumerate}
\item Ask the supervisee to either describe an alternating treatment design they have used in the past or have them describe an alternating treatments design.
\item Have supervisee look at various figures  from the articles (such as some of those below) as well as articles that did not use an alternating treatment design and have them determine which figures depict the use of an alternating treatment design. 
\item Have the supervisee look at figures in the articles below and describe what characteristics make it an alternating treatment design
\item Have the supervisee describe the pros of using an alternating treatment design and the condition in which the use of an alternating treatment design would not be desirable
\end{enumerate}
%
\subsection{Assessment}
\begin{enumerate}
\item Have supervisee describe the changing criterion design and state when it may be most appropriate, strengths of this design, as well as limitations of the changing criterion design.
\item Have supervisee create a hypothetical analysis using the changing criterion design. Have him/her state why the changing criterion design was the most effective design to display experimental control.
\item Have supervisee label the parts of a completed changing criterion design graph and describe how the graph displays experimental control. 
\end{enumerate}
%
\subsection{Assessment}
\begin{enumerate}
\item Have supervisee describe the multiple baseline design and state when it may be most appropriate, strengths of this design, as well as limitations of the multiple baseline design.
\item Have supervisee create a hypothetical analysis using the multiple baseline design. Have him/her state why the multiple baseline design was the most effective design to display experimental control.
\item Have supervisee label the parts of a completed multiple baseline design graph and describe how the graph displays experimental control. 
\end{enumerate}
%
\subsection{Assessment}
\begin{enumerate}
\item Ask the supervisee to explain the meaning of ``baseline logic'' and the reason it enables measurement of experimental control.
\item Ask the supervisee why MPD across behaviors requires similar, but functionally different behaviors in each tier of a MPD.
\item Give the supervisee three articles in which researchers based their conclusions on MPD line graph data. Ask the supervisee to interpret results of the study based on the graphic data. Compare to the conclusions written by study authors.
\end{enumerate}
%
\subsection{Assessment}
\begin{enumerate}
\item Have your supervisee design an experiment that would be best suited to use a combination of design elements.
\item Ask your supervisee to point to the sections of the graphs in Colon et al. (2012) and Alexander (1985) that reflect the types of experimental designs used. 
\item Ask your supervisee to describe what Sidman meant when he said that having rules for designing an experiment would be ``disastrous.''
\end{enumerate}
%
