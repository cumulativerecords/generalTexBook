%\clearpage \section[\fourkEight{}]{\fourkEight{}%
              \sectionmark{K-08 Establish support...}}
\sectionmark{K-08 Establish support...}
\subsection{Definition}
As a behavior analyst, it is important to conduct yourself with professionalism to your clients and their family and also to individuals from other disciplines that may support your client.  You must enlist a circle of care around your client, which will help you to better understand your client through the lenses of another discipline.  For example if you have a child who has feeding refusal, taking the time to enlist information from the nutritionist, speech pathologist, doctors and occupational therapists will help you to provide safe and effective treatment.

In addition, when providing treatment having help from people within the client's family, the community, and from the circle of care around your client will enable you to generalize skills to new environments, new people and new activities.  These individuals can be a great asset in troubleshooting and providing additional data as well.

Additionally, when your goals are met, you will leave.  In order for the continued support of your client and the maintenance of the goals, it is important to enlist individuals who will be with the client for a longer period.

Bailey and Burch (2010) provide a useful book on professional strategies. He provides several suggestions on how to establish a working relationship in the best interest of the client.  Establishing yourself as a positive reinforcer for your colleagues by demonstrating integrity, basic meeting etiquette, providing a professional image, using nontechnical language, listening to others, and bringing their opinions into the assessment. 
%
\subsection{Assessment}
\begin{enumerate}
\item Ask supervisee to describe the importance of enlisting the help of direct and indirect consumers.
\item Use behavior skill tracking to teach and practice describing interventions in nontechnical terms.
\item Discuss challenges in dealing with colleagues within different fields and strategies to help.
\end{enumerate}
%
\subsection{Relevant Literature}
\begin{refsection}
\nocite{bailey201025,
        bailey2013ethics,
        bac2014professional}
\printbibliography[heading=none]
\end{refsection} 
%                         
\subsection{Related Lessons}
\fourgSeven{}\\
\fourkNine{}\\
\fourkTen{}\\
%
%\clearpage \section[\fourkNine{}]{\fourkNine{}%
              \sectionmark{K-09 Secure the support...}}
\sectionmark{K-09 Secure the support...}
\subsection{Definition}
Foxx (1996, p. 230) stated that in programming successful behavior change interventions, ``10\% is knowing what to do; 90\% is getting people to do it... Many programs are unsuccessful because these percentages have been reversed'' (Cooper et al., 2007, p. 652). 

Being explicit yet simplistic in describing programs and protocols will help secure support from other individuals in a client's environment. If a behavior change procedure or program is too difficult, technical, or places unreasonable demands on the other individuals involved, they are less likely to implement these programs. In addition, adequate training of behavior procedures should be provided to ensure proper implementation by those interacting with the client in the natural environment. Specifically, training pertaining to the delivery of reinforcers, which maintain the individual's newly acquired behavioral repertoires.

Jarmolowicz et al. (2008) compared the effectiveness of conversational language instructions and technical language instructions when explaining how to implement a treatment to caregivers. They found that the caregivers that were given conversational language instruction implemented the treatment more accurately.
%
\subsection{Examples}
\begin{enumerate}
\item Richard is trying to generalize skills learned in the special education classroom for one of his students. He went to each teacher to explain how this will help the student in their class and answered any questions they may have about the programs. In addition, he conducted a training on the specific program and offered to consult with each teacher in order to make sure generalization was successful and the repertoire was maintained.
%
\end{enumerate}
%
\subsection{Assessment}
\begin{enumerate}
\item Have supervisee identify ways they can build a rapport with other service providers and help support them when they have a student who needs to generalize and maintain skills in new settings.
\item Have supervisee choose a particular behavior change program or strategy. Have him/her describe and explain this program to an individual who does not have a background in applied behavior analysis. 
\item Have supervisee choose a specific behavior change program. Have him/her practice explaining the benefits of this program to others in order to get them on board with implementing this program.
\item Give supervisee a complex behavior change program. Have him/her simplify this program and create guidelines and instructions that they could give to an individual who does not have knowledge of applied behavior analytic strategies and techniques. 
\end{enumerate}
%
\subsection{Relevant Literature}
\begin{refsection}
\nocite{cooper2007applied,
        jarmolowicz2008effects,
        stokes1974programming}
\printbibliography[heading=none]
\end{refsection}
%
\subsection{Related Lessons}
\fourhOne{}\\
\fourjOne{}\\
\fourkThree{}\\
\fourkFour{}\\
\fourkSix{}\\
\fourkEight{}\\
%
%\clearpage \section{\fourkSix{}}
Smith, Parker, Taubman, and Lovaas (1992) found that knowledge transfer from a staff training workshop did not generalize to the group home.  Research suggests that training is most effective if there is training and ongoing supervision in the environment where the behavior change program is occurring (Parsons \& Reed, 1995; Sarokoff \& Sturmey, 2004; Miles \& Wilder, 2009).

Codding, Feinberg, Dunn and Pace (2004) found that treatment integrity increased following a one hour performance feedback session every other week.  Social validity ratings provided favorable feedback for the frequent supervisions. 

Cooper, Heron and Heward (2007) suggest that supervision of data takers and booster training is necessary to avoid observer drift. 

Observing and graphing data will provide immediate feedback on the participant's performance.  This can lead to quick decisions, modifications if necessary, or the termination of ineffective programs.  This supervision is necessary to create the most effective interventions and troubleshoot areas for improvement. It is also important to observe the intervention in the natural environment in order to determine if the intervention is realistic and practical in the natural environment.

Summary of Rationale for Supervision of Behavior Change Agents
\begin{enumerate}
\item Provides effective knowledge transfer for individuals implementing the intervention
\item Increases motivation and treatment integrity
\item Reduces error associated with data collection
\item Allows for quick clinical decisions to modify or terminate programs which ensures the most effective treatment
\end{enumerate}
%
\subsection{Assessment}
\begin{enumerate}
\item Ask supervisee to describe why it is important to provide supervision of behavior change agents.
\item Look at data with supervisee and ask them to discuss what modification they may consider
\item Have supervisee observe intervention in the home environment and ask them to write down suggestions to increase the effectiveness of the plan.
%
\end{enumerate}
%
\subsection{Relevant Literature}
\begin{refsection}
\nocite{codding2005effects,
        cooper2007applied,
        miles2009effects,
        parsons1995training,
        sarokoff2004effects,
        smith1992transfer}
\printbibliography[heading=none]
\end{refsection} 
% 
\subsection{Related Lessons}
\fourhThree{}\\
\fourhFour{}\\
\fourkTwo{}\\
\fourkThree{}\\
\fourkFour{}\\
