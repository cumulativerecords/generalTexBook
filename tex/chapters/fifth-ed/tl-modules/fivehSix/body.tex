% This is a weird task list item. Part client-facing, part supervivision, overlaps with ethics perhaps. Weird.
\clearpage \section[\fourkFive{}]{\fourkFive{}%
              \sectionmark{K-05 Design and use systems...}}
\sectionmark{K-05 Design and use systems...}
\subsection{Definition}
Procedural Integrity - ``The extent to which the independent variable is applied exactly as planned and described and no other unplanned variables are administered inadvertently along with the planned treatment'' (Cooper Heron, \& Heward, 2007, pp. 706-707).

Treatment Drift - ``An undesirable situation in which the independent variable of an experiment is applied differently during later stages than it was at the outset of the study'' (Cooper Heron, \& Heward, 2007, p. 706).

Low treatment integrity is not only bad for research (confounding, cannot interpret the results) but can also lead to inconsistencies and poor outcomes in treatments.  These can be related to many factors such as experimenter bias (unfair advantage to see positive results), staff implementing only procedures they favor, treatment too difficult to implement, poor staff training, or staff turnover.

Systems to Avoid Treatment Drift
\begin{enumerate}
\item Precise operational definition
\item Make behavioral plan simple and easy to administer
\item Provide competency-based training (use behavior skills training)
\item Assess treatment integrity
\end{enumerate}

Assessing Treatment Integrity
\begin{enumerate}
\item Provide a brief checklist of each of the components in the treatment plan
\item May be self-monitored but important to conduct inter-rater reliability
\item Establish a schedule to complete the treatment integrity
\item Graph the percent of treatment integrity and monitor to ensure treatment drift does not occur
\end{enumerate}
%
\subsection{Assessment}
\begin{enumerate}
\item Ask supervisee to describe why procedural integrity is important.
\item Ask supervisee to create a procedural integrity checklist for a behavioral program.
\item Ask supervisee to collect data using the procedural integrity checklist in the natural environment and take inter-rater reliability measurement with yourself, discuss discrepancies, repeat if necessary.
\end{enumerate}
%
\subsection{Relevant Literature}
\begin{refsection}
\nocite{cooper2007applied,
        johnston2010strategies,
        wolery1994procedural}
\printbibliography[heading=none]
\end{refsection} 
%                         
\subsection{Related Tasks}
\fourfOne{}\\
\fourhThree{}\\
\fourhFour{}\\
\fourkThree{}\\
\fourkFour{}\\
%
\subsection{Footnotes}
Also called procedural fidelity, treatment integrity, procedural reliability, or treatment adherence.
%
\clearpage \section{\fourkSeven{}}
\subsection{Definition}
In the field of applied behavior analysis, it is crucial to have ongoing evaluation of the effectiveness of behavior programs.  This helps to ensure that the most effective treatments are being offered to a client based on ethical practices, the most current research, and the individual's needs.  

Birnbrauer (1999) lists the following steps for evaluating the effectiveness of treatment:
\begin{enumerate}
\item Describe the exact purposes of the treatment –what is it intended to achieve? 
\item Describe exactly how the treatment is conducted –there should be no mystery or secrecy about the methods and procedures being used. 
\item Describe how treatment effects were measured –what numerical data were collected and how were they collected? 
\item Show before and after data collected by independent, unbiased evaluators
\item Show follow up data –do the persons maintain gains? Do they continue to improve? Do they regress? 
\end{enumerate}

``The data obtained throughout a behavior change program or a research study are the means for that contract; they form the empirical basis for every important decision: to continue with the present procedure, to try a different intervention, or to reinstitute a previous condition'' (Cooper, Heron, \& Heward, 2007, p. 167).  It is important that if there is evidence of behavioral regression or that the treatment package is ineffective, that the team re-evaluate, make changes or adjustments, or discontinue a behavioral program entirely.  It is unethical to continue a behavioral program that is deemed ineffective. 
%
\subsection{Assessment}
\begin{enumerate}
\item Ask the supervisee to state why ongoing evaluation of a behavioral program is important.
\item Ask the supervisee to state what all important program decisions should be driven by.
\item Ask the supervisee to state who should be evaluating data.
\item Ask the supervisee under which conditions to discontinue a behavioral program.
%
\end{enumerate}
%
\subsection{Relevant Literature}
\begin{refsection}
\nocite{bailey2013ethics,
        birnbrauer1999how,
        cooper2007applied}
\printbibliography[heading=none]
\end{refsection}
%
\subsection{Related Tasks}
\fourbOne{}\\
\fourFKFourtySeven{}\\
\fourgOne{}\\
\fourhOne{}\\
\fouriOne{}\\
\fouriFive{}\\
\fouriSix{}\\
\fourjOne{}\\
\fourkSeven{}\\
