%\clearpage \section[\fourgFour{}]{\fourgFour{}%
              \sectionmark{G-04 Explain behavioral concepts...}}
\sectionmark{G-04 Explain behavioral concepts...}
\subsection{Definition}
It is important for a behavior analyst to have a strong verbal repertoire when speaking about the science of behavior analysis. Jargon used in journals, universities, and with other behavior analysts is valuable to promote effective action on the part of the listener with precise discriminative control. 

That being said, Skinner wrote that we should choose words for the effects they have on the listener (Skinner, 1957) and unless that listener has extensive training in behavior analysis, our use of technical language will ``fall on deaf ears'' and not produce effective action. When speaking with client family members, friends, or professionals from other closely related fields, it is important to remember that your verbal behavior is for the benefit of your audience. Bailey (1991) describes this phenomenon well:

``In our zeal to be scientific, we have stressed the need to match the requirements of science in our writing and publishing. Although this has given us much-needed academic credibility (faculty can be promoted and tenured by publishing in JABA) it doesn't help at all in selling our technology to the masses'' (p. 446)

During the supervision process, spend considerable amounts of time, working on precise definitions for technical terms. This benefits the supervisee in several ways. It allows him/her to behave effectively as a listener and speaker when interacting with other behavior analysts in the field. It will also promote accurate translations to nontechnical language. If a precise definition is practiced, a less technical, more layperson-friendly definition will be easier to describe and it will be more likely to be accurate.
%
\subsection{Assessment}
\begin{enumerate}
\item Ask Supervisee to give a precise definition of a behavioral term from a textbook. Then ask Supervisee to accurately describe the term in nontechnical language. Work on this for the most commonly used terms. Provide feedback after, when in a supervision meeting.
\item Observe the Supervisee describing a behavior analytic concept to another person (client, colleague, etc.) The Supervisee should be able to answer basic questions related to the topic using nontechnical language. Provide feedback after, when in a supervision meeting.
\end{enumerate}
%
\subsection{Relevant Literature}
\begin{refsection}
\nocite{bailey1991marketing,
        lindsley1991technical,
        malott1992should}
\printbibliography[heading=none]
\end{refsection}
\subsection{Related Lessons}
\fourgSix{}\\
\fouriSix{}\\
\fourjSix{}\\
\fourjSeven{}\\
\fourkOne{}\\
\fourkThree{}\\
\fourkEight{}\\
\fourkNine{}\\
%
%\clearpage \section[\fourgFive{}]{\fourgFive{}%
              \sectionmark{G-05 Describe and explain behavior...}}
\sectionmark{G-05 Describe and explain behavior...}
\subsection{Definition}
Behavior analysts must have a strong verbal repertoire when speaking about behavior analysis.  This includes using behavior-analytic language when describing and explaining behavior, including private events.  

Skinner's radical behaviorism rejected psychological models of behavior that relied on mentalistic explanations. Mentalistic approaches attributed the origination and cause of behavior to ``inner'' dimensions or mental entities (i.e., hypothetical constructs and explanatory fictions such as the unconscious or psyche).  Mentalistic explanations of behavior often neglect the consideration and analysis of controlling variables in the environment and use circular reasoning to explain the cause and effect of behavior.  Understanding the philosophy of radical behaviorism and the principles of behavior can assist behavior analysts in explaining behavior in behavior-analytic terms.

For example, you are conducting a functional behavior assessment in a school setting for a student who engages in high rates of aggression in the classroom.  The teacher tells you that the student's aggression occurs because the student is frustrated and lives in an unpleasant environment at home.  This is a mentalistic explanation.  After several observations, you have determined that when academic demands are placed, the student engages in aggression and their aggression is reinforced by escape (i.e., academic demands are removed). This is a behavior-analytic explanation of behavior that accounts for behavior as it is a function of environmental variables.  

Using behavior-analytic language to explain and describe behavior can be difficult as we are often exposed to mentalistic explanations (e.g., ``wanting to'' or ``felt like it'' as causes of behavior).  Read Malott and Trojan-Suarez (2004) for a discussion about circular reasoning and talking about behavior.

\subsection{Assessment}
\begin{enumerate}
\item Ask supervisee to explain and describe behavior in behavior-analytic terms.
\item Provide supervisee with examples of mentalistic explanations of behavior and ask supervisee to provide behavior-analytic explanations.
\item Observe supervisee explain behavior in behavior-analytic terms to a colleague or client.
%
\end{enumerate}
%
\subsection{Relevant Literature}
\begin{refsection}
\nocite{cooper2007applied,
        malott2008principles,
        moore2008conceptual}
\printbibliography[heading=none]
\end{refsection}
%
\subsection{Related Lessons}
\fourFKSeven{}\\
\fourFKEight{}\\
\fourFKThirtyOne{}\\
\fourFKThirtyThree{}\\
\fourgFour{}\\
\fouriOne{}\\
\fouriTwo{}\\
