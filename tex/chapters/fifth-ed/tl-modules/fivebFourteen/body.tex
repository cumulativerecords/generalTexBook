%%\clearpage \section[\fourdNine{}]{\fourdNine{}%
              \sectionmark{D-09 Use the verbal operants...}}
\subsection{Definition}
In the field of applied behavior analysis extensive research has been done on the development of verbal behavior.  

``Verbal behavior involves social interactions between speakers and listeners, whereby speakers gain access to reinforcement and control their environment through the behavior of listeners'' (Sundberg as cited in Cooper, Heron, \& Heward, 2007, p. 529). Verbal operants are the basic units of this exchange.  

In 1957 B.F. Skinner identified six elementary verbal operants in his book on Verbal Behavior.  These included mands, tacts, intraverbals, echoics, textuals, and transcription.  ``Skinner's analysis suggests that a complete verbal repertoire is composed of each of the different elementary operants, and separate speaker and listener repertoires'' (Sundberg as cited in Cooper, et al., 2007, p. 541).   

Since Skinner described these operants, those in the field have applied these concepts to both language assessment and training.  In order to evaluate whether or not specific language training is necessary, a variety of standardized tools have been used to test an individual's receptive and expressive language abilities.  These include but are not limited to: the Peabody Picture Vocabulary Test III (Dunn \& Dunn, 1997), the Comprehensive Receptive and Expressive Vocabulary Test (Hammill \& Newcomer, 1997), the Assessment of Basic Language and Learning Skills (ABLLS) (Partington \& Sundberg, 1998), the Verbal Behavior Milestones Assessment and Placement Program (VB-MAPP) and the CELF-4 Semel, Wiig, \& Secord, 2003).  

Not all of these tests will identify deficits in one or more of the verbal operants. Some children who may be proficient in tacting (such as labeling things in their environment such as letters and numbers) may fail to make appropriate mands for desired items (Cooper, et al., 2007).  In this case it is important for behavior analysts to use a combination of approaches or less standardized methods to assess these needs.  It may be helpful to observe the individual in their natural environment and take data on their verbal interactions.  It will be important to ask questions such as:
\begin{enumerate}
\item What is the frequency of and complexity of mands?
\item What is the frequency and complexity of tacting behavior?
\item Will the child or individual demonstrate echoic behavior when prompted?
\item Does the child or individual engage in intraverbal behavior with known caregivers?
\item Can or will the child or individual read words that are written down for him?
\item Can or will the child or individual write words that are said to him? 
\end{enumerate}
%
\subsection{Assessment}
\begin{enumerate}
\item Ask the supervisee to name the basic unit of language
\item Ask the supervisee to name all 6 of the elementary verbal operants
\item Ask the supervisee to name some of the standardized tests often used to assess language
\item Ask the supervisee to explain why these standardized tests may not provide adequate information
\item Ask the supervisee to describe how one might assess an individual's use of verbal operants if testing fails to yield enough information.
\end{enumerate}
%
\subsection{Relevant Literature}
\begin{refsection}
\nocite{cooper2007applied,
    partington1998assessment,
    semel2003clinical,
    skinner1957verbal,
    sundberg2008verbal,
    sundberg1998teaching}
\printbibliography[heading=none]
\end{refsection}
%cannot verify reference.
%Hammill, D., \& Newcomer, P.L. (1997).  Test of language development-3.  Austin, TX: Pro-Ed.
%
\subsection{Related Lessons}
\fourdTen{}\\
\fourdEleven{}\\
\fourdTwelve{}\\
\fourdThirteen{}\\
\fourdFourteen{}\\
\fourFKFourtyThree{}\\
\fourFKFourtyFour{}\\
\fourFKFourtyFive{}\\
\fourFKFourtySix{}\\
%
%\clearpage \section{\fourFKFourtyThree{}}
\subsection{Definition}  
Echoic - ``An elementary verbal operant involving a response that is evoked by a verbal discriminative stimulus that has point-to-point correspondence and formal similarity with the response'' (Cooper, Heron, \& Heward, 2007, p. 694).   
%
\subsection{Examples}
\begin{enumerate}
\item Mrs. Platypus is instructing her 3rd grade class on their math facts.  She holds up a card stating that, ``three times nine is eighteen.''  She then restates the fact asking the class to repeat.  The class says, ``three times nine is eighteen'' in unison.  Mrs. Platypus praises the students for their repetition.
\item Mr. Penguin is a kindergarten teacher.  He is working with one student on his reading skills.  He shows little Timmy the letter R.  He tells him that the letter R makes the ``rrr'' sound and asks him to repeat.  Little Timmy says, ``rrr,'' and Mr. Penguin comments, ``Nice job Timmy.''  
\item Mrs. Dodo the art teacher needs one of her students to run to the office and get some supplies. One of the children volunteers.  She tells him that she needs him to get, ``Crayons, markers, and paint.''  He repeats, ``Crayons, markers, and paint.''  ``Exactly,'' Mrs. Dodo says sending him on his way.  
\item (Non-example) Mrs. Platypus is still working on math facts with her class.  She holds up the math fact 4x9= and asks the students to give the answer.  Susie Q raises her hand and answers ``thirty-six.''  
%
\end{enumerate}
%
\subsection{Assessment}
\begin{enumerate}
\item Ask the supervisee to give the definition of an echoic
\item Ask the supervisee to give several examples of echoics
\item Ask the supervisee to give a non-example of an echoic 
%
\end{enumerate}
%
\subsection{Relevant Literature}
\begin{refsection}
\nocite{cooper2014applied,
        sundberg2008verbal,
        skinner1957verbal}
\printbibliography[heading=none]
\end{refsection} 
%
\subsection{Related Lessons}
\fourdFour{}\\
\fourdTen{}\\
%
%\clearpage \section{\fourFKFourtyFour{}}
\subsection{Definition}
Mand - ``An elementary verbal operant that is evoked by an MO and followed by specific reinforcement'' (Cooper, Heron, \& Heward, 2007, p. 699).\\

The form of the response is specific and under control of motivating operations. Response topography can vary: vocal, sign language, augmentative communication, pushing, reaching, hitting, etc.
%
\subsection{Examples}
\begin{enumerate}
\item ``I want a cookie.'' (This is a mand for an item. Mands can include verbs, use of adjectives, prepositions, pronouns etc.)
\item A child says ``watch me'' after learning how to ride a bike independently (mand for attention)
\item Asking questions like ``what's your name? or ``where's the phone?'' (mand for information)
\item Child says, ``No!'' when parent is about to use blender (mand for avoidance of an aversive)
%

%
\end{enumerate}
%
\subsection{Assessment}
\begin{enumerate}
\item Ask your Supervisee to recall how they asked for supervision
\item Ask your Supervisee to list the types of mands they would emit if they were lost in a foreign county and needed directions to a local gas station
\item Ask you supervisee to list 5 ways they use mands in an inappropriate way (eg. complain about work to get attention)
%
\end{enumerate}
%
\subsection{Relevant Literature}
\begin{refsection}
\nocite{cooper2007applied,
        laraway2003motivating,
        michael1988establishing,
        sundberg2001benefits,
        sweeney2007transferring}
\printbibliography[heading=none]
\end{refsection}
%
\subsection{Related Lessons}
\fourdNine{}\\
\fourdEleven{}\\
\fourFKTwentySeven{}\\
\fourFKTwentyEight{}\\
%
%\clearpage \section{\fourFKFourtyFive{}}
\subsection{Definition}
Tact – ``An elementary verbal operant evoked by a nonverbal discriminative stimulus and followed by generalized conditioned reinforcement'' (Cooper, Heron, \& Heward, 2007, p. 705).
%
\subsection{Examples}
\begin{enumerate}
\item Dexter walks outside with his mother and sees birds in a tree.  ``Robins,'' he says.  ``You're right. Those are robins,'' Dexter's mom says. ``Robins'' is a tact.
\item Hector is in the store shopping for Valentines Day.  He sees a variety of flowers before noticing the ones he wants to buy.  ``Red roses,'' Hector says quietly to himself. ``Red roses'' in this context is likely a tact. 
\item Chester goes to his friends Superbowl party.  Upon scanning the array of delicious apps and snacks on the counter, he hones in on one that is his favorite.  ``Ooh, buffalo chicken dip,'' he comments.  ``Buffalo chicken dip'' would be likely a tact in this context.
\item Non-example: Dexter is thinking about buying some cookies the next time he goes to the supermarket.  He writes the word ``cookies'' down on his shopping list. 
%
\end{enumerate}
%
\subsection{Assessment}
\begin{enumerate}
\item Ask the supervisee to define ``tact.''  
\item Ask the supervisee to give several examples of tacts.
\item Ask the supervisee to give a non-example of a tact. Discuss why.
%
\end{enumerate}
%
\subsection{Relevant Literature}
\begin{refsection}
\nocite{cooper2007applied,
        skinner1957verbal}
\printbibliography[heading=none]
\end{refsection}
%
\subsection{Related Lessons}
\fourdTwelve{}\\
%
%\clearpage \section{\fourFKFourtySix{}}
\subsection{Definition} 
Intraverbal – ``An elementary verbal operant that is evoked by a verbal discriminative stimulus and that does not have point-to-point correspondence with that verbal stimulus'' (Cooper, Heron \& Heward, 2007, p. 698).
%
\subsection{Examples}
\begin{enumerate}
\item A new employee shows up for his first day on the job. The man in the cubical next to him asks, ``What is your name?''  ``Harvey,'' the man replies. Saying ``Harvey'' is an intraverbal in that context.
\item Hanks boss stops his office to let him know that his sales were ``outstanding this week.''  ``Thanks. I really put in some long hours,'' Hank notes.  ``Thanks,'' is an intraverbal in that context.
\item (Non-example) The office phone rings. Harvey picks up the phone and answers ``Hello.'' There is no one on the line so he hangs up and keeps working. 
%
\end{enumerate}
%
\subsection{Assessment}
\begin{enumerate}
\item Ask the supervisee to define ``intraverbal''  
\item Ask the supervisee to give several examples of intraverbals.
\item Ask the supervisee to give a non-example of an intraverbal.
%
\end{enumerate}
%
\subsection{Relevant Literature}
\begin{refsection}
\nocite{cooper2007applied,
        partington1993teaching,
        skinner1957verbal}
\printbibliography[heading=none]
\end{refsection}
%
\subsection{Related Lessons}
\fourdThirteen{}\\
