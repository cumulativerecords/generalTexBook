%\clearpage \section[\fouriOne{}]{\fouriOne{}%
              \sectionmark{I-01 Define behavior...}}
\sectionmark{I-01 Define behavior...}
\subsection{Definition}
The importance of defining behavior in observable and measurable terms:\\
As Baer, Wolf and Risley said in 1968, ``since the behavior of an individual is composed of physical events, its scientific study requires their precise measurement'' (p. 93). In order to be scientific in our study of behavior, we must be very clear about what behavior it is we are actually studying. Therefore, the target behavior must be observable and measurable. Cooper, Heron and Heward (2007) also make the point that one of the most basic tenets of science is replication. In order for other scientists to replicate an experiment or study, the definition of the behavior under investigation and how it was measured must be transparent enough, that future replication is possible.

Technically-sound written definitions of target behaviors:\\
\begin{enumerate}
\item Cooper, Heron, and Heward, (2007), suggest that a good behavioral definition is \textit{operational}, allowing the practitioner to obtain complete information about a behavior's occurrence/non-occurrence. Operational definitions allow accurate application of the procedures.
\item Cooper, Heron and Heward (2007) also state that good definitions increase the likelihood that an accurate evaluation of the effectiveness of a study or experiment will be conducted.
\end{enumerate}

Two types of target behavior definitions:\\
\begin{enumerate}
\item Cooper, Heron, and Heward (2007, p. 65) suggest that there are two types of target behavior definitions:
\item Functional (These types of definition label responses as part of the target behavior's response class if they have the same effect upon the environment.)
\item Topographical (These types of definition look at the shape or form of the target behavior.)
\end{enumerate}

How to write behavioral definitions
\begin{enumerate}
\item Cooper, Heron, and Heward (2007) cite Hawkins and Dobes (1977) as giving three characteristics of good written target behavior definitions:
\item Objective (should refer only to observable characteristics of the behavior and environment and shouldn't utilize inferential terms, such as ``feeling angry.'')
\item Clear (the definition should be readable and unambiguous.)
\item Complete (it should outline the boundaries of what is included as an instance of a response and what is not included.)
\end{enumerate}
%
%\clearpage \section[\fouriTwo{}]{\fouriTwo{}%
              \sectionmark{I-02 Define environmental var...}}
\sectionmark{I-02 Define environmental var...}
\subsection{Definition}
This task relates to the importance of defining environmental variables in observable and measurable terms.
\begin{enumerate}
\item As Cooper, Heron, and Heward (2007) state, in order to achieve a high level of treatment integrity in an experiment, it is of utmost importance to ``develop complete and precise operational definitions of the treatment procedures'' (Cooper, Heron, \& Heward, 2007, p. 235). In the same way that it is critical to define target behavior in observable and measurable terms, so is the case with defining environmental variables. 
\item Baer, Wolf, and Risley (1968) stress that the ``technological'' dimension of Applied Behavior Analysis refers simply to the fact that ``the techniques making up a particular behavioral application are completely identified and described''(Baer, Wolf, \& Risley, 1968, p. 95). As such, the techniques, or environmental variables being manipulated, must be defined in observable and measurable terms to meet the technological dimension of applied behavior analysis (Cooper, Heron, \& Heward, 2007). 
\item However, historically, operationally defining independent variables has not been conducted to the standard required for a science of behavior that seeks to achieve the technological dimension of applied behavior analysis*. It has also not been done to the same standard as that of the dependent variables (Johnston \& Pennypacker, 1980; Peterson, Homer \& Wonderlich, 1982; Gresham, Gansle \& Noell, 1993). In 1982 Peterson, Homer and Wonderlich called for researchers to measure the independent variables in a more stringent manner. Unfortunately, an assessment of this area later on by Gresham, Gansle and Noell (1993) found that this had not been accomplished.
\item Defining environmental variables in observable and measurable terms
\item It is believed that environmental variable definitions should be written to meet the same standards as those required to be met by target behavior definitions (Gresham, Gansle \& Noell, 1993). They should be ``clear, concise, unambiguous, and objective'' (Cooper, Heron \& Heward, 2007, p. 235).
\item Gresham et al. (1993, p. 261) suggest that independent variable definitions can be made along four dimensions: spatial, verbal, physical and temporal. 
\end{enumerate}

Footnotes\\
*See Baer, Wolf \& Risley (1968) for more information on the seven dimensions of Applied Behavior Analysis.
