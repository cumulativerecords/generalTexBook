\subsection{Examples}
Gresham et al. (1993, pp. 261-262) give an example of an adequate definition of an independent variable, a time-out procedure, provided by Mace, Page, Ivancic and O'Brien (1986).
\begin{enumerate}
\item Immediately following the occurrence of a target behavior (temporal dimension), (b) the therapist said, ``No, go to time-out'' (verbal dimension), (c) led the child by the arm to a prepositioned time-out chair (physical dimension), and (d) seated the child facing the corner (spatial dimension). (e) If the child's buttocks were raised from the time-out chair or if the child's head was turned more than 45° (spatial dimension), the therapist used the least amount of force necessary to guide compliance with the time-out procedure (physical dimension). (f) At the end of 2 min (temporal dimension), the therapist turned the time-out chair 45° from the corner (physical and spatial dimensions) and walked away (physical dimension).
\item Gresham et al., (1993, p. 262) argued that a failure to define operational variables along these four dimensions, as done so by Mace, Page, Ivancic and O'Brien (1986), makes ``replication and external validation of behavior-analytic investigations difficult.''
\end{enumerate}
%
