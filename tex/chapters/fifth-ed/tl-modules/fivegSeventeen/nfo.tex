%\clearpage \section[\fourfTwo{}]{\fourfTwo{}%
              \sectionmark{F-02 Use token economies...}}
\sectionmark{F-02 Use token economies...}
\subsection{Definition}
Conditioned Reinforcer - ``A stimulus change that functions as a reinforcer because of prior pairing with one or more other reinforcers'' (Cooper, Heron, \& Heward, 2007, p. 692).

``Token economies are used as a method of strengthening a behavior, or increasing its frequency, because the tokens are a way of ‘paying' children for completing tasks and the children can then use these tokens to buy desired activities or items'' (Miltenberger, 2008, p. 513).

A token economy uses a conditioned reinforcer, or token, as currency for a student to exchange for a backup reinforcer (i.e., tangible items, edibles, activities, etc.) based on earning a certain amount of tokens for desired target behaviors. 

The strength of the token is derived from its being paired with other reinforcers (also referred to as a backup reinforcer). If the backup reinforcer loses value due to satiation, the token will subsequently lose its effectiveness.

Miltenberger (2008) listed seven components that need to be defined before implementing a token economy. These include: identifying the desired target behavior to be strengthened, identifying tokens to be used as conditioned reinforcement, identifying backup reinforcers, outlining a reinforcement schedule for token delivery, identifying the amount of tokens needed to exchange for reinforcers, identifying the time and place to exchange tokens, and identifying if a response cost contingency would be necessary for the individual. 

