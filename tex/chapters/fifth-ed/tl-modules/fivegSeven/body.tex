%\clearpage \section{\fourdFive{}}
\subsection{Definition}
Shaping – ``Using differential reinforcement to produce a series of gradually changing response classes; each response class is a successive approximation toward a terminal behavior.  Members of an existent response class are selected for differential reinforcement because they more closely resemble the terminal behavior'' (Cooper, Heron, \& Heward, 2007, p. 704).
% 
\subsection{Examples}
\begin{enumerate}
\item Bernice's infant is babbling. She has been trying to get him to say, ``mama.'' While playing with him she happens to catch him making the ``mmm'' sound. She smiles and praises him for making the vocalization. Over the next several days she continues to applaud when he makes this sound.  After a few weeks she observes the baby making a ``ma'' noise.  She praises him more enthusiastically giving him tickles. Although she still continues to commend him for making the ``mmm'' sound, the social reinforcement delivered for saying ``ma'' is differentially delivered.  Some time later she catches him babbling ``ma ma ma.''  She praises him, saying, ``You said ‘mama','' giving him big hugs and kisses. Verbal praise and affection is almost exclusively delivered for saying ``ma ma ma'' now. Eventually the baby who continually hears him mother say ``mama'' (and not ``ma ma ma'') echoes his mother when she gives the verbal model.  She demonstrates the highest level of excitement for this vocalization and the baby continues to emit this response.  
\item Petunia is pet sitting for a friend.  On her way out the door the cat, Mr. Boots, escapes outside.  Petunia tries to call the feline back indoors, but every time she gets near him, Mr. Boots runs away.  Petunia has an idea.  She places a bowl of cat food outside.  Mr. Boots goes to the bowl but only when he thinks the coast is clear.  Over the next few days, she successively moves the bowl of food closer to the front door.  On the fourth day, Petunia puts the bowl just inside of the doorway.  Mr. Boots takes the bait.  While he gobbles down the food, Petunia, who had been hiding nearby shuts the door and captures the beloved cat.  
\item (Non-example) Bernice's baby has gotten bigger.  While looking at a picture book she points out a picture of a farm animal.  She tells him that this is a cow and that the cow says, ``moo.'' The baby immediately echoes the word ``moo'' and Bernice praises him.  He continues to say ``moo'' when seeing pictures of cows in other books as well.
\end{enumerate}
% 
\subsection{Assessment}
\begin{enumerate}
\item Ask your Supervisee to identify the steps taken to shape the desired behavior of saying ``mama'' above. 
\item Ask the supervisee to identify how differential reinforcement is used to shape desired behavior.  
\item Ask your supervisee to create another example and non-example of his/her own. 
\end{enumerate}
%
\subsection{Relevant Literature}
\begin{refsection}
\nocite{cooper2007applied,
    lovaas1977autistic,
    newman2009reasonable,
    pryor1999don,
    ricciardi2006shaping,
    skinner1979shaping}
\printbibliography[heading=none]
\end{refsection}
%
\subsection{Related Lessons}
\fourdOne{}\\
\fourdFive{}\\
\fourdTwentyOne{}\\
\fourFKFourtyOne{}\\
