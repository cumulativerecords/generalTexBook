\clearpage \section[\fourgThree{}]{\fourgThree{}%
              \sectionmark{G-03 Conduct a preliminary as...}}
\sectionmark{G-03 Conduct a preliminary as...}
\subsection{Definition}
Behavioral assessment - ``A form of assessment that involves a full range of inquiry methods (observation, interview, testing, and the systematic manipulation of antecedent or consequent variables) to identify probable antecedent and consequent controlling variables. Behavioral assessment is designed to discover resources, assets, significant others, competing contingencies, maintenance and generality factors, and possible reinforcers and/or punishers that surround the potential target behavior'' (Cooper, Heron, \& Heward, 2007, p. 691).

Five Phases of a Behavioral Assessment 
Hawkins (1979) described behavioral assessment as being funnel shaped, beginning with a broad scope and then moving to a narrow focus.  
\begin{enumerate}
\item Screening and general disposition
\item Defining and quantifying problems or goals
\item  Pinpointing the target behavior
\item Monitoring progress
\item Follow-up
\end{enumerate}

The preliminary assessment consists of the first 3 phases of this model.  It is the broad gathering of information needed in order to pinpoint the target behavior. Once the target behavior is selected, a formal functional behavioral assessment is required. 
% 
Preliminary Assessment\\
\begin{enumerate}
\item Interviews (client and significant others) 
\item Checklists  
\item Standardized Tests 
\item Direct Observations  
\end{enumerate}

Social Significance\\
Before selecting a target behavior, it is important to reflect on how important the behavior change is for the client, not to others around the client.  The rational for the behavior change must be critically analyzed.  Cooper Heron, and Heward (2007) suggest the following methods for determining the social significance of the target behavior: 
\begin{enumerate}
\item Is the behavior likely to produce reinforcement in the natural environment?
\item Is the skill useful?
\item Will it increase the individual's access to new reinforcing environments?
\item Will it allow more social interaction? 
\item Is it a pivotal behavior?  
\item Is it a behavior cusp?
\item Is it age-appropriate? 
\item If it is a behavior to be eliminated?
\item What is the replacement skill? 
\item Is the identified behavior actually problematic? 
\item Is this the identified behavior just reports or is it real? 
\end{enumerate}

Prioritizing Target Behaviors\\
If a number of target behaviors are selected which are socially significant, it is then important to prioritize the target behavior ensuring dangerous behavior is targeted first.  Other guidelines are listed by Cooper, Heron, and Heward (2007) as the following: 
\begin{enumerate}
\item Pose a danger to client or others? 
\item How often does it occur? 
\item How longstanding is the problem? 
\item Will changing the behavior produce higher rates of reinforcement? 
\item What is the importance related to overall independence? 
\item Will changing the behavior reduce negative attention? 
\item Will changing the behavior produce positive attention? 
\item How likely is success of changing the behavior? 
\item How much will it cost to change the behavior? 
\end{enumerate}
 
Defining the Target Behavior\\
Before beginning, the target behavior must be objectively and concisely defined in a clear concrete observable manner. 
 
Setting the Criteria for Behavior Change \\
Goals must be socially meaningful to the person's life.   
%
\subsection{Assessment}
\begin{enumerate}
\item Ask supervisee what assessment tools can be used to do a preliminary assessment. 
\item Provide examples of targets, which are and are not socially significant and ask the supervisee to determine if these behaviors are appropriate target behaviors.  Have supervisee explain why. 
\item Provide a list of five target responses and have the supervisee prioritize them, justifying their decisions using the guidelines provided by Cooper, Heron and Heward (2007).
\end{enumerate}
%
\subsection{Relevant Literature}
\begin{refsection}
\nocite{bailey2013ethics,
        cooper2007applied,
        hawkins1979functions,
        linehan1977issues,
        van1979social}
\printbibliography[heading=none]
\end{refsection}
%
\subsection{Related Lessons} 
\fouriOne{}\\
\fouriTwo{}\\
\fourjOne{}\\
