\clearpage \section{\fourjTwelve{}}
\subsection{Definition}
Response maintenance refers to ``the extent to which a learner continues to perform the target behavior after a portion or all of the intervention responsible for the behavior's initial appearance in the learner's repertoire has been terminated'' (Cooper, Heron \& Heward, 2007, p. 703). Rusch and Kazdin (1981) note that withdrawing or gradually fading components of an individual's treatment package can support response maintenance. 

Program for behavior learned in structured environments to be maintained with contingencies in the client's natural environment. Thin reinoforcement schedules so that the natural environment can support and continue to maintain similar rates in behavior (e.g., While learning to mand, a child may be given a chip every time she asks for a chip. However, as she becomes more adept with this goal, the schedule of reinforcement should move from a continuous schedule to an intermittent schedule because the child will not always be given a chip upon request in her natural environment.)

Response maintenance can often be confused with generalization across multiple exemplars. The key difference is that response maintenance is said to occur if the response can be maintained in settings and situations in which it was previously exhibited, after generalization to that setting and/or situation has already occurred at least once in the past.  For instance, if an individual was taught how to purchase items at a store and did so successfully at some point at Starbucks, McDonald's and Target but did not exhibit this skill at Starbucks a month later, a lack of response maintenance is said to occur. If, however, the individual did not exhibit the skill at Macy's where the individual has never performed the skill, a lack of generalization is said to occur (Cooper, Heron, \& Heward, 2007). 
%
\subsection{Assessment}
\begin{enumerate}
\item Have supervisee give examples of response maintenance. 
\item In a role-play or on the job, ask supervisee how he/she would program for response maintenance. 
%
\end{enumerate}
%
\subsection{Relevant Literature}
\begin{refsection}
\nocite{cooper2007applied,
        rusch1981toward}
\printbibliography[heading=none]
\end{refsection}
%
\subsection{Related Lessons}
\fourjSix{}\\
\fourjSeven{}\\
\fourjEight{}\\
\fourjEleven{}\\
