\clearpage \section[\fouriFive{}]{\fouriFive{}%
              \sectionmark{I-05 Organize, Analyze...}}
\sectionmark{I-05 Organize, Analyze...}
\clearpage \section{\fouriFive{}}
\subsection{Definition}
Organize the data
\begin{enumerate}
\item Once data have been collected in their raw format, it is then important to organize the data into a format that is easy to analyze (Cooper, Heron \& Heward, 2007). The most effective way to do this, and the most common method utilized by Behavior Analysts, is to visually display the data in a graph (Cooper, Heron, \& Heward, 2007).
\item As Parsonson and Baer (1978, p. 134) said ``the function of the graph is to communicate... in an attractive manner, descriptions and summaries of the data that enable rapid and accurate analysis of the facts'' (cited from Cooper, Heron \& Heward, 2007, p. 128).
\item The visual formats most often used by behavior analysts are line graphs, bar graphs, cumulative records, semilogarithmic charts and scatterplots* (Cooper, Heron \& Heward, 2007).
\end{enumerate}
%
Analyze and interpret observed data\\

Behavior analysts use a systematic form of assessing graphically displayed data called visual analysis (Cooper, Heron \& Heward, 2007). 

Visual analysis encompasses examining each of three characteristics in a graphic display of data, both within and across the different conditions and phases of an experiment. These three characteristics are:

\begin{enumerate}
\item The level of the data
\item The extent and type of variability in the data 
\item The trends in the data
\end{enumerate}

Johnston and Pennypacker (1993b, p. 320) recommend that the viewer should carefully examine the graph's overall construction, paying attention to details such as the axis labels and the scaling of each axis, prior to attempting to interpret the data. They argue ``it's impossible to interpret graphic data without being influenced by various characteristics of the graph itself'' (cited from Cooper, Heron \& Heward, 2007, p. 149).
%
Visual analysis within conditions\\
\begin{enumerate}
\item Within given conditions, examination needs to occur to determine a few relevant factors (Cooper, Heron \& Heward, 2007):
\item The number of data points in each condition (in general, the more measurements of the dependent variable there are per unit of time, the more confidence one can have in the data).
\item Variability (A high degree of variability usually indicates little control has been achieved over the factors influencing behavior).
\item Level (examined in terms of its absolute value within a condition, the degree of stability/variability and the extent of change from one level to another).
\item Trend (the trend indicates whether a particular behavior has increased, decreased or has neither increased nor decreased within a condition).
\end{enumerate}

Visual analysis between conditions\\
\begin{enumerate}
\item After examining the data within each condition or phase of a study, visual analysis now proceeds to examining the data between conditions (Cooper, Heron \& Heward, 2007):
\item Comparison needs to be made between the different conditions of the level, trend and variability of the data (Cooper, Heron \& Heward, 2007, p. 154).
\item The data are examined in terms of the overall level of performance between conditions; generally when there is no overlap of data points between the highest values in one condition and the lowest values in another condition, there is a strong likelihood that the behavior changed from one condition to the next (Cooper, Heron \& Heward, 2007, p. 154).
\end{enumerate}

Once an ``examination and comparison of changes in level, trend and variability between conditions has occurred, a comparison needs to be made of performance across similar conditions'' (Cooper, Heron \& Heward, 2007, p. 155). If a behavior change is found to have occurred over the course of an intervention, the next question to be asked is, ``was the behavior change a result of the intervention?'' (Cooper, Heron \& Heward, 2007, p. 155).

\subsection{Assessment}
\begin{enumerate}
\item Ask your Supervisee to explain why it's important as a behavior analyst to organize and interpret observed data. 
\item Ask your Supervisee to organize a set of data and display it graphically in the most appropriate way. 
\end{enumerate}
%
\subsection{Relevant Literature}
\begin{refsection}
\nocite{cooper2007applied,
        fisher2014handbook,
        johnston1993strategies,
        parsonson1978analysis}
\printbibliography[heading=none]
\end{refsection}
%
\subsection{Related Tasks}
\fouraTen{}\\
\fouraEleven{}\\
\fourbFour{}\\
\fourbFive{}\\
\fourbSix{}\\
\fourbSeven{}\\
\fourbEight{}\\
\fourbNine{}\\
\fourjFifteen{}\\
%
Footnotes
* See Cooper, Heron \& Heward (2007), pages 129 – 154 for more information on these graphic displays.
%
\clearpage \section[\fouriSix{}]{\fouriSix{}%
              \sectionmark{I-06 Make recommendations...}}
\sectionmark{I-06 Make recommendations...}
\subsection{Definition}
Hawkins (1984, p. 284) (cited from Cooper, Heron \& Heward, 2007, p. 56) defined habilitation as ``the degree to which the person's repertoire maximizes short and long term reinforcers for that individual and for others, and minimizes short and long term punishers.''

When determining what behaviors to target, one can use the relevance of behavior rule (Ayllon and Azrin, 1968) as a guide. This rule states that a target behavior should only be selected if it is likely to produce reinforcement for the client in their natural environment. Another key factor is deciding if the behavior will generalize to other settings and be sustainable once the behavior change program has ended. 

Cooper et al. (2007) provide some considerations when choosing a target behavior to increase, decrease, or maintain. These include:
\begin{enumerate}
\item Does this behavior pose any danger to the client or others?
\item How many opportunities will the person have to use this new behavior? Or how often does this problem behavior occur?
\item How long-standing is the problem or skill deficit?
\item Will changing the behavior produce higher rates of reinforcement for the person?
\item What will be the relative importance of this target behavior to the future skill development and independent functioning?
\item Will changing this behavior reduce negative attention from others?
\item Will the new behavior produce reinforcement for significant others?
\item How likely is success in changing this target behavior?
\item How much will it cost to change this behavior?
\end{enumerate}
%
\subsection{Examples}
\begin{enumerate}
\item Dave has decided to implement an intervention to increase a student's compliance. He chose this because lack of compliance interferes with the student's ability to learn new skills and access reinforcement by completing their work and daily routines.
\end{enumerate}
%
\subsection{Assessment}
\begin{enumerate}
\item Write a list of potential target behaviors. Have supervisee rank the behaviors in order of social significance and give rationale for their decisions.
\item Have supervisee present a case study on a client they are familiar with, including the maladaptive behaviors in their repertoire. Have the supervisee choose two behaviors to target for intervention and state why they chose those behaviors.
\end{enumerate}
%
\subsection{Relevant Literature}
\begin{refsection}
\nocite{ayllon1968token,
        cooper2007applied,
        hawkins1984meaningful,
        hawkins1986selection,
        rosales1997behavioral}
\printbibliography[heading=none]
\end{refsection}
\subsection{Related Tasks}
\fourbOne{}\\ 
\fourgThree{}\\
\fourgFive{}\\
\fouriOne{}\\
\fourjOne{}\\
\fourjFive{}\\
\fourjEight{}\\
\fourjTen{}\\
\fourjThirteen{}\\
\fourFKTen{}\\
