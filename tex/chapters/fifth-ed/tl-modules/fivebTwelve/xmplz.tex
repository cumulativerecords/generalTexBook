%\subsection{Examples}
\begin{enumerate}
\item Roger has not slept in three days because he has been studying for his chemistry final. Sleep becomes more valuable the more deprived of sleep he gets. 
\item (Non-example) Roger lost his key to his apartment and cannot get in. The locked door serves as motivation for him to find his key to get into his apartment. The key serves as a reinforcer because his learning history identifies this as the only way to unlock his door and get into his apartment.
\end{enumerate}
%
%\subsection{Examples}
Surrogate CMOs replace and have the same effect as the motivating operation that it was previously paired with.\\

Example: A rat is placed in a distinctive environment when food deprived. This is repeated a number of times. Over time, the rat is placed in the same environment when they have not been deprived of food. The distinctive environment and it's relation to a state of food deprivation results in an increase in the value of food as a reinforcer and an increase in the frequency of behavior with a history of producing food. In this example, the distinctive environment is paired with a unconditioned motivating operation (food deprivation). Over time, the distinctive environment functions as a motivating operation in the absence of food deprivation. 

Reflexive CMOs create a circumstance in which its own removal serves as the reinforcement.\\

Example: The presence of instructional materials often precedes the presentation of instructional tasks. If an individual engages in behavior maintained by access to escape from instructional tasks, in time they may engage in escape maintained behavior in the presence of instructional materials and the removal of these materials may function as a reinforcer. In this example, the instructional materials serve as a CMO-R.

Transitive CMOs make other stimuli more effective reinforcers.\\

Example: A locked door functions as a CMO-T to establish a key as a reinforcer.
%
%\subsection{Examples}
\begin{enumerate}
\item (CMO-R) A child engages in escape-maintained problem behavior during matching to sample instruction. In time, the child engages in escape-maintained problem behavior when the materials for matching to sample are brought out, before instruction has begun. 
\item (CMO-T) You walk up to your front door and turn the knob, but the door is locked. You reach into your pocket and grab your keys and unlock the door. 
\item (CMO-S) In the presence of a stimulus that has been paired with a cold environment, the value of stimuli that produce warmth increases.
\end{enumerate}
%
%\subsection{Examples}
\begin{enumerate}
\item A child often asks their parents to play video games after school. The child's father often says ``yes'' to this request, while the child's mother says ``no'' and tells the child to get started on their homework. Over time the child continues to ask their father if they can play video games, but has stopped asking their mother. In this example the presence of the father likely functions as an SD due to the history of requests being granted in his presence, but not in the presence of their mother.
\item After playing outside for an hour, a child walks into the house and gets a drink of water. In this example, playing outside likely functions as a motivating operation, more specifically as establishing operation, in that it increases the value of water as a reinforce and increases the frequency of behavior with a history of producing water.
\item A student earns tokens throughout the school day and can trade them in for a preferred item or activity. Usually the student chooses to trade in their tokens for a small snack, accept after lunch. Usually after lunch the student chooses computer over snacks. In this example, consuming food during lunch likely functions as a motivating operation, more specifically an abolishing operation, in that it decreases the value of food as a reinforcer and decreases the frequency of behavior with a history of producing food.
\end{enumerate}
%
%\subsection{Examples}
\begin{enumerate}
\item Being deprived of food or water increases the reinforcing value of food and water (i.e., a value altering effect in which the MO functions as an EO), and there will likely be an increase in the current frequency of all behavior that has previously been reinforced with food and water (i.e., an evocative behavior-altering effect).  Conversely, if a large meal was just consumed then it is unlikely that food will be reinforcing (i.e., a value-altering effect in which the MO functions as an AO) and there will be a reduction in the current frequency of all behavior previously reinforced with food (i.e., an abative behavior-altering effect).
\item This next example can illustrate the difference between MO effects and reinforcement/punishment effects (i.e., repertoire-altering effects).  Before leaving for work you realize that it is going to be a cold day.  The heater in your car does not work well so you plan ahead by putting a blanket and extra jacket in your car to use if it becomes too cold. During your drive to work, it becomes increasingly cold so you turn on your car heater, put your extra jacket on, and lay the blanket over you so you become much warmer.  In this example, there was an increase in the current frequency of all behavior that has been reinforced by becoming warmer (i.e., an evocative behavior-altering effect).  For the rest of the winter, to avoid becoming too cold on your drive to work, you leave every morning already wearing an extra jacket and put a blanket on you as soon as you get in the car (i.e., repertoire-altering effect on future behavior).  
\end{enumerate}