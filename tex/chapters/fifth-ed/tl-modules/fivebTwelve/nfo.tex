%\clearpage \section[\foureOne{}]{\foureOne{}%
 \subsection{Definition}
While it is commonly known that behaviors are maintained by consequences, including antecedent interventions within an individual's treatment package can often expedite positive behavioral change and mitigate negative effects of consequent strategies (e.g. extinction bursts). Some antecedent strategies include motivating operations, discriminative stimuli, non-contingent reinforcement and usage of high probability request sequences.

Michael (1982, p. 149) describes motivating operations as ``a stimulus change which, (1) given the momentary effectiveness of some particular type of reinforcement (2) increases the frequency of a particular type of response (3) because that stimulus change has been correlated with an increase in the frequency with which that type of response has been followed by that type of reinforcement.''

Skinner first explored this concept, describing deprivation and satiation to be motivating variables that govern behavior. Simply put, reinforcers obtain most of their reinforcing value depending on the individual's drive to obtain that reinforcer, which is a direct result of deprivation-satiation contingencies. The motivating operation for one's behavior that has run three miles in the heat is to quench their thirst increasing the value of water as a reinforcer. Having no money to put into a vending machine to get a bottle of water is the motivating operation to ask friends for loose change. Similarly, after the person drinks an entire bottle of water, water may no longer function as a reinforcer. Behavior analysts can thereby affect behavioral change by manipulating motivating operations (e.g. challenging behavior maintained by escape from non-preferred tasks may be mitigated by giving the individual frequent breaks). 

Another antecedent strategy is effectively programming for discriminative stimuli. Skinner claimed that virtually all operant behavior falls under stimulus control, asserting that ‘‘if all behavior were equally likely to occur on all occasions, the result would be chaotic'' (Skinner, 1953, p. 108).  It is therefore important for individuals to learn to discriminate between conditions in which specific responses will be reinforced and when these responses will not. Discriminative stimuli evoke behavior because they have been correlated with increased probability of accessing a reinforcer. For instance, teaching a student to mand for a break can be problematic if the student mands for a break continuously throughout the day, thereby yielding very little on-task behavior. However, this can be possibly remedied by the availability of a break is represented by the presence of a break icon (e.g. break icon is the discriminative stimulus, signaling that if the student asks for a break, a break will be granted).  

Other antecedent strategies include usage of non-contingent reinforcement and usage of high-probability request sequences. Non-contingent reinforcement is ``an antecedent intervention in which stimuli with known reinforcing properties are delivered on a fixed-time or variable-time schedule independent of the learner's behavior'' (Cooper, Heron \& Heward, 2007, p. 489). This operates on the principle of motivating operations. By satiating an individual with wants/needs, the individual is no longer motivated to engage in responses that used to generate that want/need (e.g. giving attention to a student every five minutes may abolish attention as a reinforcer, thereby reducing the need to engage in inappropriate attention-seeking behavior). 
%
%\clearpage \section{\fourFKTwentySix{}}
%\subsection{Definition}
Unconditioned motivating operations - ``...events, operations, and stimulus conditions with value-altering motivating effects that are unlearned'' (Michael, as cited in Cooper, Heron, \& Heward, 2007, p. 377).\\
Deprivation of basic human needs such as water, food, and sleep all create ``evocative effects'' that establish these items as reinforcers.\\

Cooper et al., 2007 identifies nine unconditioned motivating operations (UMOs) including food deprivation, water deprivation, sleep deprivation, activity deprivation, oxygen deprivation, sex deprivation, becoming too warm or cold, and an increase in painful stimulation. The withholding of any of these will lead to an increase in the reinforcing value of obtaining that which has been deprived.\\

On the other hand, when there is no longer deprivation, this serves as a UMO having an abative effect on behavior, making it less likely to occur.
%
%\clearpage \section{\fourFKTwentySeven{}}
%\subsection{Definition}
Conditioned motivating operation - ``A motivating operation whose value-altering effect depends on a learning history'' (Michael, as cited in Cooper, Heron, \& Heward, 2007, p. 384).\\

Three types of conditioned motivating operations (CMOs): surrogate (CMO-S), reflexive (CMO-R), and transitive (CMO-T)\\

%\clearpage \section[\fourFKTwentyEight{}]{\fourFKTwentyEight{}%
 %\subsection{Definition}
Conditioned motivating operations consist of ``...motivating variables that alter the reinforcing effectiveness of other stimuli, objects, or events, but only as a result of an organism's learning history...'' (Cooper, Heron, \& Heward, 2007, p. 384).\\

The 3 types of conditioned motivating operations are surrogate (CMO-S), reflexive (CMO-R), and transitive (CMO-T).\\

``Any stimulus that systematically precedes the onset of painful stimulation becomes a CMO-R (reflexive- CMO), in that its occurrence will evoke any behavior that has been followed by such reinforcement'' (Cooper et al., 2007, p. 385).\\

``When an environmental variable is related to the relation between another stimulus and some form of improvement, the presence of that variable functions as a transitive CMO, or CMO-T, to establish the second condition's reinforcing effectiveness and to evoke the behavior that has been followed by that reinforcer'' (Cooper et al., 2007, p. 387).\\

Surrogate CMO's are stimuli that have been paired with another motivating operation and ``acquired a form of behavioral effectiveness by being paired with a behaviorally effective stimulus'' (Cooper et al., 2007, p. 384). There is not strong evidence for this type of CMO.
%
%\clearpage \section[\fourFKTwentyNine{}]{\fourFKTwentyNine{}%
%\subsection{Definition}
Discriminative stimulus (SD) - ``A stimulus in the presence of which responses of some type have been reinforced and in the absence of which the same type of responses have occurred and not been reinforced; this history of differential reinforcement is the reason an SD increases the momentary frequency of the behavior'' (Cooper, Heron, \& Heward, 2007, p. 694).\\

Motivating operation - ``An environmental variable that (a) alters (increases or decreases) the reinforcing or punishing effectiveness of some stimulus, object, or event; and (b) alters (increases or decreases) the current frequency of all behavior that has been reinforced or punished by that stimulus, object, or event'' (Cooper et al., 2007, p. 699).\\
%
%\clearpage \section[\fourFKThirty{}]{\fourFKThirty{}%
%\subsection{Definition}
Understanding motivating operations (MO) and reinforcement effects are critical components in the analysis of behavior.  To briefly explain the difference, MOs are antecedent variables that have behavior-altering effects in that they alter the current frequency of relevant behaviors whereas the process of reinforcement is a consequence-based process (as is extinction and punishment) said to have repertoire-altering effects in that the future frequency of the behavior that preceded the consequence is altered (Cooper, Heron, \& Heward, 2007).  While this explanation can help clarify the difference between MO effects and reinforcement effects, it is also important to understand the basic features of MOs.  Specifically, MOs have a value-altering effect or a behavior-altering effect as defined by Cooper et al., (2007).\\

The value-altering effect is either (a) an increase in the reinforcing effectiveness of some stimulus, object, or event, in which case the MO is an establishing operation (EO); or (b) a decrease in reinforcing effectiveness, in which case the MO is an abolishing operation (AO).  The behavior-altering effect is either (a) an increase in the current frequency of behavior that has been reinforced by some stimulus, object, or event, called an evocative effect; or (b) a decrease in the current frequency of behavior that has been reinforced by some stimulus, object, or event, called an abative effect (p. 375).
%
