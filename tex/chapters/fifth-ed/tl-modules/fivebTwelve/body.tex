\clearpage \section[\foureOne{}]{\foureOne{}%
              \sectionmark{E-01 Use interv... manipulation}}
\subsection{Definition}
While it is commonly known that behaviors are maintained by consequences, including antecedent interventions within an individual's treatment package can often expedite positive behavioral change and mitigate negative effects of consequent strategies (e.g. extinction bursts). Some antecedent strategies include motivating operations, discriminative stimuli, non-contingent reinforcement and usage of high probability request sequences.

Michael (1982, p. 149) describes motivating operations as ``a stimulus change which, (1) given the momentary effectiveness of some particular type of reinforcement (2) increases the frequency of a particular type of response (3) because that stimulus change has been correlated with an increase in the frequency with which that type of response has been followed by that type of reinforcement.''

Skinner first explored this concept, describing deprivation and satiation to be motivating variables that govern behavior. Simply put, reinforcers obtain most of their reinforcing value depending on the individual's drive to obtain that reinforcer, which is a direct result of deprivation-satiation contingencies. The motivating operation for one's behavior that has run three miles in the heat is to quench their thirst increasing the value of water as a reinforcer. Having no money to put into a vending machine to get a bottle of water is the motivating operation to ask friends for loose change. Similarly, after the person drinks an entire bottle of water, water may no longer function as a reinforcer. Behavior analysts can thereby affect behavioral change by manipulating motivating operations (e.g. challenging behavior maintained by escape from non-preferred tasks may be mitigated by giving the individual frequent breaks). 

Another antecedent strategy is effectively programming for discriminative stimuli. Skinner claimed that virtually all operant behavior falls under stimulus control, asserting that ‘‘if all behavior were equally likely to occur on all occasions, the result would be chaotic'' (Skinner, 1953, p. 108).  It is therefore important for individuals to learn to discriminate between conditions in which specific responses will be reinforced and when these responses will not. Discriminative stimuli evoke behavior because they have been correlated with increased probability of accessing a reinforcer. For instance, teaching a student to mand for a break can be problematic if the student mands for a break continuously throughout the day, thereby yielding very little on-task behavior. However, this can be possibly remedied by the availability of a break is represented by the presence of a break icon (e.g. break icon is the discriminative stimulus, signaling that if the student asks for a break, a break will be granted).  

Other antecedent strategies include usage of non-contingent reinforcement and usage of high-probability request sequences. Non-contingent reinforcement is ``an antecedent intervention in which stimuli with known reinforcing properties are delivered on a fixed-time or variable-time schedule independent of the learner's behavior'' (Cooper, Heron \& Heward, 2007, p. 489). This operates on the principle of motivating operations. By satiating an individual with wants/needs, the individual is no longer motivated to engage in responses that used to generate that want/need (e.g. giving attention to a student every five minutes may abolish attention as a reinforcer, thereby reducing the need to engage in inappropriate attention-seeking behavior). 
%
\subsection{Assessment}
\begin{enumerate}
\item Have your supervisee list and describe applicable antecedent interventions.
\item In a clinical setting or during role-play, have your supervisee describe what motivating operations may be affecting the client's behavior.
\item Describe a scenario and have the supervisee lists some potential antecedent strategies that can be used and have them describe why they chose these strategies.
\end{enumerate}
%
\subsection{Relevant Literature}
\begin{refsection}
\nocite{michael1982distinguishing,
        skinner1953science,
        smith1997antecedent}
\printbibliography[heading=none]
\end{refsection}
%
\subsection{Related Tasks} 
\foureNine{}\\
\fourFKTwentySix{}\\
\fourFKTwentySeven{}\\
\fourFKTwentyNine{}\\
%
\clearpage \section{\fourFKTwentySix{}}
\subsection{Definition}
Unconditioned motivating operations - ``...events, operations, and stimulus conditions with value-altering motivating effects that are unlearned'' (Michael, as cited in Cooper, Heron, \& Heward, 2007, p. 377).\\
Deprivation of basic human needs such as water, food, and sleep all create ``evocative effects'' that establish these items as reinforcers.\\

Cooper et al., 2007 identifies nine unconditioned motivating operations (UMOs) including food deprivation, water deprivation, sleep deprivation, activity deprivation, oxygen deprivation, sex deprivation, becoming too warm or cold, and an increase in painful stimulation. The withholding of any of these will lead to an increase in the reinforcing value of obtaining that which has been deprived.\\

On the other hand, when there is no longer deprivation, this serves as a UMO having an abative effect on behavior, making it less likely to occur.
%
\subsection{Examples}
\begin{enumerate}
\item Roger has not slept in three days because he has been studying for his chemistry final. Sleep becomes more valuable the more deprived of sleep he gets. 
\item (Non-example) Roger lost his key to his apartment and cannot get in. The locked door serves as motivation for him to find his key to get into his apartment. The key serves as a reinforcer because his learning history identifies this as the only way to unlock his door and get into his apartment.
%
\end{enumerate}
%
\subsection{Assessment}
\begin{enumerate}
\item Have supervisee identify at least 7 unconditioned motivating operations. Have him/her describe their reinforcer-establishing effect as well as their evocative effect. (see page 379, table 16.1 in Cooper et al., 2007)
\item Have supervisee identify UMOs that decrease reinforcer effectiveness and abate relevant behavior. Have him/her describe the reinforcer-abolishing effect and the abative effect of each UMO. (See page 380, table 16.2 in Cooper et al., 2007).
\item Have supervisee explain how to weaken the effects of a UMO. (See page 380-381 in Cooper et al., 2007).
\item Have supervisee explain the difference between motivating operations and discriminative stimuli. (See page 377 in Cooper et al., 2007).
%
\end{enumerate}
%
\subsection{Relevant Literature}
\begin{refsection}
\nocite{cooper2007applied,
        laraway2001abative,
        lotfizadeh2012motivating,
        michael1982distinguishing,
        michael2000implications,
        ulrich1962reflexive}
\printbibliography[heading=none]
\end{refsection}
%
\subsection{Related Tasks}
\fourdOne{}\\
\foureOne{}\\
\fouriTwo{}\\
\fourFKTwo{}\\
\fourFKThirteen{}\\
\fourFKFourteen{}\\
\fourFKSeventeen{}\\
\fourFKNineteen{}\\
\fourFKTwentySeven{}\\
\fourFKTwentyEight{}\\
\fourFKTwentyNine{}\\
\fourFKThirty{}\\
%
\clearpage \section{\fourFKTwentySeven{}}
\subsection{Definition}
Conditioned motivating operation - ``A motivating operation whose value-altering effect depends on a learning history'' (Michael, as cited in Cooper, Heron, \& Heward, 2007, p. 384).\\

Three types of conditioned motivating operations (CMOs): surrogate (CMO-S), reflexive (CMO-R), and transitive (CMO-T)\\



\subsection{Examples}
Surrogate CMOs replace and have the same effect as the motivating operation that it was previously paired with.\\

Example: A rat is placed in a distinctive environment when food deprived. This is repeated a number of times. Over time, the rat is placed in the same environment when they have not been deprived of food. The distinctive environment and it's relation to a state of food deprivation results in an increase in the value of food as a reinforcer and an increase in the frequency of behavior with a history of producing food. In this example, the distinctive environment is paired with a unconditioned motivating operation (food deprivation). Over time, the distinctive environment functions as a motivating operation in the absence of food deprivation. 

Reflexive CMOs create a circumstance in which its own removal serves as the reinforcement.\\

Example: The presence of instructional materials often precedes the presentation of instructional tasks. If an individual engages in behavior maintained by access to escape from instructional tasks, in time they may engage in escape maintained behavior in the presence of instructional materials and the removal of these materials may function as a reinforcer. In this example, the instructional materials serve as a CMO-R.

Transitive CMOs make other stimuli more effective reinforcers.\\

Example: A locked door functions as a CMO-T to establish a key as a reinforcer.
%
%
\subsection{Assessment}
\begin{enumerate}
\item Have supervisee describe the three types of CMOs. Have him/her give examples of each. 
\item Have supervisee explain the definitions for conditioned and unconditioned motivating operations in simple terms that someone who does not have ABA experience can understand.
\item Have supervisee explain how to weaken the effects of each of the three types of CMO. 
%
\end{enumerate}
%
\subsection{Relevant Literature}
\begin{refsection}
\nocite{catania1993coming,
        clark1958effect,
        cooper2007applied,
        endicott2007contriving,
        hesse1993establishing,
        iwata2000current,
        lotfizadeh2012motivating,
        michael1993establishing}
\printbibliography[heading=none]
\end{refsection}
%
\subsection{Related Tasks}
\fourdOne{}\\
\foureOne{}\\
\fouriTwo{}\\
\fourFKTwo{}\\
\fourFKThirteen{}\\
\fourFKFourteen{}\\
\fourFKSeventeen{}\\
\fourFKNineteen{}\\
\fourFKTwentySix{}\\
\fourFKTwentyEight{}\\
\fourFKTwentyNine{}\\
\fourFKThirty{}\\
%
\clearpage \section[\fourFKTwentyEight{}]{\fourFKTwentyEight{}%
              \sectionmark{FK-28 Transitive, reflexive...}}
\sectionmark{FK-28 Transitive, reflexive...}
\subsection{Definition}
Conditioned motivating operations consist of ``...motivating variables that alter the reinforcing effectiveness of other stimuli, objects, or events, but only as a result of an organism's learning history...'' (Cooper, Heron, \& Heward, 2007, p. 384).\\

The 3 types of conditioned motivating operations are surrogate (CMO-S), reflexive (CMO-R), and transitive (CMO-T).\\

``Any stimulus that systematically precedes the onset of painful stimulation becomes a CMO-R (reflexive- CMO), in that its occurrence will evoke any behavior that has been followed by such reinforcement'' (Cooper et al., 2007, p. 385).\\

``When an environmental variable is related to the relation between another stimulus and some form of improvement, the presence of that variable functions as a transitive CMO, or CMO-T, to establish the second condition's reinforcing effectiveness and to evoke the behavior that has been followed by that reinforcer'' (Cooper et al., 2007, p. 387).\\

Surrogate CMO's are stimuli that have been paired with another motivating operation and ``acquired a form of behavioral effectiveness by being paired with a behaviorally effective stimulus'' (Cooper et al., 2007, p. 384). There is not strong evidence for this type of CMO.
%
\subsection{Examples}
\begin{enumerate}
\item (CMO-R) A child engages in escape-maintained problem behavior during matching to sample instruction. In time, the child engages in escape-maintained problem behavior when the materials for matching to sample are brought out, before instruction has begun. 
\item (CMO-T) You walk up to your front door and turn the knob, but the door is locked. You reach into your pocket and grab your keys and unlock the door. 
\item (CMO-S) In the presence of a stimulus that has been paired with a cold environment, the value of stimuli that produce warmth increases.
%
\end{enumerate}
%
\subsection{Assessment}
\begin{enumerate}
\item Have supervisee list and define the 3 types of conditioned motivating operations. 
\item Have him/her give examples of each type of conditioned motivating operation.
\item Have supervisee explain the difference between conditioned and unconditioned motivating operations.
%
\end{enumerate}
%
\subsection{Relevant Literature}
\begin{refsection}
\nocite{carbone2007role,
        cooper2007applied,
        laraway2003motivating,
        mcgill1999establishing,
        michael2004concepts,
        michael1993establishing,
        mineka1975some,
        rosales2007contriving}
\printbibliography[heading=none]
\end{refsection}
%
\subsection{Related Tasks}
\fourdOne{}\\
\foureOne{}\\
\fourFKThirteen{}\\
\fourFKFourteen{}\\
\fourFKFifteen{}\\
\fourFKSixteen{}\\
\fourFKEighteen{}\\
\fourFKTwenty{}\\
\fourFKTwentySeven{}\\
\fourFKTwentyNine{}\\
\fourFKThirty{}\\
%
\clearpage \section[\fourFKTwentyNine{}]{\fourFKTwentyNine{}%
              \sectionmark{FK-29 Distinguish... discriminative}}
\sectionmark{FK-29 Distinguish... discriminative}
\subsection{Definition}
Discriminative stimulus (SD) - ``A stimulus in the presence of which responses of some type have been reinforced and in the absence of which the same type of responses have occurred and not been reinforced; this history of differential reinforcement is the reason an SD increases the momentary frequency of the behavior'' (Cooper, Heron, \& Heward, 2007, p. 694).\\

Motivating operation - ``An environmental variable that (a) alters (increases or decreases) the reinforcing or punishing effectiveness of some stimulus, object, or event; and (b) alters (increases or decreases) the current frequency of all behavior that has been reinforced or punished by that stimulus, object, or event'' (Cooper et al., 2007, p. 699).\\
%
\subsection{Examples}
\begin{enumerate}
\item A child often asks their parents to play video games after school. The child's father often says ``yes'' to this request, while the child's mother says ``no'' and tells the child to get started on their homework. Over time the child continues to ask their father if they can play video games, but has stopped asking their mother. In this example the presence of the father likely functions as an SD due to the history of requests being granted in his presence, but not in the presence of their mother.
\item After playing outside for an hour, a child walks into the house and gets a drink of water. In this example, playing outside likely functions as a motivating operation, more specifically as establishing operation, in that it increases the value of water as a reinforce and increases the frequency of behavior with a history of producing water.
\item A student earns tokens throughout the school day and can trade them in for a preferred item or activity. Usually the student chooses to trade in their tokens for a small snack, accept after lunch. Usually after lunch the student chooses computer over snacks. In this example, consuming food during lunch likely functions as a motivating operation, more specifically an abolishing operation, in that it decreases the value of food as a reinforcer and decreases the frequency of behavior with a history of producing food.
%
\end{enumerate}
%
\subsection{Assessment}
\begin{enumerate}
\item Have supervisee explain the difference between SD s and MOs.
\item Have the supervisee to create additional examples of SD s and MOs.
\item Provide the supervisee with examples of responses and the reinforcers for those responses. Have the supervisee describe potential ways to increase and decrease the value of the reinforcer.
%
\end{enumerate}
%
\subsection{Relevant Literature}
\begin{refsection}
\nocite{cooper2007applied,
        michael1982distinguishing}
\printbibliography[heading=none]
\end{refsection}
%
\subsection{Related Tasks}
\foureOne{}\\
\fourgEight{}\\
\fouriTwo{}\\
\fourjFour{}\\
\fourjSix{}\\
\fourjSeven{}\\
\fourkNine{}\\
\fourFKSeven{}\\
\fourFKTwentyFour{}\\
\fourFKTwentySix{}\\
\fourFKThirtyOne{}\\
\fourFKThirtyThree{}\\
%
\clearpage \section[\fourFKThirty{}]{\fourFKThirty{}%
              \sectionmark{FK-30 Distinguish... motivating}}
\sectionmark{FK-30 Distinguish... motivating}
\subsection{Definition}
Understanding motivating operations (MO) and reinforcement effects are critical components in the analysis of behavior.  To briefly explain the difference, MOs are antecedent variables that have behavior-altering effects in that they alter the current frequency of relevant behaviors whereas the process of reinforcement is a consequence-based process (as is extinction and punishment) said to have repertoire-altering effects in that the future frequency of the behavior that preceded the consequence is altered (Cooper, Heron, \& Heward, 2007).  While this explanation can help clarify the difference between MO effects and reinforcement effects, it is also important to understand the basic features of MOs.  Specifically, MOs have a value-altering effect or a behavior-altering effect as defined by Cooper et al., (2007).\\

The value-altering effect is either (a) an increase in the reinforcing effectiveness of some stimulus, object, or event, in which case the MO is an establishing operation (EO); or (b) a decrease in reinforcing effectiveness, in which case the MO is an abolishing operation (AO).  The behavior-altering effect is either (a) an increase in the current frequency of behavior that has been reinforced by some stimulus, object, or event, called an evocative effect; or (b) a decrease in the current frequency of behavior that has been reinforced by some stimulus, object, or event, called an abative effect (p. 375).
%
\subsection{Examples}
\begin{enumerate}
\item Being deprived of food or water increases the reinforcing value of food and water (i.e., a value altering effect in which the MO functions as an EO), and there will likely be an increase in the current frequency of all behavior that has previously been reinforced with food and water (i.e., an evocative behavior-altering effect).  Conversely, if a large meal was just consumed then it is unlikely that food will be reinforcing (i.e., a value-altering effect in which the MO functions as an AO) and there will be a reduction in the current frequency of all behavior previously reinforced with food (i.e., an abative behavior-altering effect).
\item This next example can illustrate the difference between MO effects and reinforcement/punishment effects (i.e., repertoire-altering effects).  Before leaving for work you realize that it is going to be a cold day.  The heater in your car does not work well so you plan ahead by putting a blanket and extra jacket in your car to use if it becomes too cold. During your drive to work, it becomes increasingly cold so you turn on your car heater, put your extra jacket on, and lay the blanket over you so you become much warmer.  In this example, there was an increase in the current frequency of all behavior that has been reinforced by becoming warmer (i.e., an evocative behavior-altering effect).  For the rest of the winter, to avoid becoming too cold on your drive to work, you leave every morning already wearing an extra jacket and put a blanket on you as soon as you get in the car (i.e., repertoire-altering effect on future behavior).  
%
\end{enumerate}
%
\subsection{Assessment}
\begin{enumerate}
\item Ask supervisee to define MOs and explain the basic characteristics.
\item Ask supervisee to discriminate between behavior-altering effects and repertoire-altering effects.
\item Ask supervisee to identify potential MOs for client's behavior.
\item Ask supervisee to provide examples of behavior-altering effects and repertoire-altering effects that are operating in a client's environment.
\end{enumerate}
%
\subsection{Relevant Literature}
\begin{refsection}
\nocite{cooper2007applied,
        iwata2000current,
        laraway2001abative,
        schlinger1987function}
\printbibliography[heading=none]
\end{refsection}
\subsection{Related Tasks}
\foureOne{}\\
\fourFKTwentyFive{}\\
\fourFKTwentySix{}\\
\fourFKTwentySeven{}\\
\fourFKTwentyEight{}\\
\fourFKTwentyNine{}\\
