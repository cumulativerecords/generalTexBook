%\clearpage \section[\fourhOne{}]{\fourhOne{}%
              \sectionmark{H-01 Select a measurement sys...}}
\sectionmark{H-01 Select a measurement sys...}
\subsection{Definition}
Assessment and treatment decisions of behavior analysts rely on data. A behavior analyst can design and implement effective treatments only when the data accurately represent the behavior of interest (validity) and have been reliably recorded as they are observed. To accomplish this goal, behavior analysts choose behaviors and dimensions of those behaviors to facilitate accurate and consistent recording within a given context for each client. Behavior analysts measure ``three fundamental properties, or dimensional quantities'' (Cooper et al., 2007, p. 75) of behavior: 
\begin{enumerate}
\item Repeatability--Behavior can be counted in the same way each time it occurs.
\item Temporal extent--Behavior can be measured in relation to time.
\item Temporal locus—Behavior occurs in relation to other behaviors.
\end{enumerate}

First, identify which of these three properties will provide the most accurate method for quantifying behavior. Then, decides what dimension of behavior to measure, such as a count of occurrences, frequency of behavior per unit of time, duration, latency, or other. Last, decide who will record the behavior and in what context. If a preferred measurement system is unlikely to be effectively implemented, then the analyst has to reconsider definitions or recording circumstances in order to obtain adequate and accurate data for decision-making. 
%
%\clearpage \section[\fouraTwelve{}]{\fouraTwelve{}%
              \sectionmark{A-12 Design... continuous}}
\sectionmark{A-12 Design... continuous}
\subsection{Definition} 
	Event recording – ``measurement procedure for obtaining a tally or count of the number of times a behavior occurs'' (Cooper, Heron, \& Heward, 2007, p. 695).
%  
%\clearpage \section[\fouraThirteen{}]{\fouraThirteen{}%
              \sectionmark{A-13 Design... discontinuous}}
\sectionmark{A-13 Design... discontinuous}
\subsection{Definition}
Time sampling – ``...refers to a variety of methods for observing and recording behavior during intervals or at specific moments in time. The basic procedure involves dividing the observation period into time intervals and then recording the presence or absence of behavior within or at the end of interval... Three forms of time sampling used often by applied behavior analysts are whole-interval recording, partial-interval recording and momentary time sampling'' (Cooper, Heron, \& Heward, 2007, p. 90). 

Whole-Interval Recording\\
Once the interval has ended, the observer records whether the behavior has occurred throughout the entire interval. Whole-interval recording tends to underestimate how much a behavior is occurring because the behavior has to be emitted for the entire interval in order to get recorded (Cooper, Heron, \& Heward, 2007).

Partial-Interval Recording\\
With the partial-interval recording method, the time of observation is again divided into intervals and a behavior is recorded as having occurred if it has occurred at some point during the interval. Data are usually reported as percentage of intervals (Cooper, Heron, \& Heward, 2007).

Momentary Time Sampling\\
With this type of measurement, a period of time is divided up into intervals and the observer records whether the behavior is occurring at the precise moment the interval ends (Cooper, Heron and Heward, 2007).

