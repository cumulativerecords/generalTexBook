%
\subsection{Examples}
\begin{enumerate}
\item A teacher conducting receptive language training tells her student ``sit down'' the student sits down and the teacher praises the student. The teacher then says ``clap hands'' the student claps hands and the teacher praises the student.
\item A teacher presents an array of fruit and says to the student, ``give me the apple.'' The student gives the teacher the apple and the teacher gives the student a token. 
\end{enumerate}
%
\subsection{Examples}
\begin{enumerate}
\item Mrs. Platypus is instructing her 3rd grade class on their math facts.  She holds up a card stating that, ``three times nine is eighteen.''  She then restates the fact asking the class to repeat.  The class says, ``three times nine is eighteen'' in unison.  Mrs. Platypus praises the students for their repetition.
\item Mr. Penguin is a kindergarten teacher.  He is working with one student on his reading skills.  He shows little Timmy the letter R.  He tells him that the letter R makes the ``rrr'' sound and asks him to repeat.  Little Timmy says, ``rrr,'' and Mr. Penguin comments, ``Nice job Timmy.''  
\item Mrs. Dodo the art teacher needs one of her students to run to the office and get some supplies. One of the children volunteers.  She tells him that she needs him to get, ``Crayons, markers, and paint.''  He repeats, ``Crayons, markers, and paint.''  ``Exactly,'' Mrs. Dodo says sending him on his way.  
\item (Non-example) Mrs. Platypus is still working on math facts with her class.  She holds up the math fact 4x9= and asks the students to give the answer.  Susie Q raises her hand and answers ``thirty-six.''  
%
\end{enumerate}
%
\subsection{Examples}
\begin{enumerate}
\item ``I want a cookie.'' (This is a mand for an item. Mands can include verbs, use of adjectives, prepositions, pronouns etc.)
\item A child says ``watch me'' after learning how to ride a bike independently (mand for attention)
\item Asking questions like ``what's your name? or ``where's the phone?'' (mand for information)
\item Child says, ``No!'' when parent is about to use blender (mand for avoidance of an aversive)
%

%
\end{enumerate}
%
\subsection{Examples}
\begin{enumerate}
\item Dexter walks outside with his mother and sees birds in a tree.  ``Robins,'' he says.  ``You're right. Those are robins,'' Dexter's mom says. ``Robins'' is a tact.
\item Hector is in the store shopping for Valentines Day.  He sees a variety of flowers before noticing the ones he wants to buy.  ``Red roses,'' Hector says quietly to himself. ``Red roses'' in this context is likely a tact. 
\item Chester goes to his friends Superbowl party.  Upon scanning the array of delicious apps and snacks on the counter, he hones in on one that is his favorite.  ``Ooh, buffalo chicken dip,'' he comments.  ``Buffalo chicken dip'' would be likely a tact in this context.
\item Non-example: Dexter is thinking about buying some cookies the next time he goes to the supermarket.  He writes the word ``cookies'' down on his shopping list. 
%
\end{enumerate}
%
\subsection{Examples}
\begin{enumerate}
\item A new employee shows up for his first day on the job. The man in the cubical next to him asks, ``What is your name?''  ``Harvey,'' the man replies. Saying ``Harvey'' is an intraverbal in that context.
\item Hanks boss stops his office to let him know that his sales were ``outstanding this week.''  ``Thanks. I really put in some long hours,'' Hank notes.  ``Thanks,'' is an intraverbal in that context.
\item (Non-example) The office phone rings. Harvey picks up the phone and answers ``Hello.'' There is no one on the line so he hangs up and keeps working. 
%
\end{enumerate}
%
