%\clearpage \section[\fourdNine{}]{\fourdNine{}%
              \sectionmark{D-09 Use the verbal operants...}}
\subsection{Definition}
In the field of applied behavior analysis extensive research has been done on the development of verbal behavior.  

``Verbal behavior involves social interactions between speakers and listeners, whereby speakers gain access to reinforcement and control their environment through the behavior of listeners'' (Sundberg as cited in Cooper, Heron, \& Heward, 2007, p. 529). Verbal operants are the basic units of this exchange.  

In 1957 B.F. Skinner identified six elementary verbal operants in his book on Verbal Behavior.  These included mands, tacts, intraverbals, echoics, textuals, and transcription.  ``Skinner's analysis suggests that a complete verbal repertoire is composed of each of the different elementary operants, and separate speaker and listener repertoires'' (Sundberg as cited in Cooper, et al., 2007, p. 541).   

Since Skinner described these operants, those in the field have applied these concepts to both language assessment and training.  In order to evaluate whether or not specific language training is necessary, a variety of standardized tools have been used to test an individual's receptive and expressive language abilities.  These include but are not limited to: the Peabody Picture Vocabulary Test III (Dunn \& Dunn, 1997), the Comprehensive Receptive and Expressive Vocabulary Test (Hammill \& Newcomer, 1997), the Assessment of Basic Language and Learning Skills (ABLLS) (Partington \& Sundberg, 1998), the Verbal Behavior Milestones Assessment and Placement Program (VB-MAPP) and the CELF-4 Semel, Wiig, \& Secord, 2003).  

Not all of these tests will identify deficits in one or more of the verbal operants. Some children who may be proficient in tacting (such as labeling things in their environment such as letters and numbers) may fail to make appropriate mands for desired items (Cooper, et al., 2007).  In this case it is important for behavior analysts to use a combination of approaches or less standardized methods to assess these needs.  It may be helpful to observe the individual in their natural environment and take data on their verbal interactions.  It will be important to ask questions such as:
\begin{enumerate}
\item What is the frequency of and complexity of mands?
\item What is the frequency and complexity of tacting behavior?
\item Will the child or individual demonstrate echoic behavior when prompted?
\item Does the child or individual engage in intraverbal behavior with known caregivers?
\item Can or will the child or individual read words that are written down for him?
\item Can or will the child or individual write words that are said to him? 
\end{enumerate}
%
\subsection{Assessment}
\begin{enumerate}
\item Ask the supervisee to name the basic unit of language
\item Ask the supervisee to name all 6 of the elementary verbal operants
\item Ask the supervisee to name some of the standardized tests often used to assess language
\item Ask the supervisee to explain why these standardized tests may not provide adequate information
\item Ask the supervisee to describe how one might assess an individual's use of verbal operants if testing fails to yield enough information.
\end{enumerate}
%
\subsection{Relevant Literature}
\begin{refsection}
\nocite{cooper2007applied,
    partington1998assessment,
    semel2003clinical,
    skinner1957verbal,
    sundberg2008verbal,
    sundberg1998teaching}
\printbibliography[heading=none]
\end{refsection}
%cannot verify reference.
%Hammill, D., \& Newcomer, P.L. (1997).  Test of language development-3.  Austin, TX: Pro-Ed.
%
\subsection{Related Lessons}
\fourdTen{}\\
\fourdEleven{}\\
\fourdTwelve{}\\
\fourdThirteen{}\\
\fourdFourteen{}\\
\fourFKFourtyThree{}\\
\fourFKFourtyFour{}\\
\fourFKFourtyFive{}\\
\fourFKFourtySix{}\\
%
%\clearpage \section{\fourdTen{}}
\subsection{Definition} 
Echoics are units of verbal behavior that are, ``evoked by a verbal discriminative stimulus that has point-to point correspondence and formal similarity with the response'' (Cooper, Heron, \& Heward, 2014, p. 694).  

Repeating words, phrases, or other auditory verbal units is common for all speakers in day-to-day speech (Sundberg, 2008).

Echoic training, sometimes referred to as vocal imitation training, is a procedure in which a caregiver or teacher emits a sound and a listener echoes or repeats what has been said (Skinner, 1957).  Reinforcement (either social, tangible or other) is often delivered after the correct utterance is given.  

Echoic training can be used to teach a variety of skills such as:
\begin{enumerate}
\item Mands-such as when you give a child a full verbal model of the appropriate way to ask for another cup of milk ``I want milk'' and he repeats this phrase.
\item Tacts- such as telling a classroom full of Spanish students that the word for dog is ``perro'' and asking them to repeat this word back to you.
\item Intraverbal behavior- such as an elementary school teaching modeling the fill in of the word ``star'' after saying ``Twinkle, twinkle little... '' and pausing (Skinner, 1957).
\end{enumerate}

When using echoic training, the trainer should:
\begin{enumerate}
\item Deliver the verbal discriminative stimulus (the word, sound or phrase you intend them to repeat).
\item Provide positive reinforcement for responses that have point-to-point similarity to the target response.  
\end{enumerate}
%
\subsection{Assessment}
\begin{enumerate}
\item Ask the supervisee to state what echoic training can be used to teach.
\item Ask the supervisee to give examples of echoic training.
\item Ask the supervisee to discuss how echoic training should be delivered.
\end{enumerate}
%
\subsection{Relevant Literature}
\begin{refsection}
\nocite{drash1999using,
    cooper2007applied,
    kodak2009acquisition,
    stock2008comparison,
    sundberg2008verbal,
    skinner1957verbal}
\printbibliography[heading=none]
\end{refsection}
%
\subsection{Related Lessons}
\fourdOne{}\\
\fourdFour{}\\
\fourFKFourtyThree{}\\
%
%\clearpage \section{\fourdEleven{}}
\subsection{Definition}
Mands are important in the development of language in children.  The development of a mands allows an individual to communicate their wants and needs, including basic needs such as food and water, to those around them.  When an early learner fails to develop a mand repertoire, they may not be able to effectively communicate with others and may not be able to access these reinforcers.  This can lead to frustration, learned helplessness, or a variety of other maladaptive behaviors such aggression, self-injurious behavior social withdrawal or tantrums (Cooper, Heron, \& Heward, 2007).

When mands fail to develop typically, it is crucial begin language training.   Cooper, Heron and Heward (2007) suggest teaching mands before all other types of verbal behavior as manding allows an individual to gain access to their wants and needs. During early training a variety of mands should be taught so that the child learns to differentiate their response based on their current needs and MO. Instructors should focus on teaching bids for edibles and tangibles before making other requests.   Sundberg \& Partington (1998) suggest that teachers should use a combination of ``prompting, fading and differential reinforcement to transfer control from stimulus variables to motivative variables'' (as cited in Cooper, Heron, \& Heward, 2007, p. 541).  It is important that both the echoic prompt and the non-verbal stimulus be faded out for mand training to be effective.  

When using mand training, the trainer should:
\begin{enumerate}
\item Establish an MO (motivating operation).  This may be done formally through preference assessment procedures or more informally through observations or caregiver report.  It is important that a child be motivated to make a request for mand training to be effective.  Note: It may be helpful to ensure that a reinforcer has been withheld prior to training to ensure that it is potent.  For instance a child who has just recently eaten is not as likely to be motivated to request food.  
\item Enrich the environment with potential reinforcers (things that the child generally seems to prefer such as foods and toys).
\item Wait for the child to initiate or show interest in the non-verbal stimulus (the child reaches for an item, emits some sort of vocalization, points to it, etc.). 
\item Use an echoic prompt to label the non-verbal stimulus.  Successively reinforce closer and closer approximations to the target verbal response and follow with specific reinforcement (the requested item).  
\item Once the client is able to imitate the verbal model in the presence of the stimulus, gradually fade out the echoic prompt to establish the response ``under the multiple control of the MO and the nonverbal stimulus'' (Sundberg, as cited in Cooper, Heron, \& Heward, 2007, pp. 541-542).  
\item Finally the presentation of the non-verbal stimulus should also be faded out so that the response is only under the control of the MO.  This helps to ensure that the individual can make the request regardless of whether or not the item is physically present within the environment. 
\item Gradually increase the verbal requirement over time so that the child is making more complex and specific requests (``I want the chocolate cookie.'').
\end{enumerate}
%
\subsection{Assessment}
\begin{enumerate}
\item Ask the supervisee to state what mand training can be used to teach.
\item Ask the supervisee to give examples of mand training.
\item Ask the supervisee to state why it is important to fade the non-verbal stimulus and the echoic prompt.  
\end{enumerate}
%
\subsection{Relevant Literature}
\begin{refsection}
\nocite{cooper2007applied,
    drash1999using,
    skinner1957verbal,
    sundberg2001contriving,
    sundberg1998teaching}
\printbibliography[heading=none]
\end{refsection}
%
\subsection{Related Lessons}
\fourFKFourtyFour{}\\
%
%\clearpage \section{\fourdTwelve{}}
\subsection{Definition}
Practitioners may use a variety of techniques to teach language when working with clients.  Tact training is one such technique in which a consumer may be taught to label ``objects, actions, properties of objects and actions, prepositional relations, abstractions, private events, and so on''  (Sundberg as cited in Cooper, Howard, \& Heron, 2007, p. 544). ``The goal of teaching is to bring a verbal response under nonverbal stimulus control''  (Sundberg as cited in Cooper, Howard, \& Heron, 2007, p.544). 

Initially, a practitioner pairs a nonverbal stimulus (such as snow falling outside of one's window) with an echoic model ``snow.''  The imitation of this verbal model is differentially reinforced.  Over time this echoic is faded out so that only the presence of the nonverbal stimulus (the snow) sets the occasion for the consumer to label ``snow'' in the absence of a verbal model. A time delay procedure, in which the practitioner gradually delays the presentation of the echoic model after the nonverbal stimulus appears, may be helpful in fading out the verbal model.  

When using tact training, the trainer should:
\begin{enumerate}
\item Ensure that the listener is attending.  Make sure that they are looking in your direction, are making eye contact and that the environment isn't too noisy or distracting.
\item Pair the presentation of nonverbal stimulus that you would like to train with an echoic model.  
\item Pause to allow the listener to process the information and wait for a response.
\item Provide differential reinforcement for responses that are closer and closer approximations to the verbal model. Note:  It may be difficult to differentially reinforce the tacting of events that cannot be shared by both instructor and student, such as private events like pain, as the instructor may not be able to adequately able to label them (Cooper, Howard, \& Heron, 2007).  For this reason it is recommended that initial tact training be done with objects or actions that can be directly observed.
\item Once the client is able to imitate the verbal model in the presence of the stimulus, gradually fade out the verbal model so that only the stimulus itself sets the occasion for the response. 
\end{enumerate}
%
\subsection{Assessment}
\begin{enumerate}
\item Ask the supervisee to state what tact training can be used to teach.
\item Ask the supervisee to give examples of tact training.
\item Ask the supervisee to discuss how tact training should be delivered.
\item Ask the supervisee to state how the verbal model can be successfully faded out.
\end{enumerate}
%
\subsection{Relevant Literature}
\begin{refsection}
\nocite{skinner1957verbal,
    sundberg2001benefits,
    sundberg1998teaching}
\printbibliography[heading=none]
\end{refsection}
%could not verify. used sundberg2001benefits instead.
%Sundberg, M. L., \& Michael, J. (2001). The value of Skinner's analysis of verbal behavior for teaching children with autism. Behavior Modification, 25, 698-724.
%
\subsection{Related Lessons}
\fourdTwelve{}\\
\fourFKFourtyFive{}\\
%
%\clearpage \section{\fourdThirteen{}}
\subsection{Definition}
``Many children with autism, developmental disabilities, or other language delays suffer from defective or nonexistent intraverbal repertoires, even though some can emit hundreds of mands, tacts, and receptive responses'' (Sundberg, as cited in Cooper, Heron, \& Heward, 2007, p. 545).  

Although typically developing children generally acquire this type of language on their own, some learners may not acquire this type of language without specific training in the skill.  In such cases, intraverbal training may be recommended.  Prior to starting intraverbal training, the learner must have a acquired a variety of pre-requisite skills such as being able to mand, tact, engage in echoic behavior or imitation, to receptively identify stimuli, and to do match to sample tasks (Sundberg \& Partington, 1998). The goal is not to teach new language, but to bring words or phrases that are currently under nonverbal stimulus control entirely under verbal stimulus control (Cooper, Heron, \& Heward, 2007).  For instance, a child who has previously learned to tact or echo the word ``cow'' when they see a picture of a cow, may learn to then say ``cow'' when his teacher has asked, ``What says moo?''

It is recommended that simple intraverbal interactions that are appropriate to the child's developmental age be taught before more complicated responses.  Fill in the blank relations are often the easiest to teach first (Cooper et al., 2007).  For instance, a learner may be taught to fill in the word ``star'' after someone has delivered the line ``Twinkle, twinkle little...''    A teacher may start by using visual stimuli and then gradually fade out these prompts as the child is successful so that only the verbal stimulus is presented.  

Since intraverbal behavior is reinforced by generalized conditioned reinforcement (i.e., social reinforcement via praise, eye contact, body language, etc.) it may be challenging to motivate some students initially to engage in the desired response.  Trainers may need to initially pair specific reinforcement (such as a crayon after the child has responded ``crayon'' when asked, ``What do you color with?'') initially and then fade this over time (Cooper et al., 2007). 

Varying both the verbal stimuli and the verbal responses over time will help to strengthen these responses (Cooper et al., 2007).   For instance a child who has learned to respond ``bear'' when asked, ``What is your favorite toy?'' may then learn to respond with more complexity such as ``blue bear'' or ``my blue bear with the purple hat.''  The teacher can also vary the verbal prompt such as asking, ``What toy do you like the most?'' which is simply another way of phrasing ``what is your favorite toy,'' and is a part of the same stimulus class.    

When using intraverbal training, the trainer should:
\begin{enumerate}
\item Be sure that the desired verbal responses are already in the child's or individual's repertoire
\item Ensure that the listener is attending.  Make sure that they are looking in your direction, are making eye contact and that the environment isn't too noisy or distracting.
\item Deliver the verbal stimulus (i.e. ``The itsy bitsy...'') and pause.  Initially pair the verbal response with a nonverbal stimulus that can be faded out over time (such as a picture of a spider or spider puppet).
\item Provide reinforcement for correct responding. Specific reinforcement such as providing edibles, access to the spider puppet, etc. should be faded over time so that social reinforcement becomes the reinforcing consequence.  
\item Once simple intraverbal relations have been established, teach the child to respond to variations to the verbal stimulus (``Who went up the water spout?'') or respond with more complexity.    
\end{enumerate}
%
\subsection{Assessment}
\begin{enumerate}
\item Ask the supervisee to state what intraverbal training can be used to teach.
\item Ask the supervisee to state the prerequisite skills that are needed to teach intraverbal behavior.  
\item Ask the supervisee to give examples of intraverbal training.
\item Ask the supervisee to discuss how intraverbal training should be delivered.
\item Ask the supervisee to state why nonverbal stimuli should be faded out.
\end{enumerate}
%
\subsection{Relevant Literature}
\begin{refsection}
\nocite{cooper2007applied,
    partington1993teaching,
    skinner1957verbal,
    sundberg1998teaching,
    vedora2009teaching}
\printbibliography[heading=none]
\end{refsection}
%
\subsection{Related Lessons}
\fourdThirteen{}\\
\fourFKFourtySix{}\\
%
%\clearpage \section{\fourdFourteen{}}
\subsection{Definition}
Listener – ``someone who provides reinforcement for verbal behavior'' (Cooper, Heron, \& Heward, 2007, p. 698)

Part of being involved in a verbal community is reinforcing the behavior of speakers. There are several methods for training someone to respond as a listener. Skills such as vocal imitation (echoic), following instructions, answering questions (intraverbal), conversation skills (intraverbal), indicating objects, etc., all require listener behavior. 

Sundberg discusses a distinction between intraverbal and listener responding. ``If the child's response were verbal, then it would classified as intraverbal... but if the response were nonverbal it would be classified as listener behavior (or often termed receptive language or receptive labeling)'' (Sundberg, 2008, pp. 11-12).

There are multiple protocols for receptive language training (cf., Fabrizio \& Moors, 2001; Leaf \& McEachin, 1999; Lovaas, 2003; Maurice, Green, \& Luce, 1996). In a review of teaching receptive language to children with autism, Pelios and Sucharzewski (2004) point out that one must consider antecedent manipulations (e.g., within-stimulus prompts, keeping stimulus short, using topographically dissimilar responses) and consequence manipulations (e.g., rich reinforcement schedules, token economies, performance based breaks).  Also, they recommend systematically programming specific antecedent and consequence manipulations and requiring specific response requirements.

\subsection{Examples}
\begin{enumerate}
\item A teacher conducting receptive language training tells her student ``sit down'' the student sits down and the teacher praises the student. The teacher then says ``clap hands'' the student claps hands and the teacher praises the student.
\item A teacher presents an array of fruit and says to the student, ``give me the apple.'' The student gives the teacher the apple and the teacher gives the student a token. 
\end{enumerate}
%
\subsection{Assessment}
\begin{enumerate}
\item Have supervisee read relevant literature on receptive language/listener training.
\item Role play with your supervisee teacher (speaker) and student (listener) examples of receptive language training protocols.
\end{enumerate}
%
\subsection{Relevant Literature}
\begin{refsection}
\nocite{cooper2007applied,
    fabrizio2001brief,
    leaf1999work,
    lovaas2003teaching,
    maurice1996behavioral,
    pelios2004teaching,
    schlinger2008listening,
    sundberg2008vb-mapp}
\printbibliography[heading=none]
\end{refsection}
%
\subsection{Related Lessons}
\fourdTen{}\\
\fourdThirteen{}\\
\fourFKFourtyThree{}\\
\fourFKFourtyFour{}\\
\fourFKFourtyFive{}\\
\fourFKFourtySix{}\\
%
%\clearpage \section{\fourFKFourtyThree{}}
\subsection{Definition}  
Echoic - ``An elementary verbal operant involving a response that is evoked by a verbal discriminative stimulus that has point-to-point correspondence and formal similarity with the response'' (Cooper, Heron, \& Heward, 2007, p. 694).   
%
\subsection{Examples}
\begin{enumerate}
\item Mrs. Platypus is instructing her 3rd grade class on their math facts.  She holds up a card stating that, ``three times nine is eighteen.''  She then restates the fact asking the class to repeat.  The class says, ``three times nine is eighteen'' in unison.  Mrs. Platypus praises the students for their repetition.
\item Mr. Penguin is a kindergarten teacher.  He is working with one student on his reading skills.  He shows little Timmy the letter R.  He tells him that the letter R makes the ``rrr'' sound and asks him to repeat.  Little Timmy says, ``rrr,'' and Mr. Penguin comments, ``Nice job Timmy.''  
\item Mrs. Dodo the art teacher needs one of her students to run to the office and get some supplies. One of the children volunteers.  She tells him that she needs him to get, ``Crayons, markers, and paint.''  He repeats, ``Crayons, markers, and paint.''  ``Exactly,'' Mrs. Dodo says sending him on his way.  
\item (Non-example) Mrs. Platypus is still working on math facts with her class.  She holds up the math fact 4x9= and asks the students to give the answer.  Susie Q raises her hand and answers ``thirty-six.''  
%
\end{enumerate}
%
\subsection{Assessment}
\begin{enumerate}
\item Ask the supervisee to give the definition of an echoic
\item Ask the supervisee to give several examples of echoics
\item Ask the supervisee to give a non-example of an echoic 
%
\end{enumerate}
%
\subsection{Relevant Literature}
\begin{refsection}
\nocite{cooper2014applied,
        sundberg2008verbal,
        skinner1957verbal}
\printbibliography[heading=none]
\end{refsection} 
%
\subsection{Related Lessons}
\fourdFour{}\\
\fourdTen{}\\
%
%\clearpage \section{\fourFKFourtyFour{}}
\subsection{Definition}
Mand - ``An elementary verbal operant that is evoked by an MO and followed by specific reinforcement'' (Cooper, Heron, \& Heward, 2007, p. 699).\\

The form of the response is specific and under control of motivating operations. Response topography can vary: vocal, sign language, augmentative communication, pushing, reaching, hitting, etc.
%
\subsection{Examples}
\begin{enumerate}
\item ``I want a cookie.'' (This is a mand for an item. Mands can include verbs, use of adjectives, prepositions, pronouns etc.)
\item A child says ``watch me'' after learning how to ride a bike independently (mand for attention)
\item Asking questions like ``what's your name? or ``where's the phone?'' (mand for information)
\item Child says, ``No!'' when parent is about to use blender (mand for avoidance of an aversive)
%

%
\end{enumerate}
%
\subsection{Assessment}
\begin{enumerate}
\item Ask your Supervisee to recall how they asked for supervision
\item Ask your Supervisee to list the types of mands they would emit if they were lost in a foreign county and needed directions to a local gas station
\item Ask you supervisee to list 5 ways they use mands in an inappropriate way (eg. complain about work to get attention)
%
\end{enumerate}
%
\subsection{Relevant Literature}
\begin{refsection}
\nocite{cooper2007applied,
        laraway2003motivating,
        michael1988establishing,
        sundberg2001benefits,
        sweeney2007transferring}
\printbibliography[heading=none]
\end{refsection}
%
\subsection{Related Lessons}
\fourdNine{}\\
\fourdEleven{}\\
\fourFKTwentySeven{}\\
\fourFKTwentyEight{}\\
%
%\clearpage \section{\fourFKFourtyFive{}}
\subsection{Definition}
Tact – ``An elementary verbal operant evoked by a nonverbal discriminative stimulus and followed by generalized conditioned reinforcement'' (Cooper, Heron, \& Heward, 2007, p. 705).
%
\subsection{Examples}
\begin{enumerate}
\item Dexter walks outside with his mother and sees birds in a tree.  ``Robins,'' he says.  ``You're right. Those are robins,'' Dexter's mom says. ``Robins'' is a tact.
\item Hector is in the store shopping for Valentines Day.  He sees a variety of flowers before noticing the ones he wants to buy.  ``Red roses,'' Hector says quietly to himself. ``Red roses'' in this context is likely a tact. 
\item Chester goes to his friends Superbowl party.  Upon scanning the array of delicious apps and snacks on the counter, he hones in on one that is his favorite.  ``Ooh, buffalo chicken dip,'' he comments.  ``Buffalo chicken dip'' would be likely a tact in this context.
\item Non-example: Dexter is thinking about buying some cookies the next time he goes to the supermarket.  He writes the word ``cookies'' down on his shopping list. 
%
\end{enumerate}
%
\subsection{Assessment}
\begin{enumerate}
\item Ask the supervisee to define ``tact.''  
\item Ask the supervisee to give several examples of tacts.
\item Ask the supervisee to give a non-example of a tact. Discuss why.
%
\end{enumerate}
%
\subsection{Relevant Literature}
\begin{refsection}
\nocite{cooper2007applied,
        skinner1957verbal}
\printbibliography[heading=none]
\end{refsection}
%
\subsection{Related Lessons}
\fourdTwelve{}\\
%
%\clearpage \section{\fourFKFourtySix{}}
\subsection{Definition} 
Intraverbal – ``An elementary verbal operant that is evoked by a verbal discriminative stimulus and that does not have point-to-point correspondence with that verbal stimulus'' (Cooper, Heron \& Heward, 2007, p. 698).
%
\subsection{Examples}
\begin{enumerate}
\item A new employee shows up for his first day on the job. The man in the cubical next to him asks, ``What is your name?''  ``Harvey,'' the man replies. Saying ``Harvey'' is an intraverbal in that context.
\item Hanks boss stops his office to let him know that his sales were ``outstanding this week.''  ``Thanks. I really put in some long hours,'' Hank notes.  ``Thanks,'' is an intraverbal in that context.
\item (Non-example) The office phone rings. Harvey picks up the phone and answers ``Hello.'' There is no one on the line so he hangs up and keeps working. 
%
\end{enumerate}
%
\subsection{Assessment}
\begin{enumerate}
\item Ask the supervisee to define ``intraverbal''  
\item Ask the supervisee to give several examples of intraverbals.
\item Ask the supervisee to give a non-example of an intraverbal.
%
\end{enumerate}
%
\subsection{Relevant Literature}
\begin{refsection}
\nocite{cooper2007applied,
        partington1993teaching,
        skinner1957verbal}
\printbibliography[heading=none]
\end{refsection}
%
\subsection{Related Lessons}
\fourdThirteen{}\\
