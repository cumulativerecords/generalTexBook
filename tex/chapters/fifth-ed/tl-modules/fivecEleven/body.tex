%\clearpage \section[\fouriFive{}]{\fouriFive{}%
              \sectionmark{I-05 Organize, Analyze...}}
\sectionmark{I-05 Organize, Analyze...}
%\clearpage \section{\fouriFive{}}
\subsection{Definition}
Organize the data
\begin{enumerate}
\item Once data have been collected in their raw format, it is then important to organize the data into a format that is easy to analyze (Cooper, Heron \& Heward, 2007). The most effective way to do this, and the most common method utilized by Behavior Analysts, is to visually display the data in a graph (Cooper, Heron, \& Heward, 2007).
\item As Parsonson and Baer (1978, p. 134) said ``the function of the graph is to communicate... in an attractive manner, descriptions and summaries of the data that enable rapid and accurate analysis of the facts'' (cited from Cooper, Heron \& Heward, 2007, p. 128).
\item The visual formats most often used by behavior analysts are line graphs, bar graphs, cumulative records, semilogarithmic charts and scatterplots* (Cooper, Heron \& Heward, 2007).
\end{enumerate}
%
Analyze and interpret observed data\\

Behavior analysts use a systematic form of assessing graphically displayed data called visual analysis (Cooper, Heron \& Heward, 2007). 

Visual analysis encompasses examining each of three characteristics in a graphic display of data, both within and across the different conditions and phases of an experiment. These three characteristics are:

\begin{enumerate}
\item The level of the data
\item The extent and type of variability in the data 
\item The trends in the data
\end{enumerate}

Johnston and Pennypacker (1993b, p. 320) recommend that the viewer should carefully examine the graph's overall construction, paying attention to details such as the axis labels and the scaling of each axis, prior to attempting to interpret the data. They argue ``it's impossible to interpret graphic data without being influenced by various characteristics of the graph itself'' (cited from Cooper, Heron \& Heward, 2007, p. 149).
%
Visual analysis within conditions\\
\begin{enumerate}
\item Within given conditions, examination needs to occur to determine a few relevant factors (Cooper, Heron \& Heward, 2007):
\item The number of data points in each condition (in general, the more measurements of the dependent variable there are per unit of time, the more confidence one can have in the data).
\item Variability (A high degree of variability usually indicates little control has been achieved over the factors influencing behavior).
\item Level (examined in terms of its absolute value within a condition, the degree of stability/variability and the extent of change from one level to another).
\item Trend (the trend indicates whether a particular behavior has increased, decreased or has neither increased nor decreased within a condition).
\end{enumerate}

Visual analysis between conditions\\
\begin{enumerate}
\item After examining the data within each condition or phase of a study, visual analysis now proceeds to examining the data between conditions (Cooper, Heron \& Heward, 2007):
\item Comparison needs to be made between the different conditions of the level, trend and variability of the data (Cooper, Heron \& Heward, 2007, p. 154).
\item The data are examined in terms of the overall level of performance between conditions; generally when there is no overlap of data points between the highest values in one condition and the lowest values in another condition, there is a strong likelihood that the behavior changed from one condition to the next (Cooper, Heron \& Heward, 2007, p. 154).
\end{enumerate}

Once an ``examination and comparison of changes in level, trend and variability between conditions has occurred, a comparison needs to be made of performance across similar conditions'' (Cooper, Heron \& Heward, 2007, p. 155). If a behavior change is found to have occurred over the course of an intervention, the next question to be asked is, ``was the behavior change a result of the intervention?'' (Cooper, Heron \& Heward, 2007, p. 155).

\subsection{Assessment}
\begin{enumerate}
\item Ask your Supervisee to explain why it's important as a behavior analyst to organize and interpret observed data. 
\item Ask your Supervisee to organize a set of data and display it graphically in the most appropriate way. 
\end{enumerate}
%
\subsection{Relevant Literature}
\begin{refsection}
\nocite{cooper2007applied,
        fisher2014handbook,
        johnston1993strategies,
        parsonson1978analysis}
\printbibliography[heading=none]
\end{refsection}
%
\subsection{Related Lessons}
\fouraTen{}\\
\fouraEleven{}\\
\fourbFour{}\\
\fourbFive{}\\
\fourbSix{}\\
\fourbSeven{}\\
\fourbEight{}\\
\fourbNine{}\\
\fourjFifteen{}\\
%
Footnotes
* See Cooper, Heron \& Heward (2007), pages 129 – 154 for more information on these graphic displays.
%
%\clearpage \section[\fourjFifteen{}]{\fourjFifteen{}%
              \sectionmark{J-15 Base decision-making...}}
\sectionmark{J-15 Base decision-making...}
\subsection{Definition}
Behavior analysts make decisions during assessment and intervention based on data. Graphic displays (e.g., line graphs, bar graphs, and cumulative graphs) aid accurate and efficient interpretation of quantitative data and facilitate communication with others. ``The primary function of a graph is to communicate without assistance from the accompanying text'' (Spriggs \& Gast, 2010, p. 167). Line graphs are most often used by behavior analysts to show effects and possible functional relations between intervention (independent variable) and a defined behavior (dependent variable). Bar graphs are often used by behavior analysts to summarize or compare discrete aspects of recorded behavior. Cumulative graphs show the rate of change in responding across time. Although they may be used with duration or latency data, they are most often used to show frequency data. Behavior analysts often use tables to summarize data or other information. ``An informative table supplements—rather than duplicates—the text'' (APA, 2010). As with graphs, a table should communicate efficiently but include enough information to be understood alone without explanations in the text.
%
\subsection{Examples}
\begin{enumerate}
\item A behavior analyst interprets the effectiveness of a constant time delay procedure for teaching a student 10 sight words using the following line graph: 
\item A behavior analyst uses a table to summarize the number of times a child chose a specific toy during seven sessions of free-play with four toys available. 
\end{enumerate}
%
\subsection{Assessment}
\begin{enumerate}
\item Ask the supervisee to interpret results of the constant time delay intervention using only the graphic data. Would the supervisee recommend CTD for teaching discrete skills to this student? Ask the supervisee to write a title for this graph.
\item Ask the supervisee to interpret information in the table to decide which toys she would place in a free time area for this child to enjoy. Would the supervisee consider adding different types of any one toy to the playtime area based on this data? If so, which type? If not, why not? Ask the supervisee to write a title for this table.
%
\end{enumerate}
%
\subsection{Relevant Literature}
\begin{refsection}
\nocite{cooper2007applied,
        spriggs2010visual,
        american2010publication}
\printbibliography[heading=none]
\end{refsection}
%
\subsection{Related Lessons}
\fouraTen{}\\
\fouraEleven{}\\
\fouraTwelve{}\\
\fouraFourteen{}\\
\fourbThree{}\\
\fourhOne{}\\
\fourhThree{}\\
\fourhFour{}\\
\fouriFive{}\\
\fouriSeven{}\\
\fourjOne{}\\
\fourkFour{}\\
\fourFKThirtyThree{}\\
%
%\clearpage \section[\fourhThree{}]{\fourhThree{}%
              \sectionmark{H-03 Select a data display...}}
\sectionmark{H-03 Select a data display...}
\subsection{Definition}
Behavior change is an ongoing process that must be continuously evaluated.  This evaluation occurs through an analysis of data that reflects the quantifiable form of the behavior of interest.  However, understanding the extent of behavior change can be difficult if one is looking at raw data alone.  As such, behavior analysts use graphic displays to analyze, interpret, and communicate the results of behavior interventions (Cooper, Heron, \& Heward, 2007).  The most common graphic displays include line graphs, bar graphs, cumulative records, Standard Celeration charts, and scatterplots.  The clinical utility of each graphic display varies so it is important to select the graphic display that will most accurately illustrate what the behavior analyst wants to understand. 

Cooper et al. (2007) outlines the following purposes for the different graphic displays.  

Line graphs are the most common form of graphic display and can be used to (1) show multiple dimensions of one behavior, (2) two or more different behaviors, (3) a behavior under different conditions, (4) changes in the target behavior relative to the manipulation of an independent variable, (5) and the behavior of multiple learners. 

Bar graphs are typically used to (1) display discrete data that cannot be captured by an underlying dimension reflected on a horizontal axis and (2) provide an easy comparison of variables during different conditions. 

Cumulative records are useful when the behavior analyst wants to (1) illustrate the total number of responses made over time, (2) the graph is used as a means to provide feedback to the learner, (3) the behavior of interest can only occur once during the specified measurement period, and (4) an analysis of a specific instance during an experiment is warranted.

The Standard Celeration Chart is a semilogarithmic chart that is used to reflect a linear measure of change across time.  Lastly, scatterplots illustrate the comparative distribution of discrete measures in a data set and can be useful to uncover relationships across different subsets of data.
%
\subsection{Examples}
\begin{enumerate}
\item If you want to see data paths across three behaviors and different intervention conditions, then a line graph is the most appropriate graphic display.
\item Your client has rapidly acquired several language targets.  A cumulative record can illustrate acquisition and is more efficient than creating a line graph for each acquired language target.
%
\end{enumerate}
%
\subsection{Assessment}
\begin{enumerate}
\item Ask supervisee to explain the utility of each type of graphic display.
\item Provide supervisee with examples of graphs and ask supervisee to identify which type of display is used and interpret the data presented in the display.
\item Ask supervisee to create at least one of each type of graphic display.
\item Provide examples of different measures of behavior and what information is desired from each set of data and ask supervisee to identify which graphic display would be most appropriate to communicate the results.
%
\end{enumerate}
%
\subsection{Relevant Literature}
\begin{refsection}
\nocite{cooper2007applied,
        parsonson1978analysis}
\printbibliography[heading=none]
\end{refsection}
%
\subsection{Related Lessons}
\fouraTen{}\\
\fouraEleven{}\\
\fouraTwelve{}\\
\fourFKFourtySeven{}\\
\fourbThree{}\\
\fourhFour{}\\
\fourhFive{}\\
