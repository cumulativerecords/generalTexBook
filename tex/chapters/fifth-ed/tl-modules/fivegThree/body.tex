\clearpage \section[\foureEleven{}]{\foureEleven{}%
              \sectionmark{E-11 Use pairing procedures...}}
\sectionmark{E-11 Use pairing procedures...}
``Stimulus events or conditions that are present or that occur just before or simultaneous with the occurrence of other reinforcers (or punishers) may acquire the ability to reinforce (or punish) behavior when they later occur on their own as consequences. Called conditioned reinforcers or conditioned punishers, these stimulus changes function as reinforcers and punishers only because of their prior pairing with other reinforcers or punishers'' (Cooper et al., 2007, p. 40).

\subsection{Definition}
Conditioned reinforcer – ``...a previously neutral stimulus change that has acquired the capability to function as a reinforcer through stimulus-stimulus pairing with one or more unconditioned or conditioned reinforcers'' (Cooper et al., 2007, p. 269).

Some common conditioned reinforcers include social praise, tokens, and money because they are often paired with other reinforcers. 

Conditioned reinforcers become stronger the more they are paired with other known reinforcers. For instance, paper money will likely not function as a reinforcer until he/she buy toys, candy, and other things that he enjoys with it. The more the child uses money to buy items that are appetitive, the more the paper money becomes a conditioned reinforcer.

The more items the conditioned reinforcer can \textit{buy}, the less sensitive to motivating operations they become. This is called generalized conditioned reinforcement. The more reinforcers that have been paired with the stimulus, the more generalized the conditioned reinforcer is.
%
\subsection{Assessment}
\begin{enumerate}
\item Have supervisee create a token economy and explain how it would be used as a conditioned reinforcer in their professional setting. 
\item Have supervisee create a list of conditioned and unconditioned reinforcers and explain the difference between the two. 
\item Have supervisee explain the concept of a generalized conditioned reinforcer and how this is different from typical conditioned reinforcers.
%
\end{enumerate}
%
\subsection{Relevant Literature}
\begin{refsection}
\nocite{cooper2007applied,
        williams1978effects,
        engelmann1975your,
        morse1977determinants}
\printbibliography[heading=none]
\end{refsection}
%
\subsection{Related Lessons}
\fourcOne{}\\
\fourdOne{}\\
\fourdTwo{}\\
\foureTwo{}\\
\fourfTwo{}\\
\fouriSeven{}\\
\fourjFour{}\\
\fourjEleven{}\\
\fourFKFourteen{}\\
\fourFKSixteen{}\\
\fourFKEighteen{}\\
\fourFKTwentyOne{}\\
\fourFKTwentySeven{}\\
\fourFKThirtyFour{}\\
