\subsection{Assessment}
\begin{enumerate}
\item Provide the supervisee with several target behaviors and their respective function. Have him/her select which differential reinforcement procedure(s) would be the most appropriate for each and why.  Review it and provide feedback. 
\item Have the supervisee describe the benefits of each differential reinforcement procedure.
\item Have the supervisee list the conditions in which the use of each variation would not be desirable. 
\item Provide supervisee with an article from the relevant literature regarding DRA or DRI and discuss the alternative or incompatible behavior the authors selected. In addition, ask them to come up with other alternative or incompatible behaviors which could have been used in the study.
\item Provide supervisee with an article from the relevant literature and discuss if the reinforcer selected by the authors is functional or arbitrary. In addition, ask the supervisee to come up with other putative reinforcers which the study could have used. Finally, discuss the pros and cons of using an arbitrary reinforcer and functional reinforcers. 
\end{enumerate}
%
\subsection{Assessment}
\begin{enumerate}
\item Have supervisee identify the advantages and disadvantages of functional communication training.
\item Have supervisee identify some common guidelines for using FCT.
\item Have supervisee describe instance when he/she used FCT in a professional setting.
\item Have supervisee describe how functional behavior assessment (FBA) and differential reinforcement of alternative behaviors (DRA) relate to the use of FCT.
%
\end{enumerate}
%
