\subsection{Examples}
\begin{enumerate}
\item DRO: Providing a toy following the absence of inappropriate vocalizations for 5 minutes which decreases inappropriate vocalizations. 
\item DRA: Providing a break for handing over a break card which increases the use of the break card in the future. 
\item DRI: Providing social attention for having hands in their own pant pockets which subsequently decreases scratching at caregivers hands.
\item DRH: A student typically only completes one math worksheet per class period. Providing a break with a preferred item contingent on finishing three math worksheets, which increases the number of worksheets completed by a student. The student only gets the preferred item when they complete three worksheets.
\item DRL: Providing attention when a student says ``excuse me'' 2 times every 10 minutes and not providing attention if the behavior occurs more frequently within that 10 minute period which maintains low rates of the behavior.
\end{enumerate}
%
\subsection{Examples}
\begin{enumerate}
\item Rob was throwing books at his teacher every time he was asked to do a math worksheet. After completing a functional analysis, Rob's teacher found throwing books was maintained by access to escape. Rob was taught to ask for a break when he was doing math instead of throwing something at his teacher. This response, paired with pre-teaching and prompt fading, helped replace the problematic behavior.
\end{enumerate}
%
