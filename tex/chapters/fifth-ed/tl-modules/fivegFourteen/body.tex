\clearpage \section[\fourdTwentyOne{}]{\fourdTwentyOne{}%
              \sectionmark{D-21 Use differential reinforcement...}}
\subsection{Definition} 
Differential Reinforcement - ``Reinforcing only those responses within a response class that meet a specific criterion along some dimension(s) (i.e., frequency, topography, duration, latency or magnitude) and placing all other behaviors in the class on extinction'' (Cooper, Heron, \& Heward, 2007, p. 693).

Five common variations of differential reinforcement are: 

Differential reinforcement of other behavior (DRO) is a procedure that provides reinforcement for the absence of problem behavior during a period of time (interval) or at a specific time (momentary) (Cooper, Heron, \& Heward, 2007).

Differential reinforcement of alternative behavior (DRA) is the reinforcement of a response that is an appropriate alternative to problem behavior (Cooper, Heron, \& Heward, 2007).

Differential reinforcement of incompatible behavior (DRI) is the reinforcement of a response that is physically incompatible with the target problem behavior (Cooper, Heron, \& Heward, 2007). 

Differential reinforcement of high rates (DRH) is reinforcement contingent upon a behavior occurring at a set high rate used to increase the overall rate of a behavior (Cooper, Heron, \& Heward, 2007).

Differential reinforcement of low rates of behavior (DRL) is reinforcement contingent upon behavior occurring at a set reduced rate used to decrease the overall rate of a behavior but not to eliminate it completely (Cooper, Heron, \& Heward, 2007).
%
\subsection{Examples}
\begin{enumerate}
\item DRO: Providing a toy following the absence of inappropriate vocalizations for 5 minutes which decreases inappropriate vocalizations. 
\item DRA: Providing a break for handing over a break card which increases the use of the break card in the future. 
\item DRI: Providing social attention for having hands in their own pant pockets which subsequently decreases scratching at caregivers hands.
\item DRH: A student typically only completes one math worksheet per class period. Providing a break with a preferred item contingent on finishing three math worksheets, which increases the number of worksheets completed by a student. The student only gets the preferred item when they complete three worksheets.
\item DRL: Providing attention when a student says ``excuse me'' 2 times every 10 minutes and not providing attention if the behavior occurs more frequently within that 10 minute period which maintains low rates of the behavior.
\end{enumerate}
%
\subsection{Assessment}
\begin{enumerate}
\item Provide the supervisee with several target behaviors and their respective function. Have him/her select which differential reinforcement procedure(s) would be the most appropriate for each and why.  Review it and provide feedback. 
\item Have the supervisee describe the benefits of each differential reinforcement procedure.
\item Have the supervisee list the conditions in which the use of each variation would not be desirable. 
\item Provide supervisee with an article from the relevant literature regarding DRA or DRI and discuss the alternative or incompatible behavior the authors selected. In addition, ask them to come up with other alternative or incompatible behaviors which could have been used in the study.
\item Provide supervisee with an article from the relevant literature and discuss if the reinforcer selected by the authors is functional or arbitrary. In addition, ask the supervisee to come up with other putative reinforcers which the study could have used. Finally, discuss the pros and cons of using an arbitrary reinforcer and functional reinforcers. 
\end{enumerate}
%
\subsection{Relevant Literature}
\begin{refsection}
\nocite{cooper2007applied,
        cowdery1990effects,
        deitz1977analysis,
        deitz1983reducing,
        hanley2001reinforcement,
        kahng2001assessment,
        lalli1995reducing,
        lindberg1999dro,
        mazaleski1993analysis,
        petscher2009review,
        rehfeldt2003functional,
        seys1978improving,
        vollmer1992differential,
        vollmer1999evaluating}
\printbibliography[heading=none]
\end{refsection}
\subsection{Related Tasks} 
\fourcOne{}\\
\fourdTwo{}\\
\fourdNineteen{}\\
\fourfSeven{}\\
\fouriSeven{}\\
\fourjTwo{}\\
%
\clearpage \section{\fourfSeven{}}
\subsection{Definition}
Functional communication training (FCT) - ``...an application of differential reinforcement of alternative behaviors (DRA) because the intervention develops an alternative communicative response as an antecedent to diminish the problem behavior'' (Fisher, Kuhn, \& Thompson, 1998, p. 543).

The alternative response can include vocalizations, sign language, communication boards and devices, picture cards, or gestures.

Carr and Durand (1985) used a two-step process to demonstrate how to deliver FCT.  First they completed a functional behavior assessment to identify the stimuli with known reinforcing properties that maintain the problem behavior, and second, they used those stimuli as reinforcers to develop an alternative behavior to replace the problem behavior.

Guidelines for the effective use of functional communication training include providing a dense schedule of reinforcement, fading prompts, and the appropriate reinforcement schedule thinning after the response is at strength.
%
\subsection{Examples}
\begin{enumerate}
\item Rob was throwing books at his teacher every time he was asked to do a math worksheet. After completing a functional analysis, Rob's teacher found throwing books was maintained by access to escape. Rob was taught to ask for a break when he was doing math instead of throwing something at his teacher. This response, paired with pre-teaching and prompt fading, helped replace the problematic behavior.
\end{enumerate}
%
\subsection{Assessment}
\begin{enumerate}
\item Have supervisee identify the advantages and disadvantages of functional communication training.
\item Have supervisee identify some common guidelines for using FCT.
\item Have supervisee describe instance when he/she used FCT in a professional setting.
\item Have supervisee describe how functional behavior assessment (FBA) and differential reinforcement of alternative behaviors (DRA) relate to the use of FCT.
%
\end{enumerate}
%
\subsection{Relevant Literature}
\begin{refsection}
\nocite{cooper2007applied,
        durand1999functional,
        carr1985reducing,
        durand1992analysis,
        fisher1998establishing,
        hanley2001reinforcement}
\printbibliography[heading=none]
\end{refsection}
%
\subsection{Related Tasks}
\fourdTwo{}\\\
\fourdThree{}\\
\fourdFour{}\\
\fourdFive{}\\
\fourdTen{}\\
\fourdEleven{}\\
\fourdTwentyOne{}\\
\foureOne{}\\
\fouriThree{}\\
\fouriFour{}\\
