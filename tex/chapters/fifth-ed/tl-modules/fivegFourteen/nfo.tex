%\clearpage \section[\fourdTwentyOne{}]{\fourdTwentyOne{}%
\subsection{Definition} 
Differential Reinforcement - ``Reinforcing only those responses within a response class that meet a specific criterion along some dimension(s) (i.e., frequency, topography, duration, latency or magnitude) and placing all other behaviors in the class on extinction'' (Cooper, Heron, \& Heward, 2007, p. 693).

Five common variations of differential reinforcement are: 

Differential reinforcement of other behavior (DRO) is a procedure that provides reinforcement for the absence of problem behavior during a period of time (interval) or at a specific time (momentary) (Cooper, Heron, \& Heward, 2007).

Differential reinforcement of alternative behavior (DRA) is the reinforcement of a response that is an appropriate alternative to problem behavior (Cooper, Heron, \& Heward, 2007).

Differential reinforcement of incompatible behavior (DRI) is the reinforcement of a response that is physically incompatible with the target problem behavior (Cooper, Heron, \& Heward, 2007). 

Differential reinforcement of high rates (DRH) is reinforcement contingent upon a behavior occurring at a set high rate used to increase the overall rate of a behavior (Cooper, Heron, \& Heward, 2007).

Differential reinforcement of low rates of behavior (DRL) is reinforcement contingent upon behavior occurring at a set reduced rate used to decrease the overall rate of a behavior but not to eliminate it completely (Cooper, Heron, \& Heward, 2007).
%
%\clearpage \section{\fourfSeven{}}
\subsection{Definition}
Functional communication training (FCT) - ``...an application of differential reinforcement of alternative behaviors (DRA) because the intervention develops an alternative communicative response as an antecedent to diminish the problem behavior'' (Fisher, Kuhn, \& Thompson, 1998, p. 543).

The alternative response can include vocalizations, sign language, communication boards and devices, picture cards, or gestures.

Carr and Durand (1985) used a two-step process to demonstrate how to deliver FCT.  First they completed a functional behavior assessment to identify the stimuli with known reinforcing properties that maintain the problem behavior, and second, they used those stimuli as reinforcers to develop an alternative behavior to replace the problem behavior.

Guidelines for the effective use of functional communication training include providing a dense schedule of reinforcement, fading prompts, and the appropriate reinforcement schedule thinning after the response is at strength.
%
