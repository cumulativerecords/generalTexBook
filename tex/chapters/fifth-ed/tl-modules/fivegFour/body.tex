%\clearpage \section{\fourdThree{}}
\subsection{Definition}
Prompts – ``...antecedent stimuli that increase the probability of a desired response'' (Piazza, \& Roane, 2014, p. 256)

Prompt fading – ``...transfer stimulus control from therapist delivered prompts to stimuli in the natural environment that should evoke appropriate responses'' (Walker, 2008 as cited in Fisher, Piazza, \& Roane, 2014, p. 412).

Prompts are used when teaching skills. Prompts can be used when teaching in task analysis, discrete trial, incidental teaching, etc. Prompt fading is important as the learner begins to show competence with the skill being taught. Fading allows the learner to become independent and meet naturalistic reinforcers for his/her behavior.

Prompts are generally divided into two categories: stimulus prompts and response prompts.

Stimulus prompts – ``...those in which some property of the criterion stimulus is altered, or other stimuli are added to or removed from the criterion stimulus'' (Etzel \& LeBlanc, 1979 cited in Fisher, Piazza, \& Roane, 2014, p. 256)
\subsection{Examples}
\begin{enumerate}
\item Response prompts – ``...addition of some behavior on the part of an instructor to evoke the desired learner behavior'' (Fisher, Piazza, \& Roane, 2014, p. 256).
\item Most-to-least prompting
\item Least-to-most prompting
\item Time delay prompts
\end{enumerate}
%
\subsection{Assessment}
\begin{enumerate}
\item Ask your supervisee to role-play several types of prompt strategies.
\item Ask your supervisee to role-play several types of prompt fading procedures.
\item Ask your supervisee to describe the transfer of stimulus control when a prompt is faded out e.g., what stimulus is controlling behavior while prompting vs what stimulus is controlling behavior after the prompt has been faded out.
\end{enumerate}
%
\subsection{Relevant Literature}
\begin{refsection}
\nocite{etzel1979simplest,
    fisher2014handbook,
    walker2008constant}
\printbibliography[heading=none]
\end{refsection}
%
\subsection{Related Lessons}
\fourdFour{}\\
\fourdFive{}\\
\fourdSix{}\\
\fourdSeven{}\\
\fourdEight{}\\
\foureOne{}\\
\foureTwo{}\\
\foureThirteen{}\\
\fourFKTwentyFour{}\\
%
%\clearpage \section{\foureTwelve{}}
\subsection{Definition}
Errorless learning - an ``approach whereby the task is manipulated to eliminate/reduce errors. Tasks are executed in such a way that the subject is unlikely to make errors'' (Fillingham, Hodgson, Sage, \& Ralph, 2003, p. 339). 

Errorless learning techniques include most-to-least prompt fading or stimulus shaping/fading techniques. Prompts are removed gradually as the individual becomes more adept with the skill, thereby reducing the likelihood of errors.  To apply errorless learning, behavioral strategies utilized may include: response prevention (e.g. only S+ is presented allowing for only correct responding or physical guidance is provided with instruction so incorrect responses are not possible); verbal prompt fading; modeling; stimulus fading (e.g. emphasizing a physical dimension of the stimuli to evoke a correct response such as by illuminating the correct selection, S+, and presenting the incorrect selection, S-, in a dimmer format); or stimulus shaping (e.g. increasing likelihood of correct responding by gradually changing the shape of the stimulus to maintain correct responding). 

The advantages of errorless learning include that it removes negative side effects involved with trial-and-error learning and that it is proven particularly effective among individuals that suffer from brain damage or have a developmental disorder. The disadvantages include cost, time-intensity, and maybe considered less natural than trial-and-error learning (Mueller, Palkovic \& Maynard, 2007).

Trial-and-error learning, being presented with stimuli in which both the correct selection (S+) and incorrect selection (S-) are available, can lead to adverse side effects due to the possibility of incorrect responding and failure to access reinforcers. Research has shown that this can result in aggression, negative emotional responses and stimulus overselectivity (Mueller et al., 2007).
%
\subsection{Assessment}
\begin{enumerate}
\item Have supervisee demonstrate the difference between trial-and-error learning and errorless learning on the job or during role-play.
\item Have supervisee describe how and when prompts will be faded to promote independent responding. 
%
\end{enumerate}
%
\subsection{Relevant Literature}
\begin{refsection}
\nocite{fillingham2003application,
        mueller2007errorless,
        terrace1963errorless}
\printbibliography[heading=none]
\end{refsection}
%
\subsection{Related Lessons} 
\fourdThree{}\\
\fourdFour{}\\
\fourFKTwentyFour{}\\
