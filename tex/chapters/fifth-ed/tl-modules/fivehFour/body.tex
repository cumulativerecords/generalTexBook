\clearpage \section[\fourjTen{}]{\fourjTen{}%
              \sectionmark{J-10 When a behavior...}}
\sectionmark{J-10 When a behavior...}
\subsection{Definition}
Differential reinforcement of alternative behavior (DRA) -  ``a procedure for decreasing problem behavior in which reinforcement is delivered for a behavior that serves as desirable alternative to the behavior targeted for reduction and withheld following instances of the problem behavior'' (Cooper, Heron \& Heward, 2007, p. 693).  

When choosing a replacement behavior, look at behavior that would serve the same function or would meet the same reinforcers for the problem behavior. This response could be using vocal-verbal behavior, exchanging a symbol, using sign language, etc. Consider the pros and cons of each mode. How quickly can the response be taught? Is it likely to be less or more effortful than the problem behavior? Will the new response meet reinforcement in natural settings?

Initially, the response to be taught should be reinforced on a continuous reinforcement schedule to ensure that the individual makes steady contact with the reinforcer and that this new replacement behavior occurs often and becomes strengthened in the individual's repertoire. Once the new response is at strength, the DRA schedule should be thinned to reflect reinforcement rates that occur in the natural environment (e.g., a child learning to request for a break in lieu of eloping will not likely be granted a break every time he/she asks for one in a typical classroom environment). 

Challenging behavior may resurge when DRA schedules are thinned. There are several techniques that can be utilized to decrease this occurrence: increase the response requirement of the alternative response (e.g., if the alternate response is to ask for a break, allow a break only after completion of a set amount of work), provide a delay for reinforcement for the alternative response (e.g., provide a break after several minutes), decrease availability of alternative response materials (e.g. if break is requested utilizing a break card, limit amount of breaks or decrease presentation), and use of a multiple schedule of reinforcement such as providing more reinforcement for completing work than asking for a break (Sweeney \& Shahan, 2013). 
%
\subsection{Assessment}
\begin{enumerate}
\item At a job or during role-play, ask supervisee what would be an appropriate alternative behavior that would allow the individual to access same or similar reinforcer. Ask about alternative responses for a variety of reinforcer types.
\item Have your supervisee list considerations when selecting an appropriate alternative replacement behavior
\item At a job or during role-play, have your supervisee design criteria for when DRA schedule should be thinned and what that process should look like.
%
\end{enumerate}
%
\subsection{Relevant Literature}
\begin{refsection}
\nocite{athens2010investigation,
        cooper2007applied,
        sweeney2013effects,
        vollmer1999evaluating}
\printbibliography[heading=none]
\end{refsection}
%
\subsection{Related Tasks}
\fourdTwo{}\\
\fourdEighteen{}\\
\fourdNineteen{}\\
\fouriSix{}\\
\fourjSix{}\\
\fourjSeven{}\\
