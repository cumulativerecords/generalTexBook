%\clearpage \section{\fourdSeven{}}
\subsection{Definition}
Task analysis - ``The process of breaking a complex skill into smaller, teachable units, or the product of which is a series of sequentially ordered steps or tasks'' (Cooper, Heron, \& Heward, 2007, p. 706).

A task analysis can be taught using any of the following chaining procedures:
\begin{enumerate}
\item Forward chaining -  behaviors are taught in sequential order; reinforcement would occur once criteria for first behavior in the task analysis is achieved
\item Backward chaining – all steps in the task analysis are done by trainer except for the final behavior; reinforcement would occur once criteria for last behavior in chain is achieved.  Subsequent teaching trials would involve providing reinforcement after the next-to-last behavior is achieved, etc.
\item  Backward chaining with leaps- similar to backwards chaining, however, some steps are in the task analysis are not taught, and perhaps just probed in order to decrease the total amount of time spent teaching a skill.  If there are some behaviors in the task analysis are mastered, a trainer could ``leap'' ahead a few steps to allow for more independence and maintenance of previously acquired skills.
\item Total task presentation – prompting would occur at any point in the task analysis when a person is unable to complete any part of steps independently.
\end{enumerate}
%
