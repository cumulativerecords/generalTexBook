%\clearpage \section{\fourdSeven{}}
\subsection{Definition}
Task analysis - ``The process of breaking a complex skill into smaller, teachable units, or the product of which is a series of sequentially ordered steps or tasks'' (Cooper, Heron, \& Heward, 2007, p. 706).

A task analysis can be taught using any of the following chaining procedures:
\begin{enumerate}
\item Forward chaining -  behaviors are taught in sequential order; reinforcement would occur once criteria for first behavior in the task analysis is achieved
\item Backward chaining – all steps in the task analysis are done by trainer except for the final behavior; reinforcement would occur once criteria for last behavior in chain is achieved.  Subsequent teaching trials would involve providing reinforcement after the next-to-last behavior is achieved, etc.
\item  Backward chaining with leaps- similar to backwards chaining, however, some steps are in the task analysis are not taught, and perhaps just probed in order to decrease the total amount of time spent teaching a skill.  If there are some behaviors in the task analysis are mastered, a trainer could ``leap'' ahead a few steps to allow for more independence and maintenance of previously acquired skills.
\item Total task presentation – prompting would occur at any point in the task analysis when a person is unable to complete any part of steps independently.
\end{enumerate}
%
\subsection{Examples}
\begin{enumerate}
\item Accessing an iPhone 6
\item Press home key (circle) 
\item Slide finger across bottom of screen, from left to right, over text that reads ``slide to unlock''
\item Enter passcode
\end{enumerate}
%
Washing a cup
\begin{enumerate}
\item Turn on tap
\item Pick up sponge and put dish soap on sponge
\item Put down dish soap
\item Pick up dirty cup, scrub outside rim, sides, and base of cup with sponge
\item Scrub inside, base, and rim of cup with sponge
\item Put sponge down
\item Rinse cup with warm water
\item Place cup in drainboard 
\item Rinse excess soap out of sponge
\item Turn off tap
\item Put sponge in holder/in sink
\end{enumerate}
%
\subsection{Assessment}
\begin{enumerate}
\item Ask your Supervisee to create a task analysis for a technology related task (e.g. faxing, scanning, etc.).
\item Ask your Supervisee to create a task analysis for eating with a utensil. 
\item Have Supervisee create a task analysis on a skill that can be observed on the job. Once complete, have Supervisee observe an individual completing the task. Supervisee should make note of any differences in sequences of behavior or steps.
\item Have Supervisee follow a task analysis for a task they are not familiar with.
\end{enumerate}
%
\subsection{Relevant Literature}
\begin{refsection}
\nocite{cooper2007applied,
    resnick1973task,
    bancroft2011comparison,
    jerome2007effects,
    slocum2011assessment}
\printbibliography[heading=none]
\end{refsection}
%
\subsection{Related Lessons}
\fourdThree{}\\
\fourdFive{}\\
\fourdSix{}\\
\fourdSeven{}