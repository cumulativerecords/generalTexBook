\subsection{Examples}
\begin{enumerate}
\item An experiment that entails exposing a participant to a condition of no programmed reinforcement for a work task (baseline) until steady state is achieved, then exposes a participate to a condition in which they earn stickers contingent on a work task (intervention) and then repeats these two conditions respectively. 
\item An experiment in which baseline consists of the reinforcement of challenging behavior and the treatment consists of differential reinforcement of an alternative/replacement behavior and both conditions are replicated at least twice. 
\end{enumerate}
%
\subsection{Examples}
\begin{enumerate}
\item An experiment that entails conducting DRA and no programmed treatment during alternating sessions to compare treatment to the no treatment. 
\item An experiment that entails conducting DRI, DRO and DRA on alternating days to compare all treatments to each other. 
\end{enumerate}
%
\subsection{Examples}
\begin{enumerate}
\item Jim created a class wide reinforcement program to increase the vocabulary test scores of a 1st grade class. He wanted to make sure that the reinforcement program was effective so he set specific score criterion for the class, to monitor their progress. Average baseline test scores were 55\% correct for the entire class. Jim set the first criterion phase at 70\% of the test questions answered correctly for the entire class. After 4 weeks, the class met these criteria for 3 consecutive tests, so Jim set the classroom performance criterion to 80\% of the test questions answered correctly. This time the class met the criterion in 3 weeks and Jim increased the criterion to 90\%. Once again, the class met these criteria for 3 consecutive tests. Jim concluded that his intervention was likely responsible for the change in test scores, since the test score reliably increased when the criteria were altered and required a greater score.
\item (Non-example) John created a reinforcement program to increase Larry's rate of answering questions during class. After 3 weeks, the data indicated that Larry was answering more questions appropriately in class. However, there was a new teacher in the classroom and other variables that may have accounted for this change. John wanted to see if the program was increasing this behavior, so he decided to remove the reinforcement program for a week to see if Larry's rate of answering questions decreased.
\end{enumerate}
%
\subsection{Examples}
\begin{enumerate}
\item Rod conducted an FBA on Billy's aggression and property destruction. Both behaviors were determined to be maintained by escape from demands. Rod decided to implement the same intervention for each behavior using a multiple baseline design because they both served the same function and a reversal would possibly reestablish the dangerous behavior after therapeutic effects were observed. 
\item (Non-example) Bob wanted to determine the effects of response blocking and redirection on hand flapping with one of his students. He implemented this procedure and once it proved effective, decided to eliminate the intervention to determine if this procedure was the likely cause of the behavior decrease. 
\end{enumerate}
%
\subsection{Examples}
\begin{enumerate}
\item Multiple probe designs are particularly useful for researchers and teachers in educational settings to efficiently demonstrate results of instructional interventions when teaching across functionally different new skills (behaviors), across multiple students (participants), or across different sets of skills (conditions).
\item Multiple probe designs might not be appropriate if assessing the effects of intervention on severe behaviors that result in injury or property destruction because of the requirement that intervention be delayed across each tier while a person continues to engage in severe behaviors with lasting consequences.
\end{enumerate}
%
