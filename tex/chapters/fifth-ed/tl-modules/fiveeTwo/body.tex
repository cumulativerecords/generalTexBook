%\clearpage \section[\fourgSeven{}]{\fourgSeven{}%
              \sectionmark{G-07 Practice within one's limits...}}
\sectionmark{G-07 Practice within one's limits...}
\subsection{Definition}
Behavior analysts follow guidelines related to boundaries of competence in 1.02 of the Behavior Analyst Certification Board professional and ethical compliance code for behavior analysts:

1.02 Boundaries of Competence.\\
(a) All behavior analysts provide services, teach, and conduct research only within the boundaries of their competence, defined as being commensurate with their education, training, and supervised experience. \\
(b) Behavior analysts provide services, teach, or conduct research in new areas (e.g., populations, techniques, behaviors) only after first undertaking appropriate study, training, supervision, and/or consultation from persons who are competent in those areas... (BACB, 2014, p.4).\\

Practicing within your area of competence, training and experience\\
If, for example, a senior therapist working within an intensive behavior intervention program for preschoolers suddenly began working with adults with phobias then they would be in violation of this ethical guideline. Cooper, Heron, and Heward (2007) go on further to say that even within one's competence area, if a situation exceeds your training or experience then a referral to another behavior analyst should be made. If there is a gap in expertise, available then workshops and conferences may be accessed.  Mentors, supervisors and colleagues can provide additional training.  

Bailey and Burch (2011) relate this ethical boundary to the Hippocratic Oath, ``Do no harm.''  The guideline addresses the responsible conduct of behavior analysts, ensures the safety of clients, and protects the integrity of the field.
%
\subsection{Assessment}
\begin{enumerate}
\item Discuss case examples from Bailey and Burch (2011).
\item Have the supervisee come up with five fictitious examples of situations where the ethical guideline 1.02 was broken and provide creative solutions to the situation.
\item Use behavior skills training to teach your supervisee how to respond to a client asking your supervisee to provide services within an area in which they had no experience.
%
\end{enumerate}
%
\subsection{Relevant Literature}
\begin{refsection}
\nocite{bailey2013ethics,
        bac2014professional,
        cooper2007applied,
        van1988right}
\printbibliography[heading=none]
\end{refsection} 
%
\subsection{Related Lessons}
\fourbTwo{}\\
\fourgTwo{}\\
\fourgSix{}\\
\fourkEight{}\\
\fourkNine{}\\
%
%\clearpage \section[\fourgFour{}]{\fourgFour{}%
              \sectionmark{G-04 Explain behavioral concepts...}}
\sectionmark{G-04 Explain behavioral concepts...}
\subsection{Definition}
It is important for a behavior analyst to have a strong verbal repertoire when speaking about the science of behavior analysis. Jargon used in journals, universities, and with other behavior analysts is valuable to promote effective action on the part of the listener with precise discriminative control. 

That being said, Skinner wrote that we should choose words for the effects they have on the listener (Skinner, 1957) and unless that listener has extensive training in behavior analysis, our use of technical language will ``fall on deaf ears'' and not produce effective action. When speaking with client family members, friends, or professionals from other closely related fields, it is important to remember that your verbal behavior is for the benefit of your audience. Bailey (1991) describes this phenomenon well:

``In our zeal to be scientific, we have stressed the need to match the requirements of science in our writing and publishing. Although this has given us much-needed academic credibility (faculty can be promoted and tenured by publishing in JABA) it doesn't help at all in selling our technology to the masses'' (p. 446)

During the supervision process, spend considerable amounts of time, working on precise definitions for technical terms. This benefits the supervisee in several ways. It allows him/her to behave effectively as a listener and speaker when interacting with other behavior analysts in the field. It will also promote accurate translations to nontechnical language. If a precise definition is practiced, a less technical, more layperson-friendly definition will be easier to describe and it will be more likely to be accurate.
%
\subsection{Assessment}
\begin{enumerate}
\item Ask Supervisee to give a precise definition of a behavioral term from a textbook. Then ask Supervisee to accurately describe the term in nontechnical language. Work on this for the most commonly used terms. Provide feedback after, when in a supervision meeting.
\item Observe the Supervisee describing a behavior analytic concept to another person (client, colleague, etc.) The Supervisee should be able to answer basic questions related to the topic using nontechnical language. Provide feedback after, when in a supervision meeting.
\end{enumerate}
%
\subsection{Relevant Literature}
\begin{refsection}
\nocite{bailey1991marketing,
        lindsley1991technical,
        malott1992should}
\printbibliography[heading=none]
\end{refsection}
\subsection{Related Lessons}
\fourgSix{}\\
\fouriSix{}\\
\fourjSix{}\\
\fourjSeven{}\\
\fourkOne{}\\
\fourkThree{}\\
\fourkEight{}\\
\fourkNine{}\\
%
%\clearpage \section[\fourgTwo{}]{\fourgTwo{}%
              \sectionmark{G-02 Consider biological/med...}}
\sectionmark{G-02 Consider biological/med...}
\subsection{Definition}
The Behavior Analyst Certification Board (BACB) instructs that a ``behavior analyst recommends seeking a medical consultation if there is any reasonable possibility that a referred behavior influenced by medical or biological variables'' in section 3.02 Medical Consultation, of the BACB professional and ethical compliance code for behavior analysts (2014)

This is relevant both in research and practice. Therefore, the first step in the assessment process should be to determine whether the problem may be due to a medical/biological issue and whether a medical evaluation has been completed (Cooper, Heron \& Heward, 2007). Failure to rule out medical needs would be unethical as it would delay potentially necessary medical treatment that may even prove life threatening dependent on the medical concerns or the severity of the challenging behavior.

Possible pain related disorders or other medical/biological disorders that restrict an individual's ability to engage in appropriate behavior should be investigated.  Some relevant behavioral topics correlated with a high likelihood of medical and biological causes are feeding disorders, toileting challenges (e.g., encopresis and incontinence), sleep problems and self-injury. Take self-injury for example; studies have shown that self-injurious behavior (SIB) has been maintained by pain attenuation which, can be categorized as automatic negative reinforcement behavior (Carr \& Smith, 1995; O'Reilly, 1997).  In detail, an increase in painful stimulation is an establishing operation (EO), thereby increasing behavior that has been reinforced by pain reduction.

Aside from the common examples presented above, it is possible that any form of challenging behavior could be a result of an underlying medical or biological issue. For example, aggressive behavior may also be related to pain related disorders which act as an EO (Carr et al., 2003; Skinner, 1953). The argument has also been made that aggressive behavior in response to painful stimulation may be respondent behavior (Ulrich \& Azrin, 1962). Furthermore, Kennedy and Meyer (1996) found that the occurrence of allergy symptoms and sleep deprivation were correlated with an increase in escape maintained challenging behavior. 

\subsection{Assessment}
\begin{enumerate}
\item Ask the supervisee what the first part of assessment should be regarding specific situations that would require collaboration with medical professionals to rule out any underlying medical issues. Have them give a rationale as to why this is important. 
\item Ask the supervisee to list a few possible medical/biological considerations that should be ruled out when treating a feeding disorder, toileting issue or sleep problem.
\item Provide supervisee with examples such as this one: A student in a school setting is engaging in severe tooth picking and the classroom teacher is calling you for advice on what to do. Then ask the supervisee what the first step of assessment should be to treating this challenging behavior. 
%
\end{enumerate}
%
\subsection{Relevant Literature}
\begin{refsection}
\nocite{bac2014professional,
        carr2003menstrual,
        carr1995biological,
        cooper2007applied,
        kennedy1996sleep,
        o1997functional,
        skinner1953science,
        ulrich1962reflexive}
\printbibliography[heading=none]
\end{refsection}
%
\subsection{Related Lessons} 
\fourgOne{}\\
\fourgThree{}\\
\fourgSix{}\\
\fourgSeven{}\\
\fourFKThirteen{}\\
\fourFKTwentySix{}\\
