\subsection{Assessment}
\begin{enumerate}
\item Discuss case examples from Bailey and Burch (2011).
\item Have the supervisee come up with five fictitious examples of situations where the ethical guideline 1.02 was broken and provide creative solutions to the situation.
\item Use behavior skills training to teach your supervisee how to respond to a client asking your supervisee to provide services within an area in which they had no experience.
%
\end{enumerate}
%
\subsection{Assessment}
\begin{enumerate}
\item Ask Supervisee to give a precise definition of a behavioral term from a textbook. Then ask Supervisee to accurately describe the term in nontechnical language. Work on this for the most commonly used terms. Provide feedback after, when in a supervision meeting.
\item Observe the Supervisee describing a behavior analytic concept to another person (client, colleague, etc.) The Supervisee should be able to answer basic questions related to the topic using nontechnical language. Provide feedback after, when in a supervision meeting.
\end{enumerate}
%
\subsection{Assessment}
\begin{enumerate}
\item Ask the supervisee what the first part of assessment should be regarding specific situations that would require collaboration with medical professionals to rule out any underlying medical issues. Have them give a rationale as to why this is important. 
\item Ask the supervisee to list a few possible medical/biological considerations that should be ruled out when treating a feeding disorder, toileting issue or sleep problem.
\item Provide supervisee with examples such as this one: A student in a school setting is engaging in severe tooth picking and the classroom teacher is calling you for advice on what to do. Then ask the supervisee what the first step of assessment should be to treating this challenging behavior. 
%
\end{enumerate}
%
