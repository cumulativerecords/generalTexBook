\clearpage \section{\fourdFifteen{}}
\subsection{Definition}
Punisher – ``A stimulus change that decreases the future frequency of behavior that immediately precedes it'' (Cooper, Heron, \& Heward, 2007, p. 702).  

Punishers can be categorized as unconditioned or conditioned. Unconditioned punishers, or unlearned punishers, are stimuli whose presentation functions as punishment without previous pairing with any other punishers. Such punishers consist of stimulation such as pain, intense odors, visual stimulation, taste, sound, or extreme temperatures (Cooper, Heron, \& Heward, 2007).  Conditioned punishers, or learned punishers, are stimuli whose presentation has previously been paired with an unconditioned punisher or a previously conditioned punisher (Cooper, Heron, \& Heward, 2007).   For example, if a person eats yogurt and immediately gags or vomits, yogurt may become a conditioned aversive and thereby a conditioned punisher by decreasing the behavior of eating yogurt and possibly other food with a similar consistency to yogurt.  As the above examples of conditioned and unconditioned punishers show, the process of punishment is a naturally occurring phenomenon that causes behavior change.  However, punishment procedures can also be an effective means for decreasing challenging behavior that is life threatening or resistant to other forms of intervention in an ethical manner.  Iwata (1988) recommends that behavior analysts view the use of punishers as a default technology to be used when other interventions have failed.  

Regarding the selection of a punisher to use in an intervention, it is important to note that punishers are idiosyncratic. A punisher for one person maybe a reinforcer for someone else, and perhaps a neutral stimulus to another. For this reason, a punisher assessment can assist in identifying stimuli that will likely function as punishers by measuring avoidance and escape behavior following the presentation with each stimulus (Fisher et al., 1994). Once potential punishers have been identified there are some factors to consider when choosing a stimulus to use in the treatment of challenging behavior. Research has indicated that the magnitude, or amount of the punisher, should be delivered at the optimum level at the outset of the intervention (Azrin \& Holz, 1966; Thompson et al., 1999).  Furthermore, in keeping with ethical considerations the selection of the least intrusive punisher(s) is recommended.  Typically intrusiveness is outlined by hierarchically arranging interventions according to the degree to which the intervention limits individual freedom, intrudes into the child's life, or produces discomfort, pain, or distress (Luiselli, 2008). Pairing procedures may be beneficial in assisting with the identification of less intrusive punishers by establishing less intrusive conditioned punishers (Vorndran \& Lerman, 2006).  Lastly, it should be noted that Lerman and Vorndran (2002) highlighted the need for further basic and applied research on punishment due to a need for identifying strategies to enhance the effectiveness of least intrusive punishment procedures. 
%
\subsection{Assessment}
\begin{enumerate}
\item Ask supervisee to give examples of an unconditioned punisher
\item Ask supervisee to give examples of a conditioned  punisher
\item Ask supervisee to identify ethical considerations regarding the use of punishment and selecting punishers 
\item Ask supervisee to list the characteristics that should be considered when selecting a punisher 
\end{enumerate}
%
\subsection{Relevant Literature}
\begin{refsection}
\nocite{azrin1966punishment,
        bac2014professional,
        cooper2007applied,
        fisher1994preliminary,
        iwata1988development,
        lerman2002status,
        luiselli2008effective,
        thompson1999effects,
        vorndran2006establishing}
\printbibliography[heading=none]
\end{refsection}
\subsection{Related Tasks}
\fourdSixteen{}\\
\fourdSeventeen{}\\
\fourjTwo{}\\
\fourjTen{}\\
\fourFKNineteen{}\\
\fourFKTwenty{}\\
\fourFKTwentyOne{}\\
\fourFKTwentyThree{}\\
%
\clearpage \section{\fourFKSeventeen{}}
\subsection{Definition}
Unconditioned reinforcer - ``A stimulus change that increases the frequency of any behavior that immediately precedes it irrespective of the organism's learning history with the stimulus. Unconditioned reinforcers are the product of the evolutionary development of the species (phylogeny) Also called primary or unlearned reinforcer'' (Cooper, Heron, \& Heward, 2007 p.707).

``...momentary effectiveness of an unconditioned reinforcer is a function of current motivating operations'' (Cooper et al., 2007, p. 39).
%
\subsection{Examples}
\begin{enumerate}
\item Food, water, oxygen, warmth, and sexual stimulation are some examples of unconditioned reinforcers.
\item A teacher gives a child a pretzel after the child does a task. The child's engagement in the task increases in the future. 
\item This is an example of unconditioned reinforcement.
%
\end{enumerate}
%
\subsection{Assessment}
\begin{enumerate}
\item Have supervisee create a list of unconditioned reinforcers. Have him/her define and describe the role satiation and deprivation plays in unconditioned reinforcement. 
\item Have supervisee give examples of unconditioned reinforcers. Have him/her describe the difference between conditioned and unconditioned reinforcement.
\item Have supervisee explain the relationship between conditioned and unconditioned reinforcers and the role unconditioned reinforcers may play in creating conditioned reinforcers.
\end{enumerate}
%
\subsection{Relevant Literature}
\begin{refsection}
\nocite{bijou1965child,
        cooper2007applied,
        gewirtz2000infant,
        malott1978behavior,
        pelaez1996infants,
        skinner1953science}
\printbibliography[heading=none]
\end{refsection}
%
\subsection{Related Tasks}
\fourcOne{}\\
\fourdOne{}\\\
\fourdTwo{}\\\
\fourdNineteen{}\\
\fourFKTwo{}\\
\fourFKThirteen{}\\
\fourFKSixteen{}\\
\fourFKNineteen{}\\
\fourFKTwentyOne{}\\
\fourFKTwentySix{}\\
\fourFKThirty{}\\
%
\clearpage \section{\fourFKEighteen{}}
\subsection{Definition}
Conditioned reinforcer - ``A stimulus change that functions as a reinforcer because of prior pairing with one or more other reinforcers; sometimes called secondary or learned reinforcers'' (Cooper et al., 2007, p. 692). 

Conditioned reinforcement – ``the operation, or process, of a response producing a conditioned reinforcer that increases the likelihood that response occurs in the future'' (Cooper et al., 2007, p. 40).
%
\subsection{Examples}
\begin{enumerate}
\item  Money, tokens, stickers.
\item A teacher says ``good job'' after a student returns their homework. The student continues to return their homework in the future. 
%
\end{enumerate}
%
\subsection{Assessment}
\begin{enumerate}
\item Have supervisee explain the differences between conditioned and unconditioned reinforcers.
\item Have supervisee explain the process of producing a conditioned reinforcer (i.e., token systems). Have him/her give an example from their professional experience.
\item Have supervisee read and summarize a journal article on the topic of conditioned reinforcement. 
\end{enumerate}
%
\subsection{Relevant Literature}
\begin{refsection}
\nocite{alessi1992models,
        cooper2007applied,
        higgins2001effects,
        michael2004concepts,
        morse1977determinants}
\printbibliography[heading=none]
\end{refsection}
%
\subsection{Related Tasks}
\fourcOne{}\\
\fourdOne{}\\
\fourdTwo{}\\
\fourdTwenty{}\\
\fourdTwentyOne{}\\
\fourfTwo{}\\
\fourjFour{}\\
\fourkFour{}\\
\fourFKTwo{}\\
\fourFKFourteen{}\\
\fourFKFifteen{}\\
\fourFKSixteen{}\\
\fourFKSeventeen{}\\
\fourFKTwentyOne{}\\
\fourFKTwentySix{}\\
\fourFKTwentySeven{}\\
%
\clearpage \section{\fourFKNineteen{}}
\subsection{Definition}
Unconditioned punisher – ``A stimulus change that decreases the frequency of any behavior that immediately precedes it irrespective of the organism's learning history with the stimulus'' (Cooper, Heron, \& Heward, 2007, p. 707).*
%
\subsection{Examples}
\begin{enumerate}
\item Bright lights, loud sounds, extreme temperatures, certain tastes (sour, bitter), physical restraint, loss of bodily support, extreme muscular efforts, etc.
%
\end{enumerate}
%
\subsection{Assessment}
\begin{enumerate}
\item Ask the supervisee to describe an example of unconditioned punishment.
\item Use the supervisee to describe the difference between unconditioned punishment and an unconditioned punisher.
\item Ask the supervisee to list as many unconditioned punishers as possible in one minute.
%
\end{enumerate}
%
\subsection{Relevant Literature}
\begin{refsection}
\nocite{cooper2007applied,
        herman1964punishment}
\printbibliography[heading=none]
\end{refsection} 
%    
%
\subsection{Related Tasks}
\fourdSeventeen{}\\
\fourdSixteen{}\\
\fourdNineteen{}\\
\foureEleven{}\\
\fourgSeven{}\\
\fourjTen{}\\
\fourFKTwenty{}\\
%
\subsection{Footnotes}
*Conditioned punishers are products of the evolutionary development of the species (Cooper, Heron, \& Heward, 2007).\\
*Conditioned punishers are also called primary or unlearned punishers (Cooper, Heron, \& Heward, 2007).\\
%
\clearpage \section{\fourFKTwenty{}}
\subsection{Definition}
Conditioned punisher – ``a stimulus that functions as a punisher as the result of being paired with unconditioned or conditioned punishers'' (Cooper, Heron, \& Heward, 2007, p. 40).

Conditioned punishment as defined by Hake and Azrin (1965) is a process that ``results when it can be shown (1) there is little or no punishment effect before the stimulus is paired with an unconditioned punisher, but (2) a punishment effect occurs after (3) the stimulus has been paired, or is being paired, with an unconditioned punisher'' (p. 279).  Evidence of conditioned punishment was suggested in early research when a reduction in a response was observed following the process of pairing a stimulus with an electric shock followed by discontinuing the shock and making the stimulus contingent upon a selected response (Hake \& Azrin, 1965).

\subsection{Examples}
\begin{enumerate}
\item Similar to classical conditioning, a tone (neutral stimulus) is repeatedly paired with an electric shock (unconditioned punisher) whenever a dog barks, in time the tone (conditioned punisher) suppresses the bark in the absence of the electric shock.
\item A child engages in aggression. A parent responds to aggression by taking away their child's favorite video game contingent on every instance of aggression.  The parent begins to pair removal of the video games with a reprimand.  The reprimand may function as a conditioned punisher if aggression continues to decrease following the presentation of a reprimand without taking away the video games. This process illustrates conditioned punishment.
\item Conditioned punishers may be referred to as learned or secondary punishers.
%
\end{enumerate}
%
\subsection{Assessment}
\begin{enumerate}
\item Ask supervisee to explain the process of conditioned punishment.
\item Ask supervisee to define a conditioned punisher.
\item Ask supervisee to provide examples of conditioned punishment and a conditioned punisher.
\item Ask supervisee to identify examples of conditioned punishment in a client's environment.
%
\end{enumerate}
%
\subsection{Relevant Literature}
\begin{refsection}
\nocite{bailey2013ethics,
        cooper2007applied,
        hake1965conditioned,
        iwata1988development}
\printbibliography[heading=none]
\end{refsection}
%
\subsection{Related Tasks}
\fourcTwo{}\\
\fourdFifteen{}\\
\fourdSixteen{}\\
\fourdSeventeen{}\\
\fourdEighteen{}\\
\fourdNineteen{}\\
\fourFKFourteen{}\\
\fourFKSeventeen{}\\
\fourFKEighteen{}\\
\fourFKNineteen{}\\
\fourFKTwentyOne{}\\
