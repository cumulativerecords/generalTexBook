\subsection{Examples}
\begin{enumerate}
\item Food, water, oxygen, warmth, and sexual stimulation are some examples of unconditioned reinforcers.
\item A teacher gives a child a pretzel after the child does a task. The child's engagement in the task increases in the future. 
\item This is an example of unconditioned reinforcement.
%
\end{enumerate}
%
%\subsection{Examples}
\begin{enumerate}
\item  Money, tokens, stickers.
\item A teacher says ``good job'' after a student returns their homework. The student continues to return their homework in the future. 
\end{enumerate}
%
%\subsection{Examples}
\begin{enumerate}
\item Bright lights, loud sounds, extreme temperatures, certain tastes (sour, bitter), physical restraint, loss of bodily support, extreme muscular efforts, etc.
%
\end{enumerate}
%\subsection{Examples}
\begin{enumerate}
\item Similar to classical conditioning, a tone (neutral stimulus) is repeatedly paired with an electric shock (unconditioned punisher) whenever a dog barks, in time the tone (conditioned punisher) suppresses the bark in the absence of the electric shock.
\item A child engages in aggression. A parent responds to aggression by taking away their child's favorite video game contingent on every instance of aggression.  The parent begins to pair removal of the video games with a reprimand.  The reprimand may function as a conditioned punisher if aggression continues to decrease following the presentation of a reprimand without taking away the video games. This process illustrates conditioned punishment.
\item Conditioned punishers may be referred to as learned or secondary punishers.
\end{enumerate}
%
