%\clearpage \section{\fourdFifteen{}}
\subsection{Definition}
Punisher – ``A stimulus change that decreases the future frequency of behavior that immediately precedes it'' (Cooper, Heron, \& Heward, 2007, p. 702).  

Punishers can be categorized as unconditioned or conditioned. Unconditioned punishers, or unlearned punishers, are stimuli whose presentation functions as punishment without previous pairing with any other punishers. Such punishers consist of stimulation such as pain, intense odors, visual stimulation, taste, sound, or extreme temperatures (Cooper, Heron, \& Heward, 2007).  Conditioned punishers, or learned punishers, are stimuli whose presentation has previously been paired with an unconditioned punisher or a previously conditioned punisher (Cooper, Heron, \& Heward, 2007).   For example, if a person eats yogurt and immediately gags or vomits, yogurt may become a conditioned aversive and thereby a conditioned punisher by decreasing the behavior of eating yogurt and possibly other food with a similar consistency to yogurt.  As the above examples of conditioned and unconditioned punishers show, the process of punishment is a naturally occurring phenomenon that causes behavior change.  However, punishment procedures can also be an effective means for decreasing challenging behavior that is life threatening or resistant to other forms of intervention in an ethical manner.  Iwata (1988) recommends that behavior analysts view the use of punishers as a default technology to be used when other interventions have failed.  

Regarding the selection of a punisher to use in an intervention, it is important to note that punishers are idiosyncratic. A punisher for one person maybe a reinforcer for someone else, and perhaps a neutral stimulus to another. For this reason, a punisher assessment can assist in identifying stimuli that will likely function as punishers by measuring avoidance and escape behavior following the presentation with each stimulus (Fisher et al., 1994). Once potential punishers have been identified there are some factors to consider when choosing a stimulus to use in the treatment of challenging behavior. Research has indicated that the magnitude, or amount of the punisher, should be delivered at the optimum level at the outset of the intervention (Azrin \& Holz, 1966; Thompson et al., 1999).  Furthermore, in keeping with ethical considerations the selection of the least intrusive punisher(s) is recommended.  Typically intrusiveness is outlined by hierarchically arranging interventions according to the degree to which the intervention limits individual freedom, intrudes into the child's life, or produces discomfort, pain, or distress (Luiselli, 2008). Pairing procedures may be beneficial in assisting with the identification of less intrusive punishers by establishing less intrusive conditioned punishers (Vorndran \& Lerman, 2006).  Lastly, it should be noted that Lerman and Vorndran (2002) highlighted the need for further basic and applied research on punishment due to a need for identifying strategies to enhance the effectiveness of least intrusive punishment procedures. 
%
%\clearpage \section{\fourFKSeventeen{}}
%\subsection{Definition}
Unconditioned reinforcer - ``A stimulus change that increases the frequency of any behavior that immediately precedes it irrespective of the organism's learning history with the stimulus. Unconditioned reinforcers are the product of the evolutionary development of the species (phylogeny) Also called primary or unlearned reinforcer'' (Cooper, Heron, \& Heward, 2007 p.707).

``...momentary effectiveness of an unconditioned reinforcer is a function of current motivating operations'' (Cooper et al., 2007, p. 39).
%
%\clearpage \section{\fourFKEighteen{}}
%\subsection{Definition}
Conditioned reinforcer - ``A stimulus change that functions as a reinforcer because of prior pairing with one or more other reinforcers; sometimes called secondary or learned reinforcers'' (Cooper et al., 2007, p. 692). 

Conditioned reinforcement – ``the operation, or process, of a response producing a conditioned reinforcer that increases the likelihood that response occurs in the future'' (Cooper et al., 2007, p. 40).
%
%\clearpage \section{\fourFKNineteen{}}
%\subsection{Definition}
Unconditioned punisher – ``A stimulus change that decreases the frequency of any behavior that immediately precedes it irrespective of the organism's learning history with the stimulus'' (Cooper, Heron, \& Heward, 2007, p. 707).*
%
\subsection{Footnotes}
*Conditioned punishers are products of the evolutionary development of the species (Cooper, Heron, \& Heward, 2007).\\
*Conditioned punishers are also called primary or unlearned punishers (Cooper, Heron, \& Heward, 2007).\\
%
%\clearpage \section{\fourFKTwenty{}}
%\subsection{Definition}
Conditioned punisher – ``a stimulus that functions as a punisher as the result of being paired with unconditioned or conditioned punishers'' (Cooper, Heron, \& Heward, 2007, p. 40).

Conditioned punishment as defined by Hake and Azrin (1965) is a process that ``results when it can be shown (1) there is little or no punishment effect before the stimulus is paired with an unconditioned punisher, but (2) a punishment effect occurs after (3) the stimulus has been paired, or is being paired, with an unconditioned punisher'' (p. 279).  Evidence of conditioned punishment was suggested in early research when a reduction in a response was observed following the process of pairing a stimulus with an electric shock followed by discontinuing the shock and making the stimulus contingent upon a selected response (Hake \& Azrin, 1965).

