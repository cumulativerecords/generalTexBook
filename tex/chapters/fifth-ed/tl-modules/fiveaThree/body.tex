\clearpage \section[\fourFKSeven{}]{\fourFKSeven{}%
              \sectionmark{FK-07 Environmental (as opposed...}}
\sectionmark{FK-07 Environmental (as opposed...}
\subsection{Definition}
Mentalism-``An approach to explaining behavior that assumes that a mental, or ‘inner,' dimension exists that differs from a behavioral dimension and that phenomena in this dimension either directly cause or at least mediate some forms of behavior, if not all'' (Cooper, Heron, Heward, 2007, p. 699).

An environmental explanation of behavior can be described by physical events in the phylogenetic or ontogenetic history of the organism that cause behavior to occur. A behavior analyst believes that all behavior is a result of these physical events and that there is no reason to believe that there are some causes of behavior outside of physical dimensions.

It can be difficult sometimes, as we learn behavior analysis, to describe behavior without the use of mentalistic explanations (e.g., the hit me because he's frustrated). This is because in non-behavior analytic cultures, where many behavior analysts spend most of their lives, behavior is described this way and there is reinforcement available from that verbal community to perpetuate mentalistic explanations of behavior. For instance, it is common for people to believe that we each are responsible for our own actions and that the choices we make are done so with ``free-will.''  Johnston (2014) mentions that, ``After a lifetime of explaining behavior in terms of such apparent freedom, it is understandably difficult to accept what appears to be a helpless or passive role...'' (p.5) 

Much of behavior in society is controlled by consequences. Johnston (2014) says ``...we assign the responsibility for behavior not to the individual but to sources of control in the physical environment. From this perspective, holding individuals responsible for their behavior by specifying the consequences for certain actions remains an important contingency because it helps manage those tendencies to act in one way or another'' (p.11).
%
%
%
%
\subsection{Assessment}
\begin{enumerate}
\item Provide scenarios for a supervisee describing repetitive problem behaviors that might lead to a conclusion that internal events are controlling variables for behavior. Tell the supervisee to write a mentalistic explanation that might explain the behavior and then identify a radical behaviorist approach to explaining the same response.
\item Present a scenario in which a supervisee is working with parents or staff who insist that their child is hitting them because she is angry or frustrated. Ask the supervisee to role play explaining to care givers that behavior analysts look at anger and frustration a different way. 
%
\end{enumerate}
%
\subsection{Relevant Literature}
\begin{refsection}
\nocite{cooper2007applied,
        skinner1953science}
\printbibliography[heading=none]
\end{refsection}
%
\subsection{Related Tasks}
\fourbOne{}\\
\fourgFour{}\\
\fourgFive{}\\
\fouriOne{}\\
\fouriTwo{}\\
\fourkTwo{}\\
\fourFKOne{}\\
\fourFKThree{}\\
\fourFKSeven{}\\
%
\clearpage \section[\fourFKEight{}]{\fourFKEight{}%
              \sectionmark{FK-08 Distinguish between rad...}}
\sectionmark{FK-08 Distinguish between rad...}
\subsection{Definition}
Radical behaviorism – ``the philosophy of a science of behavior treated as a subject matter in its own right apart from internal explanations, mental or physiological'' (Skinner, 1989, p. 122).

Methodological behaviorism – ``represents a formal and strategic agreement to regard the relation between publicly observable stimulus variables and publicly observable behavior as the appropriate subject matter for psychology as a science'' (Moore, 2008, p. 385).

The distinction between radical and methodological behaviorism can be summed up by the views on private events. Private events, or events observable by only the individual engaging in the response, are not included in the analysis of behavior by a methodological behaviorist position. Radical behaviorists consider private events to be no different than any other behavior, therefore, allowing it to be understood within the same conceptual framework understood for overt behavior.

\subsection{Assessment}
\begin{enumerate}
\item Ask your supervisee to describe the distinction between radical and methodological behaviorism.
\item Have your supervisee describe the advantages of the methodological behaviorist's view.
\item Have your supervisee describe the advantages of the radical behaviorist's view.
\end{enumerate}
%
\subsection{Relevant Literature}
\begin{refsection}
\nocite{baum2011what,
        cooper2007applied,
        moore2011review,
        moore2009radical,
        moore2008conceptual,
        skinner1989recent}
\printbibliography[heading=none]
\end{refsection}
%
\subsection{Related Tasks}
\fourbOne{}\\
\fourgFour{}\\
\fourgFive{}\\
\fourFKOne{}\\
\fourFKSeven{}\\
