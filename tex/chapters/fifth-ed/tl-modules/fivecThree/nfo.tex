%\clearpage \section{\fouraOne{}}
\subsection{Definition}
Frequency - ``A ratio of count per observation time; often expressed as count per standard unit of time and calculated by dividing the number of responses recorded by the number of standard units of time in which observations were conducted (Cooper, Heron, \& Heward, 2007, p. 85).''*
%
\subsection{Footnotes}
* Alternatively, frequency is not always defined synonymously with rate throughout the discipline of behavior analysis. Catania (2013, p. 443) defines frequency as ``total responses over a fixed time, over a session of variable duration or, in trial procedure, over a fixed number of trials.'' Cooper, Heron, \& Heward (2007) functionally defines ``count'' as Catania defines ``frequency.''
%
%\clearpage \section{\fouraTwo{}}
%
%
\subsection{Definition}
Rate - ``A ratio of count per observation time; often expressed as count per standard unit of time and calculated by dividing the number of responses recorded by the number of number of standard units of time in which observations were conducted'' (Cooper, Heron, \& Heward, 2007, p. 85).*
%
\subsection{Footnotes}
* Alternatively, rate is not always defined synonymously with frequency throughout the discipline of behavior analysis. Catania (2013) defines rate as ``responses per unit time'' (p. 458) but frequency as ``total responses over a fixed time, over a session of variable duration or, in trial procedure, over a fixed number of trials'' (p. 443) Cooper, Heron, \& Heward (2007) functionally defines ``count'' whereas Catania defines ``frequency.''
%
%\clearpage \section{\fouraSix{}}
\subsection{Definition}
Percent of occurrence - ``A ratio formed by combining the same dimensional quantities such as count or time; expressed as a number of parts per 100; typically expressed as a ratio of the number of responses of a certain type per total number of responses. A percentage represents a proportional quantity per 100'' (Cooper, Heron \& Heward, 2007, p. 701).\\

Percent of occurrence demonstrates proportional relations effectively. For example, it can be used to indicate how many times an individual engaged in a target response given a set number of opportunities available).\\

There are several important limitations. The method has no dimensional quantities (e.g. does not indicate how many target responses were emitted nor how many opportunities were given). When there are few response opportunities (e.g. fewer than 20), percent occurrence measures may skew performance (e.g. An individual  answering 1 out of 2 problems correct on a math test will receive the same score of 50\% as an individual answering 25 out of 50 problems correct). It also imposes an artificial ceiling of measurement (e.g. 100\% may be subjective; suggesting that a learner performing at 100\% cannot improve). (Cooper et al., 2007).

%\clearpage \section[\fourFKFourtySeven{}]{\fourFKFourtySeven{}%
              \sectionmark{FK-47 Identify the measur...}}
\sectionmark{FK-47 Identify the measur...}
\subsection{Definition}
According to Johnston and Pennypacker (1993), behavior has 3 fundamental dimensional quantities (properties) that can be measured.
\begin{enumerate}
\item Repeatability – Refers to the fact that a behavior can occur repeatedly through time (i.e., behavior can be counted) (e.g., count, frequency, rate)
\item Temporal extent – Refers to the fact that every instance of behavior occurs during some amount of time (i.e., when behavior occurs it can be measured in time.) (e.g., duration)
\item Temporal locus – Refers to the fact that every instance of behavior occurs at a certain point in time with respect to other events. (i.e., occurrences of behavior can be measured at points in time.) (e.g., latency, interresponse time) (As cited in Cooper, Heron, \& Heward, 2007, p. 26)
\end{enumerate}
%
