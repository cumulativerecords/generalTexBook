%\clearpage \section{\fouraOne{}}
\subsection{Definition}
Frequency - ``A ratio of count per observation time; often expressed as count per standard unit of time and calculated by dividing the number of responses recorded by the number of standard units of time in which observations were conducted (Cooper, Heron, \& Heward, 2007, p. 85).''*
%
\subsection{Examples}
\begin{enumerate}
\item Hand Raising - A student is sitting in an hour long class. The student raises his hand 3 times to ask and answer questions during the class. The bell rings once and the student goes to his next class. Frequency of hand raising is 3 per hour.
\end{enumerate}
% 
\subsection{Assessment} 
\begin{enumerate}
\item Ask your supervisee to identify the frequency of hand raising above.
\item Ask your supervisee to create another example and non-example of his/her own.
\item Have supervisee measure a frequency of a behavior on the job or in a role play.
\item Have supervisee graph the frequency measured on the job or in a role play.
\end{enumerate}
%
\subsection{Relevant Literature}
\begin{refsection}
\nocite{catania2013learning,cooper2007applied}
\printbibliography[heading=none]
\end{refsection}

\subsection{Related Lessons} 
\fouriOne{}\\
\fourhOne{}\\
\fourFKFourtySeven{}\\
%
\subsection{Footnotes}
* Alternatively, frequency is not always defined synonymously with rate throughout the discipline of behavior analysis. Catania (2013, p. 443) defines frequency as ``total responses over a fixed time, over a session of variable duration or, in trial procedure, over a fixed number of trials.'' Cooper, Heron, \& Heward (2007) functionally defines ``count'' as Catania defines ``frequency.''
%
%\clearpage \section{\fouraTwo{}}
%
%
\subsection{Definition}
Rate - ``A ratio of count per observation time; often expressed as count per standard unit of time and calculated by dividing the number of responses recorded by the number of number of standard units of time in which observations were conducted'' (Cooper, Heron, \& Heward, 2007, p. 85).*
%
\subsection{Examples}
\begin{enumerate}
\item    Eating Chips: A young child is sitting at a table where there is a bag of potato chips. They eat 8 chips, stand up, and walk to the living room to watch TV for the rest of the hour. Rate of chip eating is 8 per hour.
\item Basketball Dribbles: Child is playing basketball for 30 minutes. Dribbles 7 times and then practices foul shots. He shoots 15 times and between each shot he dribbles 3 times. Frequency of dribbling is 52 dribbles per 30 minutes.
\end{enumerate}

\subsection{Assessment} 
\begin{enumerate}
\item Ask your supervisee to identify the frequency of chip eating or basketball dribbles in examples.
\item Have supervisee measure a frequency of a behavior on the job or in a role play.
\end{enumerate}
%
\subsection{Relevant Literature}
\begin{refsection}
\nocite{catania2013learning,cooper2007applied}
\printbibliography[heading=none]
\end{refsection}

\subsection{Related Lessons}
\fouriOne{}\\
\fourhOne{}\\
\fourFKFourtySeven{}\\

\subsection{Footnotes}
* Alternatively, rate is not always defined synonymously with frequency throughout the discipline of behavior analysis. Catania (2013) defines rate as ``responses per unit time'' (p. 458) but frequency as ``total responses over a fixed time, over a session of variable duration or, in trial procedure, over a fixed number of trials'' (p. 443) Cooper, Heron, \& Heward (2007) functionally defines ``count'' whereas Catania defines ``frequency.''
%
%\clearpage \section{\fouraSix{}}
\subsection{Definition}
Percent of occurrence - ``A ratio formed by combining the same dimensional quantities such as count or time; expressed as a number of parts per 100; typically expressed as a ratio of the number of responses of a certain type per total number of responses. A percentage represents a proportional quantity per 100'' (Cooper, Heron \& Heward, 2007, p. 701).\\

Percent of occurrence demonstrates proportional relations effectively. For example, it can be used to indicate how many times an individual engaged in a target response given a set number of opportunities available).\\

There are several important limitations. The method has no dimensional quantities (e.g. does not indicate how many target responses were emitted nor how many opportunities were given). When there are few response opportunities (e.g. fewer than 20), percent occurrence measures may skew performance (e.g. An individual  answering 1 out of 2 problems correct on a math test will receive the same score of 50\% as an individual answering 25 out of 50 problems correct). It also imposes an artificial ceiling of measurement (e.g. 100\% may be subjective; suggesting that a learner performing at 100\% cannot improve). (Cooper et al., 2007).

\subsection{Examples}
\begin{enumerate}
\item  Twelve strangers walk past an elderly man on the street. He greets three of them and ignores the rest. The percent of occurrence of greeting strangers is 25\%.
\begin{itemize}
\item To compute: Divide number of greetings emitted by the man (3) by the total number of opportunities to greet (12) and multiply that product by 100 to yield a percentage (3/12= 0.25 x 100= 25\%).
\end{itemize}
\item (Non-example) Twelve strangers walk by an elderly man. He greets three of them and ignores the rest. The percent of occurrence of greeting strangers is 0.25. 
\end{enumerate}
%
\subsection{Assessment}
\begin{enumerate}
\item Provide hypothetical situations and ask your supervisee if using percent of occurrence measures are appropriate
\item Provide various hypothetical situations and ask your supervisee to calculate percent of occurrence
\item Have supervisee graph percent of occurrence measured on the job or in a role play
\end{enumerate}
%
\subsection{Relevant Literature}
\begin{refsection}
\nocite{cooper2007applied}
\printbibliography[heading=none]
\end{refsection}
%
\subsection{Related Lessons}
\fouriOne{}\\
\fourhOne{}\\
\fourFKFourtySeven\\
%
%\clearpage \section[\fourFKFourtySeven{}]{\fourFKFourtySeven{}%
              \sectionmark{FK-47 Identify the measur...}}
\sectionmark{FK-47 Identify the measur...}
\subsection{Definition}
According to Johnston and Pennypacker (1993), behavior has 3 fundamental dimensional quantities (properties) that can be measured.
\begin{enumerate}
\item Repeatability – Refers to the fact that a behavior can occur repeatedly through time (i.e., behavior can be counted) (e.g., count, frequency, rate)
\item Temporal extent – Refers to the fact that every instance of behavior occurs during some amount of time (i.e., when behavior occurs it can be measured in time.) (e.g., duration)
\item Temporal locus – Refers to the fact that every instance of behavior occurs at a certain point in time with respect to other events. (i.e., occurrences of behavior can be measured at points in time.) (e.g., latency, interresponse time) (As cited in Cooper, Heron, \& Heward, 2007, p. 26)
\end{enumerate}
%
\subsection{Assessment}
\begin{enumerate}
\item Ask Supervisee to measure their own duration related to a task (eg. give them a timer and crossword puzzle to complete)
\item Ask Supervisee to measure and calculate the rate of someone tapping their pen (or another discrete behavior) during a 10 minute meeting
\item Ask Supervisee to observe a conversation between colleagues and measure latency regarding question asking-answering.  Have Supervisee use a timer/stop watch to record latency
\item Ask Supervisee to record their own latency during a supervision meeting when asked to define a task list item, vs. a concept 
\item Ask Supervisee to measure interresponse time (IRT) by observing someone eating a meal for 5 minutes; have Supervisee record time between swallowing one bit of food and next bite and report the average IRT.
%
\end{enumerate}
%
\subsection{Relevant Literature}
\begin{refsection}
\nocite{cooper2007applied,
        johnston2010strategies,
        thomason2011response,
        worsdell2002duration}
\printbibliography[heading=none]
\end{refsection}
%
\subsection{Related Lessons}
\fouraOne{}\\
\fouraTwo{}\\
\fouraThree{}\\
\fouraFour{}\\
\fouraFive{}\\
\fouraNine{}\\
\fourdTwentyOne{}\\
