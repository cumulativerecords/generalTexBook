\clearpage \section[\fourkThree{}]{\fourkThree{}%
              \sectionmark{K-03 Design and use comp...}}
\sectionmark{K-03 Design and use comp...}
\subsection{Definition}
Rationale\\
In order for staff and family members to collect accurate data and to carry out a behavior intervention plan effectively and consistently, all staff must receive training.  Inconsistent applications of procedures may lead to slow improvement or have effects that make behavior worse than before treatment.  

Training models\\
Much research has studied competency based training for staff, teachers and parents. Parsons and Reed (1995) increased staff performance by using classroom-based instruction, observation, and feedback on the work site.  Shore et al. (1995) provided training in data collection, calculation, and review of treatment procedure, followed by training in treatment implementation in-situ with feedback and assistance.  A third phase of instructing supervisors how to train direct service providers was implemented.  Sarokoff and Sturmey (2004) provided staff with a written definition of the plan, feedback regarding their baseline performance, rehearsal with reinforcement and corrective feedback, and in-situ modeling and rehearsal for 10 minutes.  Miles and Wilder (2009) used Behavior Skills Training (BST) that involved providing a written description, reviewing baseline performance, rehearsal and feedback, and then repeating modeling and rehearsal until the staff completed three trials accurately.

BST has also been used to train individuals to conduct behavioral assessments such as functional analyses. For example, Iwata et al. (2000) and Lambert, Bloom, Clay, Kunnavatanna, and Collins (2014) trained participants to conduct functional analysis conditions with adequate fidelity. 
%
Elements of competency-based training
\begin{itemize}
\item Clear instructions
\item Modeling
\item Rehearsal
\item Feedback 
\item Repetition until skills mastered
\item Treatment Integrity data monitored
\end{itemize}
%
Threats to accuracy and reliability\\
Poor staff training can lead to inaccurate baseline, functional assessment and treatment data.  Systematic training of occurrence and nonoccurrence and other critical data collection information and providing booster training will help to minimize the following challenges:
\begin{itemize}
\item Observer drift
\item Observer reactivity
\end{itemize}
%
\subsection{Assessment}
\begin{enumerate}
\item Ask your supervisee to list elements of competency-based training. 
\item Ask your supervisee to create an example curriculum for training staff to collect ABC data collection.
\item Use behavior skills training to teach your supervisee to teach a treatment procedure to staff. 
%
\end{enumerate}
%
\subsection{Relevant Literature}
\begin{refsection}
\nocite{cooper2007applied,
        iwata2000skill,
        lambert2014training,
        miles2009effects,
        parsons1995training,
        sarokoff2004effects,
        shore1995pyramidal}
\printbibliography[heading=none]
\end{refsection} 
%
\subsection{Related Lessons}
\fourdFour{}\\
\fourdFive{}\\
\foureThree{}\\
\fourfThree{}\\
\fourkTwo{}\\
