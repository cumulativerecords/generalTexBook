\clearpage \section[\fourjFour{}]{\fourjFour{}%
              \sectionmark{J-04 Select... client preferences.}}
\sectionmark{J-04 Select... client preferences.}
\subsection{Definition}
The importance and ethical necessity of basing intervention strategies on client preferences:
\begin{enumerate}
\item As behavior analysts, it is our ethical responsibility to continually put our client's needs first, and this includes, considering which type of intervention may be more preferred by the clients we serve. As Bailey and Burch (2011) state, one of our core ethical principles is treating others with care and compassion and this encompasses giving our clients choices (Bailey \& Burch, 2011).
%
\item Historically, consideration of client preferences is an area that within behavior analysis, perhaps has not been given as much attention as it deserves. In one area of study, Hanley, Piazza, Fisher, Contrucci \& Maglieri (1997) reported that, ``few if any studies have examined the social acceptability of or consumer preferences'' for the relevant treatment options but had instead given more weight to the opinions of the caregivers as opposed to those of the client (Hanley et al., 1997, p. 460). Another interesting train of thought has been that ``choice making is often not taught'' (Bannerman, Sheldon, Sherman, \& Harchik, 1990, p. 81).
%
\item Another reason for considering clients' preferences over treatment options is it may make the intervention more successful. Data from Miltenberger, Suda, Lennox and Lindeman (1991) indicated it was very important for successful treatment, to consider client preferences when selecting interventions. Findings from many other studies have also supported this premise (e.g., Berk, 1976; Hanley, Piazza, Fisher, \& Maglieri, 2005; Mendonca \& Brehm, 1983; Perlmuter \& Montry, 1973).
\end{enumerate}

Selecting interventions based on client preferences 
\begin{enumerate}
\item There are methods reported in the literature for determining which treatment method is more preferred by a client*. As such, once it has been established that an intervention is necessary to treat a behavior, it is imperative then to consider assessing a client's preference for one treatment option over others to assist with the behavior change program. In this way, treatment is more likely to be successful, will likely have more social validity (Schwartz \& Baer, 1991) and will be meeting more of our ethical standards as behavior analysts.
\end{enumerate}
%
\subsection{Assessment}
\begin{enumerate}
\item Ask your Supervisee to explain why it's important as behavior analysts to select interventions based on client preferences. 
\item Ask your Supervisee to investigate what different methods are available for evaluating clients' preferences for different interventions and report them back to you, along with advantages and disadvantages of each method.
\end{enumerate}
%
\subsection{Relevant Literature}
\begin{refsection}
\nocite{bailey2013ethics,
        bannerman1990balancing,
        berk1976effects,
        hanley1997evaluation,
        hanley2005effectiveness,
        mendonca1983effects,
        miltenberger1991assessing,
        perlmuter1973effect}
\printbibliography[heading=none]
\end{refsection}
\subsection{Related Lessons}
\foureEight{}\\
\fouriSeven{}\\
\fourjTwo{}\\
\fourjFive{}\\
\fourjSix{}\\
\fourjSeven{}\\
\fourjEight{}\\
%
Footnotes\\
*1 See Hanley, Piazza, Fisher, Contrucci \& Maglieri (1997) and Miltenberger, Suda, Lennox \& Lindeman (1991) for more information about how to test clients' preferences for different interventions.\\
%
\clearpage \section[\fourjSix{}]{\fourjSix{}%
              \sectionmark{J-06 Select... environments.}}
\sectionmark{J-06 Select... environments.}
\subsection{Definition}
``Achieving optimal generalized outcomes requires thoughtful, systematic planning. This planning begins with two major steps: (1) selecting target behaviors that will meet natural contingencies of reinforcement, and (2) specifying all desired variations of the target behavior and the settings/situations in which it should (and should not) occur after instruction has ended'' (Cooper, Heron, \& Heward, 2007, p. 623). In other words, an intervention must be selected that will allow the client to access reinforcement in a specific environment. If that is not possible, then alternative interventions should be explored. 

Ayllon and Azrin (1968) state that an important rule of thumb is to choose interventions that will help produce reinforcement after the intervention is discontinued. The intervention should support the student until they can access naturally existing contingencies (i.e., verbal praise from a teacher) and then more intensive, contrived contingencies should be systematically faded. The goal of most intervention programs is to teach a skill and then fade support so the client can implement that skill across settings. 

Cooper, Heron, \& Heward (2007, p. 626) identify 5 strategic approaches to promote generalized behavior change.
\begin{enumerate}
\item Teach the full range of relevant stimulus conditions and response requirements (i.e., teaching sufficient stimulus and response examples based on the setting)
\item Make the instructional setting similar to the generalization setting. (i.e., program common stimuli and teach loosely)
\item Maximize the target behavior's contact with reinforcement in the generalization setting. (i.e., ask people in the generalization setting to reinforce the target behavior, teach the learner to recruit reinforcement, and teach the target behavior to levels of performance required by natural existing contingencies of reinforcement.)
\item Mediate generalization (i.e., teach self-management skills \& contrive mediating stimulus)
\item Train to generalize. (i.e., reinforce response variability and instruct learner to generalize)
\end{enumerate}
%
\subsection{Examples}
\begin{enumerate}
\item George set up a token economy for Bill that systematically increased the number of responses needed to earn a token. After some time, Bill was earning tokens for completing an entire worksheet rather than earning a token for each question answered. This allowed Bill to independently complete worksheets in his general education classroom without a paraprofessional by his side giving him tokens after each answer. 
%
\end{enumerate}
%
\subsection{Assessment}
\begin{enumerate}
\item Give supervisee a reinforcement program. Have him/her create a fading procedure for this program to increase the number of responses required to earn a token. 
\item Have supervisee list the five strategies for promoting generalized behavior change and have him/her give examples of each.
\item Have supervisee describe and differentiate between contrived contingencies and naturally existing contingencies and give several examples of each.
%
\end{enumerate}
%
\subsection{Relevant Literature}
\begin{refsection}
\nocite{ayllon1968token,
        baer1999plan,
        cooper2007applied,
        snell2006instruction,
        stokes1977implicit,
        stokes1989operant}
\printbibliography[heading=none]
\end{refsection}
%
\subsection{Related Lessons}
\fourgEight{}\\
\fourjSeven{}\\
\fourjEight{}\\
\fourjEleven{}\\
\fourjTwelve{}\\
\fourkSeven{}\\
\fourkNine{}\\
%
\clearpage \section[\fourjSeven{}]{\fourjSeven{}%
              \sectionmark{J-07 Select... constraints.}}
\sectionmark{J-07 Select... constraints.}
\subsection{Definition}
``The independent variable should be evaluated not only in terms of its effects on the dependent variable, but also in terms of its social acceptability, complexity, practicality, and cost'' (Cooper, Heron, \& Heward, 2007, p. 250).

One method for determining the feasibility of an intervention is by asking consumers (parents, teachers, administrators) to rate the social validity of the client's performance. Questions that are typically posed to consumers before interventions are implemented include asking the consumer how reasonable they feel the intervention is, asking the consumer's willingness to implement the intervention strategies, asking if the consumer would be willing to change the environment to implement the intervention, asking how disruptive the intervention may be to the natural environment, asking how costly it would be to implement the intervention, asking if there will be any discomfort in the client when implementing these procedures, and asking if carrying out the intervention will fit with the classroom or setting routines (Reimers \& Wacker, 1988 cited from Cooper et al., 2007, pp. 238-239).
%
\subsection{Examples}
\begin{enumerate}
\item Rob has decided to implement a reinforcement program based on appropriate responses rather than a fixed time DRO program. He understands that there is no paraprofessional in the classroom to help run the program and the teacher has other educational duties so she cannot run a timer and deliver reinforcement consistently enough for a rigorous DRO. 
%
\end{enumerate}
%
\subsection{Assessment}
\begin{enumerate}
\item Have supervisee come up create a list of appropriate questions to ask consumers when determining an interventions appropriateness and acceptability. Have each supervisee create his/her own treatment acceptability rating form.
\item Have supervisee list and describe various extraneous factors that must be taken into consideration before implementing an intervention. Have supervisee explain why it is important to have consumer satisfaction with an intervention program.
\end{enumerate}
%
\subsection{Relevant Literature}
\begin{refsection}
\nocite{cooper2007applied,
        hawkins1984meaningful,
        reimers1988parents,
        wolf1978social}
\printbibliography[heading=none]
\end{refsection}
%
\subsection{Related Lessons}
\fourcOne{}\\
\fourgSix{}\\
\fourgEight{}\\
\fouriOne{}\\
\fouriTwo{}\\
\fourjSix{}\\
\fourjEight{}\\
\fourkSeven{}\\
\fourkNine{}\\
\fourFKEleven{}\\
%
\clearpage \section[\fourjEight{}]{\fourjEight{}%
              \sectionmark{J-08 Select... social validity...}}
\sectionmark{J-08 Select... social validity...}
\subsection{Definition}
A distinguishing characteristic of applied behavior analysis is assessing an individual's functioning within the context of natural environments. This applied aspect focuses the behavior analyst on identifying meaningful goals and acceptable methods for intervention that will increase the individual's independence and level of functioning in natural settings. The behavior analyst sets intervention goals that comply with stated preferences of the individual client, goals of those who live and work with the individual, and consider how typical individuals function in similar environments. Analysts seek goals that are socially valid and intervention methods that are not only scientifically validated strategies for accomplishing those goals, but strategies that can be expected to be implemented consistently and with fidelity by those who will apply the strategies. Although an intervention might be effective in a clinical, controlled setting, the behavior analyst must consider intervention limitations related to ``social acceptability, complexity, practicality, and cost. Regardless of their effectiveness, treatments that are perceived by practitioners, parents, and/or clients as unacceptable or undesirable for whatever reason are unlikely to be used'' (Cooper et al, 2007, p. 250).
%
\subsection{Examples}
\begin{enumerate}
\item A child hits her six month-old sister even when their parents model and reinforce appropriate behaviors toward the baby. The parents find it difficult to avoid explaining to the child, at the same time that they block her physically, reasons her behaviors are unkind and even dangerous. The behavior analyst talks to the parents about how the parent's explanations might be reinforcing the big sister aggression. The parents and the analyst increase the opportunities they have to individually attend to the child during appropriate play throughout the day. The parents ask the analyst to help them design a structured plan to teach appropriate sibling behaviors through language, modeling, literature, role play, movies, and increased reinforcement for appropriate behaviors of the sister toward her younger sibling.
\item A man hits his head and pulls at his ears with such force that he has required emergency medical care. At the beginning of treatment, the behavior analyst recommends that the man be given access to a helmet to prevent significant injury when he is not adequately staffed to stop his behavior. His family is against the man appearing in public with a protective helmet. The behavior analyst explains the reasons such equipment might be important for protecting the man from harm when his 1:1 staff person is distracted by driving a car or interacting with clerks or others in the community. The analyst and the family agree that until interventions stops the severe self-injury, the man will participate in community activities with a helmet unless a family member accompanies staff in the community with him. 
%
\end{enumerate}
\subsection{Assessment}
\begin{enumerate}
\item A behavior analyst wants to increase a non-verbal teenager's independent functioning during daily care routines by teaching him to dress, brush teeth, and bathe independently. The boy's mother says she doesn't mind physically prompting her son through those daily care routines, but states that she hates his screaming while she does it.

\item The analyst completes a functional assessment and learns the following: The boy can complete most of the steps for dressing, brushing teeth, and bathing independently, but has not learned a consistent chain of steps for each skill. The boy screams at other times during the day when his mother uses physical prompting. Ask the supervisee to consider the social validity of the behavior analyst's goals and the preferences expressed by the boy's mother. 

\item Ask the supervisee to write at least one hypothesis to explain, based on the information above, what might be the relations between the self-care skills and the screaming behavior.

\item Ask the supervisee to explain to the mother why teaching the son chained steps for each skill is important for ending the screaming behavior in non-technical language in order to gain her support for teaching self-care routines.
\end{enumerate}
%
\subsection{Relevant Literature}
\begin{refsection}
\nocite{cooper2007applied,
        fawcett1991social,
        wolf1978social}
\printbibliography[heading=none]
\end{refsection}
%
\subsection{Related Lessons}
\fourgSix{}\\
\fourgEight{}\\
\fouriSix{}\\
\fourjFour{}\\
\fourjFive{}\\
\fourjSix{}\\
\fourjTwelve{}\\
\fourkTwo{}\\
\fourkThree{}\\
\fourkNine{}\\
% May not need below.
\clearpage \section[\fourjThree{}]{\fourjThree{}%
              \sectionmark{J-03 Select... task analysis.}}
\sectionmark{J-03 Select... task analysis.}
\subsection{Definition}
What are the options for intervention strategies when it comes to teaching a chain of behavior through a task analysis?
\begin{enumerate}
\item Once a person's baseline level has been assessed (through the single or multiple-opportunity method*) to determine what components of the task analysis he/she can perform, the appropriate intervention strategy needs to be selected. Cooper, Heron \& Heward (2007) suggest there are four appropriate intervention strategies which practitioners can choose from; forward chaining, total-task chaining, and backward chaining.
\item Cooper, Heron and Heward (2007, p. 446) argue that research to date does not suggest a clear answer to the question ``which chaining strategy to use?'' As such, it is very important to examine the results of the baseline level assessment, to consider the client and how they learn best, and what the different intervention strategies can offer in different situations, in order to select the most appropriate method.
\end{enumerate}

Total-task chaining
\begin{enumerate}
\item If the client performs quite a few steps in the task analysis but is not performing them in the correct sequence, the most appropriate method to choose would probably be total-task chaining. 
\item Total-task chaining would also be an appropriate intervention strategy to select when the client has generalized motor imitation and moderate to severe disabilities (Test et al., 1990, cited from Cooper, Heron \& Heward, 2007).
\item When the chain is quite short and not too complex, this may also be an appropriate teaching method to utilize (Cooper, Heron \& Heward, 2007).
\end{enumerate}
%
Forward chaining
\begin{enumerate}
\item This approach may be more appropriate to use when the client has demonstrated more proficiency with the first couple of steps in the chain and/or the last steps in the chain are more complex to complete. 
\item It may also be useful to use this approach when it is necessary to link smaller chains into larger ones. For example, if you have a skill such as bed making and this is made up of perhaps four/five skill clusters, forward chaining is a useful method to link the skill clusters altogether (Cooper, Heron \& Heward, 2007).
\end{enumerate}
%
Backward chaining
\begin{enumerate}
\item Backward chaining may be more appropriate to use when the client has demonstrated more proficiency with the last couple of steps in the chain and/or steps that appear earlier in the chain are more complex to complete.
\end{enumerate}
%
\subsection{Assessment}
\begin{enumerate}
\item Ask your Supervisee to identify situations in which he/she might suggest forward chaining
\item Ask your Supervisee to identify situations in which he/she might suggest backward chaining
\item Ask your Supervisee to identify situations in which he/she might suggest total-task chaining
\end{enumerate}
%
\subsection{Relevant Literature}
\begin{refsection}
\nocite{cooper2007applied,
        kazdin2012behavior,
        miltenberger2008behavior,
        test1990teaching}
\printbibliography[heading=none]
\end{refsection}
%
\subsection{Related Lessons}
\fourdSix{}\\
\fourdSeven{}\\
\fourjTwo{}\\
\fourjFive{}\\
%
Footnotes\\
* Please see 4th ed. task list item D-07 for more information on conducting task analyses.\\
* Please see 4th ed. task list item D-06 for a more detailed description of the different chaining procedures.\\
%
\clearpage \section[\fourjFive{}]{\fourjFive{}%
              \sectionmark{J-05 Select... repertoires.}}
\sectionmark{J-05 Select... repertoires.}
\subsection{Definition}
\begin{enumerate}
\item \textit{Importance of considering the client's current repertoires.} Basing intervention strategies on the client's current repertoires is a key foundation of what behavior analysts do. It is imperative that prior to implementing any type of intervention or strategy with a client, the behavior analyst is extremely clear about what the client already does and can therefore, consider possible intervention strategies. Noell, Call, and Ardoin (2011) state that ``one of the considerable challenges in teaching arises from identifying not only the behaviors that are prerequisites for the target response, but also the level of skill proficiency needed to set the occasion for teaching the target skill'' (Noell, Call, \& Ardoin, 2011, cited from Fisher, Piazza, \& Roane, 2011, p. 251).
%
\item \textit{Importance of accurate assessments.} In order to assess a client's current repertoires, it is imperative that these repertoires are properly assessed. For example, assessments to evaluate the presence of a particular skill or repertoire should take place in a variety of different environments, with many different examples of stimuli, with the antecedent presented in a variety of different ways, and with many different people presenting the skill. Novel examples of the skill should also be tested. Noell et al. (2011) emphasize this point by suggesting that ``assessment of behavior under varied conditions in a manner that tests consequences should be an element of any pre-teaching assessment'' (Noell, Call \& Ardoin, 2011, cited from Fisher, Piazza \& Roane, 2011, p. 255). 

\item \textit{After completing assessments of the client's current repertoires.} Once the assessment stage is complete, it is then appropriate to select possible intervention strategies. As Noell et al. (2011) propose it is important at this point that behavior analysts ``keep the long-term view in mind'' (Noell, Call \& Ardoin, 2011, cited from Fisher, Piazza \&Roane, 2011, p. 266).  We should be attempting to ``not bring individual operants under stimulus control'' but instead ``help clients and students develop the complex, flexible repertoires that are adaptive, that remain in contact with reinforcement, and that confer adaptive advantage and endure'' (Noell, Call \& Ardoin, 2011, cf. Fisher, Piazza \& Roane, 2011, p. 266).
%
\end{enumerate}
%
\subsection{Assessment}
\begin{enumerate}
\item Ask your Supervisee to explain why it's important as a behavior analyst to select interventions based on the client's current repertoires. 
\item Ask your Supervisee to conduct an assessment with a client, if possible, and design potential interventions or strategies. Give feedback as appropriate.
\item Have your Supervisee take the lead on assessing and developing interventions with your supervision. 
%
\end{enumerate}
%
\subsection{Relevant Literature}
\begin{refsection}
\nocite{fisher2014handbook,
        noell2011building,
        shapiro2011academic}
\printbibliography[heading=none]
\end{refsection}
%
\subsection{Related Lessons}
\fourdNine{}\\
\fourgThree{}\\
\fouriThree{}\\
\fourjThree{}\\
\fouriFour{}\\
\fourjTwo{}\\
\fourjSix{}\\
\fourjSeven{}\\
\fourjEight{}\\
