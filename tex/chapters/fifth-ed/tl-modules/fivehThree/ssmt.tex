\subsection{Assessment}
\begin{enumerate}
\item Ask your Supervisee to explain why it's important as behavior analysts to select interventions based on client preferences. 
\item Ask your Supervisee to investigate what different methods are available for evaluating clients' preferences for different interventions and report them back to you, along with advantages and disadvantages of each method.
\end{enumerate}
%
\subsection{Assessment}
\begin{enumerate}
\item Give supervisee a reinforcement program. Have him/her create a fading procedure for this program to increase the number of responses required to earn a token. 
\item Have supervisee list the five strategies for promoting generalized behavior change and have him/her give examples of each.
\item Have supervisee describe and differentiate between contrived contingencies and naturally existing contingencies and give several examples of each.
%
\end{enumerate}
%
\subsection{Assessment}
\begin{enumerate}
\item Have supervisee come up create a list of appropriate questions to ask consumers when determining an interventions appropriateness and acceptability. Have each supervisee create his/her own treatment acceptability rating form.
\item Have supervisee list and describe various extraneous factors that must be taken into consideration before implementing an intervention. Have supervisee explain why it is important to have consumer satisfaction with an intervention program.
\end{enumerate}
%
\subsection{Assessment}
\begin{enumerate}
\item A behavior analyst wants to increase a non-verbal teenager's independent functioning during daily care routines by teaching him to dress, brush teeth, and bathe independently. The boy's mother says she doesn't mind physically prompting her son through those daily care routines, but states that she hates his screaming while she does it.

\item The analyst completes a functional assessment and learns the following: The boy can complete most of the steps for dressing, brushing teeth, and bathing independently, but has not learned a consistent chain of steps for each skill. The boy screams at other times during the day when his mother uses physical prompting. Ask the supervisee to consider the social validity of the behavior analyst's goals and the preferences expressed by the boy's mother. 

\item Ask the supervisee to write at least one hypothesis to explain, based on the information above, what might be the relations between the self-care skills and the screaming behavior.

\item Ask the supervisee to explain to the mother why teaching the son chained steps for each skill is important for ending the screaming behavior in non-technical language in order to gain her support for teaching self-care routines.
\end{enumerate}
%
\subsection{Assessment}
\begin{enumerate}
\item Ask your Supervisee to identify situations in which he/she might suggest forward chaining
\item Ask your Supervisee to identify situations in which he/she might suggest backward chaining
\item Ask your Supervisee to identify situations in which he/she might suggest total-task chaining
\end{enumerate}
%
\subsection{Assessment}
\begin{enumerate}
\item Ask your Supervisee to explain why it's important as a behavior analyst to select interventions based on the client's current repertoires. 
\item Ask your Supervisee to conduct an assessment with a client, if possible, and design potential interventions or strategies. Give feedback as appropriate.
\item Have your Supervisee take the lead on assessing and developing interventions with your supervision. 
%
\end{enumerate}
%
