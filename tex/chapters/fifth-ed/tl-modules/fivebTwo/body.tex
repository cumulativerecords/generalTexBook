\clearpage \section{\fourFKEleven{}}
\subsection{Definition}  
Environment - ``The conglomerate of real circumstances in which the organism or reference part of the organism exists; behavior cannot occur in the absence of environment'' (Cooper, Heron, \& Heward, 2007, p. 694).

Stimulus - ``Any physical event, combination of events, or relation among events'' (Catania, 2013, p. 466).  

Stimulus class - ``A group of stimuli that share specified common elements along formal (e.g. size, color), temporal (e.g. antecedent or consequent), and/or functional (e.g., discriminative stimulus) dimensions'' (Cooper, Heron, \& Heward, 2007, p. 705).\\

``Any group of stimuli sharing a predetermined set of common elements in one or more of these dimensions'' (Cooper, Heron, \& Heward, 2007, p. 27).  
\subsection{Examples}
A trip to the mall
\begin{enumerate}
\item  Environment:  Willis is shopping at the local mall. The local mall would be an environment.
\item Stimulus: Willis is walking through the food court.  He smells some pizza cooking from one of the establishments and suddenly his stomach starts growling.  He stops and gets some food.  All of the things in the food court, including the smells, the changes in his stomach, and the food are stimuli.
\item Stimulus class:  At the food court, Willis will buy items that will all function as reinforcers for eating behavior. In this case, the burger, the fries, and the cookie he bought are in the same stimulus class.
%
\end{enumerate}
%
\subsection{Assessment}
\begin{enumerate}
\item Ask your supervisee to define environment, stimulus, and stimulus class.   
\item Ask your supervisee to identify the environment, the stimulus (stimuli), and the stimulus class(es) from the above examples.  Use examples of stimulus classes related to the formal, temporal, and functional dimensions. 
\item Ask your supervisee to create other examples and a non-examples of his/her own. 
\item Have your supervisee to compare and contrast these terms. 
%
\end{enumerate}
%
\subsection{Relevant Literature}
\begin{refsection}
\nocite{catania2013learning,
        cooper2007applied,
        johnston1993strategies,
        michael2004concepts}
\printbibliography[heading=none]
\end{refsection}
%
\subsection{Related Tasks}
\fourFKEleven{}\\
%
\clearpage \section{\fourFKTwelve{}}
\subsection{Definition}
Stimulus equivalence - ``The emergence of accurate responding to untrained and nonreinforced stimulus-stimulus relations following the reinforcement of responses to some stimulus-stimulus relations. A positive demonstration of reflexivity, symmetry and transitivity is necessary to meet the definition of equivalence'' (Cooper, Heron \& Heward, 2007, p. 705).'' 

\subsection{Examples}
Related definitions and examples are presented below.
\begin{enumerate}
\item Reflexivity - ``A type of stimulus-to-stimulus relation in which the student, without any prior training or reinforcement for doing so, selects a comparison stimulus that is the same as the same stimulus'' (Cooper, Heron \& Heward, 2007, p. 702).\\
 Example: Without prior reinforcement or training, when shown a picture of a dog and given a picture of the same dog, a rat, and a cow, student matches the picture of the two dogs (e.g. A=A).
\item Symmetry - ``A type of stimulus-to-stimulus relationship in which the learner, without prior training or reinforcement for doing so, demonstrates the reversibility of matched sample and comparison stimuli'' (Cooper, Heron \& Heward, 2007, p. 705).\\
 Example: Student is taught that when given the written word dog to select the picture of a dog. Without further reinforcement or training, when given the picture of the dog, student selects the written word dog (e.g. If A=B, then B=A). 
\item Transitivity - ``A derived stimulus-stimulus relation that emerges as a product of training two other stimulus-stimulus relations'' (Cooper, Heron \& Heward, 2007, p. 706).\\
 Example: Student is taught that when given the written word dog to select the picture of the dog (e.g. A=B). Student is also taught to select the picture of the dog when hearing the spoken word dog (e.g. B=C). Without further reinforcement or training, student selects the written word dog after hearing the spoken word dog (e.g. C=A). 

\item Example of Stimulus Equivalence\\
When learner responds without prior reinforcement and training that A=A (exhibiting reflexivity) and if A=B, then B must also = A (exhibiting symmetry) and finally that if A=B and B=C, then C must also equal A (exhibiting transitivity). 

\item Non-Example of Stimulus Equivalence:\\
 When learner responds without prior reinforcement and training that A=A (exhibiting reflexivity) and if A=B, then B must also equal A (exhibiting symmetry) but cannot show that if A=B and B=C, then C must also equal A (failure to exhibit transitivity). 
%
\end{enumerate}
%
\subsection{Assessment}
\begin{enumerate}
\item Have supervisees display equivalence with respect to the words ``reflexivity,'' ``transitivity,'' and ``symmetry'' in the spoken form, written form, and written definitions.
\item Have supervisee assess for stimulus equivalence on the job or during role-play 
\item Have supervisee demonstrate an example of stimulus equivalence during role-play
%
\end{enumerate}
%
\subsection{Relevant Literature}
\begin{refsection}
\nocite{cooper2007applied,
        sidman1997equivalence,
        sidman2009equivalence}
\printbibliography[heading=none]
\end{refsection}
