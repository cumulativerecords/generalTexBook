\subsection{Examples}
\begin{enumerate}
\item In evolutionary history, our ancestors ate certain foods because it had an adaptive value as it helped ensure survival (natural selection of behavior; phylogenic selection).  The food did not necessarily need to be a reinforcer but was necessary for survival.  However, in modern times, we all have food preferences and may eat food that has no nutritional value or health benefits, indicating that specific foods are eaten because of their reinforcing value (ontogenic selection).  This type of eating habit is not adaptive (e.g., think about overeating, binge eating, obesity and the subsequent health problems that can emerge from this type of eating behavior) but it is strengthened and maintained by operant conditioning, thus reflecting selection by consequences (Skinner, 1981).  
\item Cultural Selection: Pennypacker (1992) provides examples of how selection by consequences is observed in education, economics, and politics and social organization. 
%
\end{enumerate}
%
%\clearpage \section{\fourFKThree{}}
\subsection{Examples}
\begin{enumerate}
\item A window does not just randomly bang shut; it bangs shut because a gust of wind has blown it and this has exerted enough force upon the window to close it. 
\item A water pipe does not just spontaneously burst; it bursts because there is a fault somewhere in the system, which has caused a build up of water in the pipes, resulting in so much pressure within the system that the pipe has burst.
\item A client's self-injurious behavior does not suddenly decrease after days of high rates of SIB. It decreases because the sensory-blocking procedure gradually began to extinguish the behavior.
%
\end{enumerate}
%
%\clearpage \section{\fourFKFour{}}
\subsection{Examples}
\begin{enumerate}
\item Mr. Johnson, a BCBA, conducts a functional analysis to determine the function of Billy's aggressive behavior in class. He completes rating scales, interviews, and other indirect assessment procedures, but doesn't use these to guess the reinforcer for Billy's aggression. Mr. Johnson uses the indirect assessment procedures to inform his experiment. He designs a pairwise functional analysis and runs out several phases of direct observation until results are conclusive. He concludes that Billy's aggressive behavior is sensitive to attention as a maintaining variable. At the IEP meeting, Billy's parents applaud Mr. Johnson's empiricism for completing such a thorough assessment and analyzing all the possible factors before determining a function. 
\item Mr. Riley is a district BCBA and has been asked to conduct a functional behavior assessment for Mary in regards to her aggressive behavior. Mr. Riley hypothesizes that Mary is engaging in aggressive behavior to get access to her dolls because all little girls like dolls. Based on his reasoning he has already decided that Mary's aggression is maintained by access to dolls. Since he already has a strong hypothesis for the function of aggression, Mr. Riley writes a report and creates a treatment for Mary.

\end{enumerate}
%
%\clearpage \section{\fourFKFive{}}
\subsection{Examples}
\begin{enumerate}
\item A non-verbal client hits her head repeatedly for a period of days each month. Although the analyst considered multiple environmental antecedent and consequent factors that might influence the client's behavior, she first looked at the calendar to see if the client's head hitting each month corresponded to her monthly menstrual cycle. She found that head-hitting occurred the last days immediately before her period began and the first day of her period. The analyst asked a nurse to review the data and recommend a medical intervention before the analyst continued to assess the influence of external environmental factors. 
\item An analyst was asked to design strategies for staff when responding to a client's aggressive behavior after asking him to brush his teeth. He reviewed the data staff had recorded about self-care behaviors and saw that aggression was a relatively new behavior during teeth-brushing. He learned that the client became aggressive toward staff shortly after they began buying a discounted toothpaste instead of the client's usual brand. When the analyst offered the client a choice between the two brands, the client chose his old brand and aggression did not occur after asking him to brush his teeth.
\end{enumerate}
%
%\clearpage \section{\fourFKSix{}}
\subsection{Examples}
\begin{enumerate}
\item Gloria was looking for a reinforcement program for her classroom because her students were not turning their homework in on time. She consulted with the district BCBA and was able to come up with an effective and simple class-wide reinforcement program that helped her students to turn their homework on time.
\item (Non-example) Richard was a first grade teacher and wanted to represent his student's data using a scatterplot graph. However, he did not have previous training in this area and was unable to accomplish this task. He felt that this method for graphical display was too difficult to figure out.
%
\end{enumerate}
%
