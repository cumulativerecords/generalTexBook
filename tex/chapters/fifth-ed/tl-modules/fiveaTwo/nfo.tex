%\clearpage \section{\fourFKTwo{}}
\subsection{Definition} 
Selectionism - refers to selection by consequences, a scientific paradigm, which asserts that all forms of operant behavior evolve as a result of the consequences that occurred during one's lifetime.

Skinner (1981) wrote:\\

``Human behavior is the joint product of (i) the contingencies of survival responsible for the natural selection of the specific and (ii) the contingencies of reinforcement responsible for the repertoires acquired by its members, including (iii) the special contingencies maintained by the social environment. (Ultimately, of course, it is all a matter of natural selection, since operant conditioning is an evolved process, of which cultural practices are special applications.)'' (p. 502).  

Skinner's paradigm emphasizes the role of function and draws on evolutionary theory and natural selection (phylogeny). Ontogeny refers to the learning history of an individual. Skinner viewed cultural practices as an evolved process maintained by operant conditioning. Variation in behavior is required for selection by consequences, meaning the most adaptive behavioral repertoire persists because it serves a valuable function for the organism (Cooper, Heron, \& Heward, 2007). Maladaptive, unhealthy, and harmful behavior can persist because it serves a function for the individual (e.g., substance abuse, non-suicidal self-harm, etc.).

%\clearpage \section{\fourFKThree{}}
\subsection{Definition}
Determinism - the assumption that ``the universe is a lawful and orderly place in which all phenomena occur as the result of other events'' (Cooper, Heron, \& Heward, 2007, p.5).

The implication here is that events don't just occur by accident; they occur as the result of something else happening. This is an important attitude of science because if the behavior of organisms was not orderly or lawful, scientists would be unable to identify why a behavior was occurring and therefore modify it. (Fisher, Piazza, \& Roane, 2011, p. 9) 

%\clearpage \section{\fourFKFour{}}
\subsection{Definition} 
Empiricism - ``the practice of objective observation of the phenomena of interest'' (Cooper et al., 2007, p. 5).

According to Fisher (2011), scientists make observations about the world by using information available to the senses. Sensory evidence is the primary source of information and should maintain the attitude of empiricism by believing what they observe the world to be and not what they have been taught that it should be.

%\clearpage \section{\fourFKFive{}}
\subsection{Definition}
Parsimony - The concept ``that ``simple, logical explanations must be ruled out, experimentally or conceptually, before more complex, or abstract experimentations are considered'' (Cooper, Heron, \& Heward, p. 22). 

Behavior analysts attempt to identify the simplest explanation for an individual's observed responses and then apply the least complex intervention that results in improved behavior. 
%
%\clearpage \section{\fourFKSix{}}
\subsection{Definition}
Pragmatism – ``a reasonable and logical way of doing things or of thinking about problems that is based on dealing with specific situations instead of on ideas and theories'' (Merriam-Webster.com, 2015).

ABA is an inclusive approach that is easily replicable for socially significant effects by a variety of individuals that may benefit from its methodology. Jon Bailey (2000, p. 477) stated that ``It seems to me that applied behavior analysis is more relevant than ever before and that it offers our citizens, parents, teachers, and corporate and government leaders advantages that cannot be matched by any other psychological approach...'' 

``Classroom teachers, parents, coaches, workplace supervisors, and sometimes the participants themselves implemented the interventions found effective in many ABA studies. This demonstrates the pragmatic element of ABA. Although doing ABA requires far more than learning to administer a few simple procedures, it is not prohibitively complicated or arduous'' (Cooper et al., 2007, p. 19). 

In other words, the pragmatism of ABA is in its practicality and justification of methods that give it appeal to a wider audience compared to other sciences searching for ``truth.''

