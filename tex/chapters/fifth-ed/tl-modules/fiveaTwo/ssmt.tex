\subsection{Assessment}
\begin{enumerate}
\item Ask supervisee to define selectionism.
\item Ask supervisee to read and summarize the relevant literature, while highlighting examples that reflect selectionism.
\item Ask supervisee to provide an example of behavior maintained by selectionism.
%
\end{enumerate}
%
%\clearpage \section{\fourFKThree{}}
\subsection{Assessment}
Ask your Supervisee to give an explanation to the scenarios below as they relate to determinism.
\begin{enumerate}
\item A glass breaks. (Possible answers could include = someone knocks it over causing it to break, the wind blows through the window and knocks it over causing it to smash, the dog kicks the glass as he runs past it, resulting in it breaking).
\item A client begins to emit correct responses when tacting the colors purple and green, which he has previously had a low rate of correct responses for. (Possible answers could include = the intervention which is in place has resulted in the appropriate stimulus control being acquired for purple and green, leading to an increased rate of correct responding; the client's Mother has been working on the colors green and purple at home with him; the client has been observing a peer tact the colors purple and green during play).
\item A client's sleep pattern changes and he begins to refuse to go to bed at 10 pm but instead will not go to bed until 3 am. (Possible answers could include = the client has been reinforced when he has been up late as his Mother has permitted him to watch TV; the client has not been woken until 4 pm in the afternoon after not going to bed until 3.30 am the night before; a client's medication has been changed and this results in disturbed sleep and periods of insomnia). 
\end{enumerate}
%
\subsection{Assessment}
\begin{enumerate}
\item Have supervisee describe the term empiricism and how it relates to applied behavior analysis. Have him/her identify ways that they can make sure that their work is empirically based. 
\item Have supervisee read an article on decreasing problematic behavior. Have him/her identify what makes this article empirically sound. 
\item Have supervisee list practices that are not empirically based and then identify what the individual could do to make sure that they were practicing appropriate empiricism. 
\end{enumerate}
%
%\clearpage \section{\fourFKFive{}}
\subsection{Assessment}
\begin{enumerate}
\item Give the supervisee 3 scenarios and ask the supervisee to consider what might be a parsimonious (simplest that works) first approach for each situation.
\item Example of a scenario for assessment: An adult client sometimes asks to go outside before breakfast. He screams and refuses to eat when he is made to sit at the table instead of going outside. Staff believe he should eat a good breakfast as part of his regular morning routine before he begins activities. A parsimonious response from the analyst might be to suggest that staff add a choice step before breakfast in which staff ask the client if he would like to go outside for 5 minutes before he eats. 
%
\end{enumerate}
%
%\clearpage \section{\fourFKSix{}}
\subsection{Assessment}
\begin{enumerate}
\item Have supervisee explain reinforcement, punishment, mand, and tact in simple, pragmatic terms that a layperson could apply. 
\item Have supervisee identify and describe some common ABA practices and techniques that are used by professionals who have not been directly trained in ABA. Have him/her describe why these approaches represent the pragmatic nature of applied behavior analysis.  
%
\end{enumerate}
%
