\clearpage \section{\fourFKTwo{}}
\subsection{Definition} 
Selectionism - refers to selection by consequences, a scientific paradigm, which asserts that all forms of operant behavior evolve as a result of the consequences that occurred during one's lifetime.

Skinner (1981) wrote:\\

``Human behavior is the joint product of (i) the contingencies of survival responsible for the natural selection of the specific and (ii) the contingencies of reinforcement responsible for the repertoires acquired by its members, including (iii) the special contingencies maintained by the social environment. (Ultimately, of course, it is all a matter of natural selection, since operant conditioning is an evolved process, of which cultural practices are special applications.)'' (p. 502).  

Skinner's paradigm emphasizes the role of function and draws on evolutionary theory and natural selection (phylogeny). Ontogeny refers to the learning history of an individual. Skinner viewed cultural practices as an evolved process maintained by operant conditioning. Variation in behavior is required for selection by consequences, meaning the most adaptive behavioral repertoire persists because it serves a valuable function for the organism (Cooper, Heron, \& Heward, 2007). Maladaptive, unhealthy, and harmful behavior can persist because it serves a function for the individual (e.g., substance abuse, non-suicidal self-harm, etc.).

\subsection{Examples}
\begin{enumerate}
\item In evolutionary history, our ancestors ate certain foods because it had an adaptive value as it helped ensure survival (natural selection of behavior; phylogenic selection).  The food did not necessarily need to be a reinforcer but was necessary for survival.  However, in modern times, we all have food preferences and may eat food that has no nutritional value or health benefits, indicating that specific foods are eaten because of their reinforcing value (ontogenic selection).  This type of eating habit is not adaptive (e.g., think about overeating, binge eating, obesity and the subsequent health problems that can emerge from this type of eating behavior) but it is strengthened and maintained by operant conditioning, thus reflecting selection by consequences (Skinner, 1981).  
\item Cultural Selection: Pennypacker (1992) provides examples of how selection by consequences is observed in education, economics, and politics and social organization. 
%
\end{enumerate}
%
\subsection{Assessment}
\begin{enumerate}
\item Ask supervisee to define selectionism.
\item Ask supervisee to read and summarize the relevant literature, while highlighting examples that reflect selectionism.
\item Ask supervisee to provide an example of behavior maintained by selectionism.
%
\end{enumerate}
%
\subsection{Relevant Literature}
\begin{refsection}
\nocite{cooper2007applied,
        pennypacker1992behavior,
        skinner1981selection}
\printbibliography[heading=none]
\end{refsection}
%
\subsection{Related Tasks}
\fourFKFifteen{}\\
\fourFKThirtyOne{}\\
\fourFKThirtyThree{}\\
\fourFKFourtyOne{}\\
\fourFKFourtyTwo{}\\
%
\clearpage \section{\fourFKThree{}}
\subsection{Definition}
Determinism - the assumption that ``the universe is a lawful and orderly place in which all phenomena occur as the result of other events'' (Cooper, Heron, \& Heward, 2007, p.5).

The implication here is that events don't just occur by accident; they occur as the result of something else happening. This is an important attitude of science because if the behavior of organisms was not orderly or lawful, scientists would be unable to identify why a behavior was occurring and therefore modify it. (Fisher, Piazza, \& Roane, 2011, p. 9) 

\subsection{Examples}
\begin{enumerate}
\item A window does not just randomly bang shut; it bangs shut because a gust of wind has blown it and this has exerted enough force upon the window to close it. 
\item A water pipe does not just spontaneously burst; it bursts because there is a fault somewhere in the system, which has caused a build up of water in the pipes, resulting in so much pressure within the system that the pipe has burst.
\item A client's self-injurious behavior does not suddenly decrease after days of high rates of SIB. It decreases because the sensory-blocking procedure gradually began to extinguish the behavior.
%
\end{enumerate}
%
\subsection{Assessment}
Ask your Supervisee to give an explanation to the scenarios below as they relate to determinism.
\begin{enumerate}
\item A glass breaks. (Possible answers could include = someone knocks it over causing it to break, the wind blows through the window and knocks it over causing it to smash, the dog kicks the glass as he runs past it, resulting in it breaking).
\item A client begins to emit correct responses when tacting the colors purple and green, which he has previously had a low rate of correct responses for. (Possible answers could include = the intervention which is in place has resulted in the appropriate stimulus control being acquired for purple and green, leading to an increased rate of correct responding; the client's Mother has been working on the colors green and purple at home with him; the client has been observing a peer tact the colors purple and green during play).
\item A client's sleep pattern changes and he begins to refuse to go to bed at 10 pm but instead will not go to bed until 3 am. (Possible answers could include = the client has been reinforced when he has been up late as his Mother has permitted him to watch TV; the client has not been woken until 4 pm in the afternoon after not going to bed until 3.30 am the night before; a client's medication has been changed and this results in disturbed sleep and periods of insomnia). 
\end{enumerate}
%
\subsection{Relevant Literature}
\begin{refsection}
\nocite{cooper2007applied,
        delprato1992some,
        fisher2014handbook}
\printbibliography[heading=none]
\end{refsection}
%
\subsection{Related Tasks}
\fourFKOne{}\\
\fourFKFour{}\\
\fourFKFive{}\\
\fourFKSix{}\\

Footnotes\\
*Please refer to 4th Ed. FK04-FK06 for a description of the other attitudes of science.\\
%
\clearpage \section{\fourFKFour{}}
\subsection{Definition} 
Empiricism - ``the practice of objective observation of the phenomena of interest'' (Cooper et al., 2007, p. 5).

According to Fisher (2011), scientists make observations about the world by using information available to the senses. Sensory evidence is the primary source of information and should maintain the attitude of empiricism by believing what they observe the world to be and not what they have been taught that it should be.

\subsection{Examples}
\begin{enumerate}
\item Mr. Johnson, a BCBA, conducts a functional analysis to determine the function of Billy's aggressive behavior in class. He completes rating scales, interviews, and other indirect assessment procedures, but doesn't use these to guess the reinforcer for Billy's aggression. Mr. Johnson uses the indirect assessment procedures to inform his experiment. He designs a pairwise functional analysis and runs out several phases of direct observation until results are conclusive. He concludes that Billy's aggressive behavior is sensitive to attention as a maintaining variable. At the IEP meeting, Billy's parents applaud Mr. Johnson's empiricism for completing such a thorough assessment and analyzing all the possible factors before determining a function. 
\item Mr. Riley is a district BCBA and has been asked to conduct a functional behavior assessment for Mary in regards to her aggressive behavior. Mr. Riley hypothesizes that Mary is engaging in aggressive behavior to get access to her dolls because all little girls like dolls. Based on his reasoning he has already decided that Mary's aggression is maintained by access to dolls. Since he already has a strong hypothesis for the function of aggression, Mr. Riley writes a report and creates a treatment for Mary.

\end{enumerate}
%
\subsection{Assessment}
\begin{enumerate}
\item Have supervisee describe the term empiricism and how it relates to applied behavior analysis. Have him/her identify ways that they can make sure that their work is empirically based. 
\item Have supervisee read an article on decreasing problematic behavior. Have him/her identify what makes this article empirically sound. 
\item Have supervisee list practices that are not empirically based and then identify what the individual could do to make sure that they were practicing appropriate empiricism. 
\end{enumerate}
%
\subsection{Relevant Literature}
\begin{refsection}
\nocite{cooper2007applied,
        baer1968some,
        fisher2014handbook,
        schmidt1992data}
\printbibliography[heading=none]
\end{refsection}
%
\subsection{Related Tasks}
\fourbOne{}\\
\fourhOne{}\\
\fourhThree{}\\
\fouriOne{}\\
\fouriThree{}\\
\fouriFive{}\\
\fourjOne{}\\
\fourjFifteen{}\\
\fourkSeven{}\\
\fourFKTen{}\\
%
\clearpage \section{\fourFKFive{}}
\subsection{Definition}
Parsimony - The concept ``that ``simple, logical explanations must be ruled out, experimentally or conceptually, before more complex, or abstract experimentations are considered'' (Cooper, Heron, \& Heward, p. 22). 

Behavior analysts attempt to identify the simplest explanation for an individual's observed responses and then apply the least complex intervention that results in improved behavior. 
%
\subsection{Examples}
\begin{enumerate}
\item A non-verbal client hits her head repeatedly for a period of days each month. Although the analyst considered multiple environmental antecedent and consequent factors that might influence the client's behavior, she first looked at the calendar to see if the client's head hitting each month corresponded to her monthly menstrual cycle. She found that head-hitting occurred the last days immediately before her period began and the first day of her period. The analyst asked a nurse to review the data and recommend a medical intervention before the analyst continued to assess the influence of external environmental factors. 
\item An analyst was asked to design strategies for staff when responding to a client's aggressive behavior after asking him to brush his teeth. He reviewed the data staff had recorded about self-care behaviors and saw that aggression was a relatively new behavior during teeth-brushing. He learned that the client became aggressive toward staff shortly after they began buying a discounted toothpaste instead of the client's usual brand. When the analyst offered the client a choice between the two brands, the client chose his old brand and aggression did not occur after asking him to brush his teeth.
\end{enumerate}
%
\subsection{Assessment}
\begin{enumerate}
\item Give the supervisee 3 scenarios and ask the supervisee to consider what might be a parsimonious (simplest that works) first approach for each situation.
\item Example of a scenario for assessment: An adult client sometimes asks to go outside before breakfast. He screams and refuses to eat when he is made to sit at the table instead of going outside. Staff believe he should eat a good breakfast as part of his regular morning routine before he begins activities. A parsimonious response from the analyst might be to suggest that staff add a choice step before breakfast in which staff ask the client if he would like to go outside for 5 minutes before he eats. 
%
\end{enumerate}
%
\subsection{Relevant Literature}
\begin{refsection}
\nocite{cooper2007applied,
        etzel1979simplest}
\printbibliography[heading=none]
\end{refsection}
%
\subsection{Related Tasks}
\fourFKOne{}\\ %FK-01: Lawfulness of behavior.
\fourFKTwo{}\\ %* FK-02: Selectionism (phylogenic, ontogenic, cultural)
\fourFKThree{}\\%* FK-03: Determinism
\fourFKFour{}\\%* FK-04: Empiricism
\fourFKSix{}\\%* FK-06: Pragmatism
%
\clearpage \section{\fourFKSix{}}
\subsection{Definition}
Pragmatism – ``a reasonable and logical way of doing things or of thinking about problems that is based on dealing with specific situations instead of on ideas and theories'' (Merriam-Webster.com, 2015).

ABA is an inclusive approach that is easily replicable for socially significant effects by a variety of individuals that may benefit from its methodology. Jon Bailey (2000, p. 477) stated that ``It seems to me that applied behavior analysis is more relevant than ever before and that it offers our citizens, parents, teachers, and corporate and government leaders advantages that cannot be matched by any other psychological approach...'' 

``Classroom teachers, parents, coaches, workplace supervisors, and sometimes the participants themselves implemented the interventions found effective in many ABA studies. This demonstrates the pragmatic element of ABA. Although doing ABA requires far more than learning to administer a few simple procedures, it is not prohibitively complicated or arduous'' (Cooper et al., 2007, p. 19). 

In other words, the pragmatism of ABA is in its practicality and justification of methods that give it appeal to a wider audience compared to other sciences searching for ``truth.''

\subsection{Examples}
\begin{enumerate}
\item Gloria was looking for a reinforcement program for her classroom because her students were not turning their homework in on time. She consulted with the district BCBA and was able to come up with an effective and simple class-wide reinforcement program that helped her students to turn their homework on time.
\item (Non-example) Richard was a first grade teacher and wanted to represent his student's data using a scatterplot graph. However, he did not have previous training in this area and was unable to accomplish this task. He felt that this method for graphical display was too difficult to figure out.
%
\end{enumerate}
%
\subsection{Assessment}
\begin{enumerate}
\item Have supervisee explain reinforcement, punishment, mand, and tact in simple, pragmatic terms that a layperson could apply. 
\item Have supervisee identify and describe some common ABA practices and techniques that are used by professionals who have not been directly trained in ABA. Have him/her describe why these approaches represent the pragmatic nature of applied behavior analysis.  
%
\end{enumerate}
%
\subsection{Relevant Literature}
\begin{refsection}
\nocite{bailey2000futurist,
        cooper2007applied,
        heward2005focus,}
\printbibliography[heading=none]
\end{refsection}
%
\subsection{Related Tasks}
\fourbOne{}\\
\fourgFour{}\\
\fourgSix{}\\
\fourkEight{}\\
\fourkNine{}\\
