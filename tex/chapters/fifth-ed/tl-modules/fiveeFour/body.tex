%\clearpage \section[\fourgEight{}]{\fourgEight{}%
              \sectionmark{G-08 Identify and make envir...}}
\sectionmark{G-08 Identify and make envir...}
\subsection{Definition}
``Behavior analysis is a science of studying how we can arrange our environments so they make very likely the behaviors we want to be probably enough, and they make unlikely the behaviors we want to be improbable'' (Cooper, Heron, \& Heward, 2007, p. 15). The behavior analyst assesses the nature of the fit between an individual and the environment in which he/she functions using a three-part contingency (antecedent, behavior, consequences) structure of events. In this process, the analyst identifies motivating operations associated with antecedent events and the consequences that maintain problem behaviors or prevent adequate development of adaptive behaviors. This information can be used to identify relatively uncomplicated proactive environmental changes that will improve the functioning of the individual. As a result of the individual's increased access to positive reinforcement, more intrusive behavior interventions may be less necessary. 
%
\subsection{Examples}
\begin{enumerate}
\item Parents found it difficult to get their daughter ready for the school bus on time. They decided they would try giving her time to shower at night instead of in the morning. They added showering to her regular nighttime routine and the daughter not only got to the school bus on-time but began to go to sleep earlier after the expanded nighttime routine.
\item A student refused to sit in his seat at school. An occupational therapist suggested a gel filled wedge for the student's chair and the student not only remained seated in one class, but chose to use the wedge in all of his classes.
\end{enumerate}
%
\subsection{Assessment}
\begin{enumerate}
\item If a student screams and covers his ears when moderate noises occur in the classroom, what would the supervisee suggest as the simplest environmental arrangement that might be considered for that student?
\item A client throws objects at staff and pounds the wall when they sit and watch the staff's favorite television shows at night in the client's apartment. What might the staff be told to change at night before the analyst designs a program for reinforcing the client's appropriate television behavior or punishing his inappropriate behavior?
\item If an analyst determines that attention is a major positive reinforcer for problem behaviors for a client, what would the supervisee consider to be an important consequence to increase following appropriate behavior?
%
\end{enumerate}
%
\subsection{Relevant Literature}
\begin{refsection}
\nocite{cooper2007applied}
\printbibliography[heading=none]
\end{refsection}
%
\subsection{Related Lessons}
\fourbTwo{}\\
\fourdTwentyOne{}\\
\foureOne{}\\
\foureThree{}\\
\fouriThree{}\\
\fouriFive{}\\
\fouriSix{}\\
\fourjFour{}\\
\fourjFive{}\\
\fourjSix{}\\
\fourjSeven{}\\
\fourjEight{}\\
\fourjTen{}\\
\fourjTwelve{}\\
\fourkNine{}\\
\fourFKFive{}\\
\fourFKSix{}\\
\fourFKTwentyThree{}\\
\fourFKTwentySix{}\\
\fourFKThirtyThree{}\\
