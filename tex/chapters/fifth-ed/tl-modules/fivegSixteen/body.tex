%\clearpage \section{\fourdSixteen{}}
\subsection{Definition}
Punishment - ``Occurs when stimulus change immediately follows a response and decreases the future frequency of that type of behavior in similar conditions'' (Cooper, Heron, \& Heward, 2007, p. 702).

Positive Punishment: ``A behavior is followed immediately by the presentation of a stimulus that decreases the future frequency of the behavior'' (Cooper, Heron, \& Heward, 2007, p. 701).*

Negative Punishment: ``...behavior is followed immediately by the removal of a stimulus (or a decrease in the intensity of the stimulus), that decreases the future frequency of similar responses under similar conditions'' (Cooper Heron, \& Heward, 2007, p. 700). *

\subsection{Examples}
\begin{enumerate}
\item (Postive Punishment) Albert is learning to drive.  Albert drives fast and speeds along the highway.  His mother provides a firm reprimand directly following the speeding.  Albert no longer speeds when his mother is in the car. 
\item (Negative Punishment) Albert is learning to drive.  Albert drives fast and speeds along the highway.  His mother removes Albert's driving privilege for a week following the speeding.  Albert no longer speeds.
\end{enumerate}
%
\subsection{Assessment}
\begin{enumerate}
\item Ask your supervisee what the behavioral process behind ``time-out'' is. 
\item Ask your supervisee to create an example of both positive and negative punishment
\item Ask supervisee to explain the difference between negative reinforcement and negative punishment
\item Have the supervisee provide labels in the boxes
\end{enumerate}
%
\subsection{Relevant Literature}
\begin{refsection}
\nocite{bailey2013ethics,
    cooper2007applied,
    foxx1982decreasing,
    van1988right}
\printbibliography[heading=none]
\end{refsection} 
%
\subsection{Related Lessons}
\fourdOne{}\\
\fourdSeventeen{}\\
\fourdNineteen{}\\
\foureEleven{}\\
\fourgSeven{}\\
\fourjTen{}\\
\fourFKNineteen{}\\
\fourFKTwenty{}\\
%
\subsection{Footnotes}
Alternatively, Fox (1982) described positive and negative punishment as Type I punishment (contingent stimulation) and Type II punishment (contingent withdrawal of a stimulus).

An emphasis on the ethical considerations of using punishment should be introduced when punishment is first discussed.
