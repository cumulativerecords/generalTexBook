%\clearpage \section{\fourdOne{}}
\subsection{Definition}
The principle of reinforcement is in operation when immediately following a behavior, a stimulus event occurs and this serves to increase the future frequency of that behavior (Cooper, Heron \& Heward, 2007).

Positive Reinforcement\\
Positive reinforcement has occurred when a stimulus is added to the environment, or is increased in intensity, immediately following a behavior, and this serves to increase the frequency of that behavior occurring in the future (Cooper, Heron \& Heward, 2007).

Negative Reinforcement\\
Negative reinforcement has occurred when a stimulus is removed, or decreased in intensity, immediately following a behavior, and this increases the frequency of that behavior in the future (Cooper, Heron \& Heward, 2007).
%%
%\clearpage \section[\fourdNineteen{}]{\fourdNineteen{}%
\subsection{Definition}
A guideline for using extinction effectively is the simultaneous use of reinforcement procedures. Extinction used alone may result in a temporary increase in the target behavior known as extinction burst (Cooper, Heron, \& Heward, 2007). Teaching an alternative behavior may decrease the extinction burst effects and other possible side effects such as aggressive behavior (Lerman, Iwata, \& Wallace, 1999). Similarly, several studies have found that differential reinforcement procedures are most effective when used in conjunction with extinction (e.g., Fisher, Piazza, Cataldo, Harrell, Jefferson, \& Conner, 1993; Hagopian, Fisher, Sullivan, Acquisto, \& LeBlanc, 1998; Piazza, Patel, Gulotta, Sevin, \& Layer, 2002).  Using extinction with differential reinforcement ensures that concurrent access to reinforcement for inappropriate behavior is not favored, thereby increasing the likelihood of allocation to the alternative behavior targeted for increase.

Likewise, punishment procedures are most effective when used in conjunction with reinforcement-based procedures (Millenson, 1967).  The main rationale for use of reinforcement with punishment procedures is that punishment is considered an intrusive treatment procedure. In addition, punishment procedures do not teach the individual any appropriate skills. Teaching appropriate skills helps to solve these problems. Reinforcing an alternative behavior makes it more likely that punishment procedures can then be faded out as the appropriate behavior replaces the inappropriate behavior. A study conducted by Holz, Azrin and Ayllon (1963) even found that punishment was ineffective without the use of reinforcement contingencies.  Several studies have illustrated the benefits of using reinforcement in conjunction with punishment procedures (e.g., Fisher et al., 1993; Hagopian et al., 1998; Thompson, Iwata, Conners, \& Roscoe, 1999).
%
