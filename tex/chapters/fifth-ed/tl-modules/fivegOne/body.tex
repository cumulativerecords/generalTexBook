\clearpage \section{\fourdOne{}}
\subsection{Definition}
The principle of reinforcement is in operation when immediately following a behavior, a stimulus event occurs and this serves to increase the future frequency of that behavior (Cooper, Heron \& Heward, 2007).

Positive Reinforcement\\
Positive reinforcement has occurred when a stimulus is added to the environment, or is increased in intensity, immediately following a behavior, and this serves to increase the frequency of that behavior occurring in the future (Cooper, Heron \& Heward, 2007).

Negative Reinforcement\\
Negative reinforcement has occurred when a stimulus is removed, or decreased in intensity, immediately following a behavior, and this increases the frequency of that behavior in the future (Cooper, Heron \& Heward, 2007).
%%
\subsection{Examples}
\begin{enumerate}
\item Mommy Singing
\item A mommy sings a verse from ``The Wheels on the Bus'' nursery rhyme to her baby. The baby giggles immediately following the nursery rhyme being sung. The mommy is more likely to sing this nursery rhyme to her baby in the future, because the baby's giggling serves as a reinforcer.
\end{enumerate}
 
\subsection{Examples}
\begin{enumerate}
\item (Positive Reinforcement) A rat is in a cage, which has a food dispenser lever. To access food, the rat pushes the lever and food pellets are dispensed. In the future, after a few hours of food deprivation, the rat is more likely to push the lever to access food. 
\item (Negative Reinforcement) Negative reinforcement has occurred when a stimulus is removed, or decreased in intensity, immediately following a behavior, and this increases the frequency of that behavior in the future (Cooper, Heron \& Heward, 2007).
\item (Negative Reinforcement) A child asks her friend to turn down the music, as it is too loud for her. The music is immediately turned down. The child is more likely to ask for music to be turned down in the future when it is too loud.
\item (Negative Reinforcement) It is raining so you put up your umbrella and immediately reduce the amount that you are getting wet. You are more likely to put up your umbrella in the future when it is raining to avoid getting wet. 
\end{enumerate}
%
\subsection{Assessment}
\begin{enumerate}
\item Ask your Supervisees to define both positive and negative reinforcement and give an example of each type.
\item Ask your Supervisee to demonstrate an example of positive and negative reinforcement through role-playing.
\item Ask your Supervisee to identify which type of reinforcement is operating in these examples:
\begin{enumerate}
\item A Supervisee sees her BCBA Supervisor coming in to the classroom to run teaching evaluations. The Supervisee takes her client for a reinforcer break in another room and thereby avoids having a teaching evaluation completed on her. The next day, when her Supervisor comes back in to do more evaluations, she takes her client outside to play. (Answer = the Supervisee's behavior of leaving the classroom is negatively reinforced by avoiding having teaching evaluations completed on her).
\item A client regularly has challenging behavior when working with his teacher at the table, as he does not enjoy the work. However, on this occasion he has worked really well so has immediately been given a 3-minute break to run around in the playground. He continues to work well at the table after his break. (Answer = the client's hard work at the table may be being negatively reinforced. This is because his work behavior is being reinforced with removal from the table environment). 
\item A Father calls his son down to the dishes. His son comes down and does the dishes. His Father says ``Thanks, Son'' and gives him a dollar to get some candy. The next day when the son is asked to do the dishes again, he quickly comes down to do them. (Answer = the son's compliant behavior was probably positively reinforced with praise and/or money).
\end{enumerate}
\end{enumerate}
%
\subsection{Relevant Literature}
\begin{refsection}
\nocite{hall1968effects,
    hart1968effect,
    michael1975positive,
    osborne1969free,
    skinner1938behavior,
    thomas1968production}
\printbibliography[heading=none]
\end{refsection}
%
\subsection{Related Lessons}
\fourcOne{}\\
\fourdTwo{}\\
\fourdSixteen{}\\
\fourdSeventeen{}\\
\fourdNineteen{}\\
\fourdTwenty{}\\
\fourdTwentyOne{}\\
\foureTen{}\\
\foureEleven{}\\
\fourfTwo{}\\
\fourFKFourteen{}\\
\fourFKFifteen{}\\
\fourFKSeventeen{}\\
\fourFKEighteen{}\\
\fourFKNineteen{}\\
\fourFKTwenty{}\\
\fourFKTwentyOne{}\\
\fourFKTwentyThree{}\\
%
\clearpage \section[\fourdNineteen{}]{\fourdNineteen{}%
              \sectionmark{D-19 Use combinations of reinforce...}}
\subsection{Definition}
A guideline for using extinction effectively is the simultaneous use of reinforcement procedures. Extinction used alone may result in a temporary increase in the target behavior known as extinction burst (Cooper, Heron, \& Heward, 2007). Teaching an alternative behavior may decrease the extinction burst effects and other possible side effects such as aggressive behavior (Lerman, Iwata, \& Wallace, 1999). Similarly, several studies have found that differential reinforcement procedures are most effective when used in conjunction with extinction (e.g., Fisher, Piazza, Cataldo, Harrell, Jefferson, \& Conner, 1993; Hagopian, Fisher, Sullivan, Acquisto, \& LeBlanc, 1998; Piazza, Patel, Gulotta, Sevin, \& Layer, 2002).  Using extinction with differential reinforcement ensures that concurrent access to reinforcement for inappropriate behavior is not favored, thereby increasing the likelihood of allocation to the alternative behavior targeted for increase.

Likewise, punishment procedures are most effective when used in conjunction with reinforcement-based procedures (Millenson, 1967).  The main rationale for use of reinforcement with punishment procedures is that punishment is considered an intrusive treatment procedure. In addition, punishment procedures do not teach the individual any appropriate skills. Teaching appropriate skills helps to solve these problems. Reinforcing an alternative behavior makes it more likely that punishment procedures can then be faded out as the appropriate behavior replaces the inappropriate behavior. A study conducted by Holz, Azrin and Ayllon (1963) even found that punishment was ineffective without the use of reinforcement contingencies.  Several studies have illustrated the benefits of using reinforcement in conjunction with punishment procedures (e.g., Fisher et al., 1993; Hagopian et al., 1998; Thompson, Iwata, Conners, \& Roscoe, 1999).
%
\subsection{Assessment}
\begin{enumerate}
\item Provide the supervisee with articles from the relevant literature and discuss the findings and their considerations for clinical application. 
\item Ask the supervisee to list the problems associates with using extinction procedures alone 
\item Ask supervisee to list the benefits of using reinforcement based procedures with extinction
\item Ask the supervisee to list the problems associates with using punishment procedures alone
\item Ask supervisee to list the benefits of using reinforcement based procedures with punishment
\end{enumerate}
%
\subsection{Relevant Literature}
\begin{refsection}
\nocite{cooper2007applied,
        fisher1993functional,
        hagopian1998effectiveness,
        holz1963elimination,
        lerman1999side,
        millenson1967principles,
        piazza2003relative,
        thompson1999effects}
\printbibliography[heading=none]
\end{refsection}
\subsection{Related Lessons} 
\fourbTen{}\\
\fourcOne{}\\ 
\fourcTwo{}\\ 
\fourcThree{}\\
\fourdOne{}\\
\fourdTwo{}\\
\fourdFifteen{}\\
\fourdSixteen{}\\
\fourdSeventeen{}\\
\fourdEighteen{}\\
\fourdTwenty{}\\
\fourdTwentyOne{}\\
\fouriSix{}\\
\fouriSeven{}\\
\fourjTwo{}\\
\fourjTen{}\\
