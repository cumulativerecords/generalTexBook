\clearpage \section{\fourdSix{}}
\subsection{Definition}  
Behavior chain - ``A sequence of responses in which each response produces a stimulus change that functions as conditioned reinforcement for that response and as a discriminative stimulus for the next response in the chain; reinforcement for the last response in a chain maintains the reinforcing effectiveness of the stimulus changes produced by all previous responses in the chain'' (Cooper, Heron, \& Heward, 2007, p. 691).

Often behavior chains are taught using a task analysis. This involves breaking down steps of a sequence or routine and sequentially teaching them one at a time. Reinforcement is delivered following the first step, then following each succeeding step if the previous steps were completed in correct order.

According to Cooper, et al., ``a behavior chain has the following three important characteristics: (a) A behavior chain involves the performance of a specific series of discrete responses; (b) the performance of each behavior in the sequence changes the environment in such a way that it produces conditioned reinforcement for the preceding responses and serves as an SD  for the next response; and (c) the responses within the chain must be performed in a specific sequence, usually in close temporal succession'' (2007, p. 436).

``In forward chaining, behaviors are linked together beginning with the first behavior in the sequence. In backward chaining, behaviors are linked together beginning with the last behavior in the sequence'' (Cooper et al., 2007, p. 436). 

\subsection{Examples}
\begin{enumerate}
\item Brad is going to teach shoe tying to one of his students. He decides to use a forward chain and writes down all the smaller steps involved in tying ones shoes. He teaches the steps at the beginning, like crossing the laces. After the student has completed this step Brad delivers reinforcement.  Then moves down the steps in sequential order delivering reinforcement at the completion of the last step that was taught.
%
\item (Non-example) Joe decides to use video modeling to help his student learn to tie their shoes. He has them watch the video daily and then model what they learned from that video in an attempt to have them learn how to tie their shoes.
\end{enumerate}
%
\subsection{Assessment}
\begin{enumerate}
\item Have supervisee choose a target skill and create a task analysis for that skill. Then have him/her decide whether to teach this skill using forward or backward chaining and explain their rationale.
\item Have supervisee identify and describe the benefits and limitations of using both forward and backward chains. Have him/her list several behaviors/skills that they would use for each type of chaining procedure and why they chose that method.
\item Have supervisee create a task analysis procedure for a skill they can demonstrate in the group supervision setting. Have him/her model the training procedure. Finally, have him/her role play being a teacher and the supervisor being the student as they demonstrate the steps of their task analysis chain.
\end{enumerate}
%
\subsection{Relevant Literature}
\begin{refsection}
\nocite{cooper2007applied,
    catania1998learning,
    libby2008comparison,
    kayser1986comparison,
    spooner1984comparisons,
    reynolds1975primer,
    mcwilliams1990teaching,
    test1990teaching,
    snell2006instruction}
\printbibliography[heading=none]
\end{refsection}
%
\subsection{Related Lessons}
\fouraSeven{}\\
\fourdThree{}\\
\fourdFour{}\\
\fourdFive{}\\
\foureOne{}\\
\foureTwo{}\\
\fouriOne{}\\
\fourjThree{}\\
\fourFKTen{}\\
