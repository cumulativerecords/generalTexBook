%%\clearpage \section{\fourdSix{}}
\subsection{Definition}  
Behavior chain - ``A sequence of responses in which each response produces a stimulus change that functions as conditioned reinforcement for that response and as a discriminative stimulus for the next response in the chain; reinforcement for the last response in a chain maintains the reinforcing effectiveness of the stimulus changes produced by all previous responses in the chain'' (Cooper, Heron, \& Heward, 2007, p. 691).

Often behavior chains are taught using a task analysis. This involves breaking down steps of a sequence or routine and sequentially teaching them one at a time. Reinforcement is delivered following the first step, then following each succeeding step if the previous steps were completed in correct order.

According to Cooper, et al., ``a behavior chain has the following three important characteristics: (a) A behavior chain involves the performance of a specific series of discrete responses; (b) the performance of each behavior in the sequence changes the environment in such a way that it produces conditioned reinforcement for the preceding responses and serves as an SD  for the next response; and (c) the responses within the chain must be performed in a specific sequence, usually in close temporal succession'' (2007, p. 436).

``In forward chaining, behaviors are linked together beginning with the first behavior in the sequence. In backward chaining, behaviors are linked together beginning with the last behavior in the sequence'' (Cooper et al., 2007, p. 436). 

