\chapter{Preface}
%\noindent {\Huge Preface\par}
This book continues to be a labor of love. It serves as an excuse to incorporate GNU/Linux command line tools into the soul-crushing Microsoft\textregistered{} workflow of many behavioral service providers. 

Anyone is welcome to participate in the development of this book. All of the source code is publicly available, including the latest PDF version. The code is the maintainer's way of saying ``reinforcement is available'' for tinkering with the files on a personal computer. Most behavior analysts know that our community can benefit from a larger number of practitioners with computer programming competencies. Many of us have left one Ron Van Houten's talks, fired up like Christians on Sunday after a rousing sermon. We went to free websites to learn computer programming, ready to write code that would contribute to the behavior analysis community. Something about that approach did not work. We found a disconnect in terms of learning objectives. 

What was the problem? Part of the issue is that online programming tutorials train skills for computer scientists. In behavior analysis, we do not necessarily have a way to practice those skills at work. When we come home from a long day of house calls  or maybe school-based or center-based service hours, we have other things to do that get in the way of learning code for people in scientific professions. Most of us like the science but we took less science classes in college than the target audience of these online coding programs, which build from curriculum in math, physics, and chemistry. 

Practicing behavior analysts are aware tha new skills need to be maintained, so they look at their own lives in search of opportunities to practice coding skills. When can the practitioner actually use something they coded at work? Seeing the lack of opportunities to use and maintain these hard-to-acquire skills, practitioners give up. The goal is for practitioners to learn programming and remain in the industry, not have to become full-blown software developers. The result would be a loss of valuable practitioners. Putting these ideas together, the motivation to learn programming quickly fades when practitioners cannot find opportunities to practice skills. It is hardly worth, the effort needed to learn code.

If the goal of the preface was to set a dismal tone, this book already a success. There is a ray of light in the sea of misaligned contingencies. Practicing behavior analysts do not have contingencies in place for learning programming from mainstream curriculum. However, they have contingencies in place to learn behavior analysis content very well (for students) or develop very good behavior analysis curriculum at work (for supervisors). This book provides curriculum with source code. Hence, this book offers reinforcement for programming behavior for professional behavior analysis supervisors and students. 

Less formally, this book is an invitation for behavior analysis practitioners to hack the TrainABA supervision curriculum to suit individualized or cohort-wide needs. Anyone can copy, add, remove and modify text or whole sections of this book. This is possible due to the copyright license (Creative Commons 4.0 - Attribution - Sharealike - Non-commercial).

The code is maintained through Cumulative Records Documentation Society (CRDS), a 501(c)(3) nonprofit. GitHub, a major software development organization, has generously provided CRDS with lifelong sponsorship for projects like this one.

Casual readers may expect that computer programming requires a  special development environment on one's computer. It may not be surprising to consider that regular people cannot tinker with code from \textit{proprietary software}, software for which the publisher does not make source code available to the public. It may be surprising to discover that a whole world exists around free and open source software (FOSS). Such software has huge communities with values compatible with the behavior analysis community. Both communities stemmed from a scientific tradition of opening their doors to welcome new practitioners. Major publishers use proprietary software to build their books and cannot not make changes to the interface themselves. As a punk-rock inspired, wannabe indie record label, CRDS publishes stood up to proprietary software and went the hard route of using open source tools that make source code available to the masses. It is not quite two-turntables and a microphone but damn if it isn't close.

What is the real harm of using proprietary software to write a book like this? Is there a real benefit to a FOSS book for supervision curriculum? The major benefits are below.

\begin{enumerate}
\item There is no loss of quality by using FOSS tools. 

\item FOSS projects are organized. Developers must organize code nicely enough for random people to understand it or the project will not get much use. Proprietary software may or may not be organized. Nobody can see the code, so we do not know!

\item FOSS code is transparent. There is a safety aspect to FOSS code. You can see all the code so you know what it is doing. However, some people just download and run code without reading it. In these cases, the safety of users of the FOSS project is determined by the maintainers.
 
\item FOSS is customizable. The biggest benefit is that one can write code to automatically generate customized versions of a main document (or multiple documents). Behavior analysis practitioners tend to customize things, so this can be a major benefit. There are many available ideas for do-it-yourself supervision curriculum that do not automatically generate materials. The contents of this book can be used for slideshows, posters, handouts, flashcards, and study materials. The license makes it possible to share the information. These are major advantages over do-it-yourself approaches. Through the GitHub website, it easy for individual practitioners and organizations to have their own versions of the supervision curriculum. 

\item FOSS is practical. All typesetting software has a learning curve. Unlike proprietary software, which allows users to work on files in the end-user interface, FOSS code is written in a coding environment.  are much better at training  that generalize to other skills and settings. Most major publishers typeset books using expensive and complicated software. These programs are like word processors on steroids. The learning curve is high. Knowledge gained through learning the software does not generalize to other software, skills, and settings. 

\item FOSS is better in the long term. The reward for long-term development of FOSS projects is better code (more stable, easier to use, great for long-term). The reward for using proprietary software is \textit{feature creep}, where new versions add features to remain competitive but do not remove old features because it would upset legacy users. The changes were not requeted by the users, so every version comes with a new learning curve that fails to generalize to other skills and settings. ave changes to the interface or functionality that make the software hard to use. The software developers add features with new versions to compete in the industry. 

\end{enumerate}

What is the harm of using proprietary software? The harm is evident when considering these factors in context. Any software with a high learning is meant for specialized knowledge workers. In the publishing industry, people typeset as a full-time job. If it takes a few years to get the hang of it, that's okay because higher barriers to expertise increase the worker's value to the organization. What about the fancy features added each year? The new features run on new code, which takes more system resources. Eventually, new features clutter the interface. A laptop-sized monitor may no longer give enough space for the reader to see the important icons when laying out a book. Some developers solve this problem with submenus and subsubmenus, meaning there are multiple layers of hidden features. This frustrates amateur users because they have to learn the underlying logic and idiosyncratic terminology of the proprietary software. In FOSS software, the community catches the idiosyncrasies and makes information as accessible as possible.

The software is designed to run on the two dominant proprietary operating systems, Windows and MacOS. As these operating systems grow in complexity, they require their own cycle of hardware upgrades. The point word processing technology choices for this book were a form of artistic expression. These tools promote community in the century-long development of behavior analytic services as a profession. 

\section{For the Rebellion}
The act of using these tools was meant to be rebellious, if not subversive, against the way large books are usually developed. The tools came from the GNU/Linux programming environment. do-it-yourself spirit of Skinner's operant chambers. It was pure hip-hop with two turntables and a sampler. It was grunge rock singing love songs in an old garage. It was a \$200 single-subject design study at a university where other departments held out for seven figure grants.

The computers used to build this book were a nod to the traditional applied behavior analysis studies, which could be conducted for cheap with a clipboard and some doctoral students. The best example was an old Lenovo ThinkPad X200, purchased second-hand from Craigslist with cash. This was a statement against consumerism. It said no to ``upgrading'' to next-generation CPUs that ran the telemetry nightmare known as Windows 10.

For this statement, the X200 was perfect. It was golden-era hip hop in all its sound sampling glory. The laptop had a battery life of 23 minutes and came pre-installed with Hello Kitty stickers on certain keys. The stickers said things like, ``return,'' ``shift,'' and ``a.'' And yes, all stickers were placed correctly. The X200 ran a free and open source operating systems powered by GNU/Linux and approved by the Free Software Foundation.

This book was proudly typeset using \LaTeX{}, a free and open source software (FOSS). The FOSS ideology was congruent with the ideas expressed in B.F. Skinner's writings on culture. Skinner lamented the difficulties of communicating non-physical technologies to new audiences because audiences would judge the technology by its creator's moral values. Audiences evaluate physical technologies independent of their creators, which was preferred. 

Skinner's writings on culture inspired this book's maintainer resolved to learn computer programming skills for the second edition of the TrainABA Supervision Curriculum Series. The goal was to add a layer of physical (through code) technology to a book whose subject was a non-physical technology. Hence, it built on Skinner's vision of better communicating the technology of behavior analysis to general audiences. The technology for the book will always be publicly available, allowing for modifications by any users. The version control notes remain public, so anyone could see the process of creating the book. Collaboration and issue tracking is displayed publicly. In general, the distribution technology increases the likelihood that people will use current releases.

\LaTeX{} allowed modularization of the book's files. It separated files containing written content from those that generated the formatting. Any task list name, reference, or glossary item was tagged with a variable. The code ran through a compiler to assemble the PDF. The compiler's huge string of info messages was like witnessing live art. The creation process happened right before one's eyes. 

The aspect of building the book was inspired by stories of Skinner's lab at Harvard. As legends go, one could hear the symphony of operant chambers clicking and clacking throughout the halls. The stories differ but common ground is that Skinner's lab ran concurrent experiments. In this book, no live animals were used but the modular design allowed for many files to operate at the same time. \LaTeX{} runs procedurally, going step by step through every line of code and jumping to/from the files referenced. In Skinner's lab, experiments were modular such that whatever happened in one operant chamber without tampering with experiments in nearby operant chambers. For this book, the file system was modular such that any file could be added, removed, modified, or replaced without tampering with other files.
    
The books were assembled and written using Vim, a free and open source command line text editor. It was compiled using custom shell scripts in a Bash terminal. Git and GitHub were used for version control. The decision to use a command line editor was inspired by operant chambers. One need only peck at the keys to write, assemble, and distribute this book. No mouse, trackpad, eyes, or graphic user interface is involved. An experienced operator could complete the process blindfolded. The author suffered from tendonitis and frequent eyestrain from extensive computer use, so the command line technology was a welcome option.

While the specifics of the technology choices for this book may seem gratuitous, the purpose was to inspire behavior analysis professionals to indulge their curiosity in computer programming. The intended message is, roughly, that even an English major can learn to write code if the project is interesting. Perhaps others from non-programming backgrounds will consider this project as an invitation to download and tinker with code.

Please submit errors, additions, improvements, and suggested omissiongs using the GitHub Issue Tracker. Nothing posted in the modern day mead-halls of Facebook Groups will be read by those who maintain this book.

All readers may use, copy, modify, and distribute this book and its files. Hard copies are available. Custom builds are available for companies, universities, and others. Please contact CRDS for more information about how to use intellectual property for this book or make content contributions. The contact is postmaster(at)cumulativerecords.org.

The Behavior Analyst Certification Board provided a copyright license for use of the BCBA/BCaBA Task List, 4th and 5th editions. No reprinting of those materials are allowed without the express written consent of the BCBA/BCaBA. Statements of free and open source licensing of this book do not apply to the BCBA/BCaBA Task List, which is the sole property of the Behavior Analyst Certification Board. For more information, visit www.bacb.com.

Some of the content in this book builds upon work from contributors to its first edition. This book represents significant revision, rewriting, and reorganization, to the point that it is a completely different book representing original content. The first edition continues to be available through the GitHub repositories. It has been made publicly available under a Creative Commons 4.0 - Attribution - Sharealike - Noncommercial license since 2017.
\clearpage
