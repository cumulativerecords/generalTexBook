\chapter{Preface}
%\noindent {\Huge Preface\par}
This book continues to be a labor of love. It serves as an excuse to incorporate GNU/Linux command line tools into the soul-crushing Microsoft\textregistered{} workflow of many behavioral service providers. 

Anyone is welcome to participate in the development of this book and other projects like it. All of the source code, including the latest PDF version, is publicly available. 

The code is maintained through Cumulative Records Documentation Society (CRDS), a 501(c)(3) nonprofit. GitHub, a software development organization, has generously provided CRDS with lifelong sponsorship for projects like this one.

The technology choices for this book were a form of artistic expression. These tools promote community in the century-long development of behavior analytic services as a profession. The act of using these tools was meant to be rebellious, if not subversive, against the way large books are usually developed. The technology  for this book reflected the do-it-yourself spirit of Skinner's hands-on work with operant chambers. It was pure hip-hop with two turntables and a sampler. It was grunge rock singing love songs in an old garage. It was a \$200 single-subject design study at a university where other departments held out for seven figure grants.

The computers used to build this book were a nod to the traditional applied behavior analysis studies, which could be conducted for cheap with a clipboard and some doctoral students. The best example was an old Lenovo ThinkPad X200, purchased second-hand from Craigslist with cash. This was a statement against consumerism. It said no to ``upgrading'' to next-generation CPUs that ran the telemetry nightmare known as Windows 10.

For this statement, the X200 was perfect. It was golden-era hip hop in all its sound sampling glory. The laptop had a battery life of 23 minutes and came pre-installed with Hello Kitty stickers on certain keys. The stickers said things like, ``return,'' ``shift,'' and ``a.'' And yes, all stickers were placed correctly. The X200 ran a free and open source operating systems powered by GNU/Linux and approved by the Free Software Foundation.

This book was proudly typeset using \LaTeX{}, a free and open source software (FOSS). The FOSS ideology was congruent with the ideas expressed in B.F. Skinner's writings on culture. Skinner lamented the difficulties of communicating non-physical technologies to new audiences because audiences would judge the technology by its creator's moral values. Audiences evaluate physical technologies independent of their creators, which was preferred. 

Skinner's writings on culture inspired this book's maintainer resolved to learn computer programming skills for the second edition of the TrainABA Supervision Curriculum Series. The goal was to add a layer of physical (through code) technology to a book whose subject was a non-physical technology. Hence, it built on Skinner's vision of better communicating the technology of behavior analysis to general audiences. The technology for the book will always be publicly available, allowing for modifications by any users. The version control notes remain public, so anyone could see the process of creating the book. Collaboration and issue tracking is displayed publicly. In general, the distribution technology increases the likelihood that people will use current releases.

\LaTeX{} allowed modularization of the book's files. It separated files containing written content from those that generated the formatting. Any task list name, reference, or glossary item was tagged with a variable. The code ran through a compiler to assemble the PDF. The compiler's huge string of info messages was like witnessing live art. The creation process happened right before one's eyes. 

The aspect of building the book was inspired by stories of Skinner's lab at Harvard. As legends go, one could hear the symphony of operant chambers clicking and clacking throughout the halls. The stories differ but common ground is that Skinner's lab ran concurrent experiments. In this book, no live animals were used but the modular design allowed for many files to operate at the same time. \LaTeX{} runs procedurally, going step by step through every line of code and jumping to/from the files referenced. In Skinner's lab, experiments were modular such that whatever happened in one operant chamber without tampering with experiments in nearby operant chambers. For this book, the file system was modular such that any file could be added, removed, modified, or replaced without tampering with other files.
    
The books were assembled and written using Vim, a free and open source command line text editor. It was compiled using custom shell scripts in a Bash terminal. Git and GitHub were used for version control. The decision to use a command line editor was inspired by operant chambers. One need only peck at the keys to write, assemble, and distribute this book. No mouse, trackpad, eyes, or graphic user interface is involved. An experienced operator could complete the process blindfolded. The author suffered from tendonitis and frequent eyestrain from extensive computer use, so the command line technology was a welcome option.

While the specifics of the technology choices for this book may seem gratuitous, the purpose was to inspire behavior analysis professionals to indulge their curiosity in computer programming. The intended message is, roughly, that even an English major can learn to write code if the project is interesting. Perhaps others from non-programming backgrounds will consider this project as an invitation to download and tinker with code.

Please submit errors, additions, improvements, and suggested omissiongs using the GitHub Issue Tracker. Nothing posted in the modern day mead-halls of Facebook Groups will be read by those who maintain this book.

All readers may use, copy, modify, and distribute this book and its files. Hard copies are available. Custom builds are available for companies, universities, and others. Please contact CRDS for more information about how to use intellectual property for this book or make content contributions. The contact is postmaster(at)cumulativerecords.org.

The Behavior Analyst Certification Board provided a copyright license for use of the BCBA/BCaBA Task List, 4th and 5th editions. No reprinting of those materials are allowed without the express written consent of the BCBA/BCaBA. Statements of free and open source licensing of this book do not apply to the BCBA/BCaBA Task List, which is the sole property of the Behavior Analyst Certification Board. For more information, visit www.bacb.com.

Some of the content in this book builds upon work from contributors to its first edition. This book represents significant revision, rewriting, and reorganization, to the point that it is a completely different book representing original content. The first edition continues to be available through the GitHub repositories. It has been made publicly available under a Creative Commons 4.0 - Attribution - Sharealike - Noncommercial license since 2017.
\clearpage
